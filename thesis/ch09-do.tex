%%%%%%%%%%%%%%%%%%%%%%%%%%%%%%%%%%%%%%%%%%%%%%%%%%%%%%%%%%%%%%%%%%%%%%%%%%%
%
%			INVERSE SYNTHETIC APERTURE RADAR
%
%				  PhD Thesis
%
%		Stephen Simmons		simmons@ee.mu.oz.au
%
%	    Department of Electrical and Electronic Engineering
%	    University of Melbourne, Parkville, 3052, Australia
%
% Chapter 9:	Conventional Motion Estimators and the ML Solution
%
%		started first draft:	Wed 4 Jan 1995
%		finished first draft:	Fri 6 Jan 1995
%		submitted:		Fri 6 Jan 1995
%
%%%%%%%%%%%%%%%%%%%%% Copyright (C) 1995 Stephen Simmons %%%%%%%%%%%%%%%%%%


\chapter[Conventional and ML Motion Estimators]{Conventional 
Motion Estimators and the Maximum Likelihood Solution}
\label{do chp}

\bigletter Chapter \ref{sp chp} completed the analysis of the maximum
likelihood ISAR radial motion estimator which was derived in chapter
\ref{ml chp} and implemented efficiently in chapter \ref{ee chp}. 
Therefore, this chapter is an appropriate place to compare the maximum
likelihood solution with some of the conventional radial motion estimators
discussed in detail in chapter \ref{mc chp}.

The three motion estimators from chapter \ref{mc chp} reexamined here are
the cross-correlation methods of range profile alignment from section \ref{mc
sec:cc}, and the adaptive beamforming and phase gradient autofocus
algorithms of phase compensation from sections \ref{mc sec:ab} and \ref{mc
sec:pga} respectively.  The image-quality methods from section \ref{mc
sec:en} are not considered here because they are not related closely enough to
the maximum likelihood solution for the comparison to be worthwhile.

Reinterpreting conventional radial motion estimators in the light of the
maximum likelihood solution is a valuable exercise.  If there is a strong
connection between the estimator and the maximum likelihood solution, this
helps to provide a formal justification for what would otherwise be an {\em
ad hoc\/} or ``intuitively obvious'' approach.  If there is not a strong
connection between the estimator and the maximum likelihood solution, this
prompts the questions of whether there is some additional information that
the estimator uses, and whether this information could possibly be 
incorporated into the radial motion estimation model to give a better 
maximum likelihood estimator.

%%%%%%%%%%%%%%%%%%%%%%%%%%%%%%%%%%%%%%%%%%%%%%%%%%%%%%%%%%%%%%%%%%%%%%%%%%%%%
\section[The ML Estimator as a Cross-Correlation]{The Maximum Likelihood
Estimator as a Cross-Correlation}
\label{do sec:mlcc}

The maximum likelihood estimator $\rML$ from (\ref{ml eqn:J(r) uvp}) 
can be written in terms of the cross-correlation of the range profiles
formed from the two frequency responses.

From (\ref{ml eqn:J(r) uvp}), the maximum likelihood estimator is
\begin{equation}
\rML=\arg\,\min_{\r}\,\sum_{n=0}^{N-1} \left|a_n-b_n\e{jk_n\r}\right|^2
\end{equation}
Following appendix \ref{ee app:min using DFT}, this can be written in the
equivalent form
\begin{equation}
\rML=\arg\,\max_{\r}\,\real{\sum_{n=0}^{N-1}a_n\overline{b_n}\e{-jk_n\r}}
\end{equation}

The target's range profiles are the continuous-range Fourier series of the
discrete frequency responses.\footnote{See the footnote on page 
\pageref{ee ftn:crft}.} Therefore by defining the target's range profiles at
the two times as
\begin{equation}
s_1(r)=\sum_{n=0}^{N-1} a_n\,e^{jk_n r}
\end{equation}
and
\begin{equation}
s_2(r)=\sum_{n=0}^{N-1} b_n\,e^{jk_n r}
\end{equation}
the discrete frequency responses may be written in terms of the range
profiles using the formula for the discrete Fourier transform
\begin{equation}
a_n=\frac{1}{N}\sum_{n'=0}^{N-1}\left[s_1(r_{n'})\,e^{-jk_0r_{n'}}
\right]\,e^{-j\frac{2\pi}{N}nn'}
\end{equation}
\begin{equation}
b_n=\frac{1}{N}\sum_{n'=0}^{N-1}\left[s_2(r_{n'})\,e^{-jk_0r_{n'}}
\right]\,e^{-j\frac{2\pi}{N}nn'}
\end{equation}

Now define the cross-correlation of the two range profiles $s_1(t)$ and
$s_2(t)$ as
\begin{equation}
R_{12}(r)=\int_{-\frac{c}{4\df}}^{\frac{c}{4\df}}
s_1(r')\overline{s_2(r'+r)}\,dr'
\label{do eqn:R12(r)}
\end{equation}
where the limits of integration have been set to cover a total width of
$c/2\df$, equal to the extent $W_r$ of the ISAR image's range ambiguity window.  The
cross-correlation is periodic because the range profiles themselves are
periodic, so any upper and lower limits differing by $W_r$ could have been used
in this integral.

Writing the range profiles in (\ref{do eqn:R12(r)}) in terms of the discrete
frequency responses shows that
\begin{eqnarray}
R_{12}(r)
&=&\int_{-\frac{c}{4\df}}^{\frac{c}{4\df}}
\sum_{n=0}^{N-1} a_n\,e^{jk_nr'}
\sum_{n'=0}^{N-1} \overline{b_{n'}}\,e^{-jk_{n'}(r'+r)}\,dr'	\nn\\
&=&\sum_{n=0}^{N-1} \sum_{n'=0}^{N-1} a_n\overline{b_{n'}}\,e^{-jk_{n'}r}
\int_{-\frac{c}{4\df}}^{\frac{c}{4\df}}
e^{-j\dk r'(n'-n)}\,dr'	
\end{eqnarray}
where, as usual, each $k_n$ is $k_0+n\dk$. Now the integral can be
evaluated to give a Kronecker delta
\begin{eqnarray}
\int_{-\frac{c}{4\df}}^{\frac{c}{4\df}}e^{-j\dk r'(n'-n)}\,dr'	
&=&\frac{c}{2\df}\,\sinc(n'-n)\nn\\
&=&\frac{c}{2\df}\,\delta_{n'n}
\end{eqnarray}
because $n$ and $n'$ are always integral.  Then the cross-correlation is
equal to
\begin{eqnarray}
R_{12}(r)
&=&\frac{c}{2\df}\,\sum_{n=0}^{N-1} \sum_{n'=0}^{N-1} 
a_n\overline{b_{n'}}\,e^{-jk_{n'}r}\,\delta_{n'n}\nn\\
&=&\frac{c}{2\df}\,\sum_{n=0}^{N-1} a_n\overline{b_n}\,e^{-jk_nr}
\end{eqnarray}
This shows that the maximum likelihood estimator $\rML$ of $\dr$ can be
written in terms of the cross-correlation of the range profiles in 
(\ref{do eqn:R12(r)}) as
\begin{equation}
\label{do eqn:cc rML}
\rML=\arg\,\max_{\r}\,\real{R_{12}(r)}
\end{equation}

%%%%%%%%%%%%%%%%%%%%%%%%%%%%%%%%%%%%%%%%%%%%%%%%%%%%%%%%%%%%%%%%%%%%%%%%%%%%%
\section{Range Profile Alignment}
\label{do sec:rpa}

The cross-correlation method of range profile alignment presented in
section \ref{mc sec:cc} is similar to algorithms \ref{ee alg:min using DFT}
and \ref{ee alg:min using CZT} in chapter \ref{ee chp}.  These algorithms
accomplished the approximate global minimization of the cost function
$J(\r)$ used to find the maximum likelihood estimate $\rML$ of $\dr$, the
radial distance moved by the target.

In this section, it is shown that $\rML$ may be expressed as the maximum of
a cross-correlation.  This cross-correlation is slightly different from the
cross-correlation used by Chen and Andrews for range profile alignment 
\cite{Che80a}.  The reasons for this difference are discussed, and it is
demonstrated that the cross-correlation used by Chen and Andrews sacrifices
accuracy for robustness to changes in the target's aspect angle.

Equation (\ref{do eqn:cc rML}) is the form of the maximum likelihood
estimator of $\dr$ capable of estimating the target's radial movement
unambiguously to a fraction of a wavelength.  An accuracy this good is not
needed for range profile alignment.  Following the proof of algorithm
\ref{ee alg:min using CZT}, taking the absolute value of $R_{12}(r)$ rather
than the real part gives an estimate that is within a wavelength of 
$\rML$.  Therefore
\begin{equation}
\rML\approx\arg\,\max_{\r}\,\left|R_{12}(r)\right|
\end{equation}

The actual form of the cross-correlation used by Chen and Andrews 
for range profile alignment is
\begin{equation}
R'_{12}(r)=\int_{-\frac{c}{4\df}}^{\frac{c}{4\df}}
\left|s_1(r')\right|\,\left|s_2(r'+r)\right|\,dr'
\label{do eqn:R'12(r)}
\end{equation}
which is different from the $\left|R_{12}(r)\right|$ derived from the maximum
likelihood estimator.  The implications of this are discussed next.

%%%%%%%%%%%%%%%%%%%%%%%%%%%%%%%%%%%%%%%%%%%%%%%%%%%%%%%%%%%%%%%%%%%%%%%%%%%%%
\subsection{Cross-Correlation and Rotating Targets}

At first glance, there would seem to be no reason not to use the complex 
form of the cross-correlation,\footnote{Indeed Xu {\em et al.\/} did use
the complex form of the cross-correlation in \protect\cite{Xu89} and
\protect\cite{Xu90b}.  In \protect\cite{Xu90a}, they switched back to
cross-correlating the magnitudes of the range profiles.} except perhaps if
it were not known to be related to the maximum likelihood estimator. 
However, there is a simple explanation of why Chen and Andrews used the
particular form of cross-correlation that they did, and it is related to
the target's rotation and the low resolution of their fast correlation.

When the target's frequency response is measured at two different times, as
long as the target's rotation is negligible, the responses differ by a
frequency-dependent phase shift that is directly proportional to the radial
distance moved by the target.  If the target is moving and rotating at
approximately constant radial and angular velocities, this suggests two
possible strategies for measuring the target's radial position.  The first
is to measure the distance the target moves between successive range 
profiles.  The second is to choose the target's range profile at the middle
of the ISAR imaging period as a reference, and then for any other range
profile, to measure the radial distance the target has moved relative to
the reference profile.

The first strategy ensures that the target's rotation is negligible because
the time between consecutive range profiles is very short, but the estimates
of each radial movement must be made very accurately because the target only
moves a very short distance in the time between successive profiles, and
these differences must be summed to give the overall change in the target's
position.
The second strategy measures radial movements that are up to $M/2$ times
greater than the radial movements using the first strategy, so estimates
of the radial movement need not be so accurate.  The disadvantage is that
the target rotates through an angle up to $M/2$ times larger so its rotation
can no longer be considered negligible.

Before discussing which is the optimum strategy to use, the effect of the
target's rotation must be considered.  When the target does not rotate, the
effect is simple: the range profile is offset by the distance moved and the
phase changes by an amount proportional to the distance moved.  Both effects
are measured by cross-correlating the complex range profiles.

When the target moves radially and rotates, the radial motion produces the
effects described above but the rotation affects the profile slightly
differently.  Different scatterers move different radial distances
depending on their position relative to the target's centre of rotation. 
These radial distances are small relative to the size of a range bin but
large relative to the wavelength.  Therefore the phase of each scatterer in
a range bin changes by a different amount from the phases of other
scatterers in that range bin.  If the range bin contains many scatterers,
the sum of the scatterer's reflections produces a significant change in
the range bin's amplitude as well as its phase.  If, however, the range bin
contains only a single scatterer, or one scatterer with a much greater
amplitude than the rest, the amplitude of that range bin will be largely
unaffected but its phase may be changed significantly.  The overall effect
of the target's rotation is to change the phases of different range bins in
different ways but it leaves their magnitudes largely untouched.  Since
man-made targets tend to have a high dynamic range with a small number of
strong point scatterers, this suggests that the effect of the target's
rotation can be mitigated by cross-correlating the magnitudes of the range
profiles.

The disadvantage of cross-correlating the magnitudes of the profiles is
that the phase carries much of the information about the target's radial
movement.  Once the phase information has been destroyed by taking the
magnitude, the estimates of the cross-correlation are less accurate.  The
loss of accuracy from neglecting the radial movement's phase changes has to
be balanced against the increase of accuracy from removing some of the
deleterious effects of the target's rotation.

The choice of which of the two strategies mentioned above to use depends on how the
cross-correlation is implemented.  Chen and Andrews used cross-correlation
implemented as a fast correlation based on the fast Fourier transform.  The
resolution of the fast correlation is equal to the resolution of the
discrete range profiles, which in turn is equal to the width of a range bin
or $\Delta r_r=c/2N\df$.  

As an example, consider a target moving through a third of a range bin 
between each range profile.  Cross-correlating adjacent range profiles with
an accuracy of one range bin would conclude that the target has no
radial motion thoughout the ISAR imaging period.  However, using the centre
range profile as a reference, cross-correlating each range profile with the
reference would conclude that the target moves through one range bin for
every three range profiles across from the reference.  

Therefore, if the FFT-based fast correlation were used, the only viable
option would be to measure the radial motion relative to one reference range
profile.  As the target's rotation is not negligible, cross-correlation
of the magnitudes of the range profiles must be used.

If, however, the chirp-Z transform were used as in algorithm \ref{ee
alg:min using CZT}, the radial movement could be estimated down to about a
wavelength.  Now the target's rotation may be considered negligible and the
model under which the maximum likelihood estimator derived is valid.  These
are suitable conditions for the complex cross-correlation of adjacent range
profiles.  

%============================================================================
\begin{figure}\centering
\caption[Cross-correlation of range profiles for a target with a small
rotation]{In this figure, the two range profiles are from consecutive
frequency responses so the target's rotation is negligible.  The 
cross-correlation of the complex range profiles is shown in grey and the
cross-correlation of their magnitudes is in black.}
\label{do fig:cc I}

\setlength{\unitlength}{1cm}
\psset{unit=1cm}
\begin{pspicture}(-1,-1)(11,6)

\psset{xunit=1cm,yunit=1cm}
\pscustom[linewidth=0.04pt]{\code{
save
28.3 dup scale 
% Make a transformation matrix prior to drawing a graph.  This produces a
% matrix named txmat.
% Stack:  xo  yo  dx  dy  N   =>   N
% txmat = [ dx/(N-1)  0  0  dy/255  xo  yo  ]
/maketxmat {
	exch 255 div exch  		% xo yo dx dy/255 N
	dup 1 sub 4 -1 roll exch div	% xo yo dy/255 N dx/(N-1)
	[ exch 0 0                      % xo yo dy/255 N [ dx/(N-1) 0 0
	6 -1 roll 8 -2 roll ]		% N [ dx/(N-1) 0 0 dy/255 xo yo ]
	/txmat exch def			% N {txmat=[dx/(N-1) 0 0 dy/255 xo yo]}
} def
% Plot a segment of a graph.  It is assumed that the data is stored in an
% array 'dataarray' and the transformation matrix 'txmat' has been defined.
% Stack: x'(i-1) y'(i-1) i  =>  x'(i) y'(i)  
/plotsegment { dup dataarray exch get txmat transform 2 copy 6 2 roll
	newpath moveto lineto %0.2 setlinewidth 
	stroke } def
% Plot a graph.  
% Stack:  xo yo dx dy N  =>  -
% The graph extends from (x0,yo) to (xo+dx,yo+dy) with N evenly spaced
% data points.  The points are stored as N hex bytes after the command 
% 'plotgraph'.  00 is at the bottom of the y axis and FF at the top.
/plotgraph {
	gsave %  graphcolor setgray
	maketxmat dup currentfile exch string readhexstring pop
	/dataarray exch def
	0 dataarray 0 get txmat transform
	1 dup 5 -1 roll 1 sub {plotsegment} for
	pop pop grestore
} def
%n=3;o=1;a=data(:,n);b=data(:,n+o);
%m=abs(ifft(fft(abs(ifft(a))).*conj(fft(abs(ifft(b))))));
%p=abs(ifft(a.*conj(b)));plot(fftshift([m/sqrt(m'*m),p/sqrt(p'*p)]))
%fftshift(m/sqrt(m'*m)),fftshift(p/sqrt(p'*p))
% Scaled so that FF=0.5, 00=0.0
% Complex X-C offset by 1 range profile  (109,0.0036) to ( 65,0.4009)
0.5 setgray 0 0 10 6 128  plotgraph
080E1A0C0729141F07371D261331111D070D460D291610290E1C1C0C07150E2B
13280F1208311D16061A140C34263E302C442E222B3330300B2F27322A400C3D
98CD5C635924631D4130792B3643373843290F0F130E0C0B1B2820221A1C2C1D
2712050D16191311141B20191401050D09071B3720202316231E2827090D131E
% Magnitude X-C offset by 1 range profile  (110,0.0661) to ( 65,0.1365)
0 setgray 0 0 10 6 128  plotgraph
2527262626272625272623262524242425252625262625252525222421252425
24292729292827272627262B2A2D2F2F35373439393737343736393741404540
454540423A3F3D3B373A3A3A37383A34332E302E2F2B2F2D2C2C292A27262528
2627262727262626262524252423212324262424242525232523252527252527
restore
}}
% Now the axes, tick marks and their labels
\psline[linecolor=black,linewidth=1.5pt]{->}(-0.5,0)(10.5,0)
\psline[linecolor=black,linewidth=1.5pt]{->}(5,0)(5,5)
\uput[r](10.5,0){$r$}
\uput[u](5,5){$R_{12}(r)$}
\uput[d](0,0){$-\displaystyle\frac{c}{4\Delta f}$}
\uput[d](10,0){$\displaystyle\frac{c}{4\Delta f}$}
\psline[linecolor=black,linewidth=1pt]{-}(0,0)(0,0.15)
\psline[linecolor=black,linewidth=1pt]{-}(10,0)(10,0.15)
\end{pspicture}
\end{figure}
%============================================================================
%============================================================================
\begin{figure}\centering
\caption[Cross-correlation of range profiles for a target with a large
rotation]{In this figure, the two range profiles are separated by half the
ISAR imaging period so the target's rotation is not negligible.  The 
cross-correlation of the complex range profiles is shown in grey and the
cross-correlation of their magnitudes is in black.}
\label{do fig:cc II}

\setlength{\unitlength}{1cm}
\psset{unit=1cm}
\begin{pspicture}(-1,-1)(11,6)

\psset{xunit=1cm,yunit=1cm}
\pscustom[linewidth=0.04pt]{\code{
save
28.3 dup scale 
% Make a transformation matrix prior to drawing a graph.  This produces a
% matrix named txmat.
% Stack:  xo  yo  dx  dy  N   =>   N
% txmat = [ dx/(N-1)  0  0  dy/255  xo  yo  ]
/maketxmat {
	exch 255 div exch  		% xo yo dx dy/255 N
	dup 1 sub 4 -1 roll exch div	% xo yo dy/255 N dx/(N-1)
	[ exch 0 0                      % xo yo dy/255 N [ dx/(N-1) 0 0
	6 -1 roll 8 -2 roll ]		% N [ dx/(N-1) 0 0 dy/255 xo yo ]
	/txmat exch def			% N {txmat=[dx/(N-1) 0 0 dy/255 xo yo]}
} def
% Plot a segment of a graph.  It is assumed that the data is stored in an
% array 'dataarray' and the transformation matrix 'txmat' has been defined.
% Stack: x'(i-1) y'(i-1) i  =>  x'(i) y'(i)  
/plotsegment { dup dataarray exch get txmat transform 2 copy 6 2 roll
	newpath moveto lineto %0.2 setlinewidth 
	stroke } def
% Plot a graph.  
% Stack:  xo yo dx dy N  =>  -
% The graph extends from (x0,yo) to (xo+dx,yo+dy) with N evenly spaced
% data points.  The points are stored as N hex bytes after the command 
% 'plotgraph'.  00 is at the bottom of the y axis and FF at the top.
/plotgraph {
	gsave %  graphcolor setgray
	maketxmat dup currentfile exch string readhexstring pop
	/dataarray exch def
	0 dataarray 0 get txmat transform
	1 dup 5 -1 roll 1 sub {plotsegment} for
	pop pop grestore
} def
%n=3;o=71;a=data(:,n);b=data(:,n+o);
%m=abs(ifft(fft(abs(ifft(a))).*conj(fft(abs(ifft(b))))));
%p=abs(ifft(a.*conj(b)));plot(fftshift([m/sqrt(m'*m),p/sqrt(p'*p)]))
%fftshift(m/sqrt(m'*m)),fftshift(p/sqrt(p'*p))
% Scaled so that FF=0.5, 00=0.0
% Complex X-C offset by 71 range profiles  ( 85,0.0036) to (102,0.2643)
0.5 setgray 0 0 10 9 128  plotgraph
203D0F3206272E152C231523173015212C1418122C211A2D132D19231C241717
121E17201F3D0A2E2C151908101B1B150E101E1040080B0D1B2013120C0F1726
351A0B2D18181A1F131A191A09032321241621571D013807204C204945415148
3B643151476F871968224B371C580E70371837412316173B4B0F2D2224153C23
% Magnitude X-C offset by 71 range profiles  ( 68,0.705) to (104,0.1276)
0 setgray 0 0 10 9 128  plotgraph
2A2B2B2B2A2A2C28262728282A292B2A292827282728272A282A2829282B2A2A
2928282728282828282829282525282526272727282524252524252525252525
262723262426252525242827272829282A27282B28282B292B2F30363138373A
3B3A3D3739393F38413D403E403B383A3733343B3535313231312E302F2D2D2A
restore
}}
% Now the axes, tick marks and their labels
\psline[linecolor=black,linewidth=1.5pt]{->}(-0.5,0)(10.5,0)
\psline[linecolor=black,linewidth=1.5pt]{->}(5,0)(5,5)
\uput[r](10.5,0){$r$}
\uput[u](5,5){$R_{12}(r)$}
\uput[d](0,0){$-\displaystyle\frac{c}{4\Delta f}$}
\uput[d](10,0){$\displaystyle\frac{c}{4\Delta f}$}
\psline[linecolor=black,linewidth=1pt]{-}(0,0)(0,0.15)
\psline[linecolor=black,linewidth=1pt]{-}(10,0)(10,0.15)
\end{pspicture}
\end{figure}
%============================================================================


%%%%%%%%%%%%%%%%%%%%%%%%%%%%%%%%%%%%%%%%%%%%%%%%%%%%%%%%%%%%%%%%%%%%%%%%%%%%%
\subsection{Example of Large and Small Rotations}

The difference between the complex and magnitude forms of the
cross-correlation can be seen in figures \ref{do fig:cc I} and \ref{do
fig:cc II}.  These are calculated from frequency responses of the same 737
aircraft used in figure \ref{ml fig:J(r) pictures}.  In figure \ref{do
fig:cc I}, the two range profiles are from consecutive frequency responses
so the target's rotation is negligible.  The peak in the cross-correlation 
is much sharper for the cross-correlation of the complex profiles
than for the cross-correlation of their magnitudes.

In figure \ref{do fig:cc II}, the two range profiles are separated by half
the ISAR imaging period, so the target's rotation is now significant.  The
cross-correlation of the complex profiles still shows a sharp peak, but new
sharp peaks with amplitudes that are close to that of the maximum
have appeared.  These new peaks are widely separated so if any of their 
amplitudes were slightly higher, the estimate of $\dr$ would be very
inaccurate.  In comparison, the cross-correlation of the magnitudes of the
range profiles is almost unchanged in shape from that in figure \ref{do
fig:cc I}, indicating how cross-correlating the magnitudes of the profiles
has reduced the effect of the target's rotation.

The question of how much the target can rotate before its rotation is no
longer negligible for the purposes of motion estimation is considered in
section \ref{rmc sec:me}.

%%%%%%%%%%%%%%%%%%%%%%%%%%%%%%%%%%%%%%%%%%%%%%%%%%%%%%%%%%%%%%%%%%%%%%%%%%%%%
\section{Phase Compensation}
\label{do sec:pc}

Just as range profile alignment using cross-correlation is related to the
approximate global minimization algorithms of section \ref{ee sec:agm},
phase compensation using adaptive beamforming and the phase gradient
autofocus are related to the exact local minimization algorithms of section
\ref{ee sec:elm}.

As shown in equation (\ref{do eqn:cc rML}), finding the maximum likelihood
estimator of $\dr$ is equivalent to maximizing the cross-correlation 
\begin{equation}
\rML=\arg\,\max_{\r}\,\real{R_{12}(r)}
\end{equation}

Because this correlation has a very high frequency modulation, the
maximization can be carried out in two parts.  First the magnitude is
maximized, as in section \ref{do sec:rpa}.  Then the value of $r$ that
maximizes $\left|R_{12}(r)\right|$ is adjusted over a range of half a
wavelength.  The optimum value of $r$ rotates the phase so that $R_{12}(r)$ 
has no imaginary component and the real component is positive.  This second
part, to adjust the phase of the cross-correlation, is equivalent to the
operation of phase compensation algorithms.

However, dividing the maximization into two parts like this pretends that
the magnitude of the cross-correlation does not change as $r$ varies over
half a wavelength.  The magnitude does change, not by much if $r$ is close
to $\rML$, but by significantly more if $r$ is larger than a small fraction
of a range bin away from $\rML$.  From the derivation of algorithm \ref{ee
alg:min with cm} in appendix \ref{ee app:min with cm}, the value of $r$
should be adjusted so that
\begin{equation}
\imag{\sum_{n=0}^{N-1} k_na_n\overline{b_n}\e{-jk_nr}}=0
\end{equation}
Ignoring the changing magnitude for small variations in $r$ is equivalent
to solving the slightly different condition
\begin{equation}\label{do eqn:condit}
\imag{\sum_{n=0}^{N-1} a_n\overline{b_n}\e{-jk_nr}}=0
\end{equation}
These give similar estimates of $\rML$, particularly if the initial
estimate from maximizing the magnitude is close to $\rML$.  This is because
the radar has a narrow relative bandwidth so $k_n\approx\overline{k}$ for
all frequencies in the stepped-frequency waveform.

From (\ref{do eqn:cc rML}), equation (\ref{do eqn:condit}) is equivalent 
to adjusting $r$ so that
\begin{equation}
\imag{\int_{-\frac{c}{4\df}}^{\frac{c}{4\df}}
s_1(r')\overline{s_2(r'+r)}\,dr'}=0
\end{equation}
Now suppose that range profile alignment has been performed so that most of
the radial motion $\dr$ has been corrected, except for a small amount
$\delta r$ less than a range bin.  Then the value of $\delta r$ modulo half
a wavelength is found by solving the above equation for values of $r$
between $-\lambda/4$ and $\lambda/4$.  Over such a small variation in $r$,
\begin{equation}
s_2(r'+r)\approx e^{-j\overline{k}r}\,s_2(r')
\end{equation}
so that $\delta r$ is found by solving
\begin{equation}
\imag{e^{-j\overline{k}r}\,\int_{-\frac{c}{4\df}}^{\frac{c}{4\df}}
s_1(r')\overline{s_2(r')}\,dr'}=0
\end{equation}
which is equivalent to
\begin{equation}
\label{do eqn:nq mle}
\overline{k}r=
-\arg\left\{\int_{-\frac{c}{4\df}}^{\frac{c}{4\df}}
s_1(r')\overline{s_2(r')}\,dr'\right\}
\end{equation}
This is very similar to the maximum likelihood estimator of $\delta r$ on which
the phase gradient autofocus and adaptive beamforming methods of phase
compensation are based.


To see how phase gradient autofocus is related to (\ref{do eqn:nq mle}),
the phase gradient operator in (\ref{mc eqn:phidot sum}) is just one of
many phase gradient or phase difference operators that can be used with
phase gradient autofocus giving images of similar quality.  Another phase
difference operator suggested by Wahl {\em et al.\/} is \cite[eq. (5)]{Wah94}
\begin{equation}
\Delta\phi=
\arg\left\{\int_{-\frac{c}{4\df}}^{\frac{c}{4\df}}
s_1(r')\overline{s_2(r')}\,dr'\right\}
\end{equation}
which is the same as (\ref{do eqn:nq mle}). 

Therefore, the phase gradient autofocus can use an estimator of the phase
difference between profiles that is very closely related to the maximum
likelihood estimator, thus justifying its use.  

This is only a small part of the complete phase gradient autofocus
algorithm, with the shifting and windowing parts not yet discussed.  When
expressed in ISAR language instead of the phase gradient autofocus's native SAR language, these
operations help to reduce the phase noise associated with the target's
rotation.  It seems likely that the shifting and windowing parts of the
phase gradient autofocus could be used with the maximum likelihood phase
estimator to produce estimates of $\dr$ that are more robust to the
target's rotation.


Adaptive beamforming is a more difficult phase estimator to relate to the
maximum likelihood solution because the dominant scatterer does no
averaging at all to determine the reference phases in each range profile. 
The effects of noise are reduced somewhat in Haywood's modification
\cite{Hay92a} of the dominant scatterer algorithm by ensuring that the
average amplitude in the reference range bin is above a threshold, but this
is still not an optimum solution.  

To try and balance the reduction of rotation-related phase noise obtained
by selecting a reference range bin containing only a single scatterer, it
is possible that the dominant scatterer algorithm could be improved by
selecting several range bins to use as references and averaging the phase
adjustments across these ranges.  These range bins could be chosen as the
range bins with the least variance and greatest amplitude, or similar such
criteria.  This is only a tentative idea, and more development is needed to
determine whether it is worthwhile.
