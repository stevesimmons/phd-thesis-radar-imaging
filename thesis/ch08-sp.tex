%%%%%%%%%%%%%%%%%%%%%%%%%%%%%%%%%%%%%%%%%%%%%%%%%%%%%%%%%%%%%%%%%%%%%%%%%%%
%
%			INVERSE SYNTHETIC APERTURE RADAR
%
%				  PhD Thesis
%
%		Stephen Simmons		simmons@ee.mu.oz.au
%
%	    Department of Electrical and Electronic Engineering
%	    University of Melbourne, Parkville, 3052, Australia
%
% Chapter 8:	Statistical Properties of the ML Motion Estimator
%
%		started first draft:	6 Nov 1994
%		finished first draft:	12 Nov 1994
%		started second draft:	Fri 6 Jan 1995
%		submitted:		Mon 9 Jan 1995
%
%%%%%%%%%%%%%%%%%%%%% Copyright (C) 1995 Stephen Simmons %%%%%%%%%%%%%%%%%%

\chapter[Statistical Properties of the ML Motion Estimator]{Statistical
Properties of the Maximum Likelihood Motion Estimator}
\label{sp chp}

\bigletter Broadly speaking, estimators may be divided into two classes,
{\em simple\/} and {\em complex\/}.  If simple estimators are defined as
those which can be expressed as an explicit function of the measurements,
$\upWH=f(\md)$, then complex estimators may be defined as the remainder that
are not simple. Typically, complex estimators can only be expressed as
implicit functions $f(\upWH,\md)=0$ which require numerical, rather than
analytical, solutions.

When analysing the performance of simple estimators, the mean and variance
may be calculated exactly from the explicit functions giving the estimator in
terms of the measurements.  Complex estimators, by their very nature,
usually do not allow such exact performance analyses.  In such cases, 
approximate or asymptotic results for large samples are often calculated.

Since the definition of a maximum likelihood estimator takes the form of
the location of the maximum of the likelihood function, many maximum
likelihood estimators fall into the class of complex estimators.  This
makes exact analyses of their performance quite difficult.  Nevertheless,
because maximum likelihood estimators are asymptotically efficient under
fairly general regularity conditions, asymptotic results like the \CR bound
are very commonly used.

For many maximum likelihood estimators, the asymptotic behaviour is almost
trivial to derive while the corresponding non-asymptotic results are almost
impossibly complicated.  As a result, the asymptotic results are often 
applied to estimators using small data sets without checking how rapidly the
estimator's behaviour becomes asymptotic.


This chapter is a detailed examination of the statistical properties of the
maximum likelihood motion estimator $\rML$ derived in chapter~\ref{ml chp}.
The regularity conditions for the estimator's asymptotic normality are
established.  The Fisher information matrix is used to calculate the \CR
lower bound on the variance of any unbiased estimator $\rWH$ of $\dr$.  The
maximum likelihood estimator $\rML$ is shown to be unbiased, hence the \CR
bound on any $\rWH$ applies to $\rML$.  The analysis of $\rML$ concludes
with some tentative steps towards non-asymptotic statistical properties, and
a discussion of why the estimator's variance is not the most important
criterion for radar imaging.

The chapter concludes with analyses of the statistical properties of the
maximum likelihood estimators of the noise variance $\nv$ and the target's
frequency response $\sML{n}$.

%%%%%%%%%%%%%%%%%%%%%%%%%%%%%%%%%%%%%%%%%%%%%%%%%%%%%%%%%%%%%%%%%%%%%%%%%%%%%
\section{Conditions for Asymptotic Normality}
\label{sp sec:normality}

The asymptotic normality of the maximum likelihood estimator of $\dr$ will
be established using the sufficient but not necessary regularity conditions
of theorem~\ref{mp thm:normality}.  This requires that the log-likelihood
function $\ln p(\md;\uvp)$ has well-defined first and second partial
derivatives, and the expectations of the first partial derivatives be zero.

The first and second partial derivatives of $\ln p(\md;\uvp)$ are given in
appendix \ref{sp app:FIM} while deriving the Fisher information matrix.
The expectation of the first partial derivative of $\ln p(\md;\uvp)$ with
respect to $\dr$ is also zero because
\begin{eqnarray}
\E{\frac{\del\ln p(\md;\uvp)}{\del\dr}}
&=&\E{j\frac{1}{\nv}\sum_{n=0}^{N-1}k_n\left( 
b_n\conj{s_n}\e{jk_n\dr}-\conj{b_n}s_n\e{-jk_n\dr}\right)}\nn\\
&=&j\frac{1}{\nv}\sum_{n=0}^{N-1}k_n
\left(\left|s_n\right|^2-\left|s_n\right|^2\right)\nn\\
&=&0
\end{eqnarray}

Since the regularity conditions are satisfied, the maximum likelihood
estimator $\rML$ of $\dr$ is asymptotically normal.\footnote{Strictly
speaking, this conclusion is not valid because it has applied Kay's theorem
about the asymptotic normality of the scalar maximum likelihood estimator
to a vector estimator.  

Rigorous results concerning the asymptotic normality of the maximum
likelihood estimator are hard to find in the literature.  Kay states a
theorem regarding asymptotic normality \cite[thm 7.3]{Kay93} but this
begins ``If \ldots $p(\md;\uvp)$ \ldots satisfies some `regularity'
conditions \ldots'', never stating what these conditions might be.  Rao
\cite[section 5f]{Rao65} considers only the assumptions required for
asymptotic normality of a scalar maximum likelihood estimator.  Even Stuart
and Ord in {\em Kendall's Advanced Theory of Statistics\/}
\protect\cite{Stu91} do not state these regularity conditions; instead the
reader is referred to a 1943 paper by Wald.

Having said this, there is nothing about $p(\md;\uvp)$ to suggest that
$\rML$ is not asymptotically normal.  On the other hand, the $\sML{n}$ are
not asymptotically normal, and the reason for this is discussed at the end
of the chapter.}  That is, as the number of frequencies sampled in each
frequency response tends to infinity, the expected value of $\rML$ is $\dr$
\begin{equation}
\E{\rML}\to\dr\qquad\hbox{as $N\to\infty$}
\end{equation}
and the variance of $\rML$ tends to the \CR bound
\begin{equation}
\Var{\rML}\to\left(\invinfmat{\uvp}\right)_{22}\qquad\hbox{as $N\to\infty$}
\end{equation}
The \CR bound for $\rML$ is $(\invinfmat{\uvp})_{22}$, the second element 
on the diagonal of $\invinfmat{\uvp}$, because $\dr$ is the second element 
of $\uvp$.

%%%%%%%%%%%%%%%%%%%%%%%%%%%%%%%%%%%%%%%%%%%%%%%%%%%%%%%%%%%%%%%%%%%%%%%%%%%%%
\section{\CR Bounds for Unbiased Estimators of $\dr$}

In this section, a lower bound on the variance of any unbiased estimator of
the radial motion $\rWH$ of a radar target is obtained using the \CR bound. 
This derivation is done twice, once for the case when $\dr$ is the only
unknown parameter, and once for the more realistic case when $\dr$ is
estimated along with the noise variance $\nv$ and the target's
frequency response $s_n$.  A comparison of the resulting \CR bound illustrates
why assuming that $\dr$ is the only unknown gives different statistical
properties for the two estimators.

%%%%%%%%%%%%%%%%%%%%%%%%%%%%%%%%%%%%%%%%%%%%%%%%%%%%%%%%%%%%%%%%%%%%%%%%%%%%%
\subsection{The \CR Bound for Estimating $\dr$ Only}
\label{sp sec:CRB1}

The likelihood function for estimating $\dr$ only when the noise variance
$\nv$ and the target's reflectivities $s_n$ are known exactly, is given by
equation (\ref{ml eqn:p(md;dr) with J(r)})
\begin{equation}
p(\md;\dr)=\frac{2^N}{\left(\pi\nv\right)^N}\,\exp\left(-\frac{1}{2\nv}
\ds\sum_{n=0}^{N-1}\left|a_n-b_n\e{jk_n\dr}\right|^2\right)
\end{equation}
Since only a single parameter $\dr$ is unknown, the Fisher information matrix
is scalar, and its inverse is its reciprocal.  The information is
\begin{eqnarray}
I(\dr)&=&-\E{\frac{\del^2\ln p(\md;\dr)}{\del\dr^2}}\nn\\
&=&\frac{1}{\nv}\ds\sum_{n=0}^{N-1}\real{k_n^2\E{a_n\conj{b_n}}\e{-jk_n\dr}}\nn\\
&=&\frac{1}{\nv}\ds\sum_{n=0}^{N-1}k_n^2\left|s_n\right|^2
\end{eqnarray}
so the \CR bound on the variance of any unbiased estimator $\rWH$ of $\dr$
is
\begin{equation}
\Var{\rWH}\geq\frac{1}{I(\dr)}
=\frac{\nv}{\ds\sum_{n=0}^{N-1}k_n^2\left|s_n\right|^2}
\end{equation}

%%%%%%%%%%%%%%%%%%%%%%%%%%%%%%%%%%%%%%%%%%%%%%%%%%%%%%%%%%%%%%%%%%%%%%%%%%%%%
\subsection{The \CR Bound for Estimating $\dr$ With Other Parameters}
\label{sp sec:CRB2}


Chapter~\ref{ml chp} showed that the proper approach to estimating $\dr$
was to include the noise variance and the target reflectivities along with
$\dr$ in a vector of unknown parameters
\begin{equation}\label{sp eqn:uvp}
\uvp=\left[\nv, \dr, s_0, s_1, \ldots, s_{N-1}\right]^T
\end{equation}
As it happens, the expression for $\rML$ in (\ref{ml eqn:J(r) uvp}) is
independent of $\nv$ and the $s_n$, and is the same estimator as
(\ref{ml eqn:rML}) where $\dr$ was the only unknown parameter.

Although the estimators of $\dr$ for the two cases are identical, this does
not mean that their statistical performances are identical.  An estimator's
performance depends on both the function giving the estimator in terms of
the measurements and the model from which the estimator is derived.  The
model involving the multiple unknowns is different from the model involving
$\dr$ as the only unknown, so the statistical performances of the two $\rML$
estimators may well be different.  

The \CR bound on the covariance matrix $\covmat{\uvpWH}$ of any unbiased 
estimator $\uvpWH$ of $\uvp$ requires the Fisher information matrix 
$\infmat{\uvp}$.  
The elements of $\infmat{\uvp}$ are calculated in appendix~\ref{sp app:FIM}
using the rules for complex differentiation summarized in
section~\ref{mp sec:complex}.  When the elements of $\uvp$ are ordered
as in (\ref{sp eqn:uvp}), the information matrix is the $(N+2)\times(N+2)$ 
matrix
\begingroup
\def\te#1{-j\frac{1}{2}k_{#1}\conj{s_{#1}}\vphantom{\ds\frac{1}{1}}}
\def\be#1{j\frac{1}{2}k_{#1}s_{#1}\vphantom{\ds\frac{1}{1}}}
\def\re{\ds\sum_{n=0}^{N-1}k_n^2\left|s_n\right|^2}
\begin{equation}
\infmat{\uvp}=
\frac{2}{\nv}
\left[\begin{array}{cccccc}
N/\nv	&0	&0	&0	&\cdots	&0 	\\
0	&\re	&\te0	&\te1	&\cdots	&\te{N-1}\\
0	&\be{0}	&1	&0	&\cdots	&0	\\
0	&\be{1}	&0	&1	&\cdots	&0	\\
\vdots	&\vdots	&\vdots	&\vdots	&\ddots	&\vdots \\
0	&\be{N-1}&0	&0	&\cdots	&1
\end{array}\right]
\end{equation}
\endgroup
This has the block structure
\begin{equation}
\infmat{\uvp}=\frac{2}{\nv}\left[\begin{array}{c|c}
N/\nv & \vect{0} \\
\hline
\vect{0} & \vect{M}
\end{array}\right]
\end{equation}
where $\vect{M}$ is
\begin{equation}
\def\z#1{z_{#1}}
\def\zc#1{\conj{z_{#1}}}
\def\zt{\ds\sum_{n=0}^{N-1}\left|z_n\right|^2}
\vect{M}=\left[\begin{array}{ccccc}
4\zt	&\zc0	&\zc1	&\cdots	&\zc{N-1}\\
\z0	&1	&0	&\cdots	&0	\\
\z1	&0	&1	&\cdots	&0	\\
\vdots	&\vdots	&\vdots	&\ddots	&\vdots \\
\z{N-1}&0	&0	&\cdots	&1
\end{array}\right]
\end{equation}
with $z_n=j\frac{1}{2}k_ns_n$.

The inverse of $\infmat{\uvp}$ has the form
\begin{equation}
\invinfmat{\uvp}=\frac{\nv}{2}\left[\begin{array}{c|c}
\nv/N & \vect{0} \\
\hline
\vect{0} & \vect{M}^{-1}
\end{array}\right]
\end{equation}
It can be shown (in appendix~\ref{sp app:M-1}) that the inverse of $\vect{M}$ 
is
\begin{equation}
\vect{M}^{-1}=\frac{1}{3\ds\sum_{n=0}^{N-1}\left|z_n\right|^2}
\left[\begin{array}{c} 1 \\ -z_0 \\ -z_1\\ 
\vdots \\ -z_{N-1} \end{array}\right]
\left[\begin{array}{c} 1 \\ -z_0 \\ -z_1\\ 
\vdots \\ -z_{N-1} \end{array}\right]^{H}
+\mathop{\rm diag}\left[0, 1, 1, \ldots, 1\right]
\end{equation}

In particular, the diagonal elements of $\invinfmat{\uvp}$ are
\begin{equation}
\left[
\frac{\np{4}}{2N}, \frac{2\nv}{s},
\frac{\nv}{2}\left(1+\frac{k_0^2\left|s_0\right|^2}{s}\right),
\ldots,\frac{\nv}{2}\left(1+\frac{k_{N-1}^2\left|s_{N-1}\right|^2}{s}\right)
\right]
\end{equation}
where 
\begin{equation}
s=3\sum_{n=0}^{N-1}k_n^2\left|s_n\right|^2
\end{equation}
The second element on the diagonal of $\invinfmat{\uvp}$ is the \CR lower
bound on the variance of any unbiased estimator of $\dr$.  Hence
\begin{equation}
\Var{\rWH}\geq\frac{2\nv}{3\ds\sum_{n=0}^{N-1}k_n^2\left|s_n\right|^2}
\end{equation}

%%%%%%%%%%%%%%%%%%%%%%%%%%%%%%%%%%%%%%%%%%%%%%%%%%%%%%%%%%%%%%%%%%%%%%%%%%%%%
\subsection{Comparison of the \CR Bounds}

Section~\ref{sp sec:CRB1} gave the \CR bound on any unbiased estimator of
$\dr$ as
\begin{equation}\label{sp eqn:CRB1}
\Var{\rWH}\geq\frac{1}{I(\dr)}
=\frac{\nv}{\ds\sum_{n=0}^{N-1}k_n^2\left|s_n\right|^2}
\end{equation}
which applies if $\dr$ is the only unknown parameter.  
Section~\ref{sp sec:CRB2} gave a different \CR bound 
\begin{equation}\label{sp eqn:CRB2}
\Var{\rWH}\geq (\invinfmat{\uvp})_{22}
=\frac{2\nv}{3\ds\sum_{n=0}^{N-1}k_n^2\left|s_n\right|^2}
\end{equation}
which applies if $\dr$ is part of the vector $\uvp$ of unknowns in 
(\ref{sp eqn:uvp}).  

Ordinarily, the \CR bound for estimating $\dr$ would be smaller 
when $\dr$ is the only unknown than when $\dr$ is merely one of a whole
vector of unknowns.  There is a simple reason why this is not the case, and
that is that the two \CR bounds are for different models.   Eliminating the
$s_n$ from (\ref{ml eqn:std model}) to give (\ref{ml eqn:comb model})
changes the model because if the $s_n$ were genuinely known {\em a
priori\/}, there would be no need use two measurements to eliminate them. 
Using two measurements means twice the noise, hence a doubling of the \CR
bound.

If $\dr$ really were the only unknown parameter, the \CR bound for estimating
$\dr$ alone could be found using by using the likelihood function from the
standard model in (\ref{ml eqn:std model}) and assuming that every parameter
but $\dr$ was known perfectly.  The \CR bound is given by the
reciprocal of $(\infmat{\uvp})_{22}$
\begin{equation}\label{sp eqn:CRB3}
\Var{\rWH}\geq \frac{1}{(\infmat{\uvp})_{22}}
=\frac{\nv}{2\ds\sum_{n=0}^{N-1}k_n^2\left|s_n\right|^2}
\end{equation}
Now, as expected, this bound is less than the bound in (\ref{sp eqn:CRB2})
where $\dr$ was only one of many unknowns, and exactly half the bound in
(\ref{sp eqn:CRB1}) where twice as many noisy measurements as necessary were
used to estimate $\dr$.

This comparison of the \CR bounds has indicated that the scalar $\rML$ is
estimating $\dr$ based on (\ref{ml eqn:comb model}) rather than the proper
model in (\ref{ml eqn:std model}).  Therefore, from this point on, any
reference to $\rML$ is assumed to refer to $\rML$ found using vector
maximum likelihood estimation, and the appropriate \CR bound is that in
(\ref{sp eqn:CRB2}).

%%%%%%%%%%%%%%%%%%%%%%%%%%%%%%%%%%%%%%%%%%%%%%%%%%%%%%%%%%%%%%%%%%%%%%%%%%%%%
\section{Performance of $\rML$}

This section presents a performance analysis of $\rML$.  The estimator is
shown to be unbiased independently of the number of measurements, so the \CR
bound calculated in (\ref{sp eqn:CRB2}) gives a lower bound on the variance
of $\rML$.  This is followed by a brief outline of some non-asymptotic
results.

%%%%%%%%%%%%%%%%%%%%%%%%%%%%%%%%%%%%%%%%%%%%%%%%%%%%%%%%%%%%%%%%%%%%%%%%%%%%%
\subsection{Asymptotic Normality of $\rML$}

In section~\ref{sp sec:normality}, the regularity conditions for asymptotic 
normality of $\rML$ were established before the \CR bound had been
calculated.  Now that the \CR bound has been found in (\ref{sp eqn:CRB2}),
the asymptotic distribution of $\rML$ is normal with mean $\dr$ and variance
equal to the \CR bound
\begin{equation}
\rML\stackrel{d}{\sim}{\cal N}\left(\dr,
\frac{2\nv}{3\ds\sum_{n=0}^{N-1}k_n^2\left|s_n\right|^2}\right)
\end{equation}
as $N\to\infty$.

%%%%%%%%%%%%%%%%%%%%%%%%%%%%%%%%%%%%%%%%%%%%%%%%%%%%%%%%%%%%%%%%%%%%%%%%%%%%%
\subsection{Bias of $\rML$}

$\rML$ is found by locating the maximum of $J(\r)$.  Since $J(\r)$ is a
function whose period is equal to the ISAR range ambiguity window
\begin{equation}
J\left(\r\right)=J\left(\r+k\frac{c}{2\Delta f}\right)
\qquad\mbox{for any integer $k$}
\end{equation}
the support of $\r$ must be restricted to some subinterval of the real line
centred at $\r_0$ of length $c/2\Delta f$.  

The following theorem shows that $\rML$ is an unbiased estimator of $\dr$.  
The proof assumes that the support of $J(\r)$ has been located so that the
centre of the ISAR range ambiguity window $\r_0$ is at $\r=\dr$.  

%============================================================================
\begin{theorem}
The maximum likelihood estimator $\rML$ of the target's radial motion found
using (\ref{ml eqn:J(r)}) and (\ref{ml eqn:rML})
\begin{equation}
\rML=\arg\min_{\r}\,J(\r)
\end{equation}
where
\begin{equation}\label{sp eqn:J(r)}
J(\r)=\sum_{n=0}^{N-1} \left|a_n-b_n\e{jk_n\r}\right|^2
\end{equation}
is an unbiased estimator of $\dr$ when the support of $J(\r)$ is
centred at $\r_0=\dr$ and has length $c/2\Delta f$.
\end{theorem}
%============================================================================

%============================================================================
\begin{proof}
This proof shows that the probability density function of $J(\r)$ is
symmetric about $\r=\dr$, so that the probability density function of
$\rML$ is also symmetric about $\r=\dr$.  From this, the expectation of
$\rML$ is shown to be $\dr$.

Using (\ref{ml eqn:std model}) to write $J(\r)$ in terms of
\begin{eqnarray}
a_n&=&s_n+\na{n}\nn\\
b_n&=&s_n\e{-jk_n\dr}+\nb{n}    
\label{sp eqn:std model}
\end{eqnarray}
where $\na{n}$ and $\nb{n}$ are zero mean additive white Gaussian noise,
shows that
\begin{equation}
J(\r)=\sum_{n=0}^{N-1} \left|s_n\left(1-\e{-jk_n(\dr-\r)}\right)
+\left(\na{n}-\nb{n}\e{jk_n\r}\right)\right|^2
\end{equation}
When $\r$ is written as $\dr+\r$ to make the symmetry about $\r=\dr$ more
apparent,
\begin{equation}
\label{sp eqn:J(dr+r)}
J(\dr+\r)=\sum_{n=0}^{N-1} \left|s_n\left(1-\e{jk_n\r}\right)
+\left(\na{n}-\nb{n}\e{jk_n(\dr+\r)}\right)\right|^2
\end{equation}
At the point $\r$ from $\dr$ in the opposite direction,
\begin{eqnarray}
J(\dr-\r)
&=&\sum_{n=0}^{N-1} \left|s_n\left(1-\e{-jk_n\r}\right)
+\left(\na{n}-\nb{n}\e{jk_n(\dr-\r)}\right)\right|^2\nn\\
&=&\sum_{n=0}^{N-1} \left|s_n\left(1-\e{jk_n\r}\right)
-\e{jk_n\r}\left(\na{n}-\nb{n}\e{jk_n(\dr-\r)}\right)\right|^2\nn\\
&=&\sum_{n=0}^{N-1} \left|s_n\left(1-\e{jk_n\r}\right)
+\left(\nb{n}\e{jk_n\dr}-\na{n}\e{jk_n\r}\right)\right|^2\nn\\
&=&\sum_{n=0}^{N-1} \left|s_n\left(1-\e{jk_n\r}\right)
+\left(\na{n}'-\nb{n}'\e{jk_n(\dr+\r)}\right)\right|^2
\label{sp eqn:J(dr-r)}
\end{eqnarray}
where
\begin{equation}
\na{n}'=\nb{n}\e{jk_n\dr}
\end{equation}
and
\begin{equation}
\nb{n}'=\na{n}\e{-jk_n\dr}
\end{equation}
Multiplying $\na{n}$ by the constant phase shift $\e{-jk_n\dr}$ does not
change the statistical properties of $\na{n}$ because its probability
density function is independent of phase angle.  Therefore $\nb{n}'$ is
equivalent, statistically, to $\na{n}$.  Similarly, $\na{n}'$ is equivalent
to $\nb{n}$.  Since $\na{n}$ and $\nb{n}$ have identical statistical
properties, so do $\na{n}'$ and $\nb{n}'$.  

Comparing equations (\ref{sp eqn:J(dr+r)}) and (\ref{sp eqn:J(dr-r)}) shows
that they have identical statistical properties.  Therefore the probability
density function of $J(\dr+\r)$ is equal to the probability density
function of $J(\dr-\r)$.

Because $\rML$ is the location of the maximum of $J(\r)$ and the probability
density function of $J(\r)$ is symmetric about $\r=\dr$, the probability
density function of $\rML$ is also symmetric about $\r=\dr$.  Writing
$p_{\r}(\r)$ for the probability density function of $\rML$, this means
that
\begin{equation}
p_{\r}(\dr+\r)=p_{\r}(\dr-\r)
\end{equation}
Now the support of $p_{\r}(\r)$ is not the whole real line
$(-\infty,\infty)$, but the ISAR range ambiguity window 
$\left(\dr-\frac{c}{4\df},\dr+\frac{c}{4\df}\right)$.
Therefore the expectation of $\rML$ is
\begin{eqnarray}
\E{\rML}
&=&\int_{\dr-\frac{c}{4\df}}^{\dr+\frac{c}{4\df}} \r\,p_{\r}(\r)\,d\r\nn\\
&=&\int_{\dr-\frac{c}{4\df}}^{\dr+\frac{c}{4\df}} \dr\,p_{\r}(\r)\,d\r
+\int_{\dr-\frac{c}{4\df}}^{\dr+\frac{c}{4\df}} (\r-\dr)\,p_{\r}(\r)\,d\r\nn\\
&=&\dr\int_{\dr-\frac{c}{4\df}}^{\dr+\frac{c}{4\df}} \,p_{\r}(\r)\,d\r
+\int_{-\frac{c}{4\df}}^{\frac{c}{4\df}} \r\,p_{\r}(\dr+\r)\,d\r\nn\\
&=&\dr\cdot 1+0\nn\\
&=&\dr
\end{eqnarray}
Here the integral of $\r\,p_{\r}(\dr+\r)$ is zero because the integrand is
odd, being the product of an even function $p_{\r}(\dr+\r)$ and an odd 
function $\r$.
\end{proof}

This proof shows that $\rML$ is an unbiased estimator of $\dr$ when the ISAR
range ambiguity window is centred on $\dr$.  If the window were moved so that
$\dr$ was no longer at the centre, $\rML$ would appear biased.  However this
bias is the result of artificially constraining the support of $J(\r)$ when
taking the expectation of $\rML$.  The correct approach is to join the ends
of the range ambiguity window together to form a cylinder.  Then with
the expectation modified to make it suitable for periodic functions,
$\rML$ is always unbiased.

Finally, because $\rML$ is an unbiased estimator of $\dr$, the variance of 
$\rML$ is greater than the \CR bound in (\ref{sp eqn:CRB2})
\begin{equation}
\Var{\rML}\geq\frac{2\nv}{3\ds\sum_{n=0}^{N-1}k_n^2\left|s_n\right|^2}
\end{equation}

%%%%%%%%%%%%%%%%%%%%%%%%%%%%%%%%%%%%%%%%%%%%%%%%%%%%%%%%%%%%%%%%%%%%%%%%%%%%%
\subsection{Non-Asymptotic Results}

Non-asymptotic results for the statistical performance of $\rML$
are much harder to obtain than the asymptotic results.  The basic result
needed is the probability density function of $\rML$ because the variance
and other properties can be derived from it.

If $J(\r)$ is expressed in complex form as in appendix \ref{ee app:min
using DFT}, it may be possible to derive one expression for the probability
density function of the location of the maximum of the magnitude of
$J(\r)$, and another for the probability density function of the locations
of the points where the phase is a multiple of $2\pi$.  Then by combining
the two separate probability density functions, it may be possible to
obtain at least an approximate probability density function for $\rML$.

Expressions for the probability density function of the location of the
maximum likelihood frequency estimator based on a fast Fourier transform
have been obtained by Ritcey \cite{Rit87}.  This gives a discrete form of
the probability density function of $\rML$ at intervals of an ISAR range
bin, although it is not in a form that can be used
conveniently.\footnote{This suggests the general problem of finding the
probability density function of the location of the maximum of a known
bandlimited function in bandlimited noise.  See \protect\ref{co prb:blf I}
on page \protect\pageref{co prb:blf I} for a longer discussion of this.}

The other end of the scale is the probability density function of estimates
of the location of each local minimum of the cost function $J(\r)$.  This
can be used to see how clustered the estimates of $\dr$ are around
multiples of half a wavelength.  An approximate solution of this is derived
in appendix \ref{sp app:phase est}.

These results are not sufficiently complete to discuss further, except to
say that the variance of $\rML$ is not a good guide to the performance of
the radial motion estimator for radar imaging.  The variance of $\rML$
measures the total spread of estimates of $\dr$, ignoring the clustering of
the errors at intervals of half a wavelength.  But as long as the total
spread is not too wide, the clustering is more important for focusing
ISAR images than the total spread.\footnote{This is analyses in section
\protect\ref{rmc sec:ma}.}  Therefore some better measure than
variance is needed for comparing different motion estimators.

%%%%%%%%%%%%%%%%%%%%%%%%%%%%%%%%%%%%%%%%%%%%%%%%%%%%%%%%%%%%%%%%%%%%%%%%%%%%%
\section{Performance of $\nvML$}

The maximum likelihood estimator of the noise variance is given in 
(\ref{ml eqn:nvML}) as $1/4N$ times the minimum of $J(\r)$, or
\begin{equation}
\nvML=\frac{1}{4N}\sum_{n=0}^{N-1} \left|a_n-b_n\e{jk_n\rML}\right|^2
\end{equation}

If $\rML$ were known exactly, then substituting in the expressions for $a_n$
and $b_n$ in terms of the target's noisy frequency responses would show that
\begin{equation}
\E{\nvML}=\nv
\end{equation}

However, $\rML$ has to be estimated, so the actual expectation of $\nvML$
turns out to be
\begin{equation}
\E{\nvML}=\frac{\nv}{2}
\end{equation}
This is a consequence of $\rML$ being estimated from two sets of 
frequency responses.  If $\rML$ were estimated from $M$ frequency responses,
the expectation would be
\begin{equation}
\E{\nvML}=\nv\,\frac{M-1}{M}
\end{equation}
which is similar to the bias of the standard deviation of a sample.  If
$M\to\infty$, then $\nvML$ would become unbiased.  However, the averaging
does not work like this as $M$ is fixed at $2$ and the expectation of
$\nvML$ is always half the true noise variance.

Since the bias is a constant factor, an unbiased estimator of the noise
variance is given by
\begin{equation}
\widehat{\nv}=2\nvML
\end{equation}

%%%%%%%%%%%%%%%%%%%%%%%%%%%%%%%%%%%%%%%%%%%%%%%%%%%%%%%%%%%%%%%%%%%%%%%%%%%%%
\section{Performance of $\sML{n}$}

The maximum likelihood estimators $\sML{n}$ of the target's frequency
response $s_n$ at each of the $N$ frequencies are biased.  Their expectation is
\begin{eqnarray}
\E{\sML{n}}
&=&\E{\frac{1}{2}\left(a_n+b_n\e{jk_n\rML}\right)}\nn\\
&=&\E{\frac{1}{2}\left(s_n+s_n\e{-jk_n\dr}\e{jk_n\rML}\right)}\nn\\
&=&s_n\cdot\frac{1}{2}\left(1+\E{\e{-jk_n(\dr-\rML)}}\right)
\end{eqnarray}
If the motion estimator $\rML$ were exact, $\sML{n}$ would be unbiased
\begin{equation}
\E{\sML{n}}=s_n
\end{equation}
Working out the exact bias requires 
\begin{equation}
\E{\e{-jk_n(\dr-\rML)}}=\int \e{-jk_n(\dr-\r)}\,p_r(\r)\,d\r
\end{equation}
to be evaluated, where $p_r(\r)$ is the probability density function of $\rML$.  

At this stage, the probability density function of $\rML$ is unknown.  But if it is assumed that
$\rML$ is Gaussian (since it is asymptotically Gaussian) with variance
$\rMLV$, the identity \cite[7.4.6--7]{Abr65}
\begin{equation}
\infint \e{-at^2}\,\e{j2xt}\,dt=\frac{1}{2}\sqrt{\frac{\pi}{a}}\e{-x^2/a^2}
\end{equation}
shows that
\begin{equation}
\E{\sML{n}}=s_n\,\frac{1+\e{-k_n^2\rMLV/2}}{2}
\end{equation}
So when there is little noise, $\rMLV$ is close to zero and $\sML{n}$ is
nearly unbiased.  As the noise level increases, $\sML{n}$ gets more and more
biased, until at extremely high noise levels, the expected value of
$\sML{n}$ is half the true value $s_n$.

In practice, the bias of $\sML{n}$ does not matter because estimating $\rML$
does not require estimates of the $s_n$.

%%%%%%%%%%%%%%%%%%%%%%%%%%%%%%%%%%%%%%%%%%%%%%%%%%%%%%%%%%%%%%%%%%%%%%%%%%%%%
\appendix{Elements of the Fisher Information Matrix}
\label{sp app:FIM}

\begingroup
\def\FIM#1#2{-\E{\frac{\del^2\ln p(\md;\uvp)}{\del\conj{#1}\del #2}}}
\def\FPD#1{\frac{\del\ln p(\md;\uvp)}{\del #1}}

The Fisher information matrix $\infmat{\uvp}=(I_{ij})$ has elements
\begin{equation}
I_{ij}=\FIM{\ui{i}}{\ui{j}}
\end{equation}
where the vector of parameters being estimated, $\uvp$, is given in 
(\ref{sp eqn:uvp}) and the likelihood function is
\begin{equation}
p(\md;\uvp)=\frac{1}{\left(\pi\nv\right)^{2N}}\,\exp\left(
-\frac{1}{\nv}\sum_{n=0}^{N-1}
\left|\vphantom{\e{k}}a_n-s_n\right|^2+\left|b_n-s_n\e{-jk_n\dr}\right|^2
\right)
\end{equation}


The first partial derivatives of the likelihood function are
\begin{eqnarray}
\FPD{\nv}&=&-\frac{2N}{\nv}+\frac{1}{\np{4}}\sum_{n=0}^{N-1}
\left|a_n-s_n\right|^2+\left|b_n-s_n\e{-jk_n\dr}\right|^2 \\
\FPD{\dr}&=&-j\frac{1}{\nv}\sum_{n=0}^{N-1}k_n\left( 
b_n\conj{s_n}\e{jk_n\dr}-\conj{b_n}s_n\e{-jk_n\dr}\right) \\
\FPD{s_n}&=&-\frac{1}{\nv}\left[\left(\conj{s_n}-\conj{a_n}\right)
+\left(\conj{s_n}-\conj{b_n}\e{-jk_n\dr}\right)\right] \\
\FPD{\conj{s_n}}&=&-\frac{1}{\nv}\left[\left(s_n-a_n\right)
+\left(s_n-b_n\e{jk_n\dr}\right)\right] 
\end{eqnarray}

The elements on the diagonal of $\infmat{\uvp}$ are
\begin{eqnarray}
\FIM{\nv}{\nv}&=&-\E{\frac{2N}{\np{4}}-\frac{2}{\np{6}}\sum_{n=0}^{N-1}
\left|a_n-s_n\right|^2+\left|b_n-s_n\e{-jk_n\dr}\right|^2}  \nn\\
&=&-\frac{2N}{\np{4}}+\frac{2}{\np{6}}\sum_{n=0}^{N-1}\E{
\left|\na{n}\right|^2+\left|\nb{n}\right|^2}  \nn\\
&=&-\frac{2N}{\np{4}}+\frac{4N}{\np{4}} \nn\\
&=&\frac{2N}{\np{4}} \\
&&\nn\\
\FIM{\dr}{\dr}&=&\E{\frac{2}{\nv}\real{\sum_{n=0}^{N-1}
k_nb_n\conj{s_n}\e{jk_n\dr}}}  \nn\\
&=&\frac{2}{\nv}\real{\sum_{n=0}^{N-1}k_n^2\E{b_n}\conj{s_n}\e{jk_n\dr}} \nn\\
&=&\frac{2}{\nv}\sum_{n=0}^{N-1}k_n^2\left|s_n\right|^2 \\
&&\nn\\
\FIM{s_n}{s_n}&=&\E{\frac{2}{\nv}}\nn\\
&=&\frac{2}{\nv} 
\end{eqnarray}

The elements of $\infmat{\uvp}$ whose partial derivatives involve $\nv$ are
zero, as the following shows:
\begin{eqnarray}
\FIM{\nv}{\dr}&=&\E{\frac{2N}{\np{4}}\imag{
\sum_{n=0}^{N-1} k_nb_n\conj{s_n}\e{jk_n\dr}}}  \nn\\
&=&\frac{2N}{\np{4}}\imag{\sum_{n=0}^{N-1} k_n\left|s_n\right|^2}  \nn\\
&=&0 \\
&&\nn\\
\FIM{\nv}{s_n}&=&-\E{\frac{1}{\np{4}}\left[\left(\conj{s_n}-\conj{a_n}\right)
+\left(\conj{s_n}-\conj{b_n}\e{-jk_n\dr}\right)\right]} \nn\\
&=&-\frac{1}{\np{4}}\left[\E{\conj{\na{n}}}+\E{\conj{\nb{n}}}\e{-jk_n\dr}
\right] \nn\\
&=&0
\end{eqnarray}

The elements of $\infmat{\uvp}$ whose partial derivatives involve $\dr$ and
$s_n$ are
\begin{eqnarray}
\FIM{\dr}{s_n}&=&\E{-j\frac{1}{\nv} k_n\conj{b_n}\e{-jk_n\dr}} \nn\\
&=&-j\frac{k_n\conj{s_n}}{\nv} \nn\\
&&\\
\FIM{\dr}{s_n}&=&\conj{\FIM{s_n}{\dr}} \nn \\
&=&j\frac{k_ns_n}{\nv}
\end{eqnarray}

This leaves the elements whose partial derivatives involve $s_m$ and $s_n$
where $m\neq n$.  All $N^2-N$ of these are zero.
\begin{equation}
\FIM{s_m}{s_n}=0
\end{equation}
\endgroup

%%%%%%%%%%%%%%%%%%%%%%%%%%%%%%%%%%%%%%%%%%%%%%%%%%%%%%%%%%%%%%%%%%%%%%
\appendix{Inverse of Fisher's Information Matrix}
\label{sp app:M-1}

Taking the inverse of $\infmat{\uvp}$ involves inverting the submatrix
which is left when $\infmat{\uvp}$ has its first row and column removed.
The general inverse for matrices of this form is the subject of the
following theorem.

\begin{theorem} The inverse of the matrix
\begin{equation}
\vect{M}=
\left[\begin{array}{ccccc}
c		&\conj{z_0}	&\conj{z_1}	&\cdots	&\conj{z_{N-1}}\\
z_0	&1	&0	&\cdots	&0	\\
z_1	&0	&1	&\cdots	&0	\\
\vdots		&\vdots	&\vdots	&\ddots	&\vdots	\\
z_{N-1}	&0	&0	&\cdots	&1	
\end{array}\right]
\end{equation}
where 
\begin{equation}
c=4\sum_{n=0}^{N-1} \left|z_n\right|^2
\end{equation}
is
\begin{equation}
\vect{M}^{-1}=\frac{1}{3\ds\sum_{n=0}^{N-1} 
\left|z_n\right|^2}\vect{v}\,\vect{v}^H+\vect{D}
\label{sp eqn:M-1 in app}
\end{equation}
where $\vect{v}$ is
\begin{equation}
\vect{v}=\left[\begin{array}{c}
1\\ -z_0 \\-z_1\\\vdots\\-z_{N-1}\end{array}\right]
\end{equation}
and $\vect{D}$ is the diagonal matrix
\begin{equation}
\vect{D}=\mathop{\rm diag}\left[0, 1, 1, \ldots, 1\right]
\end{equation}
\end{theorem}


\begin{proof}
Note that 
\begin{equation}
\vect{M}\vect{v}\vect{v}^H=
\left(c-\sum_{n=0}^{N-1}\left|z_n\right|^2\right)
\left[\begin{array}{ccccc} 
1	&-\conj{z_0}	&-\conj{z_1}	&\cdots	&-\conj{z_{N-1}}\\
0	&0	&0	&\cdots	&0	\\
0	&0	&0	&\cdots	&0	\\
\vdots	&\vdots	&\vdots	&\ddots	&\vdots	\\
0	&0	&0	&\cdots	&0	
\end{array}\right]
\end{equation}
and that
\begin{equation}
\vect{M}\vect{D}=
\left[\begin{array}{ccccc} 
0	&\conj{z_0}	&\conj{z_1}	&\cdots	&\conj{z_{N-1}}\\
0	&1	&0	&\cdots	&0	\\
0	&0	&1	&\cdots	&0	\\
\vdots	&\vdots	&\vdots	&\ddots	&\vdots	\\
0	&0	&0	&\cdots	&1	
\end{array}\right]
\end{equation}
Therefore
\begin{equation}
\vect{M}\left(
\frac{1}{3\ds\sum_{n=0}^{N-1}\left|z_n\right|^2}\vect{v}\vect{v}^H+\vect{D}
\right)=\vect{I}
\end{equation}
This shows that $\vect{M}\vect{M}^{-1}=\vect{I}$.

To show that $\vect{M}^{-1}\vect{M}=\vect{I}$ --- the other condition 
needed for $\vect{M}^{-1}$ to be the inverse of $\vect{M}$ --- note that 
both $\vect{M}$ and $\vect{M}^{-1}$ are 
Hermitian. Then
\begin{eqnarray}
\vect{M}^{-1}\vect{M}
&=&\left[\left(\vect{M}^{-1}\vect{M}\right)^H\right]^H		\nn\\
&=&\left[\vect{M}^H\left(\vect{M}^{-1}\right)^H\right]^{H}	\nn\\
&=&\left[\vect{M}\,\vect{M}^{-1}\right]^{H}			\nn\\
&=&\vect{I}^H							\nn\\
&=&\vect{I}
\end{eqnarray}
Therefore $\vect{M}^{-1}$ in (\ref{sp eqn:M-1 in app}) is the inverse of 
$\vect{M}$.
\end{proof}

This result can also be obtained by applying the Sherman-Morrison 
formula~\cite{Gol83}
\begin{equation}
(\vect{A}+\vect{u}\vect{v}^H)^{-1}=
\vect{A}^{-1}-\frac{\vect{A}^{-1}\vect{u}\vect{v}^H\vect{A}^{-1}}
{1+\vect{v}^H\vect{A}^{-1}\vect{u}}
\end{equation}
two times in succession to
\begin{equation}
\vect{M}=\vect{I}+\vect{u}\vect{v}^H+\vect{v}\vect{u}^H
\end{equation}
where
\begin{equation}
\vect{u}=\left[\frac{c-1}{2},z_0,z_1,\ldots,z_{N-1}\right]^H
\end{equation}
and 
\begin{equation}
\vect{v}=\left[1,0,0,\ldots,0\right]^H
\end{equation}

%%%%%%%%%%%%%%%%%%%%%%%%%%%%%%%%%%%%%%%%%%%%%%%%%%%%%%%%%%%%%%%%%%%%%%
\appendix[PDF of Each Local Minimum of $J(\r)$]{Probability Density
Function of Each Local Minimum of $J(\r)$}
\label{sp app:phase est}

This appendix derives an approximate expression for the probability density 
function of estimates of the location of each local minimum of the cost
function $J(\r)$.  The variance and other properties of the estimates of
$\dr$ given by algorithm \ref{ee alg:min with cm} can be calculated
numerically from this probability density function.

%%%%%%%%%%%%%%%%%%%%%%%%%%%%%%%%%%%%%%%%%%%%%%%%%%%%%%%%%%%%%%%%%%%%%%
\subsection{Conditions for a Local Minimum of $J(\r)$}

Suppose that $\r=\ro$ is the location of a local minimum of $J(\r)$ when
noise is not present.  When noise is added to the frequency responses, the
location of the local minimum of $J(\r)$ closest to $\ro$ is now $\ro+\dro$
where $\dro$ is the error caused by the presence of noise. Then
\begin{equation}
\left.\frac{\del J(\r)}{\del \r}\right|_{\r=\ro+\dro}=0
\end{equation}
Differentiating the cost function $J(\r)$ in (\ref{ml eqn:J(r) uvp}) gives
the equation
\begin{equation}
\imag{\sum_{n=0}^{N-1} k_na_n\conj{b_n}\e{-jk_n(\ro+\dro)}}=0
\end{equation}
Substituting the noisy frequency responses in (\ref{ml eqn:std model}) into 
this gives
\begin{equation}
\imag{\sum_{n=0}^{N-1} k_n(s_n+\na{n})
(\conj{s_n}\e{jk_n\dr}+\conj{\nb{n}})\e{-jk_n(\ro+\dro)}}=0
\end{equation}
which can be expressed as
\begin{equation}
\imag{\sum_{n=0}^{N-1} k_n(s_n+\na{n})
(\conj{s_n}+\conj{\nb{n}}\e{-jk_n\dr})\e{-jk_n(\ro+\dro-\dr)}}=0
\end{equation}
This can be simplified by noting that each of the $\nb{n}$ noise terms is
the realization of a complex Gaussian random process, so the probability
density function of their phases is uniform over $\left[0,2\pi\right)$. 
Therefore shifting the phase of each $\nb{n}$ by $-k_n\dr$ gives random
variables with identical probability density functions to the original random variables.
So each $\nb{n}\e{jk_n\dr}$ can be replaced by $\nb{n}$ for notational
convenience without altering the statistical properties of $J(\r)$.  Hence
\begin{equation}
\imag{\sum_{n=0}^{N-1} k_n(s_n+\na{n})
(\conj{s_n}+\conj{\nb{n}})\e{-jk_n(\ro+\dro-\dr)}}=0
\end{equation}
Separating the initial frequency $k_0$ from the frequency steps gives
\begin{equation}\label{sp app eqn:2nd exp}
\imag{\e{-jk_0(\ro+\dro-\dr)}\sum_{n=0}^{N-1} k_n(s_n+\na{n})
(\conj{s_n}+\conj{\nb{n}})\e{-jn\dk(\ro+\dro-\dr)}}=0
\end{equation}
Since this can only be solved unambiguously for $\dro$ over an interval
containing $\ro$ whose extent is half the carrier wavelength, the magnitude
of $\dro$ is limited by
\begin{equation} 
\left|\dro\right|<\frac{c}{8f_0}
\end{equation}
Over this interval, the phase of $\e{-jn\dk\dro}$ varies 
negligibly in comparison to the phase of $\e{-jk_0\dro}$
for all values of $n$ because of the narrow relative bandwidth.
Therefore the $\dro$ term can be omitted from the second complex
exponential in (\ref{sp app eqn:2nd exp}) to give the very good approximation
\begin{equation}
\imag{\e{-jk_0(\ro+\dro-\dr)}\sum_{n=0}^{N-1} k_n(s_n+\na{n})
(\conj{s_n}+\conj{\nb{n}})\e{-jn\dk(\ro-\dr)}}=0
\end{equation}
This can be written
\begin{equation}\label{sp app eqn:d and dro}
\imag{d\,\e{-jk_0\dro}}=0
\end{equation}
where $d$ is the random variable
\begin{equation}\label{sp app eqn:d}
d=\sum_{n=0}^{N-1} k_n(s_n+\na{n})(\conj{s_n}+\conj{\nb{n}})
\e{-jk_n(\ro-\dr)}
\end{equation}
From this, if the phase of $d$ varies by $\delta\phi$, the phase of
$\e{-jk_0\dro}$ must vary by $-\delta\phi$ to keep the imaginary part of
(\ref{sp app eqn:d and dro}) zero.  So the probability density function of the
angle of $d$ completely determines the probability density function of
$\dro$.


%%%%%%%%%%%%%%%%%%%%%%%%%%%%%%%%%%%%%%%%%%%%%%%%%%%%%%%%%%%%%%%%%%%%%%
\subsection{Statistical Characteristics of $d$}

Since the noise processes $\na{n}$ and $\nb{n}$ are independent zero-mean
complex Gaussians with variance $\nv$, their expectations, variances and 
covariances are given by
\begin{eqnarray}
\E{\na{n}}=\E{\nb{n}}&=&0 \nn \\
\E{\na{m}\conj{\na{n}}}=\E{\nb{m}\conj{\nb{n}}}&=&\nv\delta_{m,n}\nn \\
\E{\na{m}\na{n}}=\E{\nb{m}\nb{n}}&=&0\nn \\
\E{\na{m}\nb{n}}=\E{\na{m}\conj{\nb{n}}}=
  \E{\conj{\na{m}}\na{n}}=\E{\conj{\nb{m}}\conj{\nb{n}}}&=&0
\end{eqnarray}
Using these, the expected value of the random variable $d$ given by
(\ref{sp app eqn:d}) is
\begin{equation}
\mu_d=\E{d}=\sum_{n=0}^{N-1} k_n\left|s_n\right|^2\e{-jk_n(\ro-\dr)}
\end{equation}
and the variance is
\begin{equation}
\sigma_d^2=\E{\left|d-\E{d}\right|^2}=\sum_{n=0}^{N-1} k_n^2 \nv \left[
2\left|s_n\right|^2 +\nv\right]
\end{equation}

$d$ can be broken into two parts $d_g$ and $d_c$, which respectively are the
Gaussian components of $d$
\begin{equation}
d_g=\sum_{n=0}^{N-1} k_n(\left|s_n\right|^2+\conj{s_n}\na{n}+s_n\conj{\nb{n}})
\e{-jk_n(\ro-\dr)}
\end{equation}
and the non-Gaussian components
\begin{equation}
d_c=\sum_{n=0}^{N-1} k_n\na{n}\conj{\nb{n}}\e{-jk_n(\ro-\dr)}
\end{equation}
It is shown in appendix \ref{sp app:Gaussians} that 
$\na{n}\conj{\nb{n}}\e{-jk_n(\ro-\dr)}$ has the probability density function 
$p_{\noise}(z)$ where
\begin{equation}
p_{\noise}(z)=\frac{2}{\pi\np{4}}{\rm K_0}\left(
\frac{2|z|}{\nv}\right)
\end{equation}
Here ${\rm K_0}(x)$ is the modified Bessel function defined in 
(\ref{sp app eqn:K0 def}).

Now, the $N$ terms of $d_c$ are independent and approximately identically
distributed, so by the central limit theorem, $d_c/N$ tends to a Gaussian
random variable as $N\rightarrow\infty$.  For stepped-frequency waveforms,
$N$ does not tend to infinity.  However, for adequate range resolution for
radar imaging, $N$ has to be sufficiently large that the Gaussian random
variable approximation is very good.

Since $d_g$ is Gaussian and $d_c$ is approximately Gaussian,
$d$ is approximately a Gaussian random variable with mean
and variance
\begin{equation}
\mu_d=\sum_{n=0}^{N-1} k_n\left|s_n\right|^2\e{-jk_n(\ro-\dr)}
\end{equation}
and 
\begin{equation}
\sigma_d^2=\sum_{n=0}^{N-1} k_n^2 \nv \left[2\left|s_n\right|^2 +\nv\right]
\end{equation}
respectively.

%%%%%%%%%%%%%%%%%%%%%%%%%%%%%%%%%%%%%%%%%%%%%%%%%%%%%%%%%%%%%%%%%%%%%%
\subsection{The Probability Density Function of $\dro$}

The probability density function of $d$, $p_d(z)$, is approximately that of a
complex Gaussian process with mean $\mu_d$ and variance $\sigma_d^2$
\begin{equation}
p_d(z)=\frac{1}{\pi\sigma_d^2}
	\e{-|z-\mu_d|^2/\sigma_d^2}
\end{equation}
This can be converted to the probability density function of $\theta$, the
angle of $d$, by changing variables to polar coordinates and integrating
over magnitudes from $0$ to $\infty$
\begin{eqnarray}
p_{\Theta}(\theta)
&=&\int_0^{\infty} p_d(r\e{j\theta}) \,r\,dr \nn \\
&=&\frac{1}{\pi\sigma_d^2}\int_0^{\infty} 
  \e{-\left|r\e{j\theta}-\mu_d\right|^2/\sigma_d^2} \,r\,dr 
\end{eqnarray}
The exponent $\left|r\e{j\theta}-\mu_d\right|^2$ can be written in a more
tractable form using
\begin{eqnarray}
\left|r\e{j\theta}-\mu_d\right|^2&=&\left|r-\mu_d\e{-j\theta}\right|^2 \nn\\
&=&\left(r-a(\theta)\right)^2+b^2(\theta)
\end{eqnarray}
where $a(\theta)$ and $b(\theta)$ are
\begin{eqnarray}
a(\theta)&=&\real{\mu_d\e{-j\theta}}\\
b(\theta)&=&\imag{\mu_d\e{-j\theta}}
\end{eqnarray}
Then
\begin{equation}\label{sp app eqn:pr(theta) 2}
p_{\Theta}(\theta)
=\frac{1}{\pi\sigma_d^2} \e{-b^2(\theta)/\sigma_d^2} 
\int_0^{\infty} r\e{-\left(r-a(\theta)\right)^2/\sigma_d^2}\,dr 
\end{equation}
Now 
\begin{eqnarray}
\lefteqn{\int_0^{\infty}r\e{-\left(r-a(\theta)\right)^2/\sigma_d^2}\,dr}
\qquad\nn\\
&=&\int_0^{\infty}\left(r-a(\theta)\right)
\e{-\left(r-a(\theta)\right)^2/\sigma_d^2}\,dr
+a(\theta)\int_0^{\infty}\e{-\left(r-a(\theta)\right)^2/\sigma_d^2}\,dr\nn\\
&=&\sigma_d^2\e{-a^2(\theta)/\sigma_d^2}+\frac{\sqrt{\pi}\,\sigma_d
a(\theta)}{2}\,\erfc{-\frac{a(\theta)}{\sigma_d}}
\end{eqnarray}
Putting these back into (\ref{sp app eqn:pr(theta) 2}) gives the probability
density function of $\theta$ as
\begin{equation}\label{sp app eqn:pr(theta) 3}
p_{\Theta}(\theta)
=\frac{1}{\pi} \e{-b^2(\theta)/\sigma_d^2} 
\left[\e{-a^2(\theta)/\sigma_d^2}+\frac{\sqrt{\pi}\,\sigma_d a(\theta)}{2}
\,\erfc{-\frac{a(\theta)}{\sqrt{2}\sigma_d}}\right]
\end{equation}
for $\theta\in\left(-\pi,\pi\right]$.

Finally, from (\ref{sp app eqn:d and dro}), 
\begin{equation}
p_{\dro}(\dro)=\frac{1}{k_0}\,p_{\Theta}\left(-\frac{\dro}{k_0}\right)
\end{equation}
which gives the probability density function of the error in finding the
local minimum due to the noise.

%%%%%%%%%%%%%%%%%%%%%%%%%%%%%%%%%%%%%%%%%%%%%%%%%%%%%%%%%%%%%%%%%%%%%%
\subsection{Conditions for $\dro$ Being Unbiased}

In general, $\dro$ is a biased estimator.  But in the special case of
$\ro=\dr$, $\dro$ is unbiased.  To see this, note that a sufficient 
condition for $\dro$ being unbiased is that 
\begin{equation}
p_{\dro}(x)=p_{\dro}(-x)
\end{equation}
When $\ro=\dro$, $\mu_d$ is real and
\begin{eqnarray}
a(\theta)&=&\mu_d\cos\theta\\
b(\theta)&=&-\mu_d\sin\theta
\end{eqnarray}
$p_{\Theta}(\theta)$ can be written as a function of functions of $\theta$
in the form
\begin{eqnarray}
p_{\Theta}(\theta)=F\left(\cos\theta,\sin^2\theta,\mu_d,\sigma_d\right)
\end{eqnarray}
Both $\cos\theta$ and $\sin^2\theta$ are even functions of $\theta$ so
$p_{\Theta}(\theta)$ is an even function of $\theta$, and
$p_{\dro}(\dro)$ is an even function of $\dro$.  Therefore $\dro$ is an
unbiased estimator when $\ro=\dr$.

%%%%%%%%%%%%%%%%%%%%%%%%%%%%%%%%%%%%%%%%%%%%%%%%%%%%%%%%%%%%%%%%%%%%%%
\appendix[Product of Two IID Complex Gaussian Processes]{Probability 
Density Function of the Product of Two Independent and Identically 
Distributed Zero Mean Complex Gaussian Random Variables}
\label{sp app:Gaussians}

% This uses slightly different definitions, so the whole appendix is
% enclosed in a begingroup/endgroup pair
\begingroup
%%%%%%%%%%%%%%%%%%%%%%%%%%%%%%%%%%%%%%%%%%%%%%%%%%%%%%%%%%%%%%%%%%%%%%
\def\np{{w}}				% Product of the noises
\def\na{{w_1}}				% Noise term 1
\def\nb{{w_2}}				% Noise term 2
\def\nv{{\sigma^2}}			% Noise variance
\def\ni#1{{\sigma^{#1}}}		% Noise s.d.to power i
\def\nsd{{\sigma}}			% Noise s.d.
\def\e#1{{e^{#1}}}			% Exponential
\def\conj#1{\overline{#1}}		% Complex conjuagete
\def\pr{{p}}		    		% Probability density function
\def\conv{{\,\star\,}}			% Convolution operator
\def\E#1{{\mathop{\rm E}\nolimits\left\{#1\right\}}}	 % Expectation
\def\Var#1{{\mathop{\rm Var}\nolimits\left\{#1\right\}}} % Variance
\def\Ko#1{{\mathop{\rm K_0}\nolimits\left(#1\right)}} % Bessel function K_0
%%%%%%%%%%%%%%%%%%%%%%%%%%%%%%%%%%%%%%%%%%%%%%%%%%%%%%%%%%%%%%%%%%%%%%%%%


It is well known that the sum of independent Gaussian random variables 
is a Gaussian random variable whose mean is the sum of the individual means
and whose variance is the sum of the individual variances.
This follows from the probability density function of the sum of two 
independent random variables being the convolution of their respective
probability density functions. 

In this appendix, the probability density function of the product of two
independent and identically distributed zero-mean complex Gaussian random
variables is derived.  The mean and variance of the product is obtained,
along with expressions for the higher order moments.\footnote{This is such a
straightforward result in the theory of noise processes that it must have
been obtained before by someone such as Rice or Middleton.  Nevertheless, no
references to this result in the signal processing literature could be
found.}

%%%%%%%%%%%%%%%%%%%%%%%%%%%%%%%%%%%%%%%%%%%%%%%%%%%%%%%%%%%%%%%%%%%%%%
\subsection{Problem Definition}

Let $\na$ and $\nb$ be independent and identically distributed complex
Gaussian random variables with zero mean and variance $\nv$.  Define the 
random variable $\np$ as the product of $\na$ and $\nb$
\begin{equation}
\np=\na\nb
\end{equation}
The probability density functions of $\na$ and $\nb$ are 
\begin{equation}
\pr_{\na}(z)=\pr_{\nb}(z)=\frac{1}{\pi\nv}\e{-|z|^2/\nv}
\end{equation}
which can be written in polar coordinates as
\begin{eqnarray}
\pr_{\na}(r_1,\theta_1)=\pr_r(r_1)\pr_{\theta}(\theta_1) \\
\pr_{\nb}(r_2,\theta_2)=\pr_r(r_2)\pr_{\theta}(\theta_2) 
\end{eqnarray}
where the magnitudes have a Rayleigh distribution on $[0,\infty)$
\begin{equation}
\pr_r(r)=\frac{2r}{\nv}\e{-r^2/\nv}
\end{equation}
and the phases have a uniform distribution on $[0,2\pi)$
\begin{equation}
\pr_{\theta}(\theta)=\frac{1}{2\pi}	
\end{equation}
By writing $\np=\na\nb$ in polar form as $\np=s\e{j\phi}$, 
\begin{eqnarray}
\phi&=&\theta_1+\theta_2 \bmod 2\pi\\
s&=&r_1r_2
\end{eqnarray}
The polar random variables $s$ and $\phi$ are independent because 
$r_1$ and $r_2$ are independent of $\theta_1$ and $\theta_2$.  The polar
probability density function of $\np$ can be written in the form 
\begin{equation}
\pr_{\np}(s,\phi)=\pr_s(s)\pr_{\phi}(\phi)
\end{equation}
When expressed in Cartesian coordinates, this becomes
\begin{equation}
\pr_{\np}(z)=\frac{1}{|z|}\,\pr_s(|z|)\,\pr_{\phi}(\arg z)
\end{equation}
where the scale factor $1/|z|$ is the Jacobian for changing variables from
polar to Cartesian coordinates.


%%%%%%%%%%%%%%%%%%%%%%%%%%%%%%%%%%%%%%%%%%%%%%%%%%%%%%%%%%%%%%%%%%%%%%
\subsection{Probability Density Function of the Phase of the $w_1w_2$}

The probability density function of $\phi=\arg(\na\nb)=\theta_1+\theta_2\bmod2\pi$ will be derived
here in two ways.  The first is a simple application of the principle
of insufficient reason, while the second uses mathematics to do it properly.

\subsubsection{Intuitive Proof}

$\na$ and $\nb$ are random phasors which may be pointing in any direction
in the complex plane with uniform probability.  So choose a reference phasor
$\alpha$ pointing in an arbitrary direction and express all angles with
respect to this phasor.  Since the probability density functions of $\theta_1$ and $\theta_2$ are 
uniform when the angles are expressed relative to $\alpha$, the probability density function of 
$\phi$ with respect to $\alpha$ must be independent of $\alpha$. This is only
possible if the probability density function of $\phi$ is constant.  So that the total probability
sums to unity, this constant must be $1/2\pi$.  Therefore
\begin{equation}
\pr_{\phi}(\phi)=\frac{1}{2\pi}
\end{equation}


\subsubsection{Rigorous Proof}

A proper proof of this uses the fact that the probability density function of the sum of two
random variables is the convolution of the two random variables' respective
probability density functions \cite[p. 37]{Dav58}.  In this case, $\pr_{\phi}(\phi)$ is the
convolution of two functions that are constant on $[0,2\pi)$ and zero
elsewhere.  Write $\pr_{\theta}(\theta)$ as
\begin{equation}
\pr_{\theta}(\theta)=\frac{1}{2\pi}\,\chi_{[0,2\pi)}(\theta)
\end{equation}
where the characteristic function $\chi_A(x)$ of a set $A$ is
\begin{equation}
\chi_A(x)=\cases{ 1 & if $x\in A$ \cr 0& if $x\not\in A$}
\end{equation}
Then $\pr_{\phi}(\phi)$ is the convolution 
\begin{eqnarray}
\pr_{\phi}(\phi)&=&\pr_{\theta}(\theta_1)\conv\pr_{\theta}(\theta_2) \nn\\
&=&\int_0^{2\pi} \pr_{\theta}(\theta)\pr_{\theta}(\phi-\theta)\,d\theta\nn\\
&=&\int_0^{2\pi} \frac{1}{4\pi^2}\chi_{[0,2\pi)}(\theta)
	\chi_{[0,2\pi)}(\phi-\theta)\,d\theta\nn\\
&=&\cases{\ds\frac{1}{4\pi^2}
\left(2\pi-|2\pi-\phi|\right) & if $\phi\in [0,4\pi)$\cr
0& otherwise}
\end{eqnarray}
This result is slightly incorrect because $\pr_{\phi}(\phi)$ is non-zero for
some $\phi$ outside $[0,2\pi)$.  The correct answer is found by adding
the probabilities for $\phi\in[2\pi,4\pi)$ to the probabilities for 
$\phi\in[0,2\pi)$ and then restricting $\phi$ to the interval $[0,2\pi)$. 
Essentially, this makes the convolution a circular convolution on
$[0,2\pi)$. Therefore
\begin{eqnarray}
\pr_{\phi}(\phi)&=&\frac{1}{4\pi^2}\left(
2\pi-|2\pi-\phi|+2\pi-|2\pi-(\phi+2\pi)|\right) \nn\\
&=&\frac{1}{4\pi^2}\left(\phi+2\pi-\phi\right) \nn\\
&=&\frac{1}{2\pi}\label{sp app eqn:pr(phi)}
\end{eqnarray}
for $\phi\in[0,2\pi)$.

This is the same answer as that obtained via the intuitive proof, but with 
the added respectability provided by a veneer of mathematics.

%%%%%%%%%%%%%%%%%%%%%%%%%%%%%%%%%%%%%%%%%%%%%%%%%%%%%%%%%%%%%%%%%%%%%%
\subsection{Probability Density Function of the Magnitude of $w_1w_2$}

The probability density function of $s=|\na\nb|=r_1r_2$ will be obtained here in two different
ways which fortunately give the same answer, which is
\begin{equation}
\pr_s(s)=\frac{4s}{\ni{4}}\Ko{\frac{2s}{\nv}}
\end{equation}
where $\Ko{x}$ is a modified Bessel function \cite[eq. 9.6.24]{Abr65}.
This has an integral representation
\begin{equation}\label{sp app eqn:K0 def}
\Ko{x}=\int_0^{\infty} \e{-x\cosh u}\,du
\end{equation}


\subsubsection{First Proof using Jacobians}

If a second random variable $t=t(r_1,r_2)$ is defined so that the pairs
$(r_1,r_2)$ and $(s,t)$ uniquely determine one another, the joint
probability density functions $\pr(s,t)$ and $\pr(r_1,r_2)$ are related by
\begin{equation}
\pr(s,t)=|J|\,\pr\left(r_1(s,t),r_2(s,t)\right)
\end{equation}
where the Jacobian $J$ is
\begin{equation}
J=\frac{\del(r_1,r_2)}{\del(s,t)}
=\left[\frac{\del(s,t)}{\del(r_1,r_2)}\right]^{-1}
\end{equation}
Then the probability density function of $s$ is
\begin{eqnarray}
\pr_s(s)&=&\int_T \pr(s,t)\,dt \nn\\
&=&\int_T |J|\,\pr\left(r_1(s,t),r_2(s,t)\right)\,dt\label{sp app eqn:int1}
\end{eqnarray}
where the set $T$ is the support of $t$.

This integration may be easy or difficult depending on the choice of
$t(r_1,r_2)$.  Also $t(r_1,r_2)$ must be chosen so that the Jacobian does 
not change sign over the domain of integration.
Note that the inverse of the Jacobian
\begin{eqnarray}
J^{-1}&=&\frac{\del(s,t)}{\del(r_1,r_2)} \nn\\
&=&\left|\begin{array}{cc}
\ds\frac{\del s}{\del r_1} & 
\ds\frac{\del s}{\del r_2} \\ 
\ds\frac{\del t}{\del r_1} & 
\ds\frac{\del t}{\del r_2}  
\end{array}\right|	\nn\\
&=& r_2\frac{\del t}{\del r_1}-r_1\frac{\del t}{\del r_2} 
\end{eqnarray}
is always non-negative when
\begin{equation}
\frac{\del t}{\del r_1}=1
\end{equation}
and
\begin{equation}
\frac{\del t}{\del r_2}=-1
\end{equation}
These conditions are satisfied when $t$ is
\begin{equation}
t(r_1,r_2)=r_1-r_2
\end{equation}
Since $r_1$ and $r_2$ are defined on $[0,\infty)$, the support of $t$ is
$T=(-\infty,\infty)$.  When $r_1$ and $r_2$ are written in terms of $s$ 
and $t$, they become
\begin{eqnarray}
r_1(s,t)&=&\frac{1}{2}\left[t+\sqrt{t^2+4s}\right]\\
r_2(s,t)&=&\frac{1}{2}\left[-t+\sqrt{t^2+4s}\right]
\end{eqnarray}
from which the Jacobian is
\begin{equation}
J=\frac{1}{r_1+r_2}=\frac{1}{\sqrt{t^2+4s}}
\end{equation}
and $r_1^2+r_2^2$ is
\begin{equation}
r_1^2+r_2^2=(r_1-r_2)^2+2r_1r_2=t^2+2s
\end{equation}

Before the integral in (\ref{sp app eqn:int1}) is evaluated, note
that because $r_1$ and $r_2$ are independent,
\begin{eqnarray}
\pr(r_1,r_2)&=&\pr_r(r_1)\pr_r(r_2)\nn\\
&=&\frac{4}{\ni{4}}r_1r_2\e{-(r_1^2+r_2^2)/\nv}\nn\\
&=&\frac{4}{\ni{4}}s\e{-(t^2+2s)/\nv}
\end{eqnarray}
Therefore
\begin{eqnarray}
\pr_s(s)&=&\int_{-\infty}^{\infty} |J|\,\pr(r_1(s,t),r_2(s,t))\,dt\nn\\
&=&\frac{4s}{\ni{4}}\int_{-\infty}^{\infty}\frac{1}{\sqrt{t^2+4s}}
\e{-(t^2+2s)/\nv} \,dt\nn\\
&=&\frac{8s}{\ni{4}}\int_0^{\infty}\frac{1}{\sqrt{t^2+4s}}
\e{-(t^2+2s)/\nv} \,dt
\end{eqnarray}
Now make the substitution $2s\cosh u=t^2+2s$.  From this,
\begin{equation}
\sqrt{t^2+4s}=2\sqrt{s}\cosh \frac{u}{2}=2\frac{dt}{du}
\end{equation}
so $\pr_s(s)$ becomes
\begin{equation}\label{sp app eqn:pr(s) 1st}
\pr_s(s)=\frac{4s}{\ni{4}}\int_0^{\infty}\e{-(2s/\nv)\cosh u} \,du
\end{equation}
which in terms of the modified Bessel function is
\begin{eqnarray}
\pr_s(s)&=&\frac{4s}{\ni{4}}\Ko{\frac{2s}{\nv}}
\end{eqnarray}


\subsubsection{Second Proof using Convolutions}

The probability density function of $s$ can be obtained more directly by noting that 
\begin{equation}
\ln s = \ln r_1+\ln r_2
\end{equation}
When $a=\ln s$, $b_1=\ln r_1$ and $b_2=\ln r_2$, $a=b_1+b_2$ so the probability density function
of $a$ is the convolution of the probability density functions of $b_1$ and $b_2$.  Then with 
$b=\ln r$,
\begin{eqnarray}
\pr_b(b)&=&\pr_r(r(b))\frac{dr}{db}\nn\\
&=&e^b\pr_r\left(e^b\right)
\end{eqnarray}
Therefore
\begin{eqnarray}
\pr_a(a)&=&\pr_b(b)\conv\pr_b(b)\nn\\
&=&\int_{-\infty}^{\infty} \pr_b(b)\pr_b(a-b)\,db\nn\\
&=&\int_{-\infty}^{\infty} \frac{4}{\ni{4}}
\e{2b}\e{-\e{2b}/\nv}\e{2(a-b)}\e{-\e{2(a-b)}/\nv}\,db\nn\\
&=&\frac{4}{\ni{4}}\e{2a}\int_{-\infty}^{\infty}
\e{-\left(\e{2b}+\e{2(a-b)}\right)/\nv}\,db
\end{eqnarray}
Convert $\pr_a(a)$ back to a probability density function with respect to $s$, giving
\begin{eqnarray}
\pr_s(s)&=&\pr_a(a(s))\frac{da}{ds}\nn\\
&=&\frac{1}{s}\pr_a(a(s))\nn\\
&=&\frac{4s}{\ni{4}}\int_{-\infty}^{\infty}
\e{-\left(\e{2b}+s^2\e{-2b}\right)/\nv}\,db
\end{eqnarray}
Now make the substitution
\begin{equation}
u=2b-\ln s
\end{equation}
so that $\e{2b}+s^2\e{-2b}$ becomes $2s\cosh u$ and $du=2db$.
Therefore
\begin{eqnarray}
\pr_s(s)&=&\frac{2s}{\ni{4}}\int_{-\infty}^{\infty}
\e{-(2s/\nv)\cosh u}\,du\nn\\
&=&\frac{4s}{\ni{4}}\int_0^{\infty}\e{-(2s/\nv)\cosh u}\,du\nn\\
&=&\frac{4s}{\ni{4}}\Ko{\frac{2s}{\nv}}\label{sp app eqn:pr(s)}
\end{eqnarray}
which agrees with (\ref{sp app eqn:pr(s) 1st}).


%%%%%%%%%%%%%%%%%%%%%%%%%%%%%%%%%%%%%%%%%%%%%%%%%%%%%%%%%%%%%%%%%%%%%%
\subsection{Probability Density Function of $w_1w_2$}

Putting equations (\ref{sp app eqn:pr(phi)}) and (\ref{sp app eqn:pr(s)})
together gives the result
\begin{equation}
\pr_{\np}(z)=\frac{1}{|z|}\cdot\frac{4|z|}{\ni{4}}\Ko{\frac{2|z|}{\nv}}\cdot
\frac{1}{2\pi}
\end{equation}
from which it is concluded that
\begin{equation}
\pr_{\np}(z)=\frac{2}{\pi\ni{4}}\Ko{\frac{2|z|}{\nv}}
\end{equation}

%%%%%%%%%%%%%%%%%%%%%%%%%%%%%%%%%%%%%%%%%%%%%%%%%%%%%%%%%%%%%%%%%%%%%%
\subsection{Mean, Variance and Higher-Order Moments}

The mean $\mu_{\np}$ and variance $\sigma_{\np}^2$ of $\np=\na\nb$ are given 
by
\begin{equation}
\mu_{\np}=\E{\np}=\int z\pr_{\np}(z)\,dz
\end{equation}
and
\begin{equation}
\sigma_{\np}^2=\E{|\np-\mu_{\np}|^2}=\int |z-\mu_{\np}|^2\pr_{\np}(z)\,dz
\end{equation}
These are particular cases of the $n$th moment.  With real random variables,
the $n$th moment is $\E{x^n}$.  Here the random variables are complex and it
is not clear which of $\E{\np^n\conj{\np}^m}$ should be chosen as the $(n+m)$th
moment.  So define all of the $\E{\np^n\conj{\np}^m}$ as the $(n+m)$th
moments, and work out a general expression applicable for any $n$ and $m$
that are positive integers or zero.  In fact, the domain of allowable $n$
and $m$ can be broadened by insisting only that $n+m$ be a positive integer
or zero and that $n-m$ be integral.  This obtains $\E{|\np|^k}$ at the
same time for odd $k$ (put $n=m=k/2$).  So
\begin{eqnarray}
\E{\np^n\conj{\np}^m}&=&\int z^n\conj{z}^m\,\pr_{\np}(z)\,dz \nn\\
&=&\int \frac{2}{\pi\ni{4}}z^n\conj{z}^m\Ko{\frac{2|z|}{\nv}}\,dz \nn\\
&=&\int_0^{\infty}\int_0^{2\pi}\frac{2}{\pi\ni{4}}
s^{n+m+1}\e{j(n-m)\phi}\Ko{\frac{2s}{\nv}}\,d\phi ds \nn\\
&=&\left(\frac{1}{2\pi}\int_0^{2\pi}\e{j(n-m)\phi}\,d\phi\right)
\left(\frac{4}{\ni{4}}\int_0^{\infty}s^{n+m+1}\Ko{\frac{2s}{\nv}}\,ds\right)
\end{eqnarray}
Because of the condition that $n-m$ be integral, the integral with respect 
to $\phi$ is unity if $n=m$, otherwise zero.  Therefore 
\begin{equation}
\E{\np^n\conj{\np}^m}=0\;\;\;\;\forall\,n\neq m
\end{equation}
Note that this means that $\mu_{\np}=0$ so each of the moments is a central
moment.

If the moment has any chance of being non-zero, $n$ must equal $m$, so put $k=n+m$ and
consider $\E{|\np|^k}$.
\begin{eqnarray}
\E{|\np|^k}&=&\frac{4}{\ni{4}}\int_0^{\infty}r^{k+1}\Ko{\frac{2r}{\nv}}\,dr\nn\\
&=&\frac{\ni{2k}}{2^k}\int_0^{\infty}u^{k+1}\Ko{u}\,du\nn\\
&=&\ni{2k}\Gamma^2\left(1+\frac{k}{2}\right)
\label{sp app eqn:Ezk}
\end{eqnarray}
This last step uses the identity \cite[eq. 11.4.22]{Abr65}
\begin{equation}
\int_0^{\infty}u^{k+1}\Ko{u}\,du=2^k\Gamma^2\left(1+\frac{k}{2}\right)
\end{equation}
Using (\ref{sp app eqn:Ezk}), $\E{1}=\Gamma^2(1)=1$ as expected, and the
variance of $w$ is
\begin{eqnarray}
\Var{\na\nb}&=&\E{|\np|^2}\nn\\
&=&\ni{4}\Gamma^2(2)\nn\\
&=&\ni{4}
\end{eqnarray}
This is not unexpected because for any two independent random variables
$z_1$ and $z_2$ with zero mean and identical variances $\sigma^2$
\begin{eqnarray}
\Var{z_1z_2}&=&\int\!\!\int |z_1z_2|^2\pr_{z_1,z_2}(z_1,z_2)\,dz_1dz_2\nn\\
&=&\int\!\!\int |z_1|^2|z_2|^2\pr_{z_1}(z_1)\,\pr_{z_2}(z_2)\,dz_1dz_2\nn\\
&=&\left(\int |z_1|^2\pr_{z_1}(z_1)\,dz_1\right)
\left(\int |z_2|^2\pr_{z_2}(z_2)\,dz_2\right)\nn\\
&=&\Var{z_1}\Var{z_2}\nn\\
&=&\sigma^4
\end{eqnarray}

%%%%%%%%%%%%%%%%%%%%%%%%%%%%%%%%%%%%%%%%%%%%%%%%%%%%%%%%%%%%%%%%%%%%%%
\subsection{Summary}

Let $\na$ and $\nb$ be independent and identically distributed complex
Gaussian random variables with zero mean and variance $\sigma^2$.  
Then $\na\nb$ has probability density function
\begin{equation}
\pr_{\np}(z)=\frac{2}{\pi\ni{4}}\Ko{\frac{2|z|}{\nv}}
\end{equation}
The mean of $\na\nb$ is zero and the variance is $\ni{4}$.

%%%%%%%%%%%%%%%%%%%%%%%%%%%%%%%%%%%%%%%%%%%%%%%%%%%%%%%%%%%%%%%%%%%%%%%%%%%%%
\endgroup
% This finishes off the appendix because of the different definitions
