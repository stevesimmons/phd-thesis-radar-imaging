%%%%%%%%%%%%%%%%%%%%%%%%%%%%%%%%%%%%%%%%%%%%%%%%%%%%%%%%%%%%%%%%%%%%%%%%%%%
%
%			INVERSE SYNTHETIC APERTURE RADAR
%
%				  PhD Thesis
%
%		Stephen Simmons		simmons@ee.mu.oz.au
%
%	    Department of Electrical and Electronic Engineering
%	    University of Melbourne, Parkville, 3052, Australia
%
% Chapter 6:	Maximum Likelihood Motion Estimation
%
%		started first draft:	Wed 5 Oct 1994
%		finished first draft:	Thu 6 Oct 1994
%		started second draft:	Mon 2 Jan 1995
%		submitted:		Mon 9 Jan 1995
%
%%%%%%%%%%%%%%%%%%%%% Copyright (C) 1995 Stephen Simmons %%%%%%%%%%%%%%%%%%

\chapter{Maximum Likelihood Motion Estimation}
\label{ml chp}

\bigletter Chapter \ref{mc chp} introduced a number of methods for
estimating the radial motion of a target during ISAR imaging.  Each
measured changes in the target's frequency responses or range profiles in
different ways, and each justified their derivation on different intuitive
grounds.  

This chapter does not present another {\em ad hoc\/} method of radial
motion estimation.  Rather it uses the full machinery of statistical
estimation theory to derive the maximum likelihood estimator $\rML$ of the
radial distance $\dr$ the target moves between two times at which its
frequency response is sampled.

The maximum likelihood estimator is obtained in two different ways.  The
first approach assumes that $\dr$ is the only unknown parameter, so the
scalar form of the maximum likelihood estimator is used to obtain $\rML$. 
The second approach uses the vector form of the maximum likelihood
estimator to estimate the value of a vector $\uvp$ of unknown parameters
which includes $\dr$ along with the measurement noise's variance $\nv$ and
the target's frequency response.  

Both approaches give the maximum likelihood estimate of $\dr$ in the form
of a cost function $J(r)$ which is minimized when $r=\rML$.  The cost
function is analysed and the properties that make it suitable for motion
estimation for ISAR imaging are discussed.

Efficient methods of finding the minimum of $J(r)$ are the subject of
chapter \ref{ee chp} and the statistical properties of $\rML$
are discussed in chapter \ref{sp chp}.  Finally, in chapter \ref{do chp}, 
some of the {\em ad hoc\/}
motion estimators from chapter \ref{mc chp} are reappraised in the light of
the maximum likelihood solution.

%%%%%%%%%%%%%%%%%%%%%%%%%%%%%%%%%%%%%%%%%%%%%%%%%%%%%%%%%%%%%%%%%%%%%%%%%%%%%
\section{The Motion Estimation Model}

The first step towards deriving any estimator involves writing down a
mathematical model expressing the relationship between the measurements and 
the unknown parameters on which the measurements depend.  

For ISAR motion estimation, the unknown parameter of most interest is the
radial change in the target's position at each discrete time step during
the ISAR imaging period.  If the solution has to allow for a target moving in a
totally arbitrary way, the fundamental problem is one of estimating a
target's radial movement between any two times at which its frequency 
response has been measured.  After briefly discussing the set of
measurements for a whole ISAR image, a model for this more fundamental
problem will be constructed.

%%%%%%%%%%%%%%%%%%%%%%%%%%%%%%%%%%%%%%%%%%%%%%%%%%%%%%%%%%%%%%%%%%%%%%%%%%%%%
\subsection{A Model for a Whole ISAR Image}

For stepped-frequency ISAR imaging, the measurements $\md$ consist of the
target's frequency responses sampled at $M$ evenly spaced times $t_m$
during the imaging period of $[-T/2,T/2]$
\begin{equation}
t_m=t_0+m\Delta t\qquad\mbox{for $m=0,1,\ldots,M-1$}
\end{equation}
where $t_0=-T/2$ and $t_{M-1}=T/2$.
Each frequency response at time $t_m$ is composed of samples of the
target's radar reflectivity $s(k_n,t_m)$ at the $N$ discrete frequencies
$k_n$ in the stepped-frequency waveform
\begin{equation}
k_n=k_0+n\dk\qquad\mbox{for $n=0,1,\ldots,N-1$}
\end{equation}
As in chapters \ref{hrr chp} to \ref{mc chp}, the $k_n$ are really twice the
wavenumber, so that $k_n=4\pi f_n/c$ where 
\begin{equation}\label{ml eqn:f0+ndf}
f_n=f_0+n\df
\end{equation}
For convenience, the single word ``frequency'' is often used to refer to
both $f$ and to $k=4\pi f/c$; the difference, if it is important, is
indicated by the context.

Using the notation and the results of chapter \ref{mc chp}, the full ISAR
motion estimation problem is to estimate the change in radial position 
\begin{equation}
\Delta r(t_m)=r(t_m)-r(t_0)
\end{equation}
of a moving target for all $M$ time steps using only the radar measurements 
\begin{equation}
s_{mn}=s(k_n,t_m)+\noise_{mn}
\end{equation}
where $s(k_n,t_m)$ is the target's frequency response at time $t_m$ and
frequency $k_n$, and $\noise_{mn}$ is the measurement noise affecting
$s_{mn}$.  The target is assumed to be rotating sufficiently
slowly\footnote{This assumption is questioned in section 
\protect\ref{rmc sec:me}.}
during the ISAR imaging period that any two samples at the same frequency
of $s(k_n,t_m)$ from different times, $t_m$ and $t_{m'}$, are related by a
phase shift depending on the radial distance moved by the target between
these times
\begin{equation}
s(k_n,t_m)=e^{-jk_n(r(t_m)-r(t_{m'}))}\,s(k_n,t_{m'})
\end{equation}

%%%%%%%%%%%%%%%%%%%%%%%%%%%%%%%%%%%%%%%%%%%%%%%%%%%%%%%%%%%%%%%%%%%%%%%%%%%%%
\subsection{The Fundamental Motion Estimation Problem}

As mentioned at the start of this chapter, the fundamental problem of
ISAR radial motion estimation is finding the radial distance $\dr$ moved by
a target between two times at which the target's frequency
response is measured.  

Suppose the target rotates a negligible amount between two times $t=0$ and
$t=T$.\footnote{From here to the end of chapter \protect\ref{sp chp}, $T$
will be generally be used to represent the time between the two frequency 
responses, not the ISAR imaging period.  This saves introducing yet another
symbol.} Then any change in its frequency response is directly attributable
to the radial distance $\Delta r$ moved by the target.  Using this model,
the measurements of the target's frequency responses at times $t=0$ and
$t=T$ are
\begin{eqnarray}
a_n&=&s_n+\na{n}\nn\\
b_n&=&s_n\e{-jk_n\dr}+\nb{n}    
\label{ml eqn:std model}
\end{eqnarray}
respectively where $s_n$ is the target's complex frequency response at frequency
$k_n$, the $n^{\rm th}$ frequency in the stepped-frequency waveform.
The phase shifts in the measurements of $b_n$ at
time $t=T$ are proportional to $\dr$, the unknown distance the target moved
during this interval.  $\na{n}$ and $\nb{n}$ are measurement noise terms
that are assumed to be independent and identically distributed zero-mean
complex Gaussian random processes with variance $\nv$
\begin{equation}
\E{\na{n}\overline{\na{n}}}=\E{\nb{n}\overline{\nb{n}}}=\nv
\end{equation}

Therefore the complete data set available for estimating $\dr$ is the 
set of $N$ pairs of frequency responses
\begin{equation}\label{ml eqn:md} 
\md=\left\{a_n,b_n \,|\, n=0,1,\ldots,N-1\right\}
\end{equation}
where $a_n$ and $b_n$ are the target's noisy frequency responses at times 
$t=0$ and $t=T$ respectively, as given in equation (\ref{ml eqn:std
model}).  In practice, the frequencies at which the $a_n$ and $b_n$ are
measured are uniformly spaced so that $k_n=k_0+n\dk$, but this is not a
necessary condition for deriving the maximum likelihood estimator of $\dr$.


Now that the measurement model in (\ref{ml eqn:std model}) and the
measurement vector $\md$ in (\ref{ml eqn:md}) have been defined, maximum likelihood estimation
can be used to find the maximum likelihood estimate $\rML$ of the unknown
radial distance $\dr$ that the target has moved between $t=0$ and $t=T$.

%%%%%%%%%%%%%%%%%%%%%%%%%%%%%%%%%%%%%%%%%%%%%%%%%%%%%%%%%%%%%%%%%%%%%%%%%%%%%
\section{Estimating $\dr$ Only}

The model in (\ref{ml eqn:std model}) has $N+1$ unknown parameters, the 
target's radial displacement $\dr$ and the target's reflectivities $s_n$ at
each of the $N$ frequencies in the stepped-frequency waveform.  By combining
the two measurement equations at each frequency to eliminate $s_n$, 
\begin{equation}\label{ml eqn:comb model}
\nb{n}=b_n-(a_n-\na{n})\e{-jk_n\dr}
\end{equation}
which now only contains one unknown parameter, namely $\dr$.

Following the description of maximum likelihood estimation in
chapter~\ref{mp chp}, the maximum likelihood estimate $\rML$ of 
$\dr$ is the value of $\dr$ which maximizes the likelihood function
$p(\md;\dr)$, the probability density function of observing the
measurements $\md$ conditional on the value of the unknown parameter $\dr$.

Since noise in the measurements at each of the $N$ frequencies is
independent, the $N$ pairs of measurements are themselves independent and
\begin{equation}\label{ml eqn:p(md;dr)}
p(\md;\dr)=\prod_{n=0}^{N-1} p(a_n,b_n;\dr)
\end{equation}

The individual likelihood functions $p(a_n,b_n;\dr)$ can be calculated 
from the probability density function of the two complex noise processes 
$\na{n}$ and $\nb{n}$
\begin{equation}
\prs{\na{n}}{z}=\prs{\nb{n}}{z}=\frac{1}{\pi\nv}\,\e{-\left|z\right|^2/\nv}
\end{equation}
by noting that $p(a_n,b_n;\dr)$ is the convolution
\begin{equation}
p(a_n,b_n;\dr)=\int_{\C}\prs{\na{n}}{z}\cdot\prs{\nb{n}}{b_n-(a_n-z)\e{-jk_n\dr}}\,dz
\end{equation}
where the subscript $\C$ on the integral sign indicates that the integral
is evaluated over the whole complex plane.  Then substituting in the
probability density functions of $\na{n}$ and $\nb{n}$ and using the identity 
\begin{equation}
\left|u\right|^2+\left|v\right|^2={\frac{1}{2}}\left(
\left|u-v\right|^2+\left|u+v\right|^2\right)
\end{equation}
the likelihood function $p(a_n,b_n;\dr)$ is
\begin{eqnarray}
p(a_n,b_n;\dr)
&=&\frac{1}{\left(\pi\nv\right)^2} \int_{\C}\e{-\left|z\right|^2/\nv}
\e{-\left|b_n-(a_n-z)\e{-jk_n\dr}\right|^2/\nv}\,dz \nn \\
&=&\frac{1}{\pi\nv}\,\e{-\left|a_n-b_n\e{jk_n\dr}\right|^2/2\nv}
\cdot\frac{1}{\pi\nv}
	\int_{\C}\e{-\left|2z-(a_n-b_n\e{jk_n\dr})\right|^2/2\nv}\,dz\nn\\
&=&\frac{2}{\pi\nv}\,\e{-\left|a_n-b_n\e{jk_n\dr}\right|^2/2\nv}
\end{eqnarray}

Applying this result to (\ref{ml eqn:p(md;dr)}) gives the likelihood
function for all $N$ pairs of measurements
\begin{eqnarray}
p(\md;\dr)&=&\prod_{n=0}^{N-1} p(a_n,b_n;\dr)\nn\\
&=&\prod_{n=0}^{N-1} \frac{2}{\pi\nv}\,
	\e{-\left|a_n-b_n\e{jk_n\dr}\right|^2/2\nv}\nn\\
&=&\frac{2^N}{\left(\pi\nv\right)^N}\,
	\exp\left(-\frac{1}{2\nv}
	\ds\sum_{n=0}^{N-1}\left|a_n-b_n\e{jk_n\dr}\right|^2\right)\nn\\
&=&\frac{2^N}{\left(\pi\nv\right)^N}\,
	\exp\left(-\frac{1}{2\nv}J(\dr)\right)
\label{ml eqn:p(md;dr) with J(r)}
\end{eqnarray}
where the cost function $J(\r)$ is
\begin{equation}\label{ml eqn:J(r)}
J(\r)=\sum_{n=0}^{N-1} \left|a_n-b_n\e{jk_n\r}\right|^2
\end{equation}

Since (\ref{ml eqn:p(md;dr) with J(r)}) shows that 
$p(\md;\dr)$ is a monotonically decreasing function of $J(\r)$,
the local maxima of $p(\md;\dr)$ are the local minima of $J(\r)$ and
the global maximum of $p(\md;\dr)$ is the global minimum of $J(\r)$.
Therefore a necessary and sufficient condition for the existence of a
maximum likelihood estimator of $\dr$ is a unique global minimum of $J(\r)$.  
A proof of the uniqueness of the global maximum of $J(\r)$ is given in 
section~\ref{ml sec:J(r)'s period}, so the maximum likelihood estimator $\rML$
exists and is given by
\begin{equation}\label{ml eqn:rML}
\rML=\arg\,\min_{\r}\,J(\r)
\end{equation}

%%%%%%%%%%%%%%%%%%%%%%%%%%%%%%%%%%%%%%%%%%%%%%%%%%%%%%%%%%%%%%%%%%%%%%%%%%%%%
\section{Estimating All Parameters}

The maximum likelihood derivation of the previous section assumed that $\dr$
was the only unknown parameter.  However this is an oversimplification of
the model because the target's frequency responses $s_n$ are always unknown
and the noise variance $\nv$ is normally unknown or has been estimated from
the radar data.  It is not apparent that it is legitimate to eliminate the
$s_n$ as in equation (\ref{ml eqn:comb model}) and assume that $\nv$ is
known when deriving $\rML$.  It is even less apparent that this is valid 
when analysing the statistical performance of $\rML$, such as while 
calculating the \CR bound.  

The correct technique is to treat all unknown parameters as components of a
vector of parameters to be estimated, and apply the vector form of maximum
likelihood estimation.  In this case, the column vector of unknown
parameters is
\begin{equation}\label{ml eqn:uvp}
\uvp=\left[\nv, \dr, s_0, s_1, \ldots, s_{N-1}\right]^T
\end{equation}
where $\nv$ and $\dr$ are real and the $s_n$ are complex.  

Solving the vector likelihood equations shows that the maximum likelihood
estimator of $\dr$ is the same as the expression obtained in 
equation~(\ref{ml eqn:rML}) when $\dr$ was considered the only unknown.  
An obvious question to ask is ``why should the vector maximum likelihood
estimator be used when the two methods of estimating $\dr$ produce the 
same answer?''  Three reasons why the vector estimator should be used are:
\begin{itemize}
\item Maximum likelihood estimation demands that the likelihood
equations maximizing $p(\md;\uvp)$ with respect to each $\ui{i}$ in $\uvp$
be solved simultaneously.  Solving just one of the likelihood equations 
(which is what the scalar unknown case does) does not guaranteed that the 
others are also solved.  
\item Finding the maximum likelihood estimates of $\nv$ and $s_n$ gives
expressions that are useful in their own right.
\item The vector and the scalar maximum likelihood estimators may have
different statistical properties even though the estimator of $\dr$ is the
same in both cases.  In particular the \CR bound is higher for the vector
estimator than for the scalar estimator.
\end{itemize}

A little thought shows that the first of these reasons is not entirely valid.
This is because if the scalar maximum likelihood estimator can be evaluated,
the neglected unknown parameters must have been eliminated.  This lack of
dependence of the unknown parameter on the other neglected unknown 
parameters is reflected in the vector case by the likelihood equation for
the one unknown parameter from the scalar case becoming decoupled from the
likelihood equations for the unknown parameters that were ignored in the
scalar case.

The second and third reasons, however, are strong enough to justify the
additional complexity of the full multiple-parameter maximum likelihood
solution.

%%%%%%%%%%%%%%%%%%%%%%%%%%%%%%%%%%%%%%%%%%%%%%%%%%%%%%%%%%%%%%%%%%%%%%%%%%%%%
\subsection{The Vector Maximum Likelihood Estimator}

The likelihood function for estimating the vector $\uvp$ of unknown
parameters from the data set $\md$ can be broken into the product of
$N$ likelihood functions of the form $p(a_n,b_n;\dr,s_n,\nv)$ because each 
of the noise terms in (\ref{ml eqn:std model}) is independent.  Then
\begin{equation}
p(\md;\uvp)=\prod_{n=0}^{N-1}p(a_n,b_n;\dr,s_n,\nv)
\end{equation}
Furthermore, because $\na{n}$ and $\nb{n}$ are independent,
$p(a_n,b_n;\dr,s_n,\nv)$ is the product
\begin{equation}
p(a_n,b_n;\dr,s_n,\nv)=p(a_n;\dr,s_n,\nv)\cdot p(b_n;\dr,s_n,\nv)
\end{equation}
Each of these can be written down in terms of the complex Gaussian
probability density function
\begin{eqnarray}
p(a_n;\dr,s_n,\nv)&=&\frac{1}{\pi\nv}\,
\exp\left(-\frac{1}{\nv}\left|a_n-s_n\right|^2\right) \\
p(b_n;\dr,s_n,\nv)&=&\frac{1}{\pi\nv}\,
\exp\left(-\frac{1}{\nv}\left|b_n-s_n\e{-jk_n\dr}\right|^2\right)
\end{eqnarray}
Then the likelihood function for frequency $k_n$ is
\begin{equation}
p(a_n,b_n;\dr,s_n,\nv)=\frac{1}{\left(\pi\nv{}\right)^2}\,\exp\left(
-\frac{1}{\nv}\left[
\left|\vphantom{\e{k}}a_n-s_n\right|^2+\left|b_n-s_n\e{-jk_n\dr}\right|^2
\right]\right)
\end{equation}
and the likelihood function for all $N$ frequencies is
\begin{eqnarray}
p(\md;\uvp)&=&\prod_{n=0}^{N-1} p(a_n,b_n;\uvp)\nn\\
&=&\frac{1}{\left(\pi\nv\right)^{2N}}\,\exp\left(
-\frac{1}{\nv}\sum_{n=0}^{N-1}
\left|\vphantom{\e{k}}a_n-s_n\right|^2+\left|b_n-s_n\e{-jk_n\dr}\right|^2
\right)
\label{ml eqn:p(md;uvp)}
\end{eqnarray}

Since $\ln x$ increases monotonically as $x$ increases, the likelihood 
function is usually written as the log-likelihood function to
remove the exponential
\begin{equation}\label{ml eqn:ln p(md;uvp)}
\ln p(\md;\uvp)=-2N\ln\left(\pi\nv\right)-\frac{1}{\nv}\sum_{n=0}^{N-1}
\left|\vphantom{\e{k}}a_n-s_n\right|^2+\left|b_n-s_n\e{-jk_n\dr}\right|^2
\end{equation}
The maximum likelihood estimator $\uvpML$ is the value of $\uvp$ which
maximizes $\ln p(\md;\uvp)$.  This is equivalent to solving the log-likelihood
equation 
\begin{equation}
\left.\frac{\del\ln p(\md;\uvp)}{\del\uvp}\right|_{\uvp=\uvpML}=0
\end{equation}
provided that $\ln p(\md|\uvp)$ has its only local extremum at the global
maximum.  If there is more than one local extremum, some other information
must be available to choose the solution of (\ref{ml eqn:ln p(md;uvp)})
which maximizes $\ln p(\md;\uvp)$.

%%%%%%%%%%%%%%%%%%%%%%%%%%%%%%%%%%%%%%%%%%%%%%%%%%%%%%%%%%%%%%%%%%%%%%%%%%%%%
\subsection{Finding $\sML{n}$}

The partial derivative of $\ln p(\md;\uvp)$ with respect to $s_n$, the
target's reflectivity at frequency $k_n$, is 
\begin{equation}
\frac{\del\ln p(\md;\uvp)}{\del s_n}=
-\frac{1}{\nv}\,\frac{\del}{\del s_n}\sum_{n'=0}^{N-1}
\left|\vphantom{\e{k}}a_{n'}-s_{n'}\right|^2
+\left|b_{n'}-s_{n'}\e{-jk_{n'}\dr}\right|^2
\end{equation}
Using the definition of the derivative of a real function of a complex
argument in (\ref{mp eqn:complex deriv}), 
\begin{eqnarray}
\frac{\del\ln p(\md;\uvp)}{\del s_n}
&=&-\frac{1}{\nv}\,\frac{\del}{\del s_n}\left(
\left|\vphantom{\e{k}}a_n-s_n\right|^2+\left|b_n-s_n\e{-jk_n\dr}\right|^2
\right)\nn\\
&=&-\frac{1}{\nv}\,\frac{\del}{\del s_n}\left(
\left|\vphantom{\e{k}}s_n-a_n\right|^2+\left|s_n-b_n\e{jk_n\dr}\right|^2
\right)\nn\\
&=&-\frac{1}{\nv}\left[
\left(\vphantom{\e{k}}\conj{s_n}-\conj{a_n}\right)
+\left(\conj{s_n}-\conj{b_n}\e{-jk_n\dr}\right)\right]\nn\\
&=&-\frac{1}{\nv}\conj{\left(
2s_n-\left(a_n+b_n\e{jk_n\dr}\right)\right)}
\end{eqnarray}
Replacing $s_n$ and $\dr$ with their maximum likelihood estimators 
$\sML{n}$ and $\rML$, then equating to zero shows that
\begin{equation}\label{ml eqn:sML}
\sML{n}=\frac{1}{2}\left(a_n+b_n\e{jk_n\rML}\right)
\end{equation}

Therefore the maximum likelihood estimate of the target's response at
frequency $k_n$ is the average of the noisy measurements at $t=0$ and $t=T$
after $b_n$ has had its phase adjusted to compensate for the target's radial
motion during the interval $[0,T]$.

%%%%%%%%%%%%%%%%%%%%%%%%%%%%%%%%%%%%%%%%%%%%%%%%%%%%%%%%%%%%%%%%%%%%%%%%%%%%%
\subsection{Finding $\rML$}

The partial derivative of $\ln p(\md;\uvp)$ with respect to $\dr$, the
radial distance the target has moved, is 
\begin{eqnarray}
\frac{\del\ln p(\md;\uvp)}{\del \dr}
&=&-\frac{1}{\nv}\,\frac{\del}{\del \dr}\sum_{n=0}^{N-1}
\left|\vphantom{\e{k}}a_n-s_n\right|^2+\left|b_n-s_n\e{-jk_n\dr}\right|^2\nn\\
&=&-\frac{1}{\nv}\sum_{n=0}^{N-1}\frac{\del}{\del \dr}
\left|b_n-s_n\e{-jk_n\dr}\right|^2
\end{eqnarray}
Since $\dr$ is real, this is the derivative of a real function of a real
variable.  So
\begin{eqnarray}
\frac{\del\ln p(\md;\uvp)}{\del \dr}
&=&-\frac{1}{\nv}\sum_{n=0}^{N-1}\frac{\del}{\del \dr}
\left|b_n-s_n\e{-jk_n\dr}\right|^2\nn\\
&=&\frac{1}{\nv}\sum_{n=0}^{N-1}jk_n\left(
b_n\conj{s_n}\e{jk_n\dr}+\conj{b_n}s_n\e{-jk_n\dr}\right)\nn\\
&=&-\frac{2}{\nv}\imag{\sum_{n=0}^{N-1}k_nb_n\conj{s_n}\e{jk_n\dr}}
\end{eqnarray}
Replacing $\dr$ with its maximum likelihood estimator $\rML$ and $s_n$ with
the expression for $\sML{n}$ from (\ref{ml eqn:sML}), then equating to zero
gives the solution of the likelihood equation as
\begin{eqnarray}
0&=&\imag{\sum_{n=0}^{N} k_n\sML{n}\conj{b_n}\e{-jk_n\rML}}\nn\\
&=&\frac{1}{2}\imag{\sum_{n=0}^{N} k_n\left(a_n+b_n\e{jk_n\rML}\right)
\conj{b_n}\e{-jk_n\rML}}\nn\\
&=&\frac{1}{2}\imag{\sum_{n=0}^{N} k_na_n\conj{b_n}\e{-jk_n\rML}}
\end{eqnarray}
because the imaginary part of $b_n\conj{b_n}$ is zero.

Therefore the maximum likelihood estimator $\rML$ is one of the solutions of 
\begin{equation}
\imag{\sum_{n=0}^{N} k_na_n\conj{b_n}\e{-jk_n\rML}}=0
\end{equation}
This has many solutions, not all of which maximize $\ln p(\md;\uvp)$.

To find which of these solutions maximizes $\ln p(\md;\uvp)$, use the
maximum likelihood estimator of $s_n$ in (\ref{ml eqn:sML}) to eliminate the
$s_n$ from (\ref{ml eqn:p(md;uvp)}).  Then
\begin{eqnarray}
\left|\vphantom{\e{k}}a_n-\sML{n}\right|^2
&=&\left|\vphantom{\e{k}}a_n-\frac{1}{2}
	\left(a_n+b_n\e{jk_n\rML}\right)\right|^2\nn\\
&=&\frac{1}{4}\left|a_n-b_n\e{jk_n\rML}\right|^2
\end{eqnarray}
and
\begin{eqnarray}
\left|b_n-\sML{n}\e{-jk_n\rML}\right|^2
&=&\left|b_n-\frac{1}{2}\left(a_n+b_n\e{jk_n\rML}\right)
	\e{-jk_n\rML}\right|^2\nn\\
&=&\frac{1}{4}\left|b_n-a_n\e{-jk_n\rML}\right|^2\nn\\
&=&\frac{1}{4}\left|a_n-b_n\e{jk_n\rML}\right|^2
\end{eqnarray}
Therefore the maximum value of $p(\md;\uvp)$ is
\begin{eqnarray}
p(\md;\uvp)
&=&\frac{1}{\left(\pi\nv\right)^{2N}}\,\exp\left(
-\frac{1}{\nv}\sum_{n=0}^{N-1}
\left|\vphantom{\e{k}}a_n-\sML{n}\right|^2+\left|b_n-\sML{n}\e{-jk_n\dr}\right|^2
\right)\nn\\
&=&\frac{1}{\left(\pi\nv\right)^{2N}}\,\exp\left(
-\frac{1}{2\nv}\sum_{n=0}^{N-1}
\left|a_n-b_n\e{jk_n\rML}\right|^2\right)
\end{eqnarray}
Thus the maximum likelihood estimator $\rML$ is found by minimizing 
\begin{equation}
\sum_{n=0}^{N-1} \left|a_n-b_n\e{jk_n\r}\right|^2
\end{equation}
which is the same cost function as $J(\r)$ in equation (\ref{ml eqn:J(r)})
derived using the scalar maximum likelihood estimator.
Therefore
\begin{equation}\label{ml eqn:J(r) uvp}
\rML=\arg\,\min_{\r}\,J(\r)
\end{equation}
where the cost function is
\begin{equation}
J(\r)=\sum_{n=0}^{N-1} \left|a_n-b_n\e{jk_n\r}\right|^2
\end{equation}

%%%%%%%%%%%%%%%%%%%%%%%%%%%%%%%%%%%%%%%%%%%%%%%%%%%%%%%%%%%%%%%%%%%%%%%%%%%%%
\subsection{Finding $\nvML$}

The final parameter to be estimated is the noise variance $\nv$.
The partial derivative of $\ln p(\md;\uvp)$ with respect to
$\nv$ is
\begin{equation}
\frac{\del\ln p(\md;\uvp)}{\del \nv}
=-\frac{2N}{\nv}+\frac{1}{\left(\nv\right)^2}
\sum_{n=0}^{N-1}
\left|\vphantom{\e{k}}a_n-s_n\right|^2+\left|b_n-s_n\e{-jk_n\dr}\right|^2
\end{equation}
Equating to zero and substituting in $\nvML$, $\sML{n}$ and $\rML$ shows that
\begin{equation}
\nvML=\frac{1}{2N}\sum_{n=0}^{N-1} 
\left|a_n-\sML{n}\right|^2+\left|b_n-\sML{n}\e{-jk_n\rML}\right|^2
\end{equation}
Using the expression for $\sML{n}$ in (\ref{ml eqn:sML}) gives the noise 
variance as $1/4N$ times the minimum of $J(\r)$
\begin{eqnarray}
\nvML
&=&\frac{1}{4N}\sum_{n=0}^{N-1} \left|a_n-b_n\e{jk_n\rML}\right|^2\nn\\
&=&\frac{1}{4N}\,J(\rML)		
\label{ml eqn:nvML}
\end{eqnarray}

%%%%%%%%%%%%%%%%%%%%%%%%%%%%%%%%%%%%%%%%%%%%%%%%%%%%%%%%%%%%%%%%%%%%%%%%%%%%%
\section{Properties of the Cost Function $J(\r)$}

%----------------------------------------------------------------------------
\begin{figure}
\caption{The cost function $J(\r)$ for a variety of simulated and
experimental radar data as a function of estimated radial motion, $r$.}
\label{ml fig:J(r) pictures}

These used a stepped-frequency radar with an initial frequency of 
$f_0=9.16$~GHz and $128$ frequency steps with a $2$~MHz spacing.
In (a) and (b), the target has no radial motion, while in (c) and (d), the
target moved $-0.58$~m between the two frequency responses.

\def\picturewidth{0.5\textwidth}
\def\captionwidth{0.37\textwidth}
\def\precaptionwidth{0.083\textwidth}
\def\postcaptionwidth{0.047\textwidth}

\def\subcaption#1#2{\hbox{\hbox to\precaptionwidth{\hfill#1\quad}
	\parbox[t]{\captionwidth}{#2}\hbox to\postcaptionwidth{}}}

\vbox{
\vphantom{I}
\hbox{\epsfxsize=\picturewidth\epsffile{../pics/ugfig1a.eps}
	\epsfxsize=\picturewidth\epsffile{../pics/ugfig1d.eps}}
\hbox{	
\subcaption{(a)}{$J(\r)$ for a simple simulated target
	with a frequency response of $s(f)\equiv 1$.}
\subcaption{(b)}{A magnified view of the global minimum of (a), showing how
	the high frequency modulation locates the minimum very accurately.}
}
\vphantom{I}
\vphantom{I}
\hbox{\epsfxsize=\picturewidth\epsffile{../pics/ugfig1b.eps}
	\epsfxsize=\picturewidth\epsffile{../pics/ugfig1c.eps}}
\hbox{
\subcaption{(c)}{$J(\r)$ for a simulation of a 737 aircraft composed of 165
	point scatterers.  This shows a similar shape to that of (a).}
\subcaption{(d)}{$J(\r)$ for a real 737 aircraft whose frequency responses
were measured while landing at Adelaide Airport.}
}
}
\end{figure}
%----------------------------------------------------------------------------

In this section, the cost function $J(\r)$ is examined in detail to discover 
why it is useful for ISAR motion estimation, and to see whether the
assumption of a unique global minimum necessary for the existence of the
maximum likelihood estimator is justified.

The exact behaviour of $J(\r)$ depends on the target's frequency response, 
which in turn depends on the target's geometry and orientation.  
However, the general features of $J(\r)$ are common to all targets, and
these will be illustrated by considering a very simple target geometry.
For this simple geometry, it will be shown that $J(\r)$ has a unique global
minimum when the frequency response is known for a continuum of frequencies.
For arbitrary targets imaged using stepped-frequency ISAR, it will be shown 
that $J(\r)$ is periodic but this does not invalidate the maximum likelihood
estimate because the period is exactly the length of the ISAR range 
ambiguity window.  Therefore, it can be safely assumed that for real targets, 
$J(\r)$ has a unique global minimum whose location is the maximum 
likelihood estimate of $\dr$.

%%%%%%%%%%%%%%%%%%%%%%%%%%%%%%%%%%%%%%%%%%%%%%%%%%%%%%%%%%%%%%%%%%%%%%
\subsection{General Features of $J(\r)$}

To illustrate the general behaviour of the cost function $J(\r)$, consider 
a simple target whose frequency responses are measured in the 
absence of noise at times $t=0$ and $t=T$ respectively to give
\begin{eqnarray}
a(f)&=&1\nn\\
b(f)&=&\e{-j\fpc f\dr}\label{ml eqn:flattarget}
\end{eqnarray}
This represents a target, with a physically unrealistic constant radar 
cross-section, which moves a distance $\dr$ between $t=0$ and $t=T$.  
For analytical simplicity, assume that the frequency responses are measured 
over a continuous frequency band from $f_0$ to $f_1=f_0+\Delta f$.  
Then equation (\ref{ml eqn:J(r)}) for $J(\r)$ becomes an integral rather than 
a summation
\begin{equation}
J(\r)=\frac{1}{\Delta f}\int_{f_0}^{f_1} 
\left|a(f)-b(f)\e{j\fpc f\r}\right|^2\;df
\end{equation}
Solving this for the frequency responses in (\ref{ml eqn:flattarget}) gives
\begin{equation}
J(\r)=2\left[1-m(\r)e(\r)\right]\label{ml eqn:J(r) for flat target}
\end{equation}
where $e(\r)$ is the low frequency envelope
\begin{equation}
e(\r)=\sinc\left(\frac{2}{c}\Delta f(\dr-\r)\right)
\end{equation}
which has zero crossings every $c/2\Delta f$, and $m(\r)$ is the high 
frequency modulation
\begin{equation}
m(\r)=\cos\left(\frac{4\pi}{c}\overline{f}(\dr-\r)\right)
\end{equation}
which has a period of $c/2\overline{f}$ where $\overline{f}=(f_0+f_1)/2$.

From (\ref{ml eqn:J(r) for flat target}), $J(\r)$ is only ever zero when both 
$e(\r)$ and $m(\r)$ are unity.  The envelope $e(\r)$ is only unity when 
$\r=\dr$ and the cosine modulation $m(\r)$ has one of its local maxima at 
$\r=\dr$.  

Therefore the global minimum of $J(\r)$ occurs when $\r$ is equal to the
radial distance $\dr$ the target moved, and so
\begin{equation}
\rML=\dr
\end{equation}

This combination of low frequency envelope and high frequency modulation
enables the global minimum of $J(\r)$ to be estimated very accurately.  The
envelope alone would give only a rough estimate of $\dr$ because
the envelope is so flat in the neighbourhood of $\r=\dr$.  The
cosinusoidal modulation alone is not suitable for estimating $\dr$
because there is no way of knowing which of the closely spaced local extrema
locates $\dr$.  But by multiplying the envelope and the modulation
together, the modulation's high frequency ensures that the global minimum 
can be located accurately, and the envelope ensures that there is no 
ambiguity in choosing which of the local minima is the global minimum.

$J(\r)$ is plotted for a number of simulations and real targets in 
figure~\ref{ml fig:J(r) pictures}.  Figure~\ref{ml fig:J(r) pictures}(a) is 
$J(\r)$ for the simple target in equation (\ref{ml eqn:J(r) for flat target}).  
The nearly solid mass of black inside the envelope indicates that the local 
minima of $J(\r)$ are very closely spaced.  The region surrounding the global 
minimum is enlarged in \ref{ml fig:J(r) pictures}(b), clearly showing how 
closely spaced the local extrema of $J(\r)$ are.  $J(\r)$ for a simulated 
aircraft and a real aircraft are presented in 
figures~\ref{ml fig:J(r) pictures}(c) and (d).  

The consistency of the overall shape of the three different cost function curves
in figure~\ref{ml fig:J(r) pictures} shows how the general form of $J(\r)$ is 
maintained, even in the presence of noise for the
real aircraft in \ref{ml fig:J(r) pictures}(d).  Note that in 
\ref{ml fig:J(r) pictures}(d), the maximum likelihood estimate of the 
signal-to-noise ratio is given by the amplitude of $J(\r)$ at the 
global minimum.

%%%%%%%%%%%%%%%%%%%%%%%%%%%%%%%%%%%%%%%%%%%%%%%%%%%%%%%%%%%%%%%%%%%%%%
\subsection{Uniqueness of $J(\r)$'s Global Minimum}
\label{ml sec:J(r)'s period}

The existence of the maximum likelihood estimator $\rML$ requires that the 
cost function $J(\r)$ has a unique global minimum.  This is because if
$J(\r)$ has a number of equally deep local minima  but no unique global 
minimum, there is no way of telling which of the local minima should be
chosen as $\rML$.

By considering stepped-frequency ISAR with uniformly spaced frequency
steps, it will be shown that $J(\r)$ is periodic, which would seem to deny
the existence of the maximum likelihood estimator.  However, the period of
$J(\r)$ is equal to the size of the ISAR range ambiguity window so the
periodicity of $J(\r)$ may be neglected and the global minimum is considered
unique.

If the frequency responses are sampled at discrete evenly-spaced
frequencies\footnote{A uniformly spaced stepped-frequency waveform is only
required for this periodicity proof and for some of the efficient ways of
evaluating the maximum likelihood estimator in chapter 
\protect\ref{ee chp}.}, $J(\r)$ is periodic.  To find the period explicitly, 
consider a target with frequency response $s_n$ at frequency $f_n$ moving a
radial distance $\dr$ between $t=0$ and $t=T$ in the absence of noise. Then
\begin{eqnarray}
a_n&=&s_n\nn\\
b_n&=&s_n\e{-j\fpc f_n\dr}\label{ml eqn:general target}
\end{eqnarray}
From this noiseless model, $J(\r)$ is
\begin{eqnarray}
J(\r)&=&2\left(\sum_{n=0}^{N-1}|s_n|^2\right)-2\real{
\e{-j\fpc f_0(\dr-\r)}\sum_{n=0}^{N-1}\left|s_n\right|^2
\e{-j\frac{4\pi}{c}(f_n-f_0)(\dr-\r)}}
\end{eqnarray}
Because $f_n-f_0=n\Delta f$ for stepped-frequency ISAR,
\begin{equation}
J(\r)=2\left(\sum_{n=0}^{N-1}\left|s_n\right|^2\right)
-2\real{m(\r)\overline{e(\r)}}
\end{equation}
where the high frequency complex modulation is
\begin{equation}
m(\r)=\e{-j\fpc f_0(\dr-\r)}
\end{equation}
and the low frequency complex envelope is
\begin{equation}\label{ml eqn:e(r)}
e(\r)=\sum_{n=0}^{N-1}\left|s_n\right|^2\e{j\frac{4\pi}{c}n\Delta f(\dr-\r)}
\end{equation}
So $J(\r)$ is minimized when the magnitude of $e(\r)$ is greatest and $e(\r)$ and
$m(\r)$ have the same phase modulo $2\pi$.

For $e(\r)$ to attain its greatest magnitude, each term in the summation 
of (\ref{ml eqn:e(r)}) must have the same phase modulo $2\pi$
\begin{equation}
\frac{4\pi}{c}n\Delta f(\dr-\r)=2u_n\pi +\phi
\end{equation}
where each of the $u_n$ is integral.  Since this has to hold for every $n$
from $0$ to $N-1$, $u_n=un$ for some integer $u$. Putting $n=0$ shows 
that $\phi=0$.  Therefore $e(\r)$'s magnitude is greatest if
\begin{equation}\label{ml eqn:kcondit}
\dr-\r=\frac{uc}{2\Delta f}
\end{equation}

Whenever this is satisfied, $m(\r)$'s phase must also be an integral multiple
of $2\pi$.  Therefore
\begin{equation}\label{ml eqn:lcondit}
uf_0(\dr-\r)=2v\pi
\end{equation}
where $v$ is integral.  Putting (\ref{ml eqn:kcondit}) and (\ref{ml eqn:lcondit}) 
together gives the condition
\begin{equation}
uf_0=v\Delta f
\end{equation}
on $u$ and $v$.  The first solution, $u=v=0$, occurs when $\dr=\r$.
At worst, the next solution is $u=\pm 1$ when $\Delta f$ is a factor of
$f_0$.  Therefore the period of $J(\r)$ is a multiple of
$$\frac{c}{2\Delta f}$$
This is exactly the size of the range ambiguity window in stepped-frequency
ISAR.  Since radial movements of exact multiples of the range ambiguity
window do not affect the final ISAR image, this shows that the 
periodic nature of $J(\r)$ can be ignored for motion estimation for 
ISAR imaging.

Therefore $\rML$ found by minimizing $J(\r)$ is indeed the maximum
likelihood estimate of $\dr$.

%%%%%%%%%%%%%%%%%%%%%%%%%%%%%%%%%%%%%%%%%%%%%%%%%%%%%%%%%%%%%%%%%%%%%%%%%%%%%
\section{Summary of Maximum Likelihood Motion Estimation}

To summarize the results of this chapter, applying maximum likelihood 
estimation to the problem of estimating the unknown parameters of a 
target moving during an interval $[0,T]$ shows that
\begin{itemize}
\item The maximum likelihood estimator of the radial distance the target
has moved during the time interval is 
$$\rML=\arg\,\min_{\r}\,J(\r)$$
where the cost function is
$$J(\r)=\sum_{n=0}^{N-1} \left|a_n-b_n\e{jk_n\r}\right|^2$$
\item The maximum likelihood estimator of the noise variance is $1/4N$ 
times the value of $J(\rML)$
$$\nvML=\frac{1}{4N}\min_{\r}\,J(\r)$$
\item The maximum likelihood estimator of the target's frequency response is
the average of the noisy measurements at times $t=0$ and $t=T$ once the 
measurements at $t=T$ have been phase-shifted to move the target back a 
radial distance of $\rML$
$$\sML{n}=\frac{1}{2}\left(a_n+b_n\e{jk_n\rML}\right)$$
\end{itemize}
