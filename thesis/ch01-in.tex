%%%%%%%%%%%%%%%%%%%%%%%%%%%%%%%%%%%%%%%%%%%%%%%%%%%%%%%%%%%%%%%%%%%%%%%%%%%
%
%			INVERSE SYNTHETIC APERTURE RADAR
%
%				  PhD Thesis
%
%		Stephen Simmons		simmons@ee.mu.oz.au
%
%	    Department of Electrical and Electronic Engineering
%	    University of Melbourne, Parkville, 3052, Australia
%
% Chapter 1:	Introduction
%
%		started first draft:
%		finished first draft:
%		submitted:		
%
%%%%%%%%%%%%%%%%%%%%% Copyright (C) 1994 Stephen Simmons %%%%%%%%%%%%%%%%%%


\chapter{Introduction}
\label{in chp}

\bigletter Radar has been used for over fifty years to locate and track
distant aircraft and ships, but it is only comparatively recently that it
has been used for forming high resolution two-dimensional images.  Since
the size of a radar system's antenna determines its angular resolution,
radar imaging systems would ordinarily require a prohibitively large
antenna to obtain a high enough angular resolution.  However, by combining
measurements made by an antenna as it moves to different locations over a
large synthetic aperture, digital signal processing creates a synthetic
aperture much larger than physical size of the radar's antenna.  This
synthetic aperture has a very high angular resolution and is used to form
the radar image.

The two forms of radar imaging are synthetic aperture radar (SAR) and
inverse synthetic aperture radar (ISAR).  In SAR imaging, the radar is
carried aboard a moving platform such as an aircraft or satellite.  The
platform's motion creates the synthetic aperture which is used to form high
resolution maps of the terrain below.  ISAR imaging recognises that the
synthetic aperture may be created just as effectively if the target being
imaged is moving and the radar is stationary.

The success of radar imaging relies on the relative position and
orientation of the radar antenna and the target being known precisely at
all times.  Radar signal processing is coherent, so it is important that
any errors in forming the synthetic aperture are less than a small fraction
of a radar wavelength.  For microwave radar systems operating at X-band,
the wavelength is about 3~cm, so this places very strict requirements on
the stability of the synthetic aperture.

The movement of the radar antenna and the target cannot be expected to
produce a synthetic aperture that is this accurate.  To prevent the radar
image becoming degraded, the relative motion between radar and target must
be continuously measured and appropriate corrections made to the synthetic
aperture.  Measuring the relative motion is called motion estimation and 
correcting the radar measurements for distortions in the synthetic aperture
is called motion compensation.

With SAR imaging, the radar moves and the target is stationary.  Therefore
an inertial nagivation unit carried with the radar can be used to measure
the position of the radar antenna at all times.  This is not accurate to a
small fraction of a wavelength, but it is good enough to correct for most
of the radar's motion.  Some further autofocusing may be required for very
high resolution SAR images to remove blurring caused by small residual
motion errors. 

In ISAR imaging, the target moves and the radar is stationary.  Because the
target's motion is not known {\em a priori\/}, it must be obtained solely
from the radar measurements that are processed to form the radar image. 
This makes ISAR imaging, particularly the motion estimation and motion
compensation parts, a much more difficult problem than SAR.  As a result,
SAR systems have been operating for more than thirty years while robust
ISAR systems are still at the research and development stage.  

A number of methods of ISAR motion estimation and motion compensation have
been developed.  These methods can generally be classed as either range
profile alignment, where range profiles of the target over time are roughly
aligned to correct for the target's large scale radial movement, or as
phase compensation, where residual motion errors less than the size of a
radar wavelength are corrected.  Range profile alignment is usually
performed by cross-correlating range profiles of the target measured at
different times.  Phase compensation algorithms, such as adaptive
beamforming and the phase gradient autofocus, compare the phases of radar
measurements made at different times to correct for slight motion errors.  

Many variations on these basic methods of motion estimation have been
developed.  All are based on intuitive notions of what constitutes the best
approach to estimating the target's motion.  The result is many {\em ad
hoc\/} methods of motion estimation.  Few make any concrete claims about
the accuracy of their estimates or analyse the conditions under which they
perform well or badly.

In this thesis, statistical estimation theory is used to obtain the maximum
likelihood solution of the ISAR motion estimation problem.  From this, the
estimator's statistical performance is characterised and connections are made
with methods of range profile alignment and phase compensation.

%%%%%%%%%%%%%%%%%%%%%%%%%%%%%%%%%%%%%%%%%%%%%%%%%%%%%%%%%%%%%%%%%%%%%%%%%%%%%
\section{Outline of Thesis}

This thesis considers the problem of estimating the radial motion of a
moving target during ISAR imaging.  Statistical estimation theory is used
to derive a new motion estimator that is the maximum likelihood solution of
the motion estimation problem.  Efficient algorithms are developed for
evaluating the estimator and its statistical performance is analysed in
detail.  Similarities between the maximum likelihood solution and other
{\em ad hoc\/} methods of motion estimation are considered.  The thesis
concludes with a detailed analysis of some of the assumptions implicitly or
explicitly made during ISAR imaging.  This places several new constraints
on the motion of the target and the resolution of the ISAR image.

Chapter \ref{mp chp} is an introduction to the principles of statistical
estimation theory, and in particular, to maximum likelihood estimation. The
various forms of the maximum likelihood estimator for scalar and vector
parameters are derived, along with the \CR bounds on any estimator's
variance.  The chapter concludes with a discussion of applying estimation
theory when the parameters being estimated are complex numbers.

Chapters \ref{hrr chp} to \ref{mc chp} serve as an introduction to high
resolution radar and radar imaging and as a detailed description of some
motion estimation algorithms that are curently used in ISAR imaging.

Chapter \ref{hrr chp} begins with an overview of radar imaging using SAR
and ISAR and a qualitative description of the requirements for achieving a
high resolution two-dimensional radar image of a moving target.  Once the 
general model for radar reflections from a distributed target has been
derived, the requirements for a high range resolution and a high
cross-range resolution are analysed to obtain some conditions on the
radar's waveforms and the the target's motion.  This demonstrates the need
for a wideband waveform such as a chirp or a stepped-frequency waveform;
both chirp and stepped-frequency waveforms are considered, but the
remainder of the thesis assumes that a stepped-frequency radar is being
used.

Chapter \ref{ii chp} defines the basic model used for ISAR imaging and
obtains ISAR imaging algorithms using range-Doppler processing and
wavenumber ($\omega-k$) processing.  The traditional view of ISAR as a
Doppler phenomenon is criticised, and this old-fashioned description of
ISAR has been included for completeness in appendix \ref{ii app:dop} at the
end of the chapter.

Chapter \ref{mc chp} emphasises the importance of motion estimation and
motion compensation in ISAR imaging.  Methods of correcting deviations in
the target's radial position and orientation are derived, and these motion
compensation algorithms are used to suggest how the target's radial
position and orientation may be estimated.  A number of radial motion
estimation algorithms are discussed.  These methods include range
profile alignment using cross-correlation and image feedback control, and
phase compensation using adaptive beamforming and the phase gradient
autofocus.

The next three chapters, chapters \ref{ml chp} to \ref{sp chp}, are a
detailed examination of a new method of radial motion estimation for ISAR
based on the principles of maximum likelihood estimation.  Expressions for
the maximum likelihood estimator are derived, algorithms for evaluating the
estimator efficiently are developed and the estimator's statistical
performance is carefully analysed.

Chapter \ref{ml chp} sets up a model for estimating the radial distance
$\dr$ moved by a target in the time between two frequency responses being
measured.  The maximum likelihood estimator of $\dr$ is derived in two
different ways based on this model.  The first assumes that the radial
distance moved by the target is the only unknown parameter, treating the
problem as one of estimating a scalar parameter.  The second form of the
maximum likelihood estimator recognises that the measurement noise variance
and the target's frequency response are also unknown, and should be
included with $\dr$ to give a vector of unknown parameters.  These two
estimators of $\dr$ have the same mathematical form, namely a cost function
$J(\r)$ whose global minimum gives the maximum likelihood estimate of
$\dr$.  The chapter also obtains expressions for the maximum likelihood
estimators of the noise variance and of the target's frequency response.

Chapter \ref{ee chp} considers possible methods of finding the global
minimum of the cost function $J(\r)$.  This is not a simple task because
$J(\r)$ has a slowly varying envelope, giving a coarse estimate of $\dr$,
and a high frequency modulation, which allows a very accurate estimate of
the fractional part of the number of half-wavelengths the target has moved. 
Reflecting these characteristics of $J(\r)$ at the two different scales,
algorithms are developed which first find the approximate location of the
minimum of $J(\r)$, and then refine this estimate until $\dr$ has been
estimated to within a fraction of a radar wavelength.  The algorithms for
finding the approximate location of the global minimum of $J(\r)$ are based
on the fast Fourier transform and the chirp-Z transform.  Two algorithms
for finding the fraction of half a wavelength the target has moved have
been developed. The first is an approximation that estimates the parameters
of a sinusoid fitted to the high frequency modulation.  The second is an
iterative scheme that uses a contraction mapping converging to the local
minima of $J(\r)$.  The accuracies of each of these algorithms in finding
the location of the maximum likelihood estimator have been specified and 
guaranteed by rigorous proofs of convergence.

Chapter \ref{sp chp} is a detailed examination of the maximum likelihood
estimators' statistical properties.  These include showing that the maximum
likelihood estimator of $\dr$ is unbiased and obtaining bounds on the 
estimator's variance for both the scalar and vector forms of the maximum
likelihood estimator.  Some suggestions are made towards obtaining
non-asymptotic expressions for the estimator's probability
density function, but complete non-asymptotic results could not be
obtained.

Chapter \ref{do chp} is a comparison between some of the conventional
methods of ISAR radial motion estimation from chapter \ref{mc chp} and the
maximum likelihood solution from chapters \ref{ml chp} to \ref{sp chp}.
The maximum likelihood estimator is rewritten as the cross-correlation of
the target's complex range profiles.  Phase gradient autofocus, adaptive
beamforming and conventional range profile alignment using
cross-correlation are shown to be closely related to modified forms 
of the maximum likelihood solution, a result which gives a sound
theoretical basis to these largely {\em ad hoc\/} methods of radial motion
estimation.  The differences between these methods and the maximum 
likelihood estimator suggest that they sacrifice accuracy for a decreased
sensitivity to phase noise caused by the target's rotation.


Chapter \ref{rmc chp}, the penultimate chapter in the thesis, is a detailed
examination of four of the approximations and assumptions that are
frequently made in ISAR imaging.  These four are the plane-wave
approximation in range-Doppler ISAR processing, the conditions under which
polar reformatting may be omitted, the assumption of negligible rotation
during radial motion estimation, and the assumption that motion
compensation need only correct the target's radial motion to the nearest
multiple of a range bin and then within half a wavelength.  All four
approximations are analysed in the same way, using a technique that can be
used to analyse any approximation in ISAR imaging.  This leads to four sets
of constraints on the ISAR image's resolution and the target's motion.  The
set of constraints arising from neglecting polar reformatting is shown to
be indentical to the constraints obtained by restricting the motion of
scatterers on the target through range and cross-range resolution cells, a
situation which appears to have nothing in common with polar reformatting. 
The analysis of a target's rotation during radial motion estimation shows
that the target should rotate sufficiently slowly that it fills no more
than one-quarter of the ISAR image in the cross-range direction.  The
analysis of the final assumption shows that provided phase compensation is
perfect, the range profile alignment can misestimate the target's radial
position by one-quarter of a range bin before the quality of the resulting
ISAR image is adversely affected.

The last chapter of the thesis, chapter \ref{co chp}, contains the 
conclusions of this research into ISAR motion estimation and
suggests some avenues for future research.


%%%%%%%%%%%%%%%%%%%%%%%%%%%%%%%%%%%%%%%%%%%%%%%%%%%%%%%%%%%%%%%%%%%%%%%%%%%%%
\section{Original Contributions}

This thesis contains the following major contributions to the field of ISAR
imaging:
\begin{itemize}
\item The whole of chapter \ref{ml chp}, including the maximum likelihood
formulation of the ISAR motion estimation problem, the two forms of the
maximum likelihood estimator of the target's radial movement and the maximum
likelihood estimator of the noise variance.

\item The whole of chapter \ref{ee chp}, including all algorithms for the
global and local minimization of the cost function $J(\r)$.  Algorithm
\ref{ee alg:min using DFT} is superficially similar to motion estimation by
cross-correlating range profiles, but it is actually quite different because
it does not take the magnitude before correlating.  Algorithm \ref{ee
alg:min using CZT} is the first time the chirp-Z transform has been used in
ISAR motion estimation.  The two algorithms for the local minimization of
the cost function are also new.  These algorithms are all proved to behave as
claimed.  Such rigorous proofs are rare in the ISAR literature.

\item The whole of chapter \ref{sp chp}, especially the results concerning
the unbiased nature of the maximum likelihood estimator of $\dr$, and the
\CR bounds.  The work on non-asymptotic results has prompted several general
questions in the field of extreme value theory.  These are stated in section
\ref{co sec:et}, and concern the probability density function of the location of the
maximum of a known bandlimited function in bandlimited noise.

\item The whole of chapter \ref{do chp}, especially the formulation of the
maximum likelihood estimator in terms of the cross-correlation of the
complex range profiles and the connections between the maximum likelihood
estimator and conventional methods of range profile alignment and phase
compensation.

\item Chapter \ref{rmc chp} is the first time that a single approach to
analysing approximations in ISAR has been used to test so many assumptions.
The constraints for neglecting polar reformatting are derived completely
differently from the usual analysis of motion through resolution cells,
leading to identical constraints.  Section \ref{rmc sec:me} is the first
time the effect of the target's rotation on radial motion estimation has
been considered.  The constraint that the target should occupy no more than
one-quarter of the cross-range window is new, as is the definition and
measurement of rotation-induced noise.  Section \ref{rmc sec:ma} is the
first analysis of the accuracy required for motion compensation, and it
justifies the use of range profile alignment followed by phase compensation.

\item The critical review of ISAR imaging in chapters \ref{hrr chp} to
\ref{mc chp} is also an important but less major contribution.
\end{itemize}

\noindent A number of papers resulting from this research are in preparation:
\begin{itemize}
\item S. E. Simmons and R. J. Evans, ``Maximum Likelihood Motion Estimation
for ISAR Imaging.'' (to be submitted to {\em IEEE Transactions on
Information Theory})
\item S. E. Simmons and R. J. Evans, ``Resolution and Motion Constraints in
ISAR.'' (to be submitted to {\em IEEE Transactions on
Signal Processing})
\item S. E. Simmons and R. J. Evans, ``ISAR Motion Estimators and the
Maximum Likelihood Solution.'' (to be submitted to {\em IEEE Transactions on
Aerospace and Electronic Systems})
\end{itemize}

\noindent Three conference papers on this and earlier research have been presented:
\begin{itemize}
\item S. E. Simmons and R. J. Evans, ``Maximum Likelihood Autofocusing of
Radar Images'', to appear in {\em Proc. Radar '95\/}, Washington D.C., 
8--11 May 1995.
\item S. E. Simmons and R. J. Evans, ``A New Approach to Motion Estimation
for ISAR Imaging'', {\em Proc. ICASSP '94\/}, Adelaide, 19--22 April 1994,
pp. V-201--V-204.
\item S. E. Simmons, D. J. Heilbronn and D. Nandagopal,	``Kalman Range 
Tracking for ISAR Motion Compensation'', {\em Proc. Radarcon '90\/},
Adelaide, 18--20 April 1990, pp. 689--694.
\end{itemize}
The ICASSP paper uses the same motion estimator as the maximum likelihood
estimator derived here, except that at the time it was not recognised as
such.  The methods of evaluating the motion estimator in that paper are far
less efficient than the methods presented in chapter \ref{ee chp} of this
thesis.  The Radarcon paper was an early attempt to derive a frequency
domain motion estimator that is not affected by $2\pi$ ambiguities.

The material in chapter \ref{mp chp} is very similar to the following 
course notes on maximum likelihood estimation:
\begin{itemize}
\item S. E. Simmons, {\em Maximum Likelihood Estimation\/} in ``Course Notes
on Estimation Theory'' prepared for the CSSIP Estimation Theory Course, 
Department of Electrical Engineering, University of Melbourne, 14--16
February 1994.
\end{itemize}

