%%%%%%%%%%%%%%%%%%%%%%%%%%%%%%%%%%%%%%%%%%%%%%%%%%%%%%%%%%%%%%%%%%%%%%%%%%%
%
%			INVERSE SYNTHETIC APERTURE RADAR
%
%				  PhD Thesis
%
%		Stephen Simmons		simmons@ee.mu.oz.au
%
%	    Department of Electrical and Electronic Engineering
%	    University of Melbourne, Parkville, 3052, Australia
%
% Chapter 11:	Conclusions
%
%		started first draft:	Fri 6 Jan 1995
%		finished first draft:	Fri 6 Jan 1995
%		submitted:		Mon 9 Jan 1995
%
%%%%%%%%%%%%%%%%%%%%% Copyright (C) 1995 Stephen Simmons %%%%%%%%%%%%%%%%%%

\chapter{Conclusions}
\label{co chp}

\bigletter This thesis has applied the formal methodology of estimation
theory to the problem of radial motion estimation for stepped-frequency
ISAR imaging. The maximum likelihood estimator of the target's radial
motion was derived, efficient algorithms were developed for evaluating the
maximum likelihood estimator and its statistical properties were
examined in detail. The maximum likelihood estimator has been compared with
other methods of ISAR motion estimation, and the differences and
similarities discussed. Finally, a number of assumptions made in ISAR
imaging have been analysed, leading to several new constraints on the
target's motion and the resolution of the ISAR image.

After discussing the conclusions in more detail, some possible topics for
further research are suggested.

%%%%%%%%%%%%%%%%%%%%%%%%%%%%%%%%%%%%%%%%%%%%%%%%%%%%%%%%%%%%%%%%%%%%%%%%%%%%%
\section{Conclusions}

This thesis has investigated the use of estimation theory to find optimal
estimators of a target's radial motion during stepped-frequency ISAR
imaging and to analyse their statistical performance.  The aim of this was
two-fold: to apply a rigorously mathematical approach to a problem that 
has traditionally been solved using more intuitive methods, and to develop
a unifying framework in whose terms the intuitive methods could be
reinterpreted and reappraised.


The first step of applying estimation theory involved specifying a model
relating the target's radial motion to measurements of its frequency
response at two different times.  This model assumed that as the target
moved, its rotation was negligible.  

The maximum likelihood estimator $\rML$ of the radial distance $\dr$ moved 
by the target was obtained in two different ways.  The first was the scalar
maximum likelihood estimator, which assumed that $\dr$ was the only unknown
parameter.  The second, the vector maximum likelihood estimator, estimated  
$\dr$ along with the measurement noise variance $\nv$ and the target's 
frequency response.  Both estimators of $\dr$ took the same form, that of a
cost function, $J(\r)$, whose global minimum is located at $\rML$.

The cost function $J(\r)$ has a slowly varying envelope modulated by a high
frequency carrier at twice the stepped-frequency waveform's average
frequency.  Conventional methods of minimizing a function could not be
applied because of the closely spaced local extrema of $J(\r)$, so
algorithms for the approximate global minimization of $J(\r)$ and the exact
local minimization of $J(\r)$ were developed.  The global algorithms were
based on the fast Fourier transform and the chirp-Z transform and provided
efficient ways of finding the approximate value of $\rML$.  The local
algorithm was iterative, and by proving that it was a contraction mapping
which converged at a rate equal to the radar's relative bandwidth, it was
shown that two iterations are usually sufficient to estimate $\rML$ for 
X-band ISAR. 

The statistical properties of $\rML$ were examined next.  It was shown that
$\rML$ is an unbiased estimator of $\dr$, and the \CR bound was used to
establish a lower bound on the estimator's variance.  Some progress was also
made towards finding the probability density function of $\rML$.

Before comparing the maximum likelihood estimator to conventional methods of
range profile alignment and phase compensation, it was shown that $\rML$
could be rewritten as the location of the maximum of the cross-correlation of the complex 
range profiles formed by Fourier-transforming the two frequency responses.
The cross-correlation methods of range profile alignment usually
cross-correlate the magnitudes of the target's range profiles.  It was shown
that taking the magnitudes of the range profiles makes the cross-correlation
less sensitive to the target's rotation at the expense of a reduced
accuracy.  This led to two strategies for range profile alignment.  

The first strategy cross-correlated consecutive pairs of complex range 
profiles.  This would be used if the cross-correlation has a high 
resolution (implemented, for example, using the chirp-Z transform).  This is 
justified because the target's rotation is negligible in the period
between the consecutive profiles.

The alternative strategy cross-correlated the magnitude of each range
profile with the magnitude of a reference profile.  This was appropriate 
when a cross-correlation with a low resolution was used 
(implemented, for example, using a fast Fourier transform) because
only radial movements greater than about a range bin
could be measured.  Correlating the complex profiles is not possible because
the interval between profiles could be much larger than under the
first strategy so the target's rotation could no longer be considered
negligible.

Phase alignment techniques of the phase gradient autofocus and adaptive
beamforming were also compared to the maximum likelihood estimator.  It was
shown that the phase gradient part of phase gradient autofocus could be
implemented using slightly sub-optimal versions of the algorithm for the 
exact local minimization of the cost function $J(\r)$.  This suggested that
the performance of the maximum likelihood motion estimator might be
improved by applying the same shift and windowing stages as the phase
gradient autofocus, although this is yet to be verified.

Finally, four assumptions commonly made during ISAR imaging were analysed. 
Constraints under which the assumptions are valid were derived by
considering the magnitude of the phase errors they cause to samples of the
target's reflectivity in the $(k_x,k_y)$-plane immediately before the ISAR
inversion.  The first assumption involved the conditions under which the
plane wave approximation of range-Doppler ISAR is valid.  The second
assumption was that polar reformatting could be omitted from the ISAR
inversion algorithm.  The constraints resulting from this assumption were
found to be identical to the constraints obtained by restricting the movement
of scatterers on the target through range and cross-range resolution cells.

The third assumption questioned whether the target's rotation during ISAR
imaging could be considered negligible for the purposes of motion
estimation.  It was found that the target's rotation caused acceptably small
phase errors between consecutive frequency responses provided the target
filled no more than one-quarter of the ISAR cross-range ambiguity window.
The term rotation-induced noise was used to describe the phase errors during
motion estimation caused by the target's motion and it was verified
experimentally how sensitive rotation-induced noise is to the target's
rotation rate.

The fourth assumption questioned whether the ISAR image would be degraded
if the small radial motion errors were only corrected using phase
alignment. It was found that if phase alignment completely corrected for
the fractional number of half-wavelengths moved by the target, the residual
radial motion error could be up to one-quarter of an ISAR range bin.

To conclude, this work has illustrated how a mathematical approach using
estimation theory can be used to obtain useful ISAR motion estimators and
provide valuable insights into other motion estimators whose derivations
otherwise could not be justified so easily.

%%%%%%%%%%%%%%%%%%%%%%%%%%%%%%%%%%%%%%%%%%%%%%%%%%%%%%%%%%%%%%%%%%%%%%%%%%%%%
\section{Future Research}

As is often the case, this research has prompted just as many questions as
it has resolved.  Here is a brief description of some unresolved issues and
some suggestions for related future research.  They are loosely divided
into those concerned with radar imaging and those concerned with estimation
theory.

%%%%%%%%%%%%%%%%%%%%%%%%%%%%%%%%%%%%%%%%%%%%%%%%%%%%%%%%%%%%%%%%%%%%%%%%%%%%%
\subsection{Radar Imaging}


\begin{itemize}
\item Test the maximum likelihood motion estimators of chapter \ref{ml chp}
and the efficient algorithms of chapter \ref{ee chp} on real radar data and
ISAR simulations to determine their effectiveness under a wide range of
conditions.

\item Perform proper theoretical analyses of the methods of range profile
alignment and phase compensation described in chapter \ref{mc chp}.  Some
theoretical results concerning their performance in noise and their
robustness to the target's rotation are long overdue.

\item As discussed in chapter \ref{do chp}, modify Steinberg's dominant scatterer
algorithm to use more than one reference range bin for phase compensation
without losing the ability to correct phase errors of rotating targets.

\item Restate the motion estimation model for chirp waveforms and derive 
the equivalent maximum likelihood motion estimator in the range domain.

\item Extend the model used to derive the maximum likelihood motion
estimator to include the effects of the target's rotation implicitly.  From
this, derive new maximum likelihood motion estimators that are better able
to deal with the target's rotation.

\item Define a measure of a motion estimator's performance that is better
than variance at describing the quality of the resulting ISAR image. 
Section \ref{rmc sec:ma} shows how motion estimation errors may extend for a
significant fraction of a range bin if they are clustered enough around
multiples of half a wavelength.  Therefore any performance measure must take into
account the total range of errors and how tightly they are clustered.

\item Consider applying the maximum likelihood motion estimators to SAR
imaging.  This would have to use many of the elements of the phase
gradient autofocus method as SAR images have strong scatterers over a much
broader cross-range extent than does ISAR.  The end result would probably
be much like the phase gradient autofocus, but able to detect errors 
between range profiles of more than half a wavelength without ambiguities.

\item Perform proper statistical analyses of the phase gradient autofocus,
and in particular, prove that it converges to a focussed SAR image.  Figure
4 of \cite{Wah94} suggests that the phase gradient autofocus converges even
for SAR scenes with very low dynamic ranges and few strong point-like
scatterers.  If this is the case, the way it converges must be much more
complicated that the simple explanation given in \cite{Eic89},
\cite{Eic89b} and \cite{Jak89}.

\item Look at the problem of angular motion estimation.

\item Consider adaptive ISAR waveforms which achieve an optimal balance
between the needs of accurate motion estimation and collecting enough data
for a high resolution image.  Because ISAR motion estimation requires lower
rotation rates than ISAR imaging, perhaps some form of interleaved 
stepped-frequency waveforms could be used.  One stepped-frequency waveform
would be used for imaging and the other for motion estimation.  Depending on
the target's rotation rate, the signal-to-noise ratio and the desired resolution of
the ISAR image, the parameters of the two stepped-frequency waveforms could
be adjusted to match the accuracy of the motion estimation with that needed
for the particular ISAR image.
\end{itemize}

%%%%%%%%%%%%%%%%%%%%%%%%%%%%%%%%%%%%%%%%%%%%%%%%%%%%%%%%%%%%%%%%%%%%%%%%%%%%%
\subsection{Estimation Theory}
\label{co sec:et}

\begin{itemize}
\item Work out more of the non-asymptotic properties of the maximum
likelihood motion estimator, especially the probability density function.

\item Generalizing the previous topic suggests the following problem, 
related to the probability density function of a maximum likelihood 
estimator:

%============================================================================
\begin{problem}[Maximum of a Bandlimited Function I]
\label{co prb:blf I}\mbox{}\par

Given a bandlimited function $f(t)$ whose values are known on an interval
$[0,T]$ and a bandlimited Gaussian noise process $\noise(t)$, determine the
probability density function $p(t,\nv)$ of the location of the maximum of
$f(t)+\noise(t)$ on $[0,T]$ as a function of the noise's variance $\nv$.
\end{problem}
%============================================================================

This problem is very simply stated, but the solution is probably not at all
simple.  When the noise process's variance is zero, $p(t,0)=\delta(t-t_0)$ 
where $t_0$ is the location of the global maximum of $f(t)$ on $[0,T]$.  As
$\nv\to\infty$, $p(t,\nv)\to 1/T$.  For some values of $\nv$ between these
two extremes, $p(t,\nv)$ will have a shape that is similar to $f(t)$
itself, in that the peaks of the probability density function will occur at
the location of the peaks of $f(t)$, and similarly for the troughs.

\item Problem \ref{co prb:blf I} leads to a more abstract problem:

%============================================================================
\begin{problem}[Maximum of a Bandlimited Function II]
\label{co prb:blf II}\mbox{}\par

Define $f(t)$ and $p(t,\nv)$ as in problem \ref{co prb:blf I}.  Determine 
the set $B$ which contains all bandlimited functions $f(t)$ such that 
\begin{equation}
p(t,\nv)=f(t)
\end{equation}
for all $t\in[0,T]$ and for some $\nv$ dependent on $f$.

Clearly $f(t)=1/T$ is a member of $B$.  Are there any non-trivial
members?
\end{problem}
%============================================================================

\item Or yet another:

%============================================================================
\begin{problem}[Maximum of a Bandlimited Function III]
\label{co prb:blf III}\mbox{}\par

Define $f(t)$ and $p(t,\nv)$ as in problem \ref{co prb:blf I}.  Is
$p(t,\nv)$ bandlimited?
\end{problem}
%============================================================================

These problems are similar to the questions considered by the branch of
statistics called extreme value theory.  However, extreme value theory has
many results about expected numbers of local extrema, expected frequencies
of level crossings, and other asymptotic properties of local phenomena, but
few about non-asymptotic properties of global phenomena.

Some of the early work of Rice \cite{Ric39,Ric44,Ric45,Ric48} and of
Middleton \cite{Mid48} concerned extreme values of random waveforms, but
they considered nothing quite like these three problems.  Surveys of extreme
value theory such as \cite{Bla73} and \cite{Wei73} do not indicate any
comparable results either.  
\end{itemize}


