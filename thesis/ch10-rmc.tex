%%%%%%%%%%%%%%%%%%%%%%%%%%%%%%%%%%%%%%%%%%%%%%%%%%%%%%%%%%%%%%%%%%%%%%%%%%%
%
%			INVERSE SYNTHETIC APERTURE RADAR
%
%				  PhD Thesis
%
%		Stephen Simmons		simmons@ee.mu.oz.au
%
%	    Department of Electrical and Electronic Engineering
%	    University of Melbourne, Parkville, 3052, Australia
%
% Chapter 10:	Resolution and Motion Constraints
%
%		started first draft:	25 December 1994
%		finished first draft:	29 December 1994
%		submitted:		Mon 9 Jan 1995
%
%%%%%%%%%%%%%%%%%%%%% Copyright (C) 1994 Stephen Simmons %%%%%%%%%%%%%%%%%%

\chapter{Resolution and Motion Constraints}
\label{rmc chp}

\bigletter A number of approximations have been made throughout this thesis
to justify various simplifying assumptions that may be needed for ISAR
inversion algorithms.  In this chapter, three approximations will be
analysed in detail.  These are:

\begin{description}
\item[Range-Doppler Processing] With range-Doppler ISAR, it is assumed that
$r_{xy}(t)$, the range of a scatterer at $(x,y)$ on the target at time $t$, 
can be written as a linear function of $x$ and $y$.

\item[Polar Reformatting] Polar reformatting can be avoided if the polar
grid on which the radar measurements give $G(k_x,k_y)$ is approximately
rectangular.  

\item[Motion Estimation] For accurate motion estimation, the target's radar 
reflections, $s(k,0)$ and
$s(k,\tau)$, at two times $t=0$ and $t=\tau$, are assumed to differ only by a
phase shift of $e^{-jk\dr}$, where $\dr$ is the radial distance the target
moves between the two times.  This is justified if the target has neglible
rotation during $[0,\tau]$.
\end{description}

The chapter also contains an analysis of the residual motion
errors remaining after motion estimation and motion compensation. 
Conceptually, this analysis of motion estimation errors belongs in 
chapter \ref{mc chp}.  However, because it uses the same techniques that
have been developed here for analysing the three assumptions
mentioned above, it has been included in section \ref{rmc sec:ma}.

For each of the three assumptions to be valid, constraints must be imposed on
the size and motion of the target and the resolution of the ISAR image.
These constraints can be checked when imaging a target to see whether the
range-Doppler approximation is justified, and if so, whether polar
reformatting can be left out without seriously compromising the image's
quality.

The classical\footnote{i.e. Incorrect.} view of ISAR as imaging by
velocity-induced Doppler shifts also imposes restrictions on the size of the
target and the image's resolution.  Breaking the target into range cells
and cross-range cells, and insisting that no scatterer moves through a
resolution cell in the range or cross-range directions, leads to constraints
such as \cite{Aus84}
\begin{eqnarray}
\Delta r_c^2&>&\frac{\lambda D_r}{4}\label{rmc eqn:prc I}\\
\Delta r_r \Delta r_c&>&\frac{\lambda D_c}{4}\label{rmc eqn:prc II}
\end{eqnarray}
Here $D_r$ and $D_c$ are the dimensions of the target in range and 
cross-range, and $\lambda$ is the wavelength at the radar's centre frequency.
Similar expressions have been obtained by Brown \cite{Bro69}, 
Walker \cite{Wal80} and Wehner \cite[eq.~(7.31)]{Weh87}, and they are also
derived in section \ref{ii sec:dop rr} on page \pageref{ii sec:dop rr}.


However, constraining the target's motion so that no point moves through
more than one range cell or one cross-range cell is an artifact of the
Doppler-shift view of ISAR.  This is dubious reasoning because these are not
fundamental limits of ISAR or even of range-Doppler ISAR. Rather they are
constraints on the need for polar reformatting, a fact that is obscured in
the Doppler-shift description of ISAR.\footnote{To be fair to Wehner, he
does mention that the existence of the blur radius is due to the lack of
polar reformatting, although he does not analyse the equivalent constraints
when polar reformatting is used. However, these conditions on polar
reformatting are obtained from the Doppler-shift view of ISAR, a model of
ISAR in which polar reformatting does not exist!  The coincidence that the
blur radius constraint is the same as that for polar reformatting 
perpertuates the myth that the blur radius is important.}

This chapter contains a proper analysis of both the constraints due to
range-Doppler processing and the constraints due to avoiding polar
reformatting.  These are followed by a derivation of the constraints due to
the need for motion estimation and motion compensation in ISAR imaging.

%%%%%%%%%%%%%%%%%%%%%%%%%%%%%%%%%%%%%%%%%%%%%%%%%%%%%%%%%%%%%%%%%%%%%%%%%%%%%
\section{Describing the Target's Size}

Since this chapter is concerned with finding some straightforward limits on
the resolution of an ISAR image and the size of a target, the target will be
described in a very simple way.  

The target is characterized by its reflectivity function $g(x,y)$, so the 
support of $g(x,y)$ will be taken to be the extent of the target.  For
simplicity, assume that the target's $x$ axis is parallel to the
radar's boresight axis and its $y$ axis is perpendicular to the boresight
axis.  This corresponds to the orientation of the rotating target at time
$t=0$ in figure \ref{ii fig:rotating target} on page \pageref{ii fig:rotating target}.  
Then the target's range extent $D_r$ and cross-range extent $D_c$ are chosen 
so that
\begin{eqnarray}
\left|x\right|&\leq&\frac{D_r}{2}	\label{rmc eqn:dr}\\
\left|y\right|&\leq&\frac{D_c}{2}	\label{rmc eqn:dc}
\end{eqnarray}
for all points $(x,y)$ on the target.

This is not a complete description of the target's size in other directions,
but the simple model of the target's size as a rectangle $D_r\times D_c$ is
sufficient.  A degree of flexibility can be used in establishing the
constraints derived here because the approximations to which they apply 
gradually become invalid.  This leads to gentle degradations in the ISAR
image that affect scatterers at the target's extremities more severely than
those at target's centre.

As an example, the sets of constraints obtained in this chapter are
derived by assuming a maximum phase variation of $\pi/2$.  This could just
as easily have been specified as maximum phase variation of $\pi/4$ or
$\pi$ or any other sensible amount.  Similarly, there is no reason why the
RMS phase variation could not have been used (except that the mathematics is
considerably more difficult).  As a consequence of this flexibility, the
limits on range and cross-range resolution throughout this chapter should be
interpreted as ``soft'' limits rather than exact limits.

So when a constraint is written in the form
\begin{equation}
\alpha>\beta
\end{equation}
it does not mean that the ISAR image is prefectly focussed if $\alpha>\beta$ 
and completely blurred 
if $\alpha\leq\beta$.  Rather, it means that the particular approximation 
is valid if $\alpha\gg\beta$, and that as $\alpha$ approaches $\beta$, the
image's quality starts to degrade.

%%%%%%%%%%%%%%%%%%%%%%%%%%%%%%%%%%%%%%%%%%%%%%%%%%%%%%%%%%%%%%%%%%%%%%%%%%%%%
\section{Range-Doppler ISAR}
\label{rmc sec:rd}

ISAR using range-Doppler processing, as described in section \ref{ii sec:rdp}, 
expresses the range of a scatterer at $(x,y)$ on the target at time $t$ as a 
time-varying linear function of $x$ and $y$.  The error in this linear
approximation can be used to derive limits on the maximum size of a target
for a given range and cross-range resolution. 

The analysis in this section applies to stationary targets rotating at
constant angular velocity.  Similar results have been obtained in different
ways by Munson and Soumekh.  Munson analysed the curvature in the wavefront 
for spotlight-mode SAR \cite[eq. (22)--(24)]{Mun83} and Soumekh considerd
phase errors due to the plane wave approximation, also for spotlight-mode SAR
\cite[eq. (A16)]{Sou92a}.  The approach taken in this section is slightly
different from these.  Here, the variation in the phase error across the
$(k_x,k_y)$ plane is used to obtain the constraints on the target and the
ISAR image.  The difference between this and Soumekh's approach is more 
conceptual than mathematical; its advantage is that the same framework can
be used to analyse any approximations in $r_{xy}(t)$, including polar
reformatting and motion estimation (which are the subjects of sections 
\ref{rmc sec:pr} and \ref{rmc sec:me}).

%%%%%%%%%%%%%%%%%%%%%%%%%%%%%%%%%%%%%%%%%%%%%%%%%%%%%%%%%%%%%%%%%%%%%%%%%%%%%
\subsection{Defining the Phase Variation $\Delta\psi_{xy}$}

From equation (\ref{ii eqn:rxy(t) rot}), the distance between a scatterer at
$(x,y)$ in the target's local coordinate system and the radar is
\begin{equation}
r_{xy}(t)=\sqrt{(r_0+x\cos\omega t-y\sin\omega t)^2
+(x\sin\omega t+y\cos\omega t)^2}
\end{equation}
at time $t$.  The range-Doppler approximation assumes that $r_{xy}(t)$ can
be written as a linear function of $x$ and $y$, as in equation (\ref{ii
eqn:rot approx}).  The error inherent in this approximation is found by
taking the first order Taylor series of $r_{xy}$ while including the
remainder.  This gives
\begin{equation}
r_{xy}(t)=r_0+x\cos\omega t-y\sin\omega t+\epsilon_{xy}(t)
\end{equation}
where $\epsilon_{xy}(t)$ is the Taylor series' remainder.

From the model (\ref{ii eqn:s(k,t) II}) of the radar measurements at 
frequency $k$ and time $t$, 
\begin{equation}
s(k,t)=\int\!\!\int g(x,y)\,e^{-jkr_{xy}(t)}\,dx\,dy
\end{equation}
the range-Doppler approximation is equivalent to using a modified model
\begin{equation}
s(k,t)=e^{-jkr_0}
\int\!\!\int g'(x,y,k,t)\,e^{-jk(x\cos\omega t-y\sin\omega t)}\,dx\,dy
\end{equation}
where the target's reflectivity $g(x,y)$ has been replaced by an apparent
reflectivity $g'(x,y,k,t)$.  This includes the frequency-varying
and time-varying phase errors due to the linear approximation of 
$r_{xy}(t)$
\begin{equation}
g'(x,y,k,t)=g(x,y)\,e^{-jk\epsilon_{xy}(t)}=g(x,y)\,e^{-j\psi_{xy}(k,t)}
\end{equation}
where the phase error in $g(x,y)$ at time $t$ and frequency $k$ has been
written $\psi_{xy}(k,t)$.  

Most ISAR inversion algorithms assume that $g(x,y)$ is independent of time
and frequency,\footnote{One that does not is described by Synder {\em et
al.}, who consider target reflectivity functions to be random processes 
\protect\cite{Sny89}.  Their approach is too demanding for the 
current state of the art in computer technology.} 
so any change in $\psi_{xy}(k,t)$ for the $N$ values of
$k$ from $k_0$ to $k_{N-1}$ or for the $M$ values of $t$ from $t_0=-T/2$ to
$t_{M-1}=T/2$ causes blurring in the final ISAR image.  Note that for any
particular point on the target, $(x,y)$, the average value of 
$\psi_{xy}(k,t)$ is unimportant because this can be absorbed into the
phase of $g(x,y)$.  What matters is the variation over all $k_n$ and $t_m$
of $\psi_{xy}(k,t)$ for the single point $(x,y)$ on the target.  
Define the total variation in $\psi_{xy}(k,t)$ for the scatterer at $(x,y)$
as $\Delta\psi_{xy}$ where
\begin{equation}
\label{rmc eqn:Dpsi}
\Delta\psi_{xy}=\max_{k,t,k',t'}\left|\psi_{xy}(k,t)-\psi_{xy}(k',t')\right|
\end{equation}
Then $\Delta\psi_{xy}$ can be directly related to the blurring in the ISAR
image in the neighbourhood of $(x,y)$ due to the linear range-Doppler
approximation.  Setting an upper limit on the variation in phase error
across the target imposes restrictions on the size of the target and the
resolution of the ISAR image.

Define $\Delta\psi$ as the maximum phase variation over the whole of the 
target,
\begin{equation}\label{rmc eqn:dphi def}
\Delta\psi=\sup_{(x,y)} \Delta\psi_{xy}
\end{equation}
where the supremum is taken for $(x,y)$ ranging over the whole of the
target.\footnote{$\Delta\psi$ is a {\em supremum\/} over $x$ and $y$ because
$x$ and $y$ are continuous variables.  $\Delta\psi_{xy}$ is a {\em
maximum\/} of $k$, $t$, $k'$ and $t'$ because these are discrete variables,
taking on one of the $M$ time steps or one of the $N$ frequency steps.}
The upper limit used in \cite{Aus84} is a quarter of a wavelength, which
corresponds to a maximum phase variation of $\pi/2$.  Munson used $\pi/4$
\cite{Mun83}.  Using a maximum of $\pi/2$ here, the maximum phase variation
must satisfy 
\begin{equation}\label{rmc eqn:dphi lim}
\Delta\psi<\frac{\pi}{2}
\end{equation}

%%%%%%%%%%%%%%%%%%%%%%%%%%%%%%%%%%%%%%%%%%%%%%%%%%%%%%%%%%%%%%%%%%%%%%%%%%%%%
\subsection{Evaluating the Phase Variation}

The range error $\epsilon_{xy}(t)$ is given by the remainder for a first
order Taylor series in two variables
\begin{equation}
\epsilon_{xy}(t)=\frac{1}{2}
\left(x\frac{\del}{\del x'}+y\frac{\del}{\del y'}\right)^2 r_{x'y'}(t)
\end{equation}
for some $x'$ between $0$ and $x$ and some $y'$ between $0$ and $y$.  
$x'$ and $y'$ depend on $(x,y)$, the point at which the Taylor series is 
evaluated, and on $t$ because the Taylor series has time-varying coefficients.  
They can be written
\begin{eqnarray}
x'&=&\alpha(t)\,x\\
y'&=&\alpha(t)\,y
\end{eqnarray}
for some $\alpha(t)$ satisfying $0<\alpha(t)<1$.  Strictly speaking,
$\alpha(t)$ should be written $\alpha(t,x,y)$, but the extra parameters will
be omitted because at the moment, the variation in $\epsilon_{xy}(t)$ with
time is being analysed for one particular point on the target.  

Evaluating the second order partial derivatives in the Taylor series'
remainder shows that
\begin{equation}
\epsilon_{xy}(t)=\frac{x^2+y^2}{2r_{x'y'}(t)}-\frac{1}{2r_{x'y'}^3(t)}
\left[x(x'+r_0\cos\omega t)+y(y'+r_0\sin\omega t)\right]^2
\end{equation}
Under the standard ISAR assumption that the size of the target is small in 
comparison to its range (an assumption that is consistent with the
constraints at the conclusion of this analysis), $r_{x'y'}(t)$ may be
replaced by $r_0$ to give a simpler expression for the range error
\begin{equation}
\label{rmc eqn:e(t)}
\epsilon_{xy}(t)=\frac{r_0}{2}\left[
\frac{x^2}{r_0^2}+\frac{y^2}{r_0^2}-
\left(
	\frac{x}{r_0}\left(\frac{x'}{r_0}+\cos\omega t\right)
       +\frac{y}{r_0}\left(\frac{y'}{r_0}+\sin\omega t\right)
\right)^2\right]
\end{equation}

Write $(x,y)$ in polar coordinates
\begin{eqnarray}
x&=&d\cos\phi\\
y&=&d\sin\phi
\end{eqnarray}
so that $(x',y')=(\alpha(t)\,d\cos\phi,\alpha(t)\,d\sin\phi)$.  Using this
in (\ref{rmc eqn:e(t)}) shows that the range error is
\begin{equation}
\epsilon_{xy}(t)=\frac{d^2}{2r_0}\left[1-\left(\frac{\alpha(t)\,d}{r_0}
+\cos(\phi-\omega t)\right)^{\!2}\right]
\end{equation}
so that
\begin{equation}
\psi_{xy}(k,t)=k\frac{d^2}{2r_0}\left[1-\left(\frac{\alpha(t)\,d}{r_0}
+\cos(\phi-\omega t)\right)^{\!2}\right]
\end{equation}

Therefore from the definition of the phase variation in (\ref{rmc
eqn:Dpsi}),
\begin{eqnarray}
\Delta\psi_{xy}
&=&\max_{k,t,k',t'}\left|\psi_{xy}(k,t)-\psi_{xy}(k',t')\right|\nn\\
&=&\max_{k,t,k',t'}\frac{d^2}{2r_0}\left|
(k-k')-k\left(\frac{\alpha(t)\,d}{r_0}+\cos(\phi-\omega t)\right)^{\!2}
\right.\nn\\
&&{}\left.\qquad+k'\left(\frac{\alpha(t)\,d}{r_0}+\cos(\phi-\omega t')
\right)^{\!2}
\right|
\end{eqnarray}
The $\alpha(t)\,d/r_0$ inside the modulus may be neglected, giving a simpler
approximate form of the phase variation.  With the approximation written as
equality, this is
\begin{equation}
\Delta\psi_{xy}=\max_{k,t,k',t'}\frac{d^2}{2r_0} 
\left|k\sin^2\left(\phi-\omega t\right)
-k'\sin^2\left(\phi-\omega t'\right)\right|
\end{equation}

From lemma \ref{rmc thm:lemma} in appendix~\ref{rmc app:lemma}, the maximum
is attained when $k$ and $k'$ are the upper and lower limits of the stepped-frequency 
waveform
\begin{eqnarray}
k&=&k_{N-1}\\
k'&=&k_0
\end{eqnarray}
and $t$ and $t'$ are respectively the maximum and minimum values of 
$\sin^2(\phi-\omega\tau)$ for $\tau\in[-T/2,T/2]$
\begin{eqnarray}
t&=&\max_{\tau}\sin^2(\phi-\omega\tau)\\
t'&=&\min_{\tau}\sin^2(\phi-\omega\tau)
\end{eqnarray}
Therefore
\begin{equation}
\Delta\psi_{xy}=\frac{d^2}{2r_0}\left[
k_{N-1}\max_{\tau}\sin^2(\phi-\omega\tau)
-k_0\min_{\tau}\sin^2(\phi-\omega\tau)
\right]
\end{equation}

Because of this dependence on $\phi$, the angle of the point $(x,y)$ from
the $x$ axis, $\Delta\psi$ in (\ref{rmc eqn:dphi def}) is best expressed as
the supremum over $d$ and $\phi$ rather than the supremum over $x$ and $y$
\begin{equation}
\Delta\psi=\sup_{d,\phi}\frac{d^2}{2r_0}\left[
k_{N-1}\max_{\tau}\sin^2(\phi-\omega\tau)
-k_0\min_{\tau}\sin^2(\phi-\omega\tau)
\right]
\end{equation}
The maximum and minimum over $\tau\in[-T/2,T/2]$ in the square brackets 
can be written as one of the following six algebraic expressions depending 
on the value of $\phi$ at which the supremum is attained:
\begin{description}
\item[Interval 1:]
If $\phi\in[0,\frac{\omega T}{2}]\cup[\pi,\pi+\frac{\omega T}{2}]$,
\begin{equation}
\Delta\psi=\sup_{d,\phi}\frac{d^2}{2r_0}
\,k_{N-1}\sin^2\left(\phi+\frac{\omega T}{2}\right)
\end{equation}

\item[Interval 2:]
If $\phi\in[\frac{\omega T}{2},\frac{\pi}{2}-\frac{\omega T}{2}]
\cup[\pi+\frac{\omega T}{2},\frac{3\pi}{2}-\frac{\omega T}{2}]$,
\begin{equation}
\Delta\psi=\sup_{d,\phi}\frac{d^2}{2r_0}\left[
k_{N-1}\sin^2\left(\phi+\frac{\omega T}{2}\right)
-k_0\sin^2\left(\phi-\frac{\omega T}{2}\right)
\right]
\end{equation}

\item[Interval 3:]
If $\phi\in[\frac{\pi}{2}-\frac{\omega T}{2},\frac{\pi}{2}]
\cup[\frac{3\pi}{2}-\frac{\omega T}{2},\frac{3\pi}{2}]$,
\begin{equation}
\Delta\psi=\sup_{d,\phi}\frac{d^2}{2r_0}\left[
k_{N-1}-k_0\sin^2\left(\phi-\frac{\omega T}{2}\right)
\right]
\end{equation}

\item[Interval 4:]
If $\phi\in[\frac{\pi}{2},\frac{\pi}{2}+\frac{\omega T}{2}]
\cup[\frac{3\pi}{2},\frac{3\pi}{2}+\frac{\omega T}{2}]$,
\begin{equation}
\Delta\psi=\sup_{d,\phi}\frac{d^2}{2r_0}\left[
k_{N-1}-k_0\sin^2\left(\phi+\frac{\omega T}{2}\right)
\right]
\end{equation}

\item[Interval 5:]
If $\phi\in[\frac{\pi}{2}+\frac{\omega T}{2},\pi-\frac{\omega T}{2}]
\cup[\frac{3\pi}{2}+\frac{\omega T}{2},2\pi-\frac{\omega T}{2}]$,
\begin{equation}
\Delta\psi=\sup_{d,\phi}\frac{d^2}{2r_0}\left[
k_{N-1}\sin^2\left(\phi-\frac{\omega T}{2}\right)
-k_0\sin^2\left(\phi+\frac{\omega T}{2}\right)
\right]
\end{equation}

\item[Interval 6:]
Finally, if $\phi\in[\pi-\frac{\omega T}{2},\pi]\cup
[2\pi-\frac{\omega T}{2},2\pi]$,
\begin{equation}
\Delta\psi=\sup_{d,\phi}\frac{d^2}{2r_0}
\,k_{N-1}\sin^2\left(\phi-\frac{\omega T}{2}\right)
\end{equation}
\end{description}

The suprema are greatest in intervals 2 and 5 because they are where the
maxima and minima of $\sin^2(\phi-\omega\tau)$ have the greatest separation
(the remaining four intervals are special cases where the gradient of
$\sin^2(\phi-\omega\tau)$ starts to change sign).  These two intervals
constitute a proportion of $1-4\omega T/2\pi$ of the whole target, which is
almost the whole target because $\omega T\ll 1$.  Therefore unless the
target's extent suddenly increases in intervals 1, 3, 4 or 6, the supremum
of $\Delta\psi$ almost certainly occurs in intervals 2 or 5.  The target's
extent does not suddenly increase because it is modelled as approximately
rectangular, so the supremum of $\Delta\psi$ may be found by considering the
form of the supremum in intervals 2 and 5 only.

In interval 2, a little algebra shows that
\begin{eqnarray}
\Delta\psi_{xy}&\leq&\frac{d^2}{2r_0}\left[\vphantom{\frac{1}{2}}
(k_{N-1}+k_0)\left|\sin\phi\cos\phi\right|\sin\omega T\right. 		\nn\\
&&\left.\mbox{}+(k_{N-1}-k_0)\left(\sin^2\phi+\sin^2\frac{\omega T}{2}\right)
\right]
\end{eqnarray}
Exactly the same expression is obtained for $\Delta\psi_{xy}$ in interval 5.

Therefore $\Delta\psi$ may be written
\begin{eqnarray}
\Delta\psi&\leq&\sup_{d,\phi}\frac{d^2}{2r_0}\left[\vphantom{\frac{1}{2}}
(k_{N-1}+k_0)\left|\sin\phi\cos\phi\right|\sin\omega T\right.		\nn\\
&&\left.\mbox{}+(k_{N-1}-k_0)\left(\sin^2\phi+\sin^2\frac{\omega T}{2}\right)
\right]
\end{eqnarray}
If $\phi$ is not close to $0$ or $\pi$, the $\sin^2\phi$ term is much
greater than the $\sin^2(\omega T/2)$ term, which may then be neglected.
Writing $d\cos\phi=x$ and $d\sin\phi=y$ gives an upper bound on $\Delta\psi$
in terms of each point's $x$ and $y$ coordinates
\begin{equation}
\Delta\psi\leq\sup_{(x,y)}\frac{1}{2r_0}\left[
(k_{N-1}+k_0)\left|xy\right|\sin\omega T+(k_{N-1}-k_0)y^2\right]
\end{equation}
The supremum is attained when $\left|y\right|=D_c/2$ and
$\left|x\right|=D_r/2$.  Since the maximum phase variation has been set at
$\pi/2$ in (\ref{rmc eqn:dphi lim}), this means that the target's extent and
the ISAR image's resolution must satisfy
\begin{equation}\label{rmc eqn:rd const}
\frac{1}{8r_0}\left[(k_{N-1}+k_0)D_rD_c\sin\omega T+
(k_{N-1}-k_0)D_c^2\right]<\frac{\pi}{2}
\end{equation}

%%%%%%%%%%%%%%%%%%%%%%%%%%%%%%%%%%%%%%%%%%%%%%%%%%%%%%%%%%%%%%%%%%%%%%%%%%%%%
\subsection{Constraints Due to Range-Doppler Processing}

Equation (\ref{rmc eqn:rd const}) is the basic constraint due to the
range-Doppler plane-wave approximation.  It is more useful, however, if it is
expressed in terms of the target and the image's resolution, not the radar
waveform's parameters.

From equations (\ref{ii eqn:rd drr}) and (\ref{ii eqn:drc rot}),
the ISAR image's resolution in the range and cross-range directions are
approximately
\begin{eqnarray}
\Delta r_r&=&\frac{2\pi}{k_{N-1}-k_0}\\
\Delta r_c&=&\frac{4\pi}{(k_0+k_{N-1})\sin\omega T}
\end{eqnarray}
Substituting these for $k_0$ and $k_{N-1}$ in (\ref{rmc eqn:rd const})
gives the condition
\begin{equation}
\frac{1}{8r_0}\left[\frac{4\pi}{\Delta r_c}D_rD_c+
\frac{2\pi}{\Delta r_r}D_c^2\right]<\frac{\pi}{2}
\end{equation}
which is
\begin{equation}
\frac{D_rD_c}{r_0\Delta r_c}+
\frac{D_c^2}{2r_0\Delta r_r}<1
\end{equation}

This is certainly satisfied if
\begin{eqnarray}
\frac{D_rD_c}{r_0\Delta r_c}&<&\frac{1}{2}\\
\frac{D_c^2}{2r_0\Delta r_r}&<&\frac{1}{2}
\end{eqnarray}

Rearranging these gives limits on the best resolution that can be used with
range-Dopler processing for a target of a given size and range.
\begin{eqnarray}
\Delta r_c&>&\frac{2D_rD_c}{r_0}		\label{rmc eqn:drc fl}\\
\Delta r_r&>&\frac{D_c^2}{r_0}			\label{rmc eqn:drr fl}
\end{eqnarray}

When designing a range-Doppler ISAR system, the first step should be using
equations (\ref{rmc eqn:drc fl}) and (\ref{rmc eqn:drr fl}) to determine the
highest resolution that the imaging geometry allows.  Then the parameters of
the stepped-frequency waveform can be selected to give the desired
resolution providing this is consistent with these limits.

These limits are consistent with the example used throughout chapter
\ref{hrr chp}, which had a resolution of 1~m by 1~m for a
target 30~m across at a range of 10~km.  The minimum cross-range resolution
is 18~cm and the minimum range resolution is 9~cm.\footnote{Note that the
analysis in this section does not mean that it is possible to obtain
resolutions of 18~cm by 9~cm.  Other constraints, principly that of a
narrow relative bandwidth, have to be violated for the limits in
(\protect\ref{rmc eqn:drc fl}) and (\protect\ref{rmc eqn:drr fl}) to be
approached.}  Both of these are substantially better than the required 
resolution of 1~m by 1~m.

One final comment about range-Doppler imaging is that the target does not
have to be in the far-field for the Taylor series to be valid.  The
target is in the far-field when
\begin{equation}
r_0>\frac{D^2}{\lambda}
\end{equation}
where $D$ is the target's extent.  For the parameters above, at X-band where
$\lambda$=3~cm, the far-field starts at a range of 30~km, whereas the
$\pi/2$ phase condition says that the target may be as close as 
$r_0=2\cdot 30^2/1=1.8$~km.

%%%%%%%%%%%%%%%%%%%%%%%%%%%%%%%%%%%%%%%%%%%%%%%%%%%%%%%%%%%%%%%%%%%%%%%%%%%%%
\section{Polar Reformatting}
\label{rmc sec:pr}

The classical but incorrect view of ISAR imaging as a temporal Doppler
process restricted by the motion of scatterers through resolution cells
leads to the bounds on $\Delta r_r$ and $\Delta r_c$ given in equations
(\ref{rmc eqn:prc I}) and (\ref{rmc eqn:prc II}).  

In this section, errors arising from omitting polar reformatting are
analysed to derive simple conditions under which acceptable ISAR images
may be formed without the complexity of polar reformatting.
These conditions are similar to those in equations (\ref{rmc eqn:prc I}) and 
(\ref{rmc eqn:prc II}) derived by considering the target's motion through 
resolution cells.  However, the more exact analysis presented here also
includes an extra factor of $D_c$ in (\ref{rmc eqn:prc I}) that has not been 
included in any other analyses of polar reformatting.

The results presented here are explicitly for ISAR imaging a uniformly 
rotating target using range-Doppler processing.  However, the same method
can also be used for targets moving in a straight line or for ISAR imaging
using $\omega-k$ processing.

%%%%%%%%%%%%%%%%%%%%%%%%%%%%%%%%%%%%%%%%%%%%%%%%%%%%%%%%%%%%%%%%%%%%%%%%%%%%%
\subsection{Defining the Phase Variation $\Delta\psi_{xy}$}

As for the analysis of phase errors due to the range-Doppler approximation,
the distance between the radar and a scatterer at $(x,y)$ is, at time $t$,
\begin{equation}
r_{xy}(t)=r_0+x\cos\omega t-y\sin\omega t+\epsilon_{xy}(t)
\end{equation}
where $\epsilon_{xy}(t)$ is the Taylor series' remainder, which is the error
due to the range-Doppler linear approximation.

Not performing polar reformatting is equivalent to assuming that the polar
grid in $(k_x,k_y)$-space is rectangular.  Therefore the point
$(k_x(k,t),k_y(k,t))$ in $(k_x,k_y)$-space is assumed to lie at
the point given by $(k,t)$ under a linear scaling, translation and a
rotation.  In matrix notation, 
\begin{equation}
\left[\begin{array}{c} k_x(k,t) \\ k_y(k,t) \end{array}\right]
\approx 
\left[\begin{array}{cc} \cos\zeta & \sin\zeta \\
 -\sin\zeta& \cos\zeta \end{array}\right]
\left[\begin{array}{cc} \alpha & 0 \\ 0 & \beta \end{array}\right]
\left(
\left[\begin{array}{c} k \\ t\end{array}\right]
-\left[\begin{array}{c} k_c \\ t_c\end{array}\right]
\right)
\end{equation}
where $(k_c,t_c)$ is the translation, $(\alpha,\beta)$ the scale factors and
$\zeta$ the angle the rectangular grid is rotated.

For the case considered here of a uniformly rotating target, the polar grid
is
\begin{eqnarray}
k_x(k,t)&=&k\cos\omega t \\
k_y(k,t)&=&k\sin\omega t
\end{eqnarray}
which is centred about the $k_x$ axis because $t$ varies from $-T/2$ to
$T/2$.  The polar grid's symmetry indicates that the optimum rectangular 
grid has its axes parallel with the $k_x$ and $k_y$ axes, thus $\zeta=0$.
The perimeters of the polar grid for a uniformly rotating target and the
assumed rectangular grid are shown in figure \ref{rmc fig:pol ref errors}.  

%============================================================================
\begin{figure}\centering
\caption{The polar grid is assumed to be rectangular when the ISAR inversion
does not use polar reformatting.}
\label{rmc fig:pol ref errors}

\setlength{\unitlength}{0.8cm}
\begin{pspicture}(-6.5,-3.2)(6.5,4)
% Allow polar coordinates
\SpecialCoor	
% First the axes and their labels
\psline[linecolor=black,linewidth=1.5pt]{->}(-6,0)(6,0)
\psline[linecolor=black,linewidth=1.5pt]{->}(0,-3)(0,3)
\uput[r](6,0){$k_x$}
\uput[u](0,3){$k_y$}
% The polar grid
\psarc[linecolor=gray,linewidth=1pt]{-}{3.8}{339}{381}
\psarc[linecolor=gray,linewidth=1pt]{-}{5.6}{339}{381}
\psline[linecolor=gray,linewidth=1pt]{-}(3.8;339)(5.6;339)
\psline[linecolor=gray,linewidth=1pt]{-}(3.8;381)(5.6;381)
% The rectangular grid
\psline[linecolor=darkgray,linewidth=1pt]{-}(3.8,0|4.7;339)(5.6,0|4.7;339)(5.6,0|4.7;381)(3.8,0|4.7;381)(3.8,0|4.7;339)
\NormalCoor
\end{pspicture}
\end{figure}
%============================================================================


Under this assumption of a rectangular grid, the phase $kr_{xy}(t)$ in
$e^{-jkr_{xy}(t)}$ can be written as
\begin{equation}
kr_{xy}(t)=kr_0+(k-k_c)\alpha x+(t-t_c)\beta y+\psi_{xy}(k,t)
+k\epsilon_{xy}(t)
\end{equation}
Here $\psi_{xy}(k,t)$ is the phase error associated with ignoring polar
reformatting and $k\epsilon_{xy}(t)$ is the phase error associated with the
range-Doppler plane-wave approximation.

The optimum rectangular grid depends on the choice of $\alpha$ and $\beta$
and $\zeta$; the choice of translation $(k_c,t_c)$ does not matter because 
this can be absorbed into the phase of $g(x,y)$.

As for the analysis of the range-Doppler approximation, define the total
variation in the phase error at a point $(x,y)$ over all $k$ and $t$ as
\begin{equation}
\Delta\psi_{xy}(\alpha,\beta,\zeta)=\max_{k,t,k',t'}
\left|\psi_{xy}(k,t)-\psi_{xy}(k',t')\right|
\end{equation}
This is a function of the rectangular grid, hence 
$\Delta\psi_{xy}(\alpha,\beta,\zeta)$ depends on $\alpha$, $\beta$ and $\zeta$.
Since the aim of the selecting the rectangular grid is to minimize the maximum
phase variation over the whole of the target, define the maximum phase
variation for a particular rectangular grid as
\begin{eqnarray}
\Delta\psi(\alpha,\beta,\zeta)
&=&\sup_{(x,y)}\Delta\psi_{xy}(\alpha,\beta,\zeta)\nn\\
&=&\sup_{(x,y)}\max_{k,t,k',t'}\left|\psi_{xy}(k,t)-\psi_{xy}(k',t')\right|
\end{eqnarray}

Finally, as the rectangular grid should be selected so that 
$\Delta\psi(\alpha,\beta,\zeta)$ is minimized, define the optimum maximum
phase variation, $\Delta\psi$, as
\begin{eqnarray}
\Delta\psi&=&\inf_{\alpha,\beta,\zeta}\Delta\psi(\alpha,\beta,\zeta)\nn\\
&=&\inf_{\alpha,\beta,\zeta}
\sup_{(x,y)}\max_{k,t,k',t'}\left|\psi_{xy}(k,t)-\psi_{xy}(k',t')\right|
\label{rmc eqn:dpsi def}
\end{eqnarray}
Now, once an upper limit on $\Delta\psi$ has been selected, such as
\begin{equation}\label{rmc eqn:dpsi cond II}
\Delta\psi<\frac{\pi}{2}
\end{equation}
substitute (\ref{rmc eqn:dpsi def}) and solve for the infinum, supremum and
maximum to obtain some constraints on the target's motion and the ISAR
image's resolution.


%%%%%%%%%%%%%%%%%%%%%%%%%%%%%%%%%%%%%%%%%%%%%%%%%%%%%%%%%%%%%%%%%%%%%%%%%%%%%
\subsection{Evaluating the Phase Variation}


For the uniformly rotating target and range-Doppler processing, the phase 
error $\psi_{xy}(k,t)$ if polar reformatting is omitted is
\begin{equation}
\psi_{xy}(k,t)=x\left(\alpha k-k\cos\omega t\right)
-y\left(\beta t-k\sin\omega t\right)
\end{equation}
As mentioned earlier, symmetry dictates that $\zeta=0$ so that the total 
phase variation $\Delta\psi_{xy}(\alpha,\beta,0)$ is less than
\begin{eqnarray}
\Delta\psi_{xy}(\alpha,\beta,0)
&\leq&\max_{k,t,k',t'} 
\left|x\right|\left|\left(\alpha k-k\cos\omega t\right)
-\left(\alpha k'-k'\cos\omega t'\right)\right|			\nn\\
&&{}+\left|y\right|\left|\left(\beta t-k\sin\omega t\right)-
\left(\beta t'-k'\sin\omega t'\right)\right| 			\nn\\
&\leq&\max_{k,t,k',t'} 
\frac{D_r}{2}\left|\left(\alpha k-k\cos\omega t\right)
-\left(\alpha k'-k'\cos\omega t'\right)\right|			\nn\\
&&{}+\frac{D_c}{2}\left|\left(\beta t-k\sin\omega t\right)-
\left(\beta t'-k'\sin\omega t'\right)\right| 
\end{eqnarray}
because $\left|x\right|<D_r/2$ and $\left|y\right|<D_c/2$ for all points on
the target.  Since this last expression is independent of $x$ and $y$, the 
supremum over $(x,y)$ has been evaluated, leaving
\begin{eqnarray}
\Delta\psi&\leq&\inf_{\alpha,\beta}\max_{k,t,k',t'} 
\frac{D_r}{2}\left|\left(\alpha k-k\cos\omega t\right)
-\left(\alpha k'-k'\cos\omega t'\right)\right|			\nn\\
&&{}+\frac{D_c}{2}\left|\left(\beta t-k\sin\omega t\right)-
\left(\beta t'-k'\sin\omega t'\right)\right| 
\end{eqnarray}
which is bounded above by an expression where the two moduli are maximized
separately
\begin{eqnarray}
\Delta\psi&\leq&\inf_{\alpha}\max_{k,t,k',t'} 
\frac{D_r}{2}\left|\left(\alpha k-k\cos\omega t\right)
-\left(\alpha k'-k'\cos\omega t'\right)\right|			\nn\\
&&{}+\inf_{\beta}\max_{k,t,k',t'} 
\frac{D_c}{2}\left|\left(\beta t-k\sin\omega t\right)-
\left(\beta t'-k'\sin\omega t'\right)\right| 
\end{eqnarray}

Split this upper bound on $\Delta\psi$ into two halves, $L_r$ and $L_c$ so
that
\begin{equation}
\Delta\psi\leq \frac{D_r}{2}\,L_r+\frac{D_c}{2}\,L_c
\end{equation}
where
\begin{equation}
L_r=\inf_{\alpha}\max_{k,t,k',t'} \left|\left(\alpha k-k\cos\omega t\right)
-\left(\alpha k'-k'\cos\omega t'\right)\right|
\end{equation}
and
\begin{equation}
L_c=\inf_{\beta}\max_{k,t,k',t'} \left|\left(\beta t-k\sin\omega t\right)-
\left(\beta t'-k'\sin\omega t'\right)\right| 
\end{equation}

Starting with $L_r$, note that $L_r$ can be written 
\begin{equation}
L_r=\inf_{\alpha}\max_{k,b,k',b'} \left|kb-k'b'\right|
\end{equation}
where the maximum includes values of $k$ and $k'$ in $[k_0,k_{N-1}]$ and 
values of $b$ and $b'$ in $B$ which is the interval
\begin{equation}
B=\left[\cos\left(\frac{\omega T}{2}\right)-\alpha,1-\alpha\right]
\end{equation}
Apply theorem \ref{rmc thm:theorem} in appendix \ref{rmc app:lemma}
with $a=k$, $a'=k'$ and $A=[k_0,k_{N-1}]$ to show that
\begin{equation}
L_r=k_{N-1}\left(1-\cos\left(\frac{\omega T}{2}\right)\right)
\approx k_{N-1}\frac{(\omega T)^2}{8}
\end{equation}
where $\alpha$ may take any value between $\cos(\omega T/2)$ and $1$.

The exact value of $L_c$ is much more difficult to obtain, so a slightly
loose upper bound on $L_c$ will be found instead of the exact value.
First, use the triangle inequality twice to show that
\begin{eqnarray}
L_c&=&\inf_{\beta}\max_{k,t,k',t'} \left|\left(\beta t-k\sin\omega t\right)-
\left(\beta t'-k'\sin\omega t'\right)\right| \nn\\
&\leq&2\inf_{\beta}\max_{k,t} \left|\beta t-k\sin\omega t\right| \nn\\
&\leq&2\left(
\max_{k,t}\left|k(\sin\omega t-\omega t)\right|+
\max_{k,t} \left|(\overline{k}-k)\omega t\right|
\right)
\end{eqnarray}
where one particular $\beta=\omega\overline{k}$ has been chosen so that the
infinum over $\beta$ can be removed.  Here $\overline{k}=(k_0+k_{N-1})/2$.
The optimum choice of $\beta$ satisfies
\begin{equation}
k_0\frac{2}{T}\sin\frac{\omega T}{2}\leq \beta \leq k_{N-1}\omega
\end{equation}
but finding it is quite difficult.  With $\beta=\omega\overline{k}$, 
the maximum occurs when $t=T/2$ and $k=k_{N-1}$ so that
\begin{eqnarray}
L_c&\leq&2\left(k_{N-1}\left(\frac{\omega T}{2}-\sin\frac{\omega T}{2}\right)
+\frac{N\Delta k}{2}\frac{\omega T}{2}\right)\nn\\
&\approx&k_{N-1}\frac{(\omega T)^3}{24}+\frac{N\Delta k\,\omega T}{2}
\end{eqnarray}

Combining these bounds on $L_r$ and $L_c$ shows that
\begin{equation}
\Delta\psi
\leq\frac{D_r}{2}\,k_{N-1}\frac{(\omega T)^2}{8}
+\frac{D_c}{2}\left(k_{N-1}\frac{(\omega T)^3}{24}
+\frac{N\Delta k\,\omega T}{2}\right)			
\end{equation}
Neglecting the $(\omega T)^3$ term, 
\begin{equation}
\Delta\psi\leq
D_r\,k_{N-1}\frac{(\omega T)^2}{16}
+D_c\frac{N\Delta k\,\omega T}{16}
\end{equation}

Since the maximum value of $\Delta\psi$ has been set to $\pi/2$,
this means that the target's extent and the ISAR image's resolution 
must satisfy
\begin{equation}\label{rmc eqn:pr const}
D_r\,k_{N-1}\frac{(\omega T)^2}{16}
+D_c\frac{N\Delta k\,\omega T}{16}<\frac{\pi}{2}
\end{equation}
if polar reformatting is not required.

%%%%%%%%%%%%%%%%%%%%%%%%%%%%%%%%%%%%%%%%%%%%%%%%%%%%%%%%%%%%%%%%%%%%%%%%%%%%%
\subsection{Constraints Due to Neglecting Reformatting}

Equation (\ref{rmc eqn:pr const}) is the basic constraint due to neglecting
polar reformatting.  As for the equivalent constraint for the range-Doppler
approximation in (\ref{rmc eqn:rd const}), it is more usefully expressed
in terms of the target and the image's resolution, rather than the radar 
waveform's parameters.  The radar's average wavelength is the only waveform
parameter that cannot be eliminated, giving
\begin{equation}
\frac{\lambda D_r}{8\Delta r_c^2}+
\frac{\lambda D_c}{8\Delta r_r\Delta r_c}<1
\end{equation}

This is satisfied if both terms on the left-hand side are less than a half,
which gives the following two constraints for neglecting polar reformatting
\begin{eqnarray}
\Delta r_c^2&>&\frac{\lambda D_r}{4}		\label{rmc eqn:prc Ia}\\
\Delta r_r\Delta r_c&>&\frac{\lambda D_c}{4}	\label{rmc eqn:prc IIa}
\end{eqnarray}
which are exactly the same as the classical constraints due to motion
through resolution cells quoted at the beginning of the chapter in 
equations (\ref{rmc eqn:prc I}) and (\ref{rmc eqn:prc II}), which is rather
a surprising result.

%%%%%%%%%%%%%%%%%%%%%%%%%%%%%%%%%%%%%%%%%%%%%%%%%%%%%%%%%%%%%%%%%%%%%%%%%%%%%
\section{Motion Estimation for Rotating Targets}
\label{rmc sec:me}

Because ISAR is designed to image moving targets, motion estimation and
motion compensation are essential parts of producing a well-focussed image.
The primary assumption of motion estimation is that the target rotates
sufficiently slowly that over a short period of time, $t\in[0,\tau]$ for
some $0\leq\tau\leq T$, measurements of the target's frequency response show
a phase change due to its radial displacement so that
\begin{equation}
s(k,\tau)=e^{-jk\dr} \,s(k,0)
\end{equation}
where $\Delta r$ is the radial distance the target moves between $t=0$ and
$t=\tau$.

In this section, this basic assumption of a target rotating sufficiently
slowly for motion estimation is analysed to find a constraint on a target's 
maximum allowable rotation rate.  It is shown that if the target is rotating
fast enough that it fills most of the ISAR image's cross-range window, it is
rotating too quickly for accurate motion estimation.  This leads to a
constraint that the cross-range extent of the target should be at most
one-quarter of the size of the image's cross-range window.

This result, one that has not appeared in the ISAR literature before, is
slightly surprising because it suggests a fundamental limitation on accurate
motion estimation for ISAR
that is much stricter than the resolution constraints due to range-Doppler
processing or ignoring polar reformatting.  Since many ISAR images have a
target that fills more than a quarter of the cross-range window, this
constraint is often violated.  When this happens, reflections from 
scatterers that are far from the centre of the target appear as random phase
shifts while reflections from those closer to the centre have phase shifts
that are more tightly concentrated about $-k\dr$.  This leads to a type of
rotation-induced noise which increases in variance as the target fills more
and more of the cross-range window.

The target motion considered in this section is that of a
uniformly-rotating target a fixed distance from the radar.  This
particular motion has been chosen because it is the same as that in the
other two analyses in this chapter.  The same method can be used with any
other target motion, and it is a simple matter to rewrite the results in
terms of a target moving in a straight line at constant speed.  Using a
target that is uniformly rotating with no radial motion emphasizes that this
effect is due to the target's rotation, not to the radial motion itself.

%%%%%%%%%%%%%%%%%%%%%%%%%%%%%%%%%%%%%%%%%%%%%%%%%%%%%%%%%%%%%%%%%%%%%%%%%%%%%
\subsection{Defining the Total Phase Variation $\Delta\psi$}


From equation (\ref{ii eqn:rxy(t) rot}), the distance between a scatterer at
$(x,y)$ in the target's local coordinate system and the radar is
\begin{equation}
r_{xy}(t)=\sqrt{(r_0+x\cos\omega t-y\sin\omega t)^2
+(x\sin\omega t+y\cos\omega t)^2}
\end{equation}
at time $t$.  This can be written slightly differently as
\begin{equation}
r_{xy}(t)=\sqrt{(r_0+x)^2+y^2-2r_0(x(1-\cos\omega t)+y\sin\omega t)}
\end{equation}

The radar measurement of the target at frequency $k$ and time $t=0$ is
\begin{equation}
s(k,0)=\int\!\!\int g(x,y)\,e^{-jkr_{xy}(0)}\,dx\,dy
\end{equation}
If the target is rotates negligibly between $t=0$ and $t=\tau$, then 
\begin{equation}
s(k,\tau)=e^{-jk\dr} \,s(k,0)
\end{equation}
which suggests that
\begin{equation}
s(k,\tau)=e^{-jk\dr}\int\!\!\int g(x,y)\,e^{-jkr_{xy}(0)}\,dx\,dy
\end{equation}
However, the target is rotating slowly, so the correct expression for
$s(k,\tau)$ is
\begin{equation}
s(k,\tau)=e^{-jk\dr}\int\!\!\int g'(x,y,k,\tau)\,e^{-jkr_{xy}(0)}\,dx\,dy
\end{equation}
where the target's reflectivity function has been modified to give
\begin{equation}
g'(x,y,k,\tau)=g(x,y)\,e^{-jk\epsilon_{xy}(\tau)}
\end{equation}
with $\epsilon_{xy}(\tau)$ being the error in the range of a scatterer at
$(0,0)$ due to the assumption that the target is not rotating.  Thus
\begin{equation}
r_{xy}(\tau)=r_{xy}(0)+\dr+\epsilon_{xy}(\tau)
\end{equation}
The variation in $\epsilon_{xy}(\tau)$ at different points $(x,y)$ on the
target means that the phase difference between $(s(k,0)$ and $s(k,\tau)$ is
not exactly $-k\dr$.  This is the cause of rotation-induced noise.

Define the total phase variation as
\begin{equation}
\Delta\psi=\sup_{xy,x'y'}\left|k\epsilon_{xy}(\tau)-k\epsilon_{x'y'}(\tau)
\right|
\end{equation}
To limit the rotation-induced noise to an acceptable level, place an upper 
limit on the total phase variation over all points on the target, so that
\begin{equation}
\Delta\psi<\frac{\pi}{2}
\end{equation}

%%%%%%%%%%%%%%%%%%%%%%%%%%%%%%%%%%%%%%%%%%%%%%%%%%%%%%%%%%%%%%%%%%%%%%%%%%%%%
\subsection{Evaluating the Total Phase Variation}

For the uniformly-rotating target, $\epsilon_{xy}(\tau)$ is
\begin{eqnarray}
\epsilon_{xy}(\tau)
&=&\sqrt{(r_0+x)^2+y^2-2r_0(x(1-\cos\omega\tau)+y\sin\omega\tau}
-\sqrt{(r_0+x)^2+y^2}	\nn\\
&\approx&x(1-\cos\omega\tau)+y\sin\omega\tau
\end{eqnarray}
so that
\begin{equation}
\Delta\psi=\sup_{xy,x'y'}\,k\,\left|
(x-x')(1-\cos\omega\tau)+(y-y')\sin\omega\tau\right|
\end{equation}
This supremum is attained for $(x,y)$ and $(x',y')$ at opposite ends of the
target.  The maximum value of $\left|x-x'\right|$ is $D_r$, the target's 
range extent, and the maximum value of $\left|y-y'\right|$ is $D_c$, the 
target's cross-range extent. Therefore
\begin{equation}
\Delta\psi=k D_r(1-\cos\omega\tau)+k D_c\sin\omega\tau
\end{equation}

The time $t=\tau$, at which the measurement of $s(k,\tau)$ was made, is
between $0$ and $T$, the period over which the data for the ISAR image is
acquired.  Therefore $\left|\omega\tau\right|\leq\left|\omega T\right|\ll 1$
and $\Delta\psi$ can be written
\begin{equation}
\Delta\psi=\frac{k}{2}D_r(\omega\tau)^2+k D_c\omega\tau
\end{equation}

Since $\tau$ is some fraction of the total imaging period, write
\begin{equation}
\tau=\eta T\qquad\mbox{for some $\eta\in[0,1]$}
\end{equation}
Then using $\Delta\psi<\pi/2$, the constraint becomes
\begin{equation}\label{rmc eqn:me const}
\frac{k \eta^2}{2}D_r(\omega T)^2+k \eta D_c\omega T<\frac{\pi}{2}
\end{equation}

%%%%%%%%%%%%%%%%%%%%%%%%%%%%%%%%%%%%%%%%%%%%%%%%%%%%%%%%%%%%%%%%%%%%%%%%%%%%%
\subsection{Constraints Due to Motion Estimation}

Equation (\ref{rmc eqn:me const}), the basic constraint due to the
target's rotation during motion estimation, can be written 
\begin{equation}
\pi\eta^2\frac{\lambda D_r}{\Delta r_c^2}
+2\pi\eta\frac{D_c}{\Delta r_c}<\frac{\pi}{2}
\end{equation}

The first term is negligible in comparison with the second as $\Delta r_c\gg
\lambda$.  Therefore the single constraint due to the need for reliable
motion estimation is
\begin{equation}
\Delta r_c>4\eta D_c
\end{equation}

From this, a small value of $\eta$ is needed for a high cross-range
resolution.  The smallest $\eta$ can be is $1/M$, where $M$ is the number of
time steps at which the target's frequency response is sampled.
The width $W_c$ of the ISAR image's cross-range ambiguity window is 
\begin{equation}
W_c=M\Delta r_c
\end{equation}
so the constraint on the target's size is
\begin{equation}
D_c<\frac{W_c}{4}
\end{equation}
which indicates that the target can occupy no more than a quarter of the
cross-range ambiguity window.

%%%%%%%%%%%%%%%%%%%%%%%%%%%%%%%%%%%%%%%%%%%%%%%%%%%%%%%%%%%%%%%%%%%%%%%%%%%%%
\subsection{Rotation-Induced Noise}

When a target is rotating slowly, all scatterers move approximately the same
radial distance as the target moves.  The phase shifts due to each
scatterer's radial motion add coherently so they can easily be detected for 
motion compensation.

%============================================================================
\begin{figure}\centering
\caption[Rotation-induced noise for a simulated target]{Rotation-induced 
noise as a function of a simulated target's angular velocity.  The shaded
horizontal line indicates the expected variance of the rotation-induced
noise when the target is rotating very quickly and motion estimation becomes
random.  The shaded vertical line shows the rotation rate when the simulated
target completely fills the ISAR cross-range window.}
\label{rmc fig:rin}

\setlength{\unitlength}{1cm}
\psset{unit=1cm}
\begin{pspicture}(-1,-1)(11,7)
% data: f:\phd\data\sim\737rot_f.mat
% results:
\psset{xunit=0.5cm,yunit=7cm}
\savedata{\mydata}[{
(1,0.0486) (2,0.0969) (3,0.2222) (4,0.3689) (5,0.4286)
(6,0.4545) (7,.4119) (8,0.4683) (9,0.5196) (10,0.4560) 
(11,0.5169) (12,0.4149) (13,0.4253) (14,0.4386) (15,0.4545) 
(16,0.4960) (17,0.5588) (18,0.4673) (19,0.5050) (20,0.4500)
}]\dataplot[plotstyle=line,showpoints=false]{\mydata} 

\savedata{\mydata}[{
(1,0.0243) (2,0.0967) (3,0.3783) (4,0.4321) (5,0.4902)
(6,0.4761) (7,0.4172) (8,0.3702) (9,0.3273) (10,0.3027)
(11,0.4929) (12,0.4267) (13,0.3535) (14,0.2989) (15,0.2936)
(16,0.3052) (17,0.3587) (18,0.4301) (19,0.5232) (20,0.6134)
}]\dataplot[plotstyle=line,showpoints=false]{\mydata}

\savedata{\mydata}[{
(1,0.0155) (2,0.0661) (3,0.1550) (4,0.2905) (5,0.5450) (6,0.5314) (7,0.5548) 
(8,0.5758) (9,0.5670) (10,0.5785) (11,0.3975) (12,0.3914) (13,0.4039) (14,0.4208) 
(15,0.4325) (16,0.4418) (17,0.4563) (18,0.4928) (19,0.5525) (20,0.6333)
}]\dataplot[plotstyle=line,showpoints=false]{\mydata} 

\savedata{\mydata}[{
(1,0.0210) (2,0.0603) (3,0.1293) (4,0.2335) (5,0.3713) (6,0.3685) (7,0.4831) 
(8,0.5392) (9,0.5058) (10,0.4589) (11,0.4205) (12,0.4095) (13,0.4099) (14,0.5089) 
(15,0.4656) (16,0.3478) (17,0.3184) (18,0.3424) (19,0.3895) (20,0.4539)
}]\dataplot[plotstyle=line,showpoints=false]{\mydata} 

\savedata{\mydata}[{
(1,0.0186) (2,0.0644) (3,0.1268) (4,0.1990) (5,0.2876) (6,0.2685) (7,0.4276) 
(8,0.4490) (9,0.4963) (10,0.4187) (11,0.3867) (12,0.3617) (13,0.3962) (14,0.3872) 
(15,0.4167) (16,0.4715) (17,0.5499) (18,0.6373) (19,0.6445) (20,0.6644)
}]\dataplot[plotstyle=line,showpoints=false]{\mydata} 

\savedata{\mydata}[{
(1,0.0211) (2,0.0814) (3,0.1713) (4,0.3912) (5,0.4301) (6,0.5027) (7,0.5905) 
(8,0.6963) (9,0.6899) (10,0.4491) (11,0.4471) (12,0.4371) (13,0.4127) (14,0.3852) 
(15,0.3789) (16,0.5099) (17,0.5195) (18,0.5514) (19,0.5957) (20,0.6667)
}]\dataplot[plotstyle=line,showpoints=false]{\mydata} 

\savedata{\mydata}[{
(1,0.0211) (2,0.0616) (3,0.1590) (4,0.2263) (5,0.3065) (6,0.4583) (7,0.4827) 
(8,0.3179) (9,0.3975) (10,0.4710) (11,0.3868) (12,0.3308) (13,0.3083) (14,0.5454) 
(15,0.5815) (16,0.5745) (17,0.4760) (18,0.3662) (19,0.3958) (20,0.4214)
}]\dataplot[plotstyle=line,showpoints=false]{\mydata} 

\savedata{\mydata}[{
(1,0.0241) (2,0.0920) (3,0.1847) (4,0.4444) (5,0.4041) (6,0.3565) (7,0.3133) 
(8,0.2760) (9,0.2607) (10,0.5014) (11,0.4727) (12,0.3980) (13,0.3535) (14,0.3124) 
(15,0.3044) (16,0.3444) (17,0.5229) (18,0.7040) (19,0.7951) (20,0.7539)
}]\dataplot[plotstyle=line,showpoints=false]{\mydata} 

\savedata{\mydata}[{
(1,0.0185) (2,0.0776) (3,0.1722) (4,0.5707) (5,0.4853) (6,0.5115) (7,0.5534) 
(8,0.6387) (9,0.6484) (10,0.6833) (11,0.3849) (12,0.3955) (13,0.4088) (14,0.4155) 
(15,0.4180) (16,0.4241) (17,0.4510) (18,0.5213) (19,0.6215) (20,0.6729)
}]\dataplot[plotstyle=line,showpoints=false]{\mydata} 

\savedata{\mydata}[{
(1,0.0227) (2,0.0709) (3,0.1582) (4,0.2852) (5,0.3796) (6,0.4584) (7,0.5177) 
(8,0.5500) (9,0.5154) (10,0.4879) (11,0.4765) (12,0.4727) (13,0.4701) (14,0.4600) 
(15,0.3463) (16,0.3093) (17,0.3219) (18,0.3589) (19,0.4190) (20,0.4938)
}]\dataplot[plotstyle=line,showpoints=false]{\mydata} 

\savedata{\mydata}[{
(1,0.0148) (2,0.0541) (3,0.1117) (4,0.1898) (5,0.3023) (6,0.4370) (7,0.4703) 
(8,0.5218) (9,0.4629) (10,0.4329) (11,0.3900) (12,0.3743) (13,0.4117) (14,0.4160) 
(15,0.4465) (16,0.4895) (17,0.5469) (18,0.5767) (19,0.6376) (20,0.6714)
}]\dataplot[plotstyle=line,showpoints=false]{\mydata} 

\savedata{\mydata}[{
(1,0.0205) (2,0.0776) (3,0.1624) (4,0.2657) (5,0.4374) (6,0.5327) (7,0.6127) 
(8,0.6810) (9,0.6570) (10,0.7009) (11,0.4702) (12,0.4271) (13,0.3787) (14,0.3511) 
(15,0.3524) (16,0.5143) (17,0.5407) (18,0.5788) (19,0.6143) (20,0.6381)
}]\dataplot[plotstyle=line,showpoints=false]{\mydata} 

\savedata{\mydata}[{
(1,0.0214) (2,0.0900) (3,0.2029) (4,0.3829) (5,0.4762) (6,0.5006) (7,0.5547) 
(8,0.5068) (9,0.4468) (10,0.3884) (11,0.3554) (12,0.5073) (13,0.4150) (14,0.3429) 
(15,0.2769) (16,0.2898) (17,0.3276) (18,0.3914) (19,0.4788) (20,0.5224)
}]\dataplot[plotstyle=line,showpoints=false]{\mydata} 

\savedata{\mydata}[{
(1,0.0136) (2,0.0562) (3,0.1315) (4,0.2523) (5,0.4353) (6,0.5070) (7,0.5127) 
(8,0.5291) (9,0.5051) (10,0.4501) (11,0.4436) (12,0.4066) (13,0.4071) (14,0.4254) 
(15,0.4476) (16,0.4633) (17,0.4764) (18,0.4947) (19,0.5361) (20,0.5892)
}]\dataplot[plotstyle=line,showpoints=false]{\mydata} 

\savedata{\mydata}[{
(1,0.0251) (2,0.0844) (3,0.1874) (4,0.3387) (5,0.4471) (6,0.5076) (7,0.5223) 
(8,0.5544) (9,0.5529) (10,0.5090) (11,0.5031) (12,0.4971) (13,0.5109) (14,0.4234) 
(15,0.2953) (16,0.2967) (17,0.3215) (18,0.3718) (19,0.4453) (20,0.6229)
}]\dataplot[plotstyle=line,showpoints=false]{\mydata} 

\savedata{\mydata}[{
(1,0.0133) (2,0.0515) (3,0.1142) (4,0.2109) (5,0.3520) (6,0.6419) (7,0.5427) 
(8,0.4877) (9,0.4759) (10,0.4307) (11,0.3977) (12,0.4289) (13,0.4126) (14,0.4260) 
(15,0.4556) (16,0.4853) (17,0.5056) (18,0.6738) (19,0.6289) (20,0.6224)
}]\dataplot[plotstyle=line,showpoints=false]{\mydata} 

\savedata{\mydata}[{
(1,0.0193) (2,0.0734) (3,0.1548) (4,0.2571) (5,0.3803) (6,0.3918) (7,0.6462) 
(8,0.6815) (9,0.7793) (10,0.7287) (11,0.6529) (12,0.5423) (13,0.3315) (14,0.3212) 
(15,0.3567) (16,0.4240) (17,0.5014) (18,0.6088) (19,0.6275) (20,0.7077)
}]\dataplot[plotstyle=line,showpoints=false]{\mydata} 

\savedata{\mydata}[{
(1,0.0207) (2,0.0575) (3,0.1141) (4,0.1944) (5,0.3034) (6,0.4683) (7,0.3644) 
(8,0.4275) (9,0.5033) (10,0.4261) (11,0.3866) (12,0.3567) (13,0.3477) (14,0.3491) 
(15,0.5319) (16,0.4923) (17,0.3845) (18,0.3456) (19,0.3719) (20,0.4142)
}]\dataplot[plotstyle=line,showpoints=false]{\mydata} 

\savedata{\mydata}[{
(1,0.0224) (2,0.0802) (3,0.1551) (4,0.2374) (5,0.3266) (6,0.2865) (7,0.2542) 
(8,0.2425) (9,0.4343) (10,0.5224) (11,0.5373) (12,0.4116) (13,0.3606) (14,0.3360) 
(15,0.3516) (16,0.4130) (17,0.5538) (18,0.7772) (19,0.8331) (20,0.6082)
}]\dataplot[plotstyle=line,showpoints=false]{\mydata} 

\savedata{\mydata}[{
(1,0.0212) (2,0.0828) (3,0.1779) (4,0.4466) (5,0.4696) (6,0.4974) (7,0.5870) 
(8,0.6626) (9,0.6677) (10,0.4012) (11,0.4100) (12,0.4183) (13,0.4176) (14,0.4089) 
(15,0.4007) (16,0.4141) (17,0.4726) (18,0.5811) (19,0.6872) (20,0.7808)
}]\dataplot[plotstyle=line,showpoints=false]{\mydata} 

% Now the axes, tick marks and their labels
\psaxes[linecolor=black, linewidth=1pt, Ox=0, Dx=5, dx=5, Oy=0.0, Dy=0.1,
dy=0.1, tickstyle=top, axesstyle=frame]{->}(20,0.9)

\psline[linecolor=white, linewidth=2.5pt](0,0.55)(20,0.55)
\psline[linecolor=gray, linewidth=1.5pt](0,0.55)(20,0.55)
\psline[linecolor=white, linewidth=2.5pt](4.8,0)(4.8,0.9)
\psline[linecolor=gray, linewidth=1.5pt](4.8,0)(4.8,0.9)

\rput[t](10,-0.15){angular velocity (degrees per second)}
\rput[b]{90}(-3,0.45){rotation-induced noise variance}

\end{pspicture}
\end{figure}
%============================================================================

As the target rotates faster, differences in the radial movement of
scatterers at the outside of the target become greater than a wavelength. 
The phase shifts due to these scatterers' radial motion add incoherently,
causing motion-induced noise.  Near the centre of the target, differences in
the scatterers' ranges are still small enough for their phase shifts to add
coherently.  As the target rotates more quickly, a smaller component of the
radar reflection contains information that is useful for motion estimation.
Therefore, the apparent signal-to-noise ratio for motion
compensation decreases as the target's angular velocity increases.

This motion-induced noise can be observed in ISAR simulations of a rotating 
target with no background noise.  Using the maximum likelihood estimator of
the signal-to-noise ratio in (\ref{ml eqn:nvML}), the apparent
signal-to-noise ratio is
\begin{equation}
\nvML=\frac{1}{4N}\sum_{n=0}^{N-1} 
\left|s(k_n,0)-s(k_n,\tau)\e{jk_n\rML}\right|^2
\end{equation}
where $\rML$ is the maximum likelihood estimate of the radial distance the
target has moved between $t=0$ and $t=\tau$.  Since the simulation has no
background noise, a non-zero $\nvML$ indicates noise caused by the target's 
rotation.

The rotation-induced noise's variance increases as the target rotates
faster.  The upper limit of the noise variance's expectation is
\begin{equation}
\frac{1}{4N}\sum_{n=0}^{N-1}\left|s(k_n,0)\right|^2+\left|s(k_n,\tau)\right|^2
\end{equation}
when the phase changes from all scatterers add incoherently and motion
estimation becomes random.

This is illustrated in figure~\ref{rmc fig:rin} for twenty estimates of the
rotation-induced noise for target angular velocities ranging from $1^o$ per
second to $20^o$ per second.  The horizontal line at a noise variance
of about $0.55$ indicates the noise variance for which motion compensation 
gives totally random results.  This shows that motion compensation is not
possible when this simulated target is rotating faster than about
$5^o$ per second.

Since the target, a simulated 737 aircraft, is $30$~metres by $26$ metres, 
and the stepped-frequency radar had a centre frequency of $9.19$~GHz, the 
target fills the ISAR cross-range window when $\omega=4.8^o$ per 
second.  At this rate, the rotation-induced noise is very high and motion
estimation will perform poorly.

For motion estimation to work well for this particular simulation, the 
target's rotation rate should be kept below about $1.2^o$ per second, when
the ISAR image fills about a quarter of the cross-range window.  This may
require sampling the target's frequency response more frequently so that the 
motion estimation is accurate, but using only a subset of the data for
forming the ISAR image.


%%%%%%%%%%%%%%%%%%%%%%%%%%%%%%%%%%%%%%%%%%%%%%%%%%%%%%%%%%%%%%%%%%%%%%%%%%%%%
\section{Accuracy of Motion Estimation}
\label{rmc sec:ma}

Many discussions of motion estimation for radar imaging and of focussing
phased-array radars quote a benchmark of $\lambda/10$ as the conventional
tolerance required.  This figure is mentioned by Steinberg
\cite{Tah76} and by Xu {\em et al.\/} \cite{Xu89,Xu90a,Xu90b}.

In this section, a similar analysis to that used throughout this chapter is
employed to show that for a maximum phase error of $\pi/2$, the magnitude of
the residual phase errors after motion compensation must be less than
$\lambda/8$.

This simple result is then extended to show that the residual range errors 
can be substantially larger than $\lambda/8$ without having a maximum phase
error greater than $\pi/2$ providing they are clustered
around multiples of $\lambda/2$.  Since the radial motion estimates produced
by the maximum likelihood motion estimator are clustered around multiples of
$\lambda/2$, this justifies using the maximum likelihood motion estimator
for ISAR motion compensation,  and it also indicates that simple measures of
motion estimation errors such as variance are inadequate for ISAR.

%%%%%%%%%%%%%%%%%%%%%%%%%%%%%%%%%%%%%%%%%%%%%%%%%%%%%%%%%%%%%%%%%%%%%%%%%%%%%
\subsection{Simple Analysis of Residual Errors}

Suppose that motion estimation and motion compensation have been applied to
the radar measurements $s(k,t)$.  If the motion compensation were perfect,
the measurements would be
\begin{equation}
s(k,t)=\int\!\!\int g(x,y)\,e^{-jkr_{xy}(t)}\,dx\,dy
\end{equation}
where $r_{xy}(t)$ is the range of point $(x,y)$ on the target at time $t$
expected by the ISAR inversion algorithm.  However, no motion compensation
is perfect so the actual measurements are different from the ideal 
\begin{equation}
s(k,t)=\int\!\!\int g(x,y)\,e^{-jkr_e(t)}\,e^{-jkr_{xy}(t)}\,dx\,dy
\end{equation}
where $r_e(t)$ is the residual motion error at time $t$.

This is equivalent to applying the ISAR inversion to a target reflectivity
function $g'(x,y,k,t)$ that varies with time and frequency according to the
severity of the motion estimation error at that time
\begin{equation}
g'(x,y,k,t)=g(x,y)\,e^{-jkr_e(t)}
\end{equation}
Define the maximum phase variation in $g(x,y)$ as
\begin{equation}
\Delta\psi=\max_{k,t,k't,}\left|kr_e(t)-k'r_e(t')\right|
\end{equation}
Then by setting an upper limit on the maximum phase variation, such as 
\begin{equation}
\Delta\psi<\frac{\pi}{2}
\end{equation}
bounds on $r_e(t)$ can be established.

In general, $r_e(t)$ is just as likely to be positive as negative.  Using
theorem \ref{rmc thm:theorem} in appendix \ref{rmc app:lemma},
\begin{equation}
\Delta\psi<2k_{N-1}\max_{\tau} \left|r_e(\tau)\right|
\end{equation}
Since $k_{N-1}\approx 8\pi/\lambda$ where $\lambda$ is the average
wavelength of the frequencies in the stepped-frequency waveform, 
this shows that the residual motion errors must satisfy
\begin{equation}
\left|r_e(t)\right|<\frac{\lambda}{16}\qquad\forall t\in[-T/2,T/2]
\end{equation}
which is equivalent to
\begin{equation}\label{rmc eqn:ma cond I}
\left|r_e(t)-r_e(t')\right|<\frac{\lambda}{8}\qquad\forall t,t'\in[-T/2,T/2]
\end{equation}
These limits are similar to the benchmark accuracy of $\lambda/10$ quoted
above.


%%%%%%%%%%%%%%%%%%%%%%%%%%%%%%%%%%%%%%%%%%%%%%%%%%%%%%%%%%%%%%%%%%%%%%%%%%%%%
\subsection{Better Analysis of Residual Errors}

The analysis of residual motion estimation errors in (\ref{rmc eqn:ma cond
I}) is inadequate because only the residual error modulo $2\pi$ is
important.\footnote{The modulo $2\pi$ is not important in the other three
assumptions analysed in this chapter because for all three, $\Delta\psi$ is
a supremum over all points $(x,y)$ on the target.  Since $x$ and $y$ are
continuous variables, taking the phase variation modulo $2\pi$ would give
$\Delta\psi=2\pi$ if any phase variations greater than $2\pi$ were
present.}  One example of motion estimation where omitting the modulo $2\pi$ 
condition  does matter is
the maximum likelihood motion estimator in chapter \ref{ml chp}, where the
residual motion errors are tightly clustered around small multiples of
$\lambda/2$.

When modulo $2\pi$ phase errors are included, the definition of $\Delta\psi$
is modified to read
\begin{equation}
\Delta\psi=\min_n\max_{k,t,k't,}\left|kr_e(t)-k'r_e(t')+2n\pi\right|
\end{equation}
where the minimum is taken over all integers $n$.

Now suppose that the range errors $r_e(t)$ are clustered around multiples of
$\lambda/2$, so that $r_e(t)\in R$ where the set $R$ is the union of
intervals of width $2\delta_i$ centred about $r_i=i\lambda/2$, where $i$ is
any integer
\begin{equation}
R=\bigcup_{i=-\infty}^{\infty} 
\left[i\frac{\lambda}{2}-\delta_i, i\frac{\lambda}{2}+\delta_i\right]
\end{equation}
%The range errors are assumed to have a probability distribution that
%is symmetrical about $r_e=0$, so that $\delta_{-i}=\delta_i$ for all $i$.

Since the phase error $\psi_e(k,t)$ corresponding to a motion error of 
$r_e(t)$ at frequency $k$ is
\begin{equation}
\psi_e(k,t)=kr_e(t)
\end{equation}
the set of phase errors resulting from motion errors in $R$ and frequencies
in $[k_0,k_{N-1}]$ can be found using similar considerations to lemma
\ref{rmc thm:lemma} and theorem \ref{rmc thm:theorem} in appendix 
\ref{rmc app:lemma}.  This set of phase errors $\Psi$ is
\begin{eqnarray}
\Psi&=&\left(
  \bigcup_{i=-\infty}^{-1}\left[
    k_{N-1}\left(i\frac{\lambda}{2}-\delta_i\right), 
    k_0\left(i\frac{\lambda}{2}+\delta_i\right)
  \right]
\right)\cup\left[-k_{N-1}\delta_0,k_{N-1}\delta_0\right]\nn\\
&&\qquad\cup\left(
  \bigcup_{i=1}^{\infty}\left[
    k_0\left(i\frac{\lambda}{2}-\delta_i\right), 
    k_{N-1}\left(i\frac{\lambda}{2}+\delta_i\right)
  \right]
\right)
\end{eqnarray}

Then the maximum modulo $2\pi$ phase variation will always be less than
$\pi/2$ if $\Psi$ is a subset of $\Psi_0$ where
\begin{equation}
\Psi_0=\bigcup_{i=-\infty}^{\infty} 
\left[2i\pi-\frac{\pi}{4}, 2i\pi+\frac{\pi}{4}\right]
\end{equation}
This leads to three sets of conditions on $\delta_i$, depending on whether
$i<0$, $i=0$ or $i>0$.

\begin{description}
\item[Case (i):] If $i<0$, $\delta_i$ must be chosen so that
\begin{equation}
\left[k_{N-1}\left(i\frac{\lambda}{2}-\delta_i\right), 
      k_0\left(i\frac{\lambda}{2}+\delta_i\right)\right]
\subset
\left[2i\pi-\frac{\pi}{4}, 2i\pi+\frac{\pi}{4}\right]
\end{equation}
This imposes the following two conditions on the interval's endpoints
\begin{equation}
k_{N-1}\left(i\frac{\lambda}{2}-\delta_i\right)\geq 2i\pi-\frac{\pi}{4}
\end{equation}
\begin{equation}
k_0\left(i\frac{\lambda}{2}+\delta_i\right)\leq 2i\pi+\frac{\pi}{4}
\end{equation}
Subtracting them shows that each $\delta_i$ must satisfy
\begin{equation}
\delta_i\leq\frac{\frac{\pi}{2}+i(k_{N-1}-k_0)\frac{\lambda}{2}}{k_{N-1}+k_0}
\end{equation}

\item[Case (ii):] If $i=0$, $\delta_0$ must be chosen so that
\begin{equation}
\left[-k_{N-1}\delta_0,k_{N-1}\delta_0\right]
\subset
\left[-\frac{\pi}{4},\frac{\pi}{4}\right]
\end{equation}
which shows that $\delta_0$ must satisfy
\begin{equation}
\delta_0\leq\frac{\pi}{4k_{N-1}}
\end{equation}

\item[Case (iii):] If $i>0$, $\delta_i$ must be chosen so that
\begin{equation}
\left[k_0\left(i\frac{\lambda}{2}-\delta_i\right), 
      k_{N-1}\left(i\frac{\lambda}{2}+\delta_i\right)\right]
\subset
\left[2i\pi-\frac{\pi}{4}, 2i\pi+\frac{\pi}{4}\right]
\end{equation}
This imposes the following two conditions on the interval's endpoints
\begin{equation}
k_0\left(i\frac{\lambda}{2}-\delta_i\right)\geq 2i\pi-\frac{\pi}{4}
\end{equation}
\begin{equation}
k_{N-1}\left(i\frac{\lambda}{2}+\delta_i\right)\leq 2i\pi+\frac{\pi}{4}
\end{equation}
Subtracting them shows that each $\delta_i$ must satisfy
\begin{equation}
\delta_i\leq\frac{\frac{\pi}{2}-i(k_{N-1}-k_0)\frac{\lambda}{2}}{k_{N-1}+k_0}
\end{equation}
\end{description}

Now that all three cases have been considered, putting their separate 
conditions together shows that
\begin{equation}
\delta_i\leq\cases{
	\ds\frac{\pi}{4k_{N-1}}			& if $i=0$ \cr
	\ds\frac{\frac{\pi}{2}-\left|i\right|(k_{N-1}-k_0)\frac{\lambda}{2}}
	  {k_{N-1}+k_0}				& otherwise
}
\end{equation}
%============================================================================
\begin{figure}\centering
\caption[Allowable residual motion estimation errors.]{The allowable values 
of the residual motion estimation errors $r_e$ lie in small intervals
centred at multiples of $\lambda/2$.  The width of each interval (twice
$\delta_i$) is shown plotted against the position of its centre.}
\label{rmc fig:motion errors}

\psset{unit=0.6cm}
\begin{pspicture}(-7,-1.25)(7,8)
% First the axes and their labels
\psline[linecolor=black,linewidth=1.5pt,ticks=none,labels=none]{->}(-6,0)(6,0)
\psline[linecolor=black,linewidth=1.5pt,ticks=none,labels=none]{->}(0,0)(0,6)
\uput[r](6,0){$r_e$}
\uput[u](0,6){width}
% And the labels with ticks
\uput[l](0,5){$\frac{\lambda}{8}$}
\psline[linecolor=black,linewidth=1pt]{-}(0,5)(0.15,5)
\uput[d](5,0){$\frac{1}{4}\Delta r_r$}
\psline[linecolor=black,linewidth=1pt]{-}(5,0)(5,-0.15)
\uput[d](-5,0){$-\frac{1}{4}\Delta r_r$}
\psline[linecolor=black,linewidth=1pt]{-}(-5,0)(-5,-0.15)
% Draw in the widths
\psclip{\pspolygon[linestyle=none](-5,0)(0,5)(5,0)}
\multido{\nx=-4.5+0.5}{19}{%
	\psline[linecolor=gray,linewidth=1.2pt]{-}(\nx,0)(\nx,5)
}
\endpsclip
% Finish with an arrow indicating that the intervals are lambda/2 apart
\psline[linecolor=black,linewidth=1pt]{->}(1,2)(1.5,2)
\psline[linecolor=black,linewidth=1pt]{<-}(2,2)(2.5,2)
\uput[u](1.75,2){$\frac{\lambda}{2}$}
\end{pspicture}
\end{figure}
%============================================================================ 
Now 
\begin{equation}
k_{N-1}-k_0\approx \frac{4\pi N\Delta f}{c}
=4\pi\frac{N\Delta f}{\lambda\overline{f}} 
\end{equation}
and
\begin{equation}
2k_{N-1}\approx k_{N-1}+k_0 =\frac{8\pi}{\lambda}
\end{equation}
so that each $\delta_i$ must satisfy
\begin{equation}
\delta_i\leq \frac{\lambda}{16}\left(1-4\frac{N\Delta f}{\overline{f}}
\left|i\right|\right)
\end{equation}

Since each $\delta_i$ must be positive, this sets an upper limit on
$\left|i\right|$
\begin{equation}
\left|i\right|\leq\frac{\overline{f}}{4N\Delta f}
\end{equation}
From this, the greatest allowable range error $r_e(t)$ can be calculated.
Using the formula for the ISAR image's range resolution $\Delta r_r$ from
(\ref{ii eqn:rd drr}), the residual error at the maximum values of $i$ has 
absolute value $\Delta r_r/4$, so the range errors may vary between
\begin{equation}
-\Delta r_r/4 \leq r_e(t) \leq \Delta r_r/4
\end{equation}
providing the $r_e(t)$ become increasingly tightly clustered about multiples
of $\lambda/2$ the further from $r_e=0$ they are.  This is shown in figure
\ref{rmc fig:motion errors}, where the length of each interval has been 
plotted against its centre.

Note that when using the maximum likelihood motion estimator $\rML$, 
the spread of residual motion estimation errors may be as wide as half the ISAR
image's range resolution, which is a quarter the width
of the central lobe of the envelope of $J(r)$ (which can be seen in
figure \ref{ml fig:J(r) pictures}(a)).

%%%%%%%%%%%%%%%%%%%%%%%%%%%%%%%%%%%%%%%%%%%%%%%%%%%%%%%%%%%%%%%%%%%%%%%%%%%%%
\section{Summary of the Constraints}

To summarize the results of this chapter, sets of constraints have
been obtained for the three assumptions commonly made for ISAR imaging.
These constraints involve the radar's average wavelength $\lambda$, the
target's extent in the range and cross-range directions, $D_r$ and $D_c$,
the ISAR image's resolutions in the range and cross-range directions,
$\Delta r_r$ and $\Delta r_c$, and the width $W_c$ of the ISAR image's
cross-range window.

\begin{description}
\item[Range-Doppler Processing]  The approximation of a spherical wave as
a plane-wave in range-Doppler ISAR is valid providing
$$\Delta r_c>\frac{2D_rD_c}{r_0} $$
and
$$\Delta r_r>\frac{D_c^2}{r_0}	$$

\item[Polar Reformatting]  The polar reformatting step of the ISAR inversion
algorithm can be omitted if
$$\Delta r_c^2>\frac{\lambda D_r}{4}	   $$
and
$$\Delta r_r\Delta r_c>\frac{\lambda D_c}{4}  $$
which are identical to the classical constraints obtained by
limiting the motion of all scatterers to a single resolution cell.

\item[Motion Estimation]  Accurate motion estimation necessary for ISAR 
imaging requires that 
$$D_c<\frac{W_c}{4}$$
\end{description}

Finally, the analysis of the motion estimation accuracy required for well-focussed ISAR
images shows that residual motion errors after motion compensation may
be significantly greater than the conventional limit of $\lambda/10$
providing the errors are nearly integral multiples of $\lambda/2$.

%%%%%%%%%%%%%%%%%%%%%%%%%%%%%%%%%%%%%%%%%%%%%%%%%%%%%%%%%%%%%%%%%%%%%%%%%%%%%
\appendix{Theorems Concerning Maxima}
\label{rmc app:lemma}

%============================================================================
\begin{lemma}
\label{rmc thm:lemma}
Let $A=[A_0,A_1]$ be an interval of the real line where 
\begin{equation}
A_1\geq A_0\geq 0
\end{equation}
Similarly, let $B=[B_0,B_1]$ be an interval of the real line where
\begin{equation}
B_1\geq B_0\geq 0
\end{equation}
Then
\begin{equation}
\sup_{\scriptstyle a,a'\in A\atop\scriptstyle b,b'\in B}
\left|ab-a'b'\right|=A_1B_1-A_0B_0
\end{equation}
\end{lemma}
%============================================================================

%============================================================================
\begin{proof} For any $a\in A$ and any $b\in B$,\footnote{This is a very
laborious proof for a lemma which seems intuitively obvious.  This is
because there are two classes of obvious results; those that are obvious and
true, and those that are obvious and false.  Sometimes, to be sure
which class a result belongs to, it is necessary to follow through
every single step.}

\begin{eqnarray}
&A_1\geq a\geq A_0\geq 0&\\
&B_1\geq b\geq B_0\geq 0&
\end{eqnarray}
Therefore
\begin{equation}
A_1B_1\geq aB_1\geq ab\geq A_0b\geq A_0B_0\geq 0
\end{equation}
for all $a\in A$ and $b\in B$.  Therefore, for any $a,a'\in A$ and any
$b,b'\in B$,
\begin{equation}
A_1B_1\geq ab\qquad\mbox{and}\qquad a'b'\geq A_0B_0\geq 0
\end{equation}
Subtracting $a'b'$ from $ab$ shows that
\begin{equation}
A_1B_1-A_0B_0\geq ab-a'b'\geq A_0B_0-A_1B_1
\end{equation}
so that 
\begin{equation}
A_1B_1-A_0B_0\geq \left|ab-a'b'\right|
\end{equation}
Therefore
\begin{equation}
A_1B_1-A_0B_0\geq\sup_{\scriptstyle a,a'\in A\atop\scriptstyle b,b'\in B}
\left|ab-a'b'\right|
\end{equation}

To show that this upper bound is attained, let $a=A_1$, $b=B_1$, $a'=A_0$
and $b'=B_0$.  Therefore
\begin{equation}
A_1B_1-A_0B_0=\sup_{\scriptstyle a,a'\in A\atop\scriptstyle b,b'\in B}  
\left|ab-a'b'\right|
\end{equation}
and the lemma is proved.
\end{proof}
%============================================================================


%============================================================================
\begin{theorem}
\label{rmc thm:theorem}
Let $A=[A_0,A_1]$ be an interval of the real line where 
\begin{equation}
A_1\geq A_0\geq 0
\end{equation}
Similarly, let $B=[B_0,B_1]$ be an interval of the real line where
\begin{equation}
B_1\geq B_0
\end{equation}
where now $B_0$ and $B_1$ may be negative.  Then for $s$ real
\begin{equation}
\inf_s\sup_{\scriptstyle a,a'\in A\atop\scriptstyle b,b'\in B}
\left|a(b-s)-a'(b'-s)\right|=A_1(B_1-B_0)
\end{equation}
and this is attained for any value of $s$ in $B$.
\end{theorem}
%============================================================================

%============================================================================
\begin{proof}
First note that the intervals $A$ and $B$ are finite and bounded, hence the
space $A^2\times B^2$ from which each $(a,a',b,b')$ comes is compact.  
Therefore for all finite $s$, the supremum of $\left|a(b-s)-a'(b'-s)\right|$ 
is attained for some $(a,a',b,b')$ in $A^2\times B^2$.  Now suppose $s\in S$
for some sufficiently large closed set $S$.  Then the infinum of 
$\sup\left|a(b-s)-a'(b'-s)\right|$ is also attained for some $s\in
S$.\footnote{This artificial compact set $S$ is needed to ensure that the 
infinum is attained, not just approached as $s\to\pm\infty$.  The conclusion
of the theorem that the infinum is attained for every $s\in B$ is an 
{\em a posteriori\/} justification for choosing any closed finite set $S$
such that $B\subset S$.}

Initially, consider
\begin{equation}
\sup_{\scriptstyle a,a'\in A\atop\scriptstyle b,b'\in B}
\left|a(b-s)-a'(b'-s)\right|
\end{equation}
for a fixed real $s$.  There are three cases to consider depending on the
value of $s$ for which the infinum is attained; $s\leq B_0$, $s\geq B_1$ and
$B_0\leq s\leq B_1$.
\begin{description}
\item[Case (i):] $s\leq B_0$\hfill\mbox{}\linebreak
For every $b,b'\in B$, $b-s$ and $b'-s$ are positive.  Using the lemma
\ref{rmc thm:lemma} shows that
\begin{eqnarray}
\sup_{\scriptstyle a,a'\in A\atop\scriptstyle b,b'\in B}
\left|a(b-s)-a'(b'-s)\right|
&=&A_1(B_1-s)-A_0(B_0-s)\nn\\
&=&(A_1B_1-A_0B_0)-(A_1-A_0)s\nn\\
&\geq&(A_1B_1-A_0B_0)-(A_1-A_0)B_0\nn\\
&=&A_1(B_1-B_0)
\end{eqnarray}

\item[Case (ii):] $s\geq B_1$\hfill\mbox{}\linebreak
For every $b,b'\in B$, $s-b$ and $s-b'$ are positive.  Using the lemma
\ref{rmc thm:lemma} again shows that
\begin{eqnarray}
\sup_{\scriptstyle a,a'\in A\atop\scriptstyle b,b'\in B}
\left|a(b-s)-a'(b'-s)\right|
&=&A_1(s-B_0)-A_0(s-B_1)\nn\\
&=&(A_0B_1-A_1B_0)+(A_1-A_0)s\nn\\
&\geq&(A_0B_1-A_1B_0)+(A_1-A_0)B_1\nn\\
&=&A_1(B_1-B_0)
\end{eqnarray}

\item[Case (iii):] $B_0\leq s \leq B_1$\hfill\mbox{}\linebreak 
If the supremum is attained with $s\leq b,b'\leq B_1$, apply case (i) with
$B_0$ replaced by $s$ to show that
\begin{eqnarray}
\sup_{\scriptstyle a,a'\in A\atop\scriptstyle b,b'\in B}
\left|a(b-s)-a'(b'-s)\right|
&=&A_1(B_1-s)-A_0(s-s)\nn\\
&=&A_1(B_1-s)\nn\\
&\leq&A_1(B_1-B_0)
\end{eqnarray}
If the supremum is attained with $B_0\leq b,b'\leq s$, apply case (ii) with
$B_1$ replaced by $s$ to show that
\begin{eqnarray}
\sup_{\scriptstyle a,a'\in A\atop\scriptstyle b,b'\in B}
\left|a(b-s)-a'(b'-s)\right|
&=&A_1(s-B_0)-A_0(s-s)\nn\\
&=&A_1(s-B_0)\nn\\
&\leq&A_1(B_1-B_0)
\end{eqnarray}
The third possibility is that the supremum is attained with $B_0\leq b\leq
s\leq b'\leq B_1$.  For all $b$ and $b'$ between these limits
\begin{eqnarray}
\left|a(b-s)-a'(b'-s)\right|&=&a(s-b)+a'(b'-s)\nn\\
&\leq&A_1(s-B_0)+A_1(B_1-s)\nn\\
&=&A_1(B_1-B_0)
\end{eqnarray}
This upper bound on the supremum is attained when $a=a'=A_1$, $b=B_0$ and
$b'=B_1$.  Therefore, if $B_0\leq b\leq s\leq b'\leq B_1$
\begin{equation}
\sup_{\scriptstyle a,a'\in A\atop\scriptstyle b,b'\in B}
\left|a(b-s)-a'(b'-s)\right|=A_1(B_1-B_0)
\end{equation}
\end{description}

Since all three cases have an identical upper bound, the supremum must be
less than or equal to this common upper bound
\begin{equation}
\sup_{\scriptstyle a,a'\in A\atop\scriptstyle b,b'\in B}
\left|a(b-s)-a'(b'-s)\right|\leq A_1(B_1-B_0)
\end{equation}
so that
\begin{equation}
\inf_s\sup_{\scriptstyle a,a'\in A\atop\scriptstyle b,b'\in B}
\left|a(b-s)-a'(b'-s)\right|\leq A_1(B_1-B_0)
\end{equation}

Setting $a=a'=A_1$, $b=B_0$ and $b'=B_1$ for any $s\in B$ shows that
this bound is reached with equality.  Therefore
\begin{equation}
\inf_s\sup_{\scriptstyle a,a'\in A\atop\scriptstyle b,b'\in B}
\left|a(b-s)-a'(b'-s)\right|=A_1(B_1-B_0)
\end{equation}
as required.
\end{proof}
%============================================================================

