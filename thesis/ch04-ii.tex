%%%%%%%%%%%%%%%%%%%%%%%%%%%%%%%%%%%%%%%%%%%%%%%%%%%%%%%%%%%%%%%%%%%%%%%%%%%
%
%			INVERSE SYNTHETIC APERTURE RADAR
%
%				  PhD Thesis
%
%		Stephen Simmons		simmons@ee.mu.oz.au
%
%	    Department of Electrical and Electronic Engineering
%	    University of Melbourne, Parkville, 3052, Australia
%
% Chapter 4:	ISAR Imaging
%
%		started first draft:	13 Dec 1994
%		finished first draft:	22 Dec 1994
%		submitted:		
%
%%%%%%%%%%%%%%%%%%%%% Copyright (C) 1994 Stephen Simmons %%%%%%%%%%%%%%%%%%

\chapter{ISAR Imaging}
\label{ii chp}

\bigletter The analysis in chapter~\ref{hrr chp} of radar reflections from a
moving target concluded with the simple model in equation 
(\ref{hrr eqn:s(f,t)}).  This expressed the target's frequency response at
any time as a function of its motion and its spatial radar reflectivity.  In
ISAR, the frequency responses are known but the target's motion and
reflectivity are not.  An inversion algorithm is needed to specify how to
estimate a target's reflectivity from measurements of its frequency
responses over a period of time.  Because the target's motion creates the
synthetic aperture that ISAR needs, the complexity of an ISAR algorithm is
directly related to the complexity of the target's motion.  

In this chapter, some efficient ISAR inversion algorithms are derived for
certain assumptions about the form of the target's motion.  Constraining the
target's motion allows approximations based on Fourier transforms, which can
be evaluated efficiently using the fast Fourier transform. Range-Doppler
ISAR processing is derived for uniformly rotating targets, and for targets
moving in a straight line at constant speed.  A more recent method of ISAR
inversion, called wavenumber processing, is derived for targets moving in a
straight line.  Some other approaches to ISAR, and imaging using 
superresolution spectral estimators, are briefly discussed.

As the motion of a target may not exactly match the assumptions built into
the ISAR inversion algorithm, motion estimation is needed to measure how the
target moves and compensate for it.  A number of methods for motion
estimation are discussed in chapter~\ref{mc chp} and a new method of motion
estimation, the subject of this thesis, is derived in chapters \ref{ml chp}
to \ref{sp chp}.  

Traditionally, ISAR has been described in terms of the Doppler effect.  This
gives a misleading view of the ISAR imaging mechanism, and should be
rejected in favour of the modern signal processing approach which allows a
more complete model of the ISAR imaging process and explicitly states the 
approximations that have been made.  The Doppler description is so common in
the ISAR literature, however, that it has been included for the sake of
completeness.  Its proper place in the modern view of ISAR is reflected in
its location, appendix \ref{ii app:dop} at the end of the chapter.


%%%%%%%%%%%%%%%%%%%%%%%%%%%%%%%%%%%%%%%%%%%%%%%%%%%%%%%%%%%%%%%%%%%%%%%%%%%%%
\section{The ISAR Imaging Model}

The model for a moving target's frequency response, developed in 
chapter~\ref{hrr chp}, is
\begin{equation}\label{ii eqn:s(f,t)}
s(f,t)=\int\!\!\int\frac{g(x,y)}{r_{xy}^2(t)}\,e^{-j\fpc f r_{xy}(t)}\,dx\,dy
\end{equation}
where $g(x,y)$ is the target's unknown radar reflectivity, measured in 
a Cartesian coordinate system which moves with the target.  $r_{xy}(t)$ is
the distance from the radar to the point $(x,y)$ on the target at time $t$. 
For convenience, the coordinate system's origin is usually taken to be the
centre of the target.

%%%%%%%%%%%%%%%%%%%%%%%%%%%%%%%%%%%%%%%%%%%%%%%%%%%%%%%%%%%%%%%%%%%%%%%%%%%%%
\subsection{Simplifying the Model}

Writing the frequency response in terms of $f$ leads to a factor of $4\pi/c$
in the argument of the complex exponential.  The notation may be simplified
slightly by using the wavenumber $k$ instead of the frequency $f$.  This
still leaves a factor of $2$ because ranges are measured as round-trip 
distances.  When this factor of $2$ is also included in the wavenumber,
\begin{equation}
k=\fpc f
\end{equation}
and the frequency response becomes
\begin{equation}\label{ii eqn:s(k,t) I}
s(k,t)=\int\!\!\int\frac{g(x,y)}{r_{xy}^2(t)}\,e^{-jkr_{xy}(t)}\,dx\,dy
\end{equation}
To emphasize this point, the wavenumber is $2\pi/\lambda=2\pi f/c$, but
throughout this thesis, the symbol $k$ is used to represent twice the
wavenumber, so $k=4\pi f/c$.  This is equivalent to pretending that the
scatterers on the target are radar sources and that the frequency has 
doubled or the speed of light has halved.\footnote{\protect\label{ii ftn:rr}
Cafforio {\em et al.\/} describe this as the ``radiating reflector'' 
model \protect\cite{Caf91}.  Its counterpart in seismic imaging is the
``exploding reflector'' model which sounds much more exciting (see the
footnote on page \pageref{ii ftn:er}).}  For convenience, both $f$ and
$k=4\pi f/c$ will be called frequencies; the context will determine which is
intended.

ISAR is designed for high-resolution imaging of small targets that are a
long way from the radar, so the $r_{xy}^2(t)$ term in the denominator of the
integrand of (\ref{ii eqn:s(k,t) I}) varies little over the whole of the 
target.  Therefore, $r_{xy}(t)$ may be regarded as a constant, $r_0$, which
is the distance between the radar and the centre of the target at time
$t=0$. Being a constant, the $1/r_0^2$ can be included with $g(x,y)$ to give
an alternative model
\begin{equation}\label{ii eqn:s(k,t) II}
s(k,t)=\int\!\!\int g(x,y)\,e^{-jkr_{xy}(t)}\,dx\,dy
\end{equation}
In extreme cases where the target's range changes appreciably during the 
imaging period, this approximation may not be appropriate.  But if the 
target's size is still small compared with its minimum range, the $r_{xy}(t)$
in the denominator may be replaced by $r_0(t)$, the range of the target's
centre at time $t$.  This gives a third form of the target's frequency 
response 
\begin{equation}\label{ii eqn:s(k,t) III}
s(k,t)=\frac{1}{r_0^2(t)}\int\!\!\int g(x,y)\,e^{-jkr_{xy}(t)}\,dx\,dy
\end{equation}
Note that similar approximations cannot be made for the $r_{xy}(t)$ in the
argument of the complex exponential because the complex exponential is
sensitive to $kr_{xy}(t)$ modulo $2\pi$.

%%%%%%%%%%%%%%%%%%%%%%%%%%%%%%%%%%%%%%%%%%%%%%%%%%%%%%%%%%%%%%%%%%%%%%%%%%%%%
\subsection{Measurements for the ISAR Inversion}

When a stepped-frequency waveform containing $N$ evenly spaced frequencies
\begin{equation}
f_n=f_0+n\Delta f\qquad\mbox{for $n=0,1,\ldots,N-1$}
\end{equation}
is transmitted at a time $t$ and the backscattererd signal sampled, the 
target's frequency response $s(k,t)$ is measured for the $N$ uniformly spaced 
values of $k$ in the stepped-frequency waveform
\begin{equation}
k_n=k_0+n\,\Delta k\qquad n=0,1,\ldots, N-1
\end{equation}
This is repeated at each of $M$ uniformly spaced times $t_m$ over the time 
interval $[-T/2,T/2]$ 
\begin{equation}
t_m=\left(\frac{m}{M-1}-\frac{1}{2}\right)T\qquad\mbox{for $m=0,1,\ldots,
M-1$}
\end{equation}
to give the full set of measurements used for ISAR imaging
\begin{equation}
s(k_n,t_m)\qquad\mbox{for $n=0,1,\ldots,N-1$ and $m=0,1,\ldots,M-1$}
\end{equation}

Some ISAR inversion algorithms that use this set of measurements will now be
derived.  The two methods examined are range-Doppler processing, where
$r_{xy}(t)$ is approximated using a linear function of $x$ and $y$, and
wavenumber processing, where the circular nature of the radar's wavefront is
exploited to give a more exact inversion.


%%%%%%%%%%%%%%%%%%%%%%%%%%%%%%%%%%%%%%%%%%%%%%%%%%%%%%%%%%%%%%%%%%%%%%%%%%%%%
\section{Range-Doppler Processing}
\label{ii sec:rdp}

The simplest formulation of ISAR is range-Doppler processing.  This uses the
first-order Taylor series for $r_{xy}(t)$ expanded about the centre of the
target  to show that the measured $s(k,t)$ are samples of the Fourier
transform of the target's reflectivity $g(x,y)$.  

The points in the spatial frequency domain at which the Fourier transform of
$g(x,y)$ is sampled depend on the target's motion.  Consequently, any
range-Doppler ISAR inversion depends on the way the target moves.  This is
analysed for two types of target motion; first for a uniformly rotating
target, then for a target moving in a straight line with constant speed.

%%%%%%%%%%%%%%%%%%%%%%%%%%%%%%%%%%%%%%%%%%%%%%%%%%%%%%%%%%%%%%%%%%%%%%%%%%%%%
\subsection{Uniformly Rotating Target}

Consider a target, such as an object mounted on a turntable in a radar test
range \cite{Che80b} or a ship pitching and rolling at anchor, which rotates
with a uniform angular velocity a fixed distance from the radar.

%============================================================================
\begin{figure}\centering
\caption[Imaging geometry for range-Doppler ISAR with a rotating 
target.]{Imaging geometry for range-Doppler ISAR with a target rotating at
an angular speed $\omega$ situated a fixed distance $r_0$ from the radar.}
\label{ii fig:rotating target}
\setlength{\unitlength}{1cm}
\psset{unit=1cm}
\begin{pspicture}(-0.625,-2.5)(12,3)
% Dotted lines to indicate the axes
\psline[linecolor=lightgray,linewidth=1pt]{-}(0,0)(12,0)
\psline[linecolor=lightgray,linewidth=1pt]{-}(10,-2)(10,2)
\psline[linecolor=lightgray,linewidth=1pt]{-}(0,-2)(0,0.5)
% The target 
% This is a little messy because EPSF makes the box zero height.
% Note that from the unit=1cm at the start, 2.5cm/1cm=2.5
\rput{25}(10,0){\put(0,-1.25){\makebox(0,2.5)[t]{
	\epsfysize=2.5cm\epsffile{../pics/plane.eps}
}}}
% The target's axes
\rput{25}(10,0){
	% Solid lines to indicate the target's axes
	\psline[linecolor=black,linewidth=1.5pt]{->}(-2,0)(2,0)
	\uput[r](2,0){$x$}
	\psline[linecolor=black,linewidth=1.5pt]{->}(0,-2)(0,2)
	\uput[u](0,2){$y$}
}
% An arrow and a label to indicate the total rotation
\psarc[linecolor=black,linewidth=1pt]{->}(10,0){1}{0}{25}
\uput[r](11,0.2){$\omega t$}
% Finally the arrows and their labels
\pcline[linecolor=black,linewidth=1pt]{<->}(0,-1.5)(10,-1.5)
\lput*{:U}{$r_0$}
\cnode*(9,0.8){2pt}{Ntar}
\uput[ul](9,0.8){$(x,y)$}
\pnode(0,0){Nrad}
\ncline[linecolor=black,linewidth=1pt]{<->}{Nrad}{Ntar}
\lput*{:U}{$r_{xy}(t)$}
% This is a little fiddle to cover up a gap in the LOS caused by the
% $r_{xy(t)} label
\psline[linecolor=lightgray,linewidth=1pt]{-}(3,0)(6,0)
% Now something for the radar at the left
\psarc*[fillcolor=darkgray](0,0){0.625}{128}{232}
\psellipse[fillstyle=solid,fillcolor=gray](-0.375,0)(0.15,0.5)
\psline{cc-cc}(-0.375,-0.475)(0,0)
\psline{cc-cc}(-0.5,0.25)(0,0)
\psline{cc-cc}(-0.25,0.25)(0,0)
\end{pspicture}
\end{figure}
%============================================================================

The imaging geometry is shown in figure~\ref{ii fig:rotating target}, where
the target is located on the radar's boresight axis a distance $r_0$ from
the radar antenna.  The target rotates at constant angular velocity
$\omega$.  The target's coordinate system has its origin at the centre of
rotation and it rotates with the target.  At time $t=0$, the target's $x$ 
axis is aligned with the boresight axis.  

The angle at time $t$ between the target's $x$ axis and the boresight axis 
is $\omega t$.  Using Pythagoras' theorem, the distance at time $t$ 
from the radar to a scatterer at $(x,y)$ is
\begin{equation}\label{ii eqn:rxy(t) rot}
r_{xy}(t)=\sqrt{(r_0+x\cos\omega t-y\sin\omega t)^2
+(x\sin\omega t+y\cos\omega t)^2}
\end{equation}
If the size of the target is very small in comparison to its distance from
the radar, so that
\begin{equation}\label{ii eqn:size condition}
\sqrt{x^2+y^2}\ll r_0
\end{equation}
for all points $(x,y)$ on the target, $r_{xy}(t)$ may be approximated by a
first order Taylor series in $x$ and $y$, expanded about the centre of the 
target at $(0,0)$
\begin{equation}\label{ii eqn:rot approx}
r_{xy}(t)\approx r_0+x\cos\omega t-y\sin\omega t
\end{equation}
Substituting this Taylor series into the measurement model in
(\ref{ii eqn:s(k,t) II}) shows that
\begin{equation}
\label{ii eqn:rot meas}
s(k,t)=e^{-jkr_0}\int\!\!\int g(x,y)\,
	e^{-jk(x\cos\omega t-y\sin\omega t)}\,dx\,dy
\end{equation}
is approximately $G(k_x,k_y)$, the two-dimensional Fourier transform 
of $g(x,y)$ given by
\begin{equation}\label{ii eqn:G(kx,ky)}
G(k_x,k_y)=\int\!\!\int g(x,y)\,e^{-j(xk_x+yk_y)}\,dx\,dy
\end{equation}
Make a transformation in (\ref{ii eqn:rot meas}) from the $(k,t)$ measurement
space to the spatial frequency $(k_x,k_y)$-space where
\begin{eqnarray}
k_x&=&k\cos\omega t \\
k_y&=&-k\sin\omega t 
\end{eqnarray}
to show that $s(k,t)$ is $G(k_x,k_y)$ sampled at points on a polar grid
in $(k_x,k_y)$-space according to
\begin{equation}
s(k,t)=e^{-jkr_0}G(k\cos\omega t,-k\sin\omega t)
\end{equation}
This is illustrated in figure \ref{ii fig:RD rot grid}.  

The inverse Fourier transform required to obtain $g(x,y)$ from $G(k_x,k_y)$
can be implemented efficiently using a fast Fourier transform (FFT) only if
the points at which $G(k_x,k_y)$ is known lie on a rectangular grid.  The
grid in figure \ref{ii fig:RD rot grid} is approximately rectangular because
the radar has a narrow relative bandwidth.  However, for an accurate ISAR
image of the target, the samples of $G(k_x,k_y)$ have to be resampled so
that the grid on which the resampled points lie is rectangular.  

%============================================================================
\begin{figure}\centering
\caption[The points at which $G(k_x,k_y)$ is known for range-Doppler ISAR 
with a uniformly rotating target.]{The points at which $G(k_x,k_y)$ is known
for range-Doppler ISAR with a uniformly rotating target lie on a small
segment of a polar grid.  They lie at the intersection of concentric circles
(whose radii are the $k_n$ of the stepped-frequency waveform) and straight
lines (whose orientations are those of the target at each time $t_m$).}
\label{ii fig:RD rot grid}

\setlength{\unitlength}{0.8cm}
\psset{unit=0.8cm}
\begin{pspicture}(-6.5,-3.2)(6.5,4)
% Allow polar coordinates
\SpecialCoor	
% First the axes and their labels
\psline[linecolor=black,linewidth=1.5pt]{->}(-6,0)(6,0)
\psline[linecolor=black,linewidth=1.5pt]{->}(0,-3)(0,3)
\uput[r](6,0){$k_x$}
\uput[u](0,3){$k_y$}
% First the lines
\multido{\nang=339+6}{8}{%
  \psline[linecolor=gray,linewidth=0.75pt]{-}(3.5;\nang)(5.75;\nang)
}
% Now the circles
\multido{\nr=3.8+0.3}{7}{%
  \psarc[linecolor=gray,linewidth=0.75pt]{-}{\nr}{336}{384}
}
% Finally all the dots.
\multido{\nang=339+6}{8}{\multido{\nr=3.8+0.3}{7}{\qdisk(\nr;\nang){2pt}}}
\NormalCoor
\end{pspicture}
\end{figure}
%============================================================================

The $e^{-jkr_0}$ phase term centres the target's image in the range direction
and may be neglected.  This gives the following algorithm for range-Doppler
processing applied to a uniformly rotating target:

%============================================================================
\begin{algorithm}[Range-Doppler ISAR---uniformly rotating target]
\label{ii alg:RD rotating}\mbox{}\par

To obtain a uniformly rotating target's spatial radar reflectivity $g(x,y)$
from uniformly spaced measurements of the target's frequency response
$s(k,t)$ over a period of time:
\begin{enumerate}
\item Use the radar data or {\em a priori\/} information to estimate the 
target's angular speed $\omega$.

\item Associate the measurements $s(k,t)$ with samples of the target's
reflectivity $G(k_x,k_y)$ using
\begin{equation}
G(k\cos\omega t,-k\sin\omega t)=s(k,t)
\end{equation}

\item Resample the $G(k\cos\omega t,-k\sin\omega t)$ from the polar grid of
figure~\ref{ii fig:RD rot grid} to give $G(k_x,k_y)$ uniformly sampled on a 
rectangular grid.

\item Take the two-dimensional inverse Fourier transform of the
rectangularly-sampled $G(k_x,k_y)$ to give an estimate $\widehat{g}(x,y)$
of the target's spatial reflectivity $g(x,y)$
\begin{equation}
\widehat{g}(x,y)={\cal F}^{-1}\left\{G(k_x,k_y)\right\}
\end{equation}

\item Display $\log\left|\widehat{g}(x,y)\right|$ as the target's ISAR image.

\end{enumerate}
\end{algorithm}
%============================================================================


%%%%%%%%%%%%%%%%%%%%%%%%%%%%%%%%%%%%%%%%%%%%%%%%%%%%%%%%%%%%%%%%%%%%%%%%%%%%%
\subsection{Target Moving in a Straight Line}

The second basic motion that has an efficient inversion using 
range-Doppler processing is that of a target moving at uniform speed along a
straight line at an angle to the boresight axis.  

The geometry of the radar and target is shown in figure 
\ref{ii fig:moving target}.  The target moves at constant speed
$v$ along a straight line at an angle $\theta$ to the boresight
axis.  $\theta$ must not be close to $0^o$ or $180^o$ because the
cross-range component of the target's motion is needed to change its aspect
angle and so generate the synthetic aperture.   The target is observed for
the time interval $[-T/2,T/2]$, and at $t=0$ the target lies a distance
$r_0$ from the radar on the boresight axis.  The coordinate system for
$g(x,y)$ has its origin at the target's geometric centre and the $x$ axis
points in the direction the target is moving.

%============================================================================
\begin{figure}\centering
\caption[Imaging geometry for range-Doppler ISAR with a target moving in a
straight line.]{Imaging geometry for range-Doppler ISAR with a target moving
in a straight line at constant speed $v$.  The target crosses the boresight
axis at $t=0$ at an angle $\theta$ and a distance $r_0$ from the radar.}
\label{ii fig:moving target}

\setlength{\unitlength}{1cm}
\psset{unit=1cm}
\begin{pspicture}(-0.625,-1.5)(12,5.5)
\SpecialCoor
% Dotted lines to indicate the axes
\psline[linecolor=lightgray,linewidth=1pt]{-}(0,0)(12,0)
\psline[linecolor=lightgray,linewidth=1pt]{-}(10,-1)(10,0)
\psline[linecolor=lightgray,linewidth=1pt]{-}(0,-1)(0,0.5)
% The target and its axes
\rput{100}(10,0){
	\psline[linecolor=lightgray,linewidth=1pt]{-}(0,0)(3,0)
	\psline[linecolor=lightgray,linewidth=1pt]{-}(0,-2)(0,0)
	% This is a little messy because EPSF makes the box zero height.
	% Note that from the unit=1cm at the start, 2.5cm/1cm=2.5
	\put(3,-1.25){\makebox(0,2.5)[t]{
		\epsfysize=2.5cm\epsffile{../pics/plane.eps}
	}}
	% Solid lines to indicate the target's axes
	\psline[linecolor=black,linewidth=1.5pt]{->}(1,0)(5,0)
	\uput[r](5,0){$x$}
	\psline[linecolor=black,linewidth=1.5pt]{->}(3,-2)(3,2)
	\uput[u](3,2){$y$}
	\pcline[linecolor=black,linewidth=1pt]{|->}(0,-1.5)(3,-1.5)
	\lput*{:U}{$vt$}
}
% An arc and a label to indicate the total rotation
\psarc[linecolor=black,linewidth=1pt]{-}(10,0){0.5}{0}{100}
\uput*[r](10.4,0.4){$\theta$}
% Finally the arrows and their labels
\pcline[linecolor=black,linewidth=1pt]{<->}(0,-0.5)(10,-0.5)
\lput*{:U}{$r_0$}
\pnode(9.48,2.95){Ncentar}
\cnode*(9.1,1.55){2pt}{Ntar}
\uput[ul](Ntar){$(x,y)$}
\pnode(0,0){Nrad}
\ncline[linecolor=black,linewidth=1pt]{<->}{Nrad}{Ntar}
\lput*{:U}{$r_{xy}(t)$}
\ncline[linecolor=black,linewidth=1pt]{<->}{Nrad}{Ncentar}
\lput*{:U}{$r_0(t)$}
% Now something for the radar at the left
\psarc*[fillcolor=darkgray](0,0){0.625}{128}{232}
\psellipse[fillstyle=solid,fillcolor=gray](-0.375,0)(0.15,0.5)
\psline{cc-cc}(-0.375,-0.475)(0,0)
\psline{cc-cc}(-0.5,0.25)(0,0)
\psline{cc-cc}(-0.25,0.25)(0,0)
\end{pspicture}
\end{figure}
%============================================================================

The distance between the radar and the point at $(x,y)$ on the target at
time $t$ is
\begin{equation}\label{ii eqn:str rxy(t)}
r_{xy}(t)=\sqrt{(x+vt+r_0\cos\theta)^2+(y-r_0\sin\theta)^2}
\end{equation}
Let $r_0(t)$ be the distance of the target's centre from the radar at time
$t$
\begin{equation}
r_0(t)=\sqrt{(vt+r_0\cos\theta)^2+r_0^2\sin^2\theta}
\end{equation}
so that $r_0=r_0(0)$.

%============================================================================
\begin{figure}\centering
\caption[The points at which $G(k_x,k_y)$ is known for range-Doppler ISAR
with a target moving in a straight line.]{The points at which $G(k_x,k_y)$
is known for range-Doppler ISAR with a target moving in a straight line at
constant speed lie on a small segment of a polar grid.  They lie at the
intersection of concentric circles (whose radii are the $k_n$ of the
stepped-frequency waveform) and straight lines (whose orientations are those
of the target at each time $t_m$).}
\label{ii fig:RD str grid}

\setlength{\unitlength}{0.8cm}
\psset{unit=0.8cm}
\begin{pspicture}(-6.5,-1)(6.5,6.8)
% Allow polar coordinates
\SpecialCoor	
% First the axes and their labels
\psline[linecolor=black,linewidth=1.5pt]{->}(-6,0)(6,0)
\psline[linecolor=black,linewidth=1.5pt]{->}(0,-1)(0,6)
\uput[r](6,0){$k_x$}
\uput[u](0,6){$k_y$}
% The circles
\multido{\nr=3.8+0.3}{7}{%
  \psarc[linecolor=gray,linewidth=0.75pt]{-}{\nr}{27}{76}
}
% Then the lines
\multido{\ntan=0.3+0.2}{8}{%
\psline[linecolor=gray,linewidth=0.75pt]{-}%
(!1 \ntan\space atan dup sin exch cos 3.5 mul exch 3.5 mul)%
(!1 \ntan\space atan dup sin exch cos 5.75 mul exch 5.75 mul)%
}
% Finally all the dots.
\multido{\nt=0.3+0.2}{8}{\multido{\nr=3.8+0.3}{7}{\qdisk(!1 \nt\space 
atan dup sin exch cos \nr\space mul exch \nr\space mul){2pt}}}
\NormalCoor
\end{pspicture}
\end{figure}
%============================================================================

Assume as before that the size of the target is very small in comparison to
$r_0(t)$ for $t\in[-T/2,T/2]$.  This justifies expanding $r_{xy}(t)$ in a 
first order Taylor series about the centre of the target, giving
\begin{equation}\label{ii eqn:str approx}
r_{xy}(t)\approx r_0(t)+\frac{x(r_0\cos\theta+vt)-yr_0\sin\theta}{r_0(t)}
\end{equation}
Writing
\begin{equation}
\cos\phi(t)=\frac{r_0\cos\theta+vt}{r_0(t)}
\end{equation}
and
\begin{equation}
\sin\phi(t)=-\frac{r_0\sin\theta}{r_0(t)}
\end{equation}
allows $r_{xy}(t)$ to be written as
\begin{equation}
r_{xy}(t)=r_0(t)+x\cos\phi(t)+y\sin\phi(t)
\end{equation}

Substituting this into the measurement model (\ref{ii eqn:s(k,t) II}) shows 
that
\begin{equation}
s(k,t)=e^{-jkr_0(t)}\int\!\!\int g(x,y)\,
	e^{-jk(x\cos\phi(t)+y\sin\phi(t))}\,dx\,dy
\end{equation}
is approximately $G(k_x,k_y)$, the two-dimensional Fourier transform 
of $g(x,y)$ from (\ref{ii eqn:G(kx,ky)}).  To see this more clearly, make
a slightly different transformation from the $(k,t)$ measurement space to the 
spatial frequencies $(k_x,k_y)$
\begin{eqnarray}
k_x&=&k\cos\phi(t) \nn\\
&=&k\frac{r_0\cos\theta+vt}{r_0(t)}\\
k_y&=&k\sin\phi(t) \nn\\
&=&-k\frac{r_0\sin\theta}{r_0(t)}
\end{eqnarray}

This shows that $s(k,t)$ and $G(k_x,k_y)$ are related by
\begin{equation}
s(k,t)=e^{-jkr_0(t)}\,G\left(k\frac{r_0\cos\theta+vt}{r_0(t)},
-k\frac{r_0\sin\theta}{r_0(t)}\right)
\end{equation}
Therefore the set of measurements of $s(k,t)$ gives $G(k_x,k_y)$ on a polar
grid in $(k_x,k_y)$-space that is the intersection of concentric circles of
radius $k_n$ for $n=0,1,\ldots,N-1$ with radial lines at angles of
$\phi(t_m)$, which is the target's aspect angle at time $t_m$. This is
illustrated in figure \ref{ii fig:RD str grid}, where the cotangents of the
angles of the radial lines are evenly spaced.

So except for differences in the polar reformatting and a phase function 
$e^{-jkr_0(t)}$ which cannot be neglected because it is not a linear 
function of time, the inversion algorithm for targets moving in a straight
line is identical to the inversion in algorithm \ref{ii alg:RD rotating}.

%============================================================================
\begin{algorithm}[Range-Doppler ISAR---straight line motion]
\label{ii alg:RD straight line}\mbox{}\par

To obtain the spatial radar reflectivity $g(x,y)$ of a target with constant
velocity from uniformly spaced measurements of the target's frequency
response $s(k,t)$ over a period of time:
\begin{enumerate}
\item Use the radar data or {\em a priori\/} information to estimate the 
target's speed $v$ and heading $\theta$.

\item Associate the measurements $s(k,t)$ with samples of the target's
reflectivity $G(k_x,k_y)$ using
\begin{equation}
G\left(k\frac{r_0\cos\theta+vt}{r_0(t)},-k\frac{r_0\sin\theta}{r_0(t)}\right)
=e^{jkr_0(t)}\,s(k,t)
\end{equation}
where
\begin{equation}
r_0(t)=\sqrt{(vt+r_0\cos\theta)^2+r_0^2\sin^2\theta}
\end{equation}

\item Resample the $G\left(k\frac{r_0\cos\theta+vt}{r_0(t)},
-k\frac{r_0\sin\theta}{r_0(t)}\right)$ from the grid of
figure~\ref{ii fig:RD str grid} to give $G(k_x,k_y)$ uniformly 
sampled on a rectangular grid.

\item Take the two-dimensional inverse Fourier transform of the
rectangularly-sampled $G(k_x,k_y)$ to give an estimate $\widehat{g}(x,y)$
of the target's spatial reflectivity $g(x,y)$
\begin{equation}
\widehat{g}(x,y)={\cal F}^{-1}\left\{G(k_x,k_y)\right\}
\end{equation}

\item Display $\log\left|\widehat{g}(x,y)\right|$ as the target's ISAR image.

\end{enumerate}
\end{algorithm}
%============================================================================

%%%%%%%%%%%%%%%%%%%%%%%%%%%%%%%%%%%%%%%%%%%%%%%%%%%%%%%%%%%%%%%%%%%%%%%%%%%%%
\subsection{Resolution of Range-Doppler Processing}

The resolution of the resulting ISAR image using range-Doppler processing
depends on the spacing of the known values of $G(k_x,k_y)$ after
polar reformatting so exact expressions for the range and cross-range resolution
can only be calculated once the resampling scheme has been specified.

Nevertheless, since the stepped-frequency waveform has a narrow relative
bandwidth and the change in the target's aspect angle is slight, the grids
in $(k_x,k_y)$-space are nearly rectangular and simple approximate range and
cross-range resolutions can be derived.

The resolution in range and cross-range, $\Delta r_r$ and $\Delta r_c$
respectively, are related to the radial and transverse extents, $\Delta k_r$
and $\Delta k_t$ respectively, of the polar grid in $(k_x,k_y)$ space 
according to the formula for the resolution of a Fourier transform
\begin{eqnarray}
\Delta r_r&=&\frac{2\pi}{\Delta k_r}\\
\Delta r_c&=&\frac{2\pi}{\Delta k_t}
\end{eqnarray}

For both types of target motions considered, the radial extent of the
$(k,t)$ polar grid is from $k_0$ to $k_{N-1}$, which suggests $\Delta
k_r=(N-1)\Delta k$.  However, the Fourier transform is discrete (implemented
as an FFT for efficiency) so values of $k$ should be taken as extending from
$k_0-\Delta k/2$ to $k_{N-1}+\Delta k/2$, a total spread of $N\Delta k$. 
This gives a range resolution of 
\begin{equation}\label{ii eqn:rd drr}
\Delta r_r=\frac{2\pi}{N\Delta k}=\frac{c}{2N\Delta f}
\end{equation}

The transverse extent of the grid is at its narrowest when $k=k_0$.  For the
uniformly rotating target imaged over the period $[-T/2,T/2]$, the grid
covers angles from $-\omega T/2$ to $\omega T/2$.  At a radial distance
$k_0$, this is a transverse extent of
\begin{equation}
\Delta k_t=2k_0\sin(\omega T/2)\approx k_0\omega T
\end{equation}
giving an approximate cross-range resolution for a uniformly rotating
target of
\begin{equation}\label{ii eqn:drc rot}
\Delta r_c=\frac{2\pi}{\Delta k_t}\approx\frac{c}{2f_0\omega T}
\end{equation}

For the target moving in a straight line at speed $v$ during the imaging
period $[-T/2,T/2]$, the grid covered angles whose tangents varied from 
\begin{equation}
\tan\phi(-T/2)=-\frac{r_0\sin\theta}{r_0\cos\theta-\frac{1}{2}vT}
\end{equation}
to
\begin{equation}
\tan\phi(T/2)=\frac{r_0\sin\theta}{r_0\cos\theta+\frac{1}{2}vT}
\end{equation}
so the total angular spread is 
\begin{eqnarray}
\Delta\phi&=&\phi(T/2)-\phi(-T/2)\nn\\
&=&\arctan\left(\frac{r_0\sin\theta}{r_0\cos\theta+\frac{1}{2}vT}\right)
+\arctan\left(\frac{r_0\sin\theta}{r_0\cos\theta-\frac{1}{2}vT}\right)
\end{eqnarray}
Using the identity
\begin{equation}
\tan(x+y)=\frac{\tan x + \tan y}{1-\tan x\,\tan y}
\end{equation}
and some algebra shows that
\begin{equation}
\Delta\phi=\arctan\left(\frac{r_0vT\sin\theta}{r_0^2-\frac{1}{4}v^2T^2}\right)
\end{equation}
If $r_0\gg vT$, this is approximately
\begin{equation}
\Delta\phi\approx\arctan\left(\frac{vT\sin\theta}{r_0}\right)
\approx\frac{vT\sin\theta}{r_0}
\end{equation}
from which the approximate cross-range resolution for a target moving at an
angle $\theta $ to the boresight axis is
\begin{equation}
\Delta r_c=\frac{2\pi}{\Delta k_t}\approx\frac{c}{2f_0\frac{vT}{r_0}\sin\theta}
\end{equation}


%%%%%%%%%%%%%%%%%%%%%%%%%%%%%%%%%%%%%%%%%%%%%%%%%%%%%%%%%%%%%%%%%%%%%%%%%%%%%
\section{Wavenumber Processing}

A relatively recent inversion for SAR and ISAR imaging writes the spherical
wave $e^{-jkr_{xy}}/r_{xy}$ as the superposition of plane waves
\begin{equation}\label{ii eqn:pw decomp}
\frac{\ds e^{-jkr_{xy}}}{r_{xy}}\approx
-j\sqrt{\frac{k}{2\pi r_{xy}}}\,
\int_{-k}^k \frac{1}{\sqrt{k^2-k_x^2}}e^{-j(xk_x+y\sqrt{k^2-k_x^2})}\,dk_x
\end{equation}
where $r_{xy}=\sqrt{x^2+y^2}$ is the distance from the source to the radar. 
This expression, derived in appendix \ref{ii app:pw}, assumes that the
target is more than ten wavelengths from the radar in the $y$ direction. 
Radar targets are usually tens of thousands of wavelengths away from the
antenna so this condition is easily met if the $y$ axis is not perpendicular
to the radar's boresight axis.

For many years, theoretical physicists have used an eigendecomposition of a
spherical wave into plane wave eigenfunctions to solve scattering
and diffraction problems \cite[sec. 9.3]{Mor53b}.  In 1978, Stolt analysed
seismic imaging using range migration%%%%%%%%%%%%%%%%%%%%%%%%%%%%%%%%%%%%%%
\footnote{\label{ii ftn:er}Range migration in seismic imaging causes
reflections from a subsurface layer at an angle to the horizontal to appear
in an incorrect position.  Seismic imaging may be considered equivalent to
firing sources located at the subsurface layer and measuring the arrival
times of the wavefronts at the surface.  Propagating the wavefield measured
at the surface back in time using the wave equation gives the location of
the subsurface layer.

This model of imaginary sources at a subsurface layer is called the 
``exploding reflector'' model \cite[ch. 2]{Hay85b}.  Notice how similar this
is to radar imaging, where the target is modelled as radiating reflectors,
as mentioned in the footnote on page~\protect\pageref{ii ftn:rr}.}%%%%%%%%%%
%%%%%
and found an equivalent expression to (\ref{ii eqn:pw decomp}) \cite{Sto78}.
From his work, changing coordinates from $(k_x,k_y)$ to 
$(k_x,\sqrt{k^2-k_x^2})$ has come to be called the Stolt transformation. 
Similar inversions have been developed for tomography with diffracting
sources, such as medical imaging using ultrasound \cite[ch. 6]{Hay85b}.  

In the last few years, radar imaging has been reformulated using this 
decomposition \cite{Caf91} instead of range-Doppler processing.  A
comparison of SAR using range-Doppler processing and this new method, called
wavenumber processing or $\omega-k$ processing, is given in \cite{Bam92}.
The description of $\omega-k$ processing for ISAR presented here closely
follows that of Soumekh for SAR and ISAR imaging \cite{Sou94}.


%%%%%%%%%%%%%%%%%%%%%%%%%%%%%%%%%%%%%%%%%%%%%%%%%%%%%%%%%%%%%%%%%%%%%%%%%%%%%
\subsection{Imaging Geometry}


Soumekh's work is primarily concerned with SAR imaging, so his description
of ISAR is in the context of imaging a slowly moving target against the
stationary background of a SAR image.  The development of his imaging model
and inversion in \cite{Yan93} and \cite{Sou94} is presented here in a form
suitable for ISAR-only imaging.  Consequently, the radar is assumed to be
stationary.


The geometry of the radar and target is the same as that in figure
\ref{ii fig:moving target} on page \pageref{ii fig:moving target} used for 
range-Doppler imaging of a target moving in a straight line at constant
speed.  From equation (\ref{ii eqn:str rxy(t)}), the distance at time $t$
from the point  $(x,y)$ on the target to the radar is
\begin{equation}
r_{xy}(t)=\sqrt{(x+vt+r_0\cos\theta)^2+(y-r_0\sin\theta)^2}
\end{equation}
Make the substitutions
\begin{eqnarray}
x_0=-r_0\cos\theta\\
y_0=r_0\sin\theta
\end{eqnarray}
so that
\begin{equation}
r_{xy}(t)=\sqrt{(x+vt-x_0)^2+(y-y_0)^2}
\end{equation}
Then the measurement of the target's reflectivity made 
at time $t$ and at frequency $f$ is given by equation (\ref{ii eqn:s(k,t) I})
\begin{eqnarray}
s(k,t)
&=&\int\!\!\int g(x,y) \frac{\ds e^{-jkr_{xy}(t)}}{\ds r_{xy}^2(t)}\,dx\,dy\nn\\
&\approx&\frac{1}{r_0}\int\!\!\int g(x,y)
\frac{\ds e^{-jkr_{xy}(t)}}{\ds r_{xy}(t)}\,dx\,dy
\label{ii eqn:Soumekh s(k,t) a}
\end{eqnarray}
where the limits of integration include the whole of the target.

%%%%%%%%%%%%%%%%%%%%%%%%%%%%%%%%%%%%%%%%%%%%%%%%%%%%%%%%%%%%%%%%%%%%%%%%%%%%%
\subsection{$\omega-k$ ISAR Inversion}

The spherical wave in the integrand of (\ref{ii eqn:Soumekh s(k,t) a})
may be written using the decomposition in (\ref{ii eqn:pw decomp}).  
This gives
\begin{eqnarray}
s(k,t)&\approx&-j\sqrt{\frac{k}{2\pi r_0^3}}\int\!\!\int g(x,y)
\int_{-k}^k \frac{1}{\sqrt{k^2-k_x^2}}				\nn\\
&&{}\times e^{-j((x_0-x-vt)k_x+(y_0-y)\sqrt{k^2-k_x^2})}\,dk_x\,dx\,dy
\end{eqnarray}
This approximation is only valid if $y_0-y\gg 0$ for all scatterers on the
target.  This implies that $r_0$ must be significantly larger than the size 
of the target and that $\theta$, the angle the target's path makes with the
boresight axis, must not be close to $0^{\circ}$ or $180^{\circ}$. 
Equivalently, the target must be far from the radar and have a velocity
vector with a non-zero cross-range component.  Note that these conditions 
are less restrictive than in range-Doppler ISAR which assume that the 
spherical wavefront at the target is flat.

Absorb the constants into $g(x,y)$ and write the ``$\approx$'' as 
equality.\footnote{In the preface to \protect\cite{Sou94}, Soumekh states
that his method is ``an approximation-free inversion''.  Yang and Soumekh,
in \protect\cite{Yan93} describe it as a ``SAR inversion without any need
for approximations.''  While these are slight exaggerations, it is true that
ISAR inversions based on $\omega-k$ processing are substantially better 
than range-Doppler processing.}  This shows that
\begin{eqnarray}
s(k,t)&=&\int_{-k}^k \frac{1}{\sqrt{k^2-k_x^2}}
\,e^{-j(x_0k_x+y_0\sqrt{k^2-k_x^2})}	\nn\\
&&{}\times G(-k_x,-\sqrt{k^2-k_x^2})\,e^{jvtk_x}\,dk_x
\label{ii eqn:s(k,t)}
\end{eqnarray}
where $G(k_x,k_y)$ is the two-dimensional Fourier transform of $g(x,y)$
\begin{equation}
G(k_x,k_y)=\int\!\!\int g(x,y) \,e^{-j(k_xx+k_yy)}\,dx\,dy
\end{equation}
The $e^{jvtk_x}$ term in the integrand indicates that (\ref{ii eqn:s(k,t)})
is an inverse Fourier transform.  Taking the Fourier transform of 
(\ref{ii eqn:s(k,t)}) with respect to $t$ gives
\begin{eqnarray}
s(k,\omega)
&=&\int s(k,t) e^{-j\omega t}\,dt		\nn\\
&=&
\frac{e^{-j(x_0\omega/v+y_0\sqrt{k^2-\omega^2/v^2})}}
{\sqrt{k^2-\omega^2/v^2}} 
\,G(-\omega/v,-\sqrt{k^2-\omega^2/v^2})
\end{eqnarray}
The $e^{-j(x_0\omega/v+y_0\sqrt{k^2-\omega^2/v^2})}$ term just shifts the
target to the centre of the ISAR image.  Neglecting that,
\begin{equation}\label{ii eqn:Soumekh inversion}
G(-\omega/v,-\sqrt{k^2-\omega^2/v^2})=\sqrt{k^2-\omega^2/v^2}\,s(k,\omega)
\end{equation}

%============================================================================
\begin{figure}\centering
\caption[The points at which $G(k_x,k_y)$ is known for $\omega-k$ ISAR with
a target moving in a straight line.]{The points at which $G(k_x,k_y)$ is
known for $\omega-k$ ISAR with a target moving in a straight line at
constant speed lie at the
intersection of concentric circles (whose radii are the $k_n$ of the
stepped-frequency waveform) and parallel straight lines.
Compare this with the range-Doppler grids in figures
\protect\ref{ii fig:RD rot grid} and \protect\ref{ii fig:RD str grid} 
where the straight lines are radial.}
\label{ii fig:WK grid}

\psset{unit=0.8cm}
\setlength{\unitlength}{0.8cm}
\begin{pspicture}(-6.5,-6.5)(6.5,2.4)
% Allow polar coordinates
\SpecialCoor	
% First the axes and their labels
\psline[linecolor=black,linewidth=1.5pt]{->}(-6,0)(6,0)
\psline[linecolor=black,linewidth=1.5pt]{->}(0,-6)(0,1.5)
\uput[r](6,0){$k_x$}
\uput[u](0,1.5){$k_y$}
% First the parallel lines
\multido{\nn=-1.05+0.30}{8}{%
	\psline[linecolor=gray,linewidth=0.6pt]{-}(\nn,-3.65)(\nn,-5.75)
}

% Now the circles
\psclip{\psline[linestyle=none](-1.3,-3.65)(1.3,-3.65)(1.3,-5.75)(-1.3,-5.75)}
\multido{\nr=3.8+0.3}{7}{%
	\psarc[linecolor=gray,linewidth=0.6pt]{-}{\nr}{210}{330}
}
\endpsclip
% Finally all the dots.  The \nr0 is a hack so the \nr won't gobble up the
% following space, giving 4.00 when \nr=4.0
\SpecialCoor
\multido{\nn=-1.05+0.30}{8}{%
  \multido{\nr=3.8+0.3}{7}{%
    \qdisk(!\nn0 \nr0 dup mul \nn0 dup mul sub sqrt neg){2pt}
  }
}
\NormalCoor
\end{pspicture}
\end{figure}
%============================================================================

This gives the following algorithm for ISAR imaging using $\omega-k$
processing:

%============================================================================
\begin{algorithm}[$\omega-k$ ISAR Inversion]
\label{ii alg:w-k}\mbox{}\par

To obtain the spatial radar reflectivity $g(x,y)$ of a target with constant
velocity from uniformly spaced measurements of the target's frequency
response $s(k,t)$ over a period of time:
\begin{enumerate}
\item Use the radar data or {\em a priori\/} information to estimate the 
target's speed $v$.

\item Take the Fourier transform (in the form of an FFT) of the $s(k,t)$ 
with respect to $t$ to give $s(k,\omega)$.

\item Multiply $s(k,\omega)$ by $\sqrt{k^2-\omega^2/v^2}$ to give $G(k_x,k_y)$
sampled as shown in figure~\ref{ii fig:WK grid}.
\begin{equation}
G(-\omega/v,-\sqrt{k^2-\omega^2/v^2})=\sqrt{k^2-\omega^2/v^2}\,s(k,\omega)
\end{equation}

\item Resample $G(-\omega/v,-\sqrt{k^2-\omega^2/v^2})$ in the $k_y$
direction to give $G(k_x,k_y)$ uniformly sampled on a rectangular grid.

\item Take the two-dimensional inverse Fourier transform of the
rectangularly-sampled $G(k_x,k_y)$ to give an estimate $\widehat{g}(x,y)$
of the target's spatial reflectivity $g(x,y)$
\begin{equation}
\widehat{g}(x,y)={\cal F}^{-1}\left\{G(k_x,k_y)\right\}
\end{equation}

\item Display $\log\left|\widehat{g}(x,y)\right|$ as the target's ISAR image.
\end{enumerate}
\end{algorithm}
%============================================================================

Actually the image produced by this algorithm is $g(x-x_0,y-y_0)$ because
the linearly increasing phase term in (\ref{ii eqn:Soumekh inversion}) has
been neglected.  Shifting the image by $(x_0,y_0)$ will centre it.

Modifications to the $\omega-k$ inversion have been proposed for cases when
the target's frequency response is sampled unevenly or for SAR imaging where
the target terrain is three-dimensional \cite{Fly92}.  Soumekh and Choi
describe estimating and correcting noisy SAR data with multiplicative noise
affecting phase only, or both phase and magnitude \cite{Sou92b}.  Soumekh
also presents inversions using his method that are suitable for tomographic
imaging and geophysical imaging \cite{Sou94}.

%%%%%%%%%%%%%%%%%%%%%%%%%%%%%%%%%%%%%%%%%%%%%%%%%%%%%%%%%%%%%%%%%%%%%%%%%%%%%
\section{Other Approaches to ISAR Imaging}

This section is a very brief discussion of some other approaches to ISAR
imaging and ISAR inversion algorithms.  These include comments that have
appeared in the literature about eliminating polar reformatting, ISAR
inversion using tomography, and achieving a higher resolution from less
data with superresolution spectral estimation methods.

%%%%%%%%%%%%%%%%%%%%%%%%%%%%%%%%%%%%%%%%%%%%%%%%%%%%%%%%%%%%%%%%%%%%%%%%%%%%%
\subsection{Polar Reformatting During Sampling}

ISAR inversion algorithms use Fourier transforms to convert the target's
frequency response $G(k_x,k_y)$ to an estimate of its spatial radar
reflectivity $g(x,y)$.  Polar reformatting is needed because sampling
$s(k,t)$ on a rectangular grid in $(k,t)$-space gives $G(k_x,k_y)$ at points
that lie on a polar grid in $(k_x,k_y)$-space while FFTs need samples on a
rectangular grid, not a polar grid.

One way to avoid polar reformatting altogether has been suggested by Harris
\cite{Har90a}.  If the target's motion is known {\em a priori\/}, it is in
principle possible to sample  $s(k,t)$ non-uniformly in such a way that the
equivalent samples of $G(k_x,k_y)$ lie on uniformly spaced rectangular grid
so that an FFT can be applied directly to the measurements without the need
for polar reformatting.

A slightly different sampling scheme for SAR has been suggested by Lawton in
\cite{Law88}.  This gives $G(k_x,k_y)$ on a grid in $(k_x,k_y)$-space that
is evenly spaced in the $k_x$ direction and has a linearly increasing
spacing in the $k_y$ direction, a bit like a chirp.  The inversion can be
done without reformatting using the Bluestein chirp form of the FFT
\cite{Blu70}.

The success of methods such as these depends on how accurately the position
and orientation of the target can be predicted at each time and frequency that
$s(k,t)$ is sampled.  Any residual errors in the target's predicted location
and orientation will still require polar resampling to correct.  No ISAR
images of real targets using these methods have been published, so it is
difficult to judge whether they are feasible or just theoretical
curiosities.

%%%%%%%%%%%%%%%%%%%%%%%%%%%%%%%%%%%%%%%%%%%%%%%%%%%%%%%%%%%%%%%%%%%%%%%%%%%%%
\subsection{Tomographic ISAR}

Similarities in the geometry of ISAR imaging, where a target uniformly 
rotates, and X-ray tomography, where a detector rotates around a patient,
suggest that tomographic inversions may be applicable to radar imaging.
This has been examined by Bernfeld \cite{Ber84}, Snyder {\em et al.}
\cite{Sny86}, Munson {\em et al.\/} \cite{Mun83} and Mensa {\em et al.\/}
\cite{Men83,Men91} among others.  

One important difference between tomography and ISAR is that tomography
requires projections of a full $180^o$ view of the target.  
This is difficult for ISAR because the orientation-independence of a radar
target's reflectivity function $g(x,y)$ is based on the assumption that the
target's aspect angle changes by at most only a few degrees.

Experiments on ISAR imaging have been conducted by Gerlach using tomographic
methods for targets viewed over very wide aspect angles \cite{Ger89}. Using
filtered backprojection to image a model target on a turntable, he found
that individual scatterers on the target generally only persisted for aspect
angle changes considerably less than $60^o$.  This degraded the point spread
function of each scatterer, causing streaks in the ISAR image.  The
orientation of each scatterer's streak depended on the range of aspect
angles for which that scatterer was a strong reflector.

%%%%%%%%%%%%%%%%%%%%%%%%%%%%%%%%%%%%%%%%%%%%%%%%%%%%%%%%%%%%%%%%%%%%%%%%%%%%%
\subsection{Superresolution Techniques}
\label{ii sec:sr}

ISAR imaging requires a two-dimensional fast Fourier transform to convert the measurements
$G(k_x,k_y)$ of the target taken in the spatial frequency domain to the
radar's reflectivity $g(x,y)$ in the spatial domain.  This is essentially a
spectral estimation problem, and the resolution	of the final ISAR image is
limited to the inherent resolution of the Fourier transform.

Since modern spectral estimation methods can achieve a higher resolution
than classical Fourier-based spectral estimation, they have been applied to
ISAR imaging, where collectively they are known as superresolution
algorithms.  Some examples of superresolution imaging are papers by
Gupta \cite{Gup94} and Nuthalapati \cite{Nut92} using two-dimensional linear
prediction, Odendaal {\em et al.\/} using the MUSIC algorithm \cite{Ode94},
Farina {\em et al.\/} using autoregressive and minimum variance methods
\cite{Far94}, Hua {\em et al.\/} using the matrix-pencil algorithm
\cite{Hua93,Hua94}, and Haywood and Evans using system identification
\cite{Hay92b,Hay92c}.

The supposed advantage of superresolution estimators is that they can
achieve a much higher resolution than conventional Fourier processing, or
achieve the same resolution with much less data.  

While the superresolution estimators mentioned above may achieve a higher
resolution than Fourier processing with less data, they only work if the
data fits their model, which is the superposition of a small number of
two-dimensional sinusoids.  This assumes that the target is composed of a
small number of ideal point scatterers (as in equation 
(\ref{hrr eqn:pt sc pr(t)})), and that polar reformatting has been
done exactly. 

The superresolution approach was criticized by Rihaczek in 1981, who pointed out
the problems caused by radar targets having many scattering centres which
are subject to shadowing and masking, and which interfere with one another 
\cite{Rih81a}.  In his reply to a comment by Jackson \cite{Jac81}, Rihaczek 
said \cite{Rih81b} 
\begin{quote}\singlespaced
``If the practical radar task were to generate an image of an array of ideal
point scatterers, I might have a different opinion. \ldots The fact is that
the superresolution methods did not fail because of any lack of signal
processing techniques; they failed because they were based on models of the
target which were simply inadequate for real targets. \ldots What we really
need is improved target models.''  
\end{quote}

This criticism was made in 1981, and the superresolution papers cited above
date from 1992 to 1994.  They are still using the same simple point
scatterer model that Rihaczek  called ``inadequate'', and the results of an
extra eleven to thirteen years' research are perhaps best summed up by one
sentence from the conclusion of \cite{Ode94}: ``At this stage, it is not
possible to claim a certain degree of improvement in resolution using the
MUSIC algorithm, as opposed to the Fourier transform.''  This also applies
to other modern spectral estimators, so the current verdict on 
superresolution ISAR must be that it is not yet demonstrably superior to
ordinary Fourier processing for real targets under uncontrolled conditions.

%%%%%%%%%%%%%%%%%%%%%%%%%%%%%%%%%%%%%%%%%%%%%%%%%%%%%%%%%%%%%%%%%%%%%%%%%%%%%
\appendix{Decomposition of a Spherical Wave into Plane Waves}
\label{ii app:pw}

This appendix contains a derivation of the eigendecomposition of a spherical
wave into plane waves.  This result is well-known in theoretical physics, 
but it has only recently been used among the signal processing community.  

This forms the basis for SAR and ISAR inversions using wavenumber
processing.  Soumekh's book \cite{Sou94} contains a derivation of this
result, but it mixes up the physics and engineering conventions for
diverging waves\footnote{Physicists use $e^{ikx}$ to represent a wave moving
in the $+x$ direction whereas engineers use the conjugate, $e^{-jkx}$.  In
equation (\protect\ref{ii eqn:2D gf}), the solution corresponding to the
diverging wave would be $i\pi H_0^{(1)}(kr)$ to a physicist, but its 
conjugate $-j\pi H^{(2)}_0(kr)$ to an engineer.  As well using $i$ rather
than $j$ for $\sqrt{-1}$, physicists define Fourier transforms and inverse 
transforms the oppposite way from engineers.} and neglects many constants
and slowly-varying magnitude functions, making it hard to rederive.  To
complicate matters, some of his earlier papers on which the book is based,
such as \cite{Sou91} and \cite{Sou92a}, use slightly different derivations
which are not easy to follow because of typographical errors and because 
contour integrals giving integral representations of functions do not have
their contours specified.

The decomposition presented here is a rederivation of that in \cite{Sou94}
with the approximations clearly stated and all constants and slowly-varying
magnitude functions included.

Consider a source at the origin emitting an electromagnetic wave, 
$\psi(t,\vect{r})$.  This wave diverges from the origin according to the wave
equation
\begin{equation}
\nabla^2\psi(t,\vect{r})-\frac{1}{c^2}\psi(t,\vect{r})=0
\end{equation}
where $c$ is the speed of light.  If the wave is harmonic with frequency
$\omega$, $\psi(t,\vect{r})$ may be written
\begin{equation}
\psi(t,\vect{r})=e^{j\omega t}h(\vect{r})
\end{equation}
where $h(\vect{r})$ is the phasor representation of the wave's 
spatial dependence.  Substituting this separable expression for the wave
into the wave equation gives $h(\vect{r})$ as a solution of the
homogeneous Hemholtz equation
\begin{equation}
\nabla^2h(\vect{r})+k^2 h(\vect{r})=0
\end{equation}
where $k=\omega/c$ is the wavenumber.  This may be solved for the
case of a source at the origin using the Green's function $G(\vect{r})$
\cite[Ch. 7]{Mor53a}, which is a solution of the inhomogeneous Hemholtz
equation
\begin{equation}
\nabla^2G(\vect{r})+k^2G(\vect{r})=-4\pi\delta(\vect{r})
\end{equation}
The solutions of this corresponding to outward-moving waves are, in three
dimensions, a spherical wave
\begin{equation}
G_3(\vect{r})=\frac{\ds e^{-jk\left|\vect{r}\right|}}{\left|\vect{r}\right|}
\end{equation}
and in two dimensions, a Hankel function with circular symmetry
\begin{equation}\label{ii eqn:2D gf}
G_2(\vect{r})=-j\pi H_0^{(2)}(k\left|\vect{r}\right|)
\end{equation}
where the Hankel function $H_{\nu}^{(2)}(z)$ is a solution to 
\begin{equation}
z^2\frac{d^2w}{dz^2}+z\frac{dw}{dz}+(z^2-\nu^2)w=0
\end{equation}
The two real linearly independent solutions of this are the Bessel functions
of the first and second kind, $J_{\nu}(z)$ and $Y_{\nu}(z)$.  A different
set of linearly independent solutions is given by the Hankel functions,
$H_{\nu}(z)=H_{\nu}^{(1)}(z)$ and $H_{\nu}^{(2)}(z)$, which are related to
$J_{\nu}(z)$ and $Y_{\nu}(z)$ by\footnote{This is analogous to the way
$e^{jx}$ and $e^{-jx}$ are related to $\sin x$ and $\cos x$.} \cite{Abr65}
\begin{eqnarray}
H_{\nu}^{(1)}(z)&=&J_{\nu}(z)+jY_{\nu}(z)\\
H_{\nu}^{(2)}(z)&=&J_{\nu}(z)-jY_{\nu}(z)
\end{eqnarray}

The two-dimensional and three-dimensional Green's functions have
asymptotically equal phases but different magnitudes.  From equation (9.2.4)
of \cite{Abr65} 
\begin{equation}
H_0^{(2)}(z)\sim j\sqrt{\frac{2}{\pi z}}\,e^{-jz}
\qquad\hbox{as $\left|z\right|\to\infty$}
\end{equation}
Using this, the two- and three-dimensional Green's functions are
asymptotically related by
\begin{equation}\label{ii eqn:G2 and G3}
G_3(\left|\vect{r}\right|)\sim \sqrt{\frac{k}{2\pi\left|\vect{r}\right|}}
\,G_2(\left|\vect{r}\right|)
\end{equation}
as $k\left|\vect{r}\right|\to\infty$.  

The rationale behind writing the spherical wave as an asymptotic Hankel
function is that it is relatively straightforward to derive an integral
representation for the Hankel function that is valid in a two-dimensional
plane.  This integral representation can be used to describe the spherical
waves reflected from a radar target because ISAR imaging considers targets
to be two-dimensional, lying in a plane parallel to the radar's line-of-sight.

The Fourier transform of the two-dimensional inhomogeneous Hemholtz equation
is
\begin{equation}\label{ii eqn:G(k)}
-(k_x^2+k_y^2)G_2(\vect{k})+ k^2G_2(\vect{k})=-4\pi
\end{equation}
where $\vect{k}=(k_x,k_y)$ are the spatial frequencies that are the
Fourier-domain counterparts of $\vect{r}=(x,y)$, and $G_2(\vect{r})$ is the
inverse Fourier transform of $G(\vect{k})$
\begin{equation}
G_2(\vect{r})=\frac{1}{4\pi^2}\infint\infint G_2(\vect{k})\,
e^{j(xk_x +yk_y)}\,dk_x\,dk_y
\end{equation}
Since (\ref{ii eqn:G(k)}) shows that 
\begin{equation}
G_2(\vect{k})=\frac{4\pi}{k_x^2+k_y^2-k^2}
\end{equation}
the two-dimensional Green's function has the following integral 
representation (this is similar to (7.2.42) of \cite{Mor53a})
\begin{equation}
G_2(\left|\vect{r}\right|)
=\frac{1}{\pi}\infint\infint \frac{1}{k_x^2+k_y^2-k^2}\,
e^{j(xk_x+yk_y)}\,dk_x\,dk_y
\end{equation}
where $\vect{r}=(x,y)$.

The double integral may be simplified by evaluating one of the integrals 
explicitly using contour integration.  Then the remaining single integral
gives the desired decomposition of a spherical wave into plane waves.

The choice of which integral to evaluate depends on where in the $(x,y)$
plane the spherical decomposition is to be used.  If the spherical wave is
measured at a point where $\left|x\right|\gg 0$, the contour integration may
be performed with respect to $k_x$.  If the spherical wave is measured at a
point where $\left|y\right|\gg 0$, the contour integration may be performed
with respect to $k_y$.  Since the ISAR imaging geometry in figure~\ref{ii
fig:moving target} assumes that the radar is far from the target in the
positive $y$ direction, the contour integral will be performed here with
respect to $k_y$.

The inner integral has poles at $k_y=\pm\sqrt{k^2-k_x^2}$.  Integrating over
a contour including only the singularity at $k_y=+\sqrt{k^2-k_x^2}$ gives a
wave that is converging when $y>0$ and diverging when $y<0$. Integrating
over a contour including only the other singularity at
$k_y=-\sqrt{k^2-k_x^2}$ gives a wave that is converging when $y<0$ and
diverging when $y>0$.  Since the desired solution is wave that diverges when
measured in the upper half-plane, only the pole at $k_y=-\sqrt{k^2-k_x^2}$
should be included in the contour integration.  This contour is illustrated
in figure~\ref{ii fig:contour}.  Using the residue theorem,
\begin{eqnarray}
\infint\frac{1}{k_y^2-(k^2-k_x^2)}\,e^{jyk_y}\,dk_y
&=&2\pi j \left. \frac{\ds e^{jyk_y}}{k_y-\sqrt{k^2-k_x^2}}
	\right|_{k_y=-\sqrt{k^2-k_x^2}}\nn\\
&=&-j\frac{\ds \pi e^{-jy\sqrt{k^2-k_x^2}}}{\sqrt{k^2-k_x^2}}
\label{ii eqn:contour int}
\end{eqnarray}
Substituting this for the inner integral of the Green's function's integral
representation shows that
\begin{equation}
G_2(\left|\vect{r}\right|)
=-j\infint \frac{1}{\sqrt{k^2-k_x^2}}\,e^{j(xk_x-y\sqrt{k^2-k_x^2})}\,dk_x
\end{equation}
From the relationship in (\ref{ii eqn:G2 and G3}), this is also an
asymptotic integral representation for $G_3(\left|r\right|)$  
\begin{equation}
G_3(\left|\vect{r}\right|)\sim -j\sqrt{\frac{k}{2\pi\left|\vect{r}\right|}}
\infint \frac{1}{\sqrt{k^2-k_x^2}}\,e^{j(xk_x-y\sqrt{k^2-k_x^2})}\,dk_x
\end{equation}

Since $\sqrt{k^2-k_x^2}$ is imaginary for $\left|k_x\right|>k$, it makes
sense to split the integral representation for $G_3(\left|r\right|)$ into an
integral for $\left|k_x\right|<k$ and an integral for $\left|k_x\right|>k$.

Let $I_1(\left|\vect{r}\right|)$ be the integral for $\left|k_x\right|<k$
and $I_2(\left|\vect{r}\right|)$ be the integral for $\left|k_x\right|>k$.
Make the substitution $u=\sqrt{k_x^2-k^2}$ to show that
\begin{eqnarray}
\left|I_2(\left|\vect{r}\right|)\right|
&=&\sqrt{\frac{2k}{\pi\left|\vect{r}\right|}} \left| \int_k^{\infty} 
\frac{1}{\sqrt{k_x^2-k^2}}
\,\cos(xk_x)\,e^{-y\sqrt{k_x^2-k^2}}\,dk_x\right|\nn\\
&\leq&\sqrt{\frac{2}{\pi k\left|\vect{r}\right|}}\int_0^{\infty} 
e^{-yu}\,du\nn\\
&=&\frac{1}{y}\sqrt{\frac{2}{\pi k\left|\vect{r}\right|}}
\end{eqnarray}
 
This suggests that $\left|I_2(\left|\vect{r}\right|)\right|$ falls off
faster than $1/y^{3/2}$ whereas $\left|G_3(\left|r\right|)\right|$ falls off 
as $1/y$.  So for measurements made far from the origin in the $y$ direction,
$I_2(\left|\vect{r}\right|)$ is negligible in comparison to 
$I_1(\left|\vect{r}\right|)$.  Kak says that $I_2(\left|\vect{r}\right|)$ is
an evanescent component which is usually of no signficance more than ten 
wavelengths from the source \cite[p. 367]{Hay85b}.  Since radar targets are 
often thousands of times further from the antenna than this, this justifies
tightening the limits of integration from $(-\infty,\infty)$ to $(-k,k)$.

Therefore the decomposition of a spherical wave into plane waves 
useful for ISAR imaging is
\begin{equation}
\frac{\ds e^{-jk\left|\vect{r}\right|}}{\left|\vect{r}\right|}=
G_3(\left|\vect{r}\right|)\sim -j\sqrt{\frac{k}{2\pi\left|\vect{r}\right|}}
\int_{-k}^k \frac{1}{\sqrt{k^2-k_x^2}}\,e^{j(xk_x-y\sqrt{k^2-k_x^2})}\,dk_x
\end{equation}
which is valid if the radar's $y$ coordinate is much greater than the
target's greatest $y$ coordinate.  This can be made slightly neater by
changing the sign of $k_x$ giving the final form of the plane wave
decomposition
\begin{equation}
\frac{\ds e^{-jk\left|\vect{r}\right|}}{\left|\vect{r}\right|}
\sim -j\sqrt{\frac{k}{2\pi\left|\vect{r}\right|}}
\int_{-k}^k \frac{1}{\sqrt{k^2-k_x^2}}\,e^{-j(xk_x+y\sqrt{k^2-k_x^2})}\,dk_x
\end{equation}

%============================================================================
\begin{figure}\centering
\caption[The contour in complex $k_y$-space corresponding to an outwardly 
propagating wave.]{This indicates the contour in complex $k_y$-space the
integral in equation (\protect\ref{ii eqn:contour int}) must take to give
the solution in the half-plane $y>0$ corresponding to an outwardly
propagating wave.}
\label{ii fig:contour}

\setlength{\unitlength}{0.6cm}
\begin{pspicture}(-4,-3.5)(4,4)
% First the axes and their labels
\psline[linecolor=black,linewidth=1.5pt]{->}(-3.3,0)(3.3,0)
\psline[linecolor=black,linewidth=1.5pt]{->}(0,-3.3)(0,3.3)
\uput[r](3.3,0){$\real{k_y}$}
\uput[u](0,3.3){$\imag{k_y}$}
\uput[u](-1.2,0){$-\sqrt{k^2-k_x^2}$}
\uput[d](1,0){$\sqrt{k^2-k_x^2}$}
\qdisk(1,0){2pt}
\qdisk(-1,0){2pt}
% Now the contour with arrows from right to left
\psline[linecolor=darkgray,linewidth=1pt]{->}(-3,0)(-2,0)
\psline[linecolor=darkgray,linewidth=1pt]{-}(-2,0)(-1.3,0)
\psline[linecolor=darkgray,linewidth=1pt]{-}(-0.7,0)(0.7,0)
\psline[linecolor=darkgray,linewidth=1pt]{-}(1.3,0)(2,0)
\psline[linecolor=darkgray,linewidth=1pt]{>-}(2,0)(3,0)
\psarc[linecolor=darkgray,linewidth=1pt](1,0){0.3}{0}{180}
\psarc[linecolor=darkgray,linewidth=1pt](-1,0){0.3}{180}{360}
\end{pspicture}
\end{figure}
%============================================================================

A slightly different derivation\footnote{The decomposition in
\protect\cite{Sou91} is hard to follow because the slowly changing amplitude
functions and constants are omitted throughout, factors of $\frac{1}{2}$ are
written as $.5$ to give terms like $j.5\phi$, and all paths in the contour
integrals have been left out.  Furthermore, in his 1992 paper
\protect\cite{Sou92a}, this derivation is repeated with a typographical
error which renders $jkr\cos\phi+j\frac{1}{2}\phi$ as
$jkr\cdot\cos\phi+j\cdot 5\phi$, where the first ``$\cdot$'' is a
multiplication sign and the second ``$\cdot$'' is the decimal point in $0.5$
with the zero omitted.  

To make matters worse, the source for equation (2) of \protect\cite{Sou92a}
is quoted as equation (5.3.89) of the 1968 edition of Morse and Feshbach
\protect\cite{Mor53a,Mor53b}.  However, there is no 1968 edition of Morse
and Feshbach, only the original 1953 edition. Equation (5.3.89) in the 1953
edition is an integral equation of the Mathieu function $Se_{2m}(z,q)$, and
has no relevance for Hankel functions.  Perhaps the (5.3.89) should read
(5.3.69), an equation which is relevant.  If so, the 1968 is a mistake and
should be corrected in \protect\cite{Sou91} and \protect\cite{Sou92b} as
well as \protect\cite{Sou92a}.  Alternatively, if the 1968 is correct, the
references may mean Morse and Ingard's {\em Theoretical Acoustics}, instead
of Morse and Feshbach, in which case, equation (5.3.89) may actually be
relevant.}is presented in appendix A of Soumekh's 1991 paper \cite{Sou91}
using an integral representation of the Hankel function of order 
$\frac{1}{2}$ because of the exact relationship
\begin{equation}
\frac{\ds e^{jk\left|\vect{r}\right|}}{\left|\vect{r}\right|}
=jk h_0(k\left|\vect{r}\right|)=
j\sqrt{\frac{\pi k}{2\left|\vect{r}\right|}} 
H_{\frac{1}{2}}(k\left|\vect{r}\right|)
\end{equation}

%%%%%%%%%%%%%%%%%%%%%%%%%%%%%%%%%%%%%%%%%%%%%%%%%%%%%%%%%%%%%%%%%%%%%%%%%%%%%
\appendix{The Doppler-Shift Formulation of ISAR}
\label{ii app:dop}

Many early papers about ISAR describe scatterers being resolved in
cross-range due to differences in the Doppler shifts of their reflections. 
This description is intuitively appealing, but completely incorrect because
it is the change in the scatterer's position that alters the round-trip
distance and is manifested as a phase shift.  The radial velocity of each
scatterer does cause a Doppler shift, but as discussed in section 
\ref{hrr sec:doppler}, the Doppler shift is usually negligible.  

This appendix has been included for completeness, not because it correctly
describes ISAR.  With this disclaimer in mind, here is an historically 
accurate but technically incorrect description of ISAR cross-range processing
using the Doppler shift.  The treatment follows Wehner \cite{Weh87}, and is
very similar to the 1969 paper by Brown \cite{Bro69} and the 1980 work of
Chen and Andrews \cite{Che80a}.

%%%%%%%%%%%%%%%%%%%%%%%%%%%%%%%%%%%%%%%%%%%%%%%%%%%%%%%%%%%%%%%%%%%%%%%%%%%%%
\subsection{Cross-Range Processing}

Consider a target a fixed distance $r_0$ from the radar rotating at constant
angular speed $\omega$ at time $t=0$, as shown in figure~\ref{ii fig:rotating target} 
on page \pageref{ii fig:rotating target}.  The scatterer at a distance $x$ in
cross-range from the target's centre of rotation is moving directly towards
the radar with instantaneous speed
\begin{equation}
v(x)=\omega x
\end{equation}
For a radar transmitting at a frequency $f_0$, the reflection has a Doppler
frequency shift $f_D$ directly proportional to the scatterer's cross-range
position
\begin{equation}
f_D(x)=\frac{2f_0}{c}v(x)=x\frac{2\omega f_0}{c}
\end{equation}

If two scatterers the same range from the radar are separated by $d$ in
cross-range, their Doppler frequencies differ by
\begin{equation}\label{ii eqn:dfd}
\Delta f_D=d\frac{2\omega f_0}{c}
\end{equation}
The target's reflections are sampled for a total time $T$, called the
coherent Doppler integration time, and the different Doppler frequencies are
detected using a Fourier transform.  Close scatterers are resolvable if the
difference in their Doppler shifts is greater than the resolution of the
Fourier transform
\begin{equation}
\Delta f_D\geq\frac{1}{T}
\end{equation}
Combining this with (\ref{ii eqn:dfd}) and writing the cross-range
resolution $\Delta r_c$ for $d$ shows that
\begin{equation}
\Delta r_c=\frac{c}{2f_0\omega T}
\end{equation}
which is the same as the approximate expression in equation 
(\ref{ii eqn:drc rot}).

%%%%%%%%%%%%%%%%%%%%%%%%%%%%%%%%%%%%%%%%%%%%%%%%%%%%%%%%%%%%%%%%%%%%%%%%%%%%%
\subsection{Resolution Restrictions}
\label{ii sec:dop rr}

The Doppler-shift analysis assumes that each scatterer can be found in the
same range cell for the whole of the Doppler integration time.  For large
targets or rapidly rotating targets, this may not be so.  If a scatterer
spends only a fraction of the Doppler integration time in a range cell, the
cross-range resolution of that scatterer in that range cell will be
decreased accordingly.  

Blurring in the final ISAR image depends on the number of range and
cross-range cells that each scatterer moves through.  Suppose the target's
size is $D_r$ in the range direction and $D_c$ in the cross-range direction.
If no scatterer is allowed to move through more than one range cell, the
total angle through which the target rotates, $\Delta\theta$, must satisfy
\begin{equation}
\frac{D_c}{2}\Delta\theta<\Delta r_r
\end{equation}
Using $\Delta\theta=\omega T$ and $\lambda=c/f_0$ shows that this is 
equivalent to the condition
\begin{equation}
\Delta r_r\Delta r_c>\frac{\lambda D_c}{4}
\end{equation}
If no scatterer is allowed to move through more than one cross-range cell, 
$\Delta\theta$ must satisfy
\begin{equation}\label{ii eqn:prc I}
\frac{D_r}{2}\Delta\theta<\Delta r_c
\end{equation}
which is equivalent to the condition
\begin{equation}\label{ii eqn:prc II}
\Delta r_c^2>\frac{\lambda D_r}{4}
\end{equation}

