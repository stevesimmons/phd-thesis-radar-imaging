%%%%%%%%%%%%%%%%%%%%%%%%%%%%%%%%%%%%%%%%%%%%%%%%%%%%%%%%%%%%%%%%%%%%%%%%%%%
%
%			INVERSE SYNTHETIC APERTURE RADAR
%
%				  PhD Thesis
%
%		Stephen Simmons		simmons@ee.mu.oz.au
%
%	    Department of Electrical and Electronic Engineering
%	    University of Melbourne, Parkville, 3052, Australia
%
% Chapter 3:	High Resolution Radar
%
%		started first draft:	13 Dec 1994
%		finished first draft:	17 Dec 1994
%		submitted:		
%
%%%%%%%%%%%%%%%%%%%%% Copyright (C) 1994 Stephen Simmons %%%%%%%%%%%%%%%%%%

\chapter{High Resolution Radar}
\label{hrr chp}

\bigletter This chapter is an introduction to high resolution radar in
general and radar imaging using inverse synthetic aperture radar in
particular.  The constraints on radar imaging using synthetic apertures are
examined by considering the implications of scaling up an optical imaging
system operating in the visible spectrum to operate at radar wavelengths
while maintaining the same angular resolution.  This suggests that a
stationary radar may be used to obtain a high resolution radar image of a
moving target if the target's motion forms a suitable synthetic aperture.

After discussing the validity of representing distributed targets as
non-interacting point scatterers, the radar waveforms and target motions
necessary for high resolution images are derived.  This justifies the simple
model that will be used for radar imaging for the remainder of the thesis.


%%%%%%%%%%%%%%%%%%%%%%%%%%%%%%%%%%%%%%%%%%%%%%%%%%%%%%%%%%%%%%%%%%%%%%%%%%%%%
\section{An Overview of SAR and ISAR}


When a radar transmitter points its antenna in the direction of a 
target\footnote{Because most radar engineers have a military background, the
word ``target'' means any object of interest.}, the transmitter's power is
distributed according to the antenna's radiation pattern.  Most power is
directed in the approximate direction the antenna is pointing (this
direction is called the antenna's boresight axis, or its line-of-sight), and
the angular width of this main beam determines the radar's angular
resolution.

The radiation pattern of an antenna is wavelength-dependent and is given 
by the Fourier transform of its aperture \cite{Bra86}.  For a one-dimensional
aperture of length $D$, the relative power radiated at an angle $\theta$ to 
the aperture's normal is
\begin{equation}
P(\theta)=\left|\sinc\left(\frac{D}{\lambda}\sin\theta\right)\right|^2
\end{equation}
For a circular antenna with diameter $D$, the relative power radiated at an 
angle $\theta$ to the boresight axis is
\begin{equation}
P(\theta)=\left|\jinc\left(\frac{D}{\lambda}\sin\theta\right)\right|^2
\end{equation}
where $\jinc(x)=J_1(\pi x)/\pi x$ is the Bessel function-equivalent of 
$\sinc(x)=\sin(\pi x)/x$.  

Radiation patterns like these are characterized by a strong central lobe and
weaker sidelobes.  The angular width $\Delta\theta$ of the mainlobe is determined by the 
angular separation of the points at which the transmitted power density is half the 
power density on the boresight axis.  This can be written as \cite[eq. (6.11)]{Sko90}
\begin{equation}\label{hrr eqn:ant bw}
\Delta\theta=K\frac{\lambda}{D}
\end{equation}
where the beamwidth factor $K$ is a constant, usually close to unity, which
depends on the antenna's shape and the uniformity of its illumination.

Taking $K=1$, the illumination from an antenna with a diameter $D$ spreads to
a width of $R\lambda/D$ a distance $R$ from the radar.  A 60~cm antenna
operated at X-band, where the wavelength is $\lambda=3$~cm, has a main 
lobe that is 500~m across at a distance of 10~km.  This suggests that such a
radar would have a cross-range resolution of about 500~m.

However, with more sophisticated processing, it is possible to resolve
radar scatterers on targets with a resolution better than 1~m.
Scatterers at different distances from the radar are resolved by measuring
differences in the times at which echoes from them are received.  Scatterers
with identical ranges but separated in the cross-range direction may be
resolved if there is some relative motion between the radar and the target 
which changes these scatterers' ranges at different rates.

This necessity for some relative motion between the radar and the target to
change the target's aspect angle leads to synthetic aperture radar (SAR) and 
inverse synthetic aperture radar (ISAR), the two basic forms of high 
resolution radar imaging.

In SAR imaging, the radar is carried aboard a moving platform, usually an
aircraft or a satellite and the target is the surface of the Earth beneath or
to the side of the platform.  SAR is widely used for remote sensing because
it is able to produce ground maps with a width of 100~km and a resolution 
of 10~m from hundreds of kilometres away.

ISAR imaging uses a stationary radar to form an image of a moving target,
usually an aircraft or a ship.  The target's changing aspect angle may be
a result of it moving in a straight line or turning during manoeuvres.  If 
ocean waves cause a ship to pitch or roll, this can also form a synthetic 
aperture for radar imaging.  ISAR is used for target identification, so a
resolution of 1~m or better is often needed.

The principles behind SAR and ISAR are identical, but their implementations
are slightly different because of differences in the target's and the
radar's motions, the target's range and the desired resolution.  Since the
relative motion between radar and target is crucial to the imaging process,
their relative positions must be known to within a fraction of a radar
wavelength.  Motion estimation is used to estimate any unwanted components 
of the relative motion.  The effects of any unwanted movement are corrected
using motion compensation before the radar image is formed.

Motion estimation is more difficult for ISAR imaging than for SAR because
the target's motion is not known {\em a priori\/}.  With SAR imaging, 
the radar is the one that moves so an inertial navigation unit or
global positioning system (GPS) can provide estimates of its position at all 
times.


%%%%%%%%%%%%%%%%%%%%%%%%%%%%%%%%%%%%%%%%%%%%%%%%%%%%%%%%%%%%%%%%%%%%%%%%%%%%%
\section{Deriving ISAR from Optical Imaging}
\label{hrr sec:qual}


Tutorials about ISAR tend to begin ``Consider a target that is slowly rotating
at a distance $R$ from the radar \ldots''.  This does little to answer
the question of why ISAR imaging is done like that, except for the circular
reply ``because it works''.\footnote{Polya emphasizes the need to appreciate
the {\em why\/} as well as the {\em how\/}.  As a student, he asked ``Yes,
the solution seems to work, it appears to be correct; but how is it possible
to invent such a solution?''\protect\cite[p. vi]{Pol57}.  This section aims
to illustrate how ISAR is part of a natural progression from ordinary
optical imaging.}  It is instructive to look at the requirements of radar
imaging a little more directly, and see how these affect the design of a
radar imaging system.  This qualitative derivation of ISAR begins with the
human eye, an optical imaging system with which we all have direct
experience.  By considering the implications of scaling up such an optical
system to operate at radar wavelengths, ISAR is placed in its proper context
among imaging systems.

Consider a typical ISAR scenario where an X-band radar operating at 10~GHz
(a wavelength of 3~cm) forms an image of an aircraft that is 10~km away. To
be useful for target identification, the image needs a spatial resolution of
about 1~m.  Consequently, the radar imaging system must be capable of
resolving individual radar scatterers on the target with an angular
separation of $\Delta\theta=\frac{1\ \rm m}{10\ \rm km}=10^{-4}$~radians.

An angular resolution of $10^{-4}$~radians is comparable to the 
diffraction-limited resolving power of the human eye.  The eye's aperture, 
determined by the size of the pupil, has a diameter which varies between 2~mm 
and 8~mm.  The diffraction-limited angular resolution of any optical system 
is given by Rayleigh's criterion \cite{Goo68}
\begin{equation}
\Delta\theta=\frac{\lambda}{D}
\end{equation}
where $\lambda$ is the wavelength of the light and $D$ the diameter of the 
aperture.  This is the same form as the beamwidth of an antenna in equation 
(\ref{hrr eqn:ant bw}).  Taking the pupil's diameter as 5~mm and the 
wavelength as 500~nm, that of blue-green light, the eye's angular resolution 
is $\Delta\theta=\frac{500\ \rm nm}{5\ \rm mm}= 10^{-4}$~radians.

From a signal processing perspective, imaging occurs in two fundamental
steps, sampling and processing.  Light is reflected from the target and 
strikes the imaging system, where it is sampled over the aperture.  These 
samples are processed to form the image.  In an optical imaging system, such 
as the eye, the lens is both the sampler and the signal
processor.  Under the Fraunhofer approximation, the light striking the lens
is the spatial frequency domain representation of the target's
reflectivity.  The lens acts as a Fourier transformer, which focuses an 
image of the target onto the image plane (onto the retina in the case of
the eye).

In principle, a radar imaging system could operate in the same way as this
signal processing view of an optical system.  A radar transmitter could
illuminate the target with X-band radiation, some of which would be
scattered back towards a receiver which samples the reflected wavefront. The
eye's 5~mm aperture at 500~nm is equivalent to an aperture some 300~m in
diameter at the X-band wavelength of 3~cm.  To achieve the
diffraction-limited resolution of $\Delta\theta=10^{-4}$ radians, the
backscattered signal must be sampled over the whole aperture with sensor
elements spaced no more than $\lambda/2$ apart \cite{Ste83b}.  Approximately
$4(\Delta\theta)^{-2}=4\times 10^8$ sensors are needed, which is clearly 
impractical.

If the spacing between sensors is increased above the $\lambda/2$
minimum, the total number of sensors may be greatly reduced.  The
disadvantage of such a thinned array is that it cannot distinguish signals
whose angles of incidence differ by 
\begin{equation}
\Delta\phi=\arcsin\left(n\frac{\lambda}{d}\right)
\end{equation}
for some integer $n$ where $d$ is the distance between elements in the
array.  This is the same effect that causes grating lobes on a diffraction
grating.

An example of a thinned array is Raytheon's Cobra Dane radar. This is an
L-band phased array radar with a 28.5~m diameter steerable antenna.  The
antenna's diameter is $120\lambda$ at a wavelength of 24~cm, giving a
theoretical angular resolution of $0.5^o$.  An unthinned array this size
would require more than 45,000 elements.  The Cobra Dane array is thinned, and has
only 15,360 elements \cite{Sko90}.

Experiments on highly-thinned arrays have been performed by Steinberg.  His
radio camera is a radar imaging system that achieves a high angular
resolution with a very small number of receiving elements \cite{Ste81,Ste83}.
The closely spaced sidelobes that normally accompany a thinned array are
reduced in amplitude by selecting the positions of the receiving elements at
random. Extensive discussions of the properties of random arrays may be
found in Steinberg's books, \cite{Ste76} and \cite{Ste83b}, and the papers
\cite{Att89}, \cite{Ste81} and \cite{Tah76}.

An alternative to using many receivers in a large array is to have a smaller
number of receivers that move over the whole of the array's aperture.  Over
a period of time, the received signal is sampled over the entire aperture. 
By correcting for any change in the target's position or orientation during
the sampling period, the data is equivalent to that collected at one instant
by a much larger number of receivers.  Now the reduction in the number of
receivers comes at the expense of a much longer data acquisition time.

The distance between samples can be made larger than $\lambda/2$ provided
each receiver has an antenna whose beamwidth given by (\ref{hrr eqn:ant bw})
is smaller than the spacing between the thinned array's lobes.  This
requires the distance between samples over the aperture to be smaller than
the receiving antennas' diameters.\footnote{This is similar to the problem
of a tracking radar deciding whether a detection is due to a weak target
in the mainlobe or a strong target in a sidelobe.  A secondary receiver,
called a guard horn, with a very wide mainlobe, is used.  The ambiguity is
resolved by comparing the strengths of the detected target given 
by the main antenna and the guard horn \protect\cite{Sti83}.}

Radio astronomy using very-long-baseline interferometry (VLBI) is a very
good example of this time-sharing approach to imaging \cite{Coh73}.  A small
number, usually between two and thirty, of widely separated radio telescopes
are linked electronically to simulate an aperture whose size is equal to
their separation.  Over a period lasting up to twelve hours, the Earth's
rotation moves the radio telescopes so that samples can be acquired over
most of the aperture, thus reducing the level of the sidelobes in the
synthesized antenna's radiation pattern.  An impressive application of VLBI
is the baseline of 8,000~km obtained by arraying radars in California,
Massachusetts and the Crimea.  Earth-rotation aperture synthesis \cite{Fom73}
gives an angular resolution better than $10^{-8}$ radians at a wavelength 
of 10~cm.

Because the angle of arrival of a plane wave is established from differences
in the wave's phase at different receivers, it takes at least two receivers
for VLBI.  A single receiver can be used only when the incoming radar
signals maintain coherence over the whole time interval required to sample
the aperture. Then the single receiver's measurement at the first position
in the aperture can be used as a phase reference for measurements made at
different positions within the aperture.  Astronomical radar sources,
however, do not maintain coherence over the length of time that
Earth-rotation synthesis requires, so a single receiver cannot be used in 
interferometric radio astronomy.

A fundamental difference between the imaging of interstellar objects and
targets such as aircraft and ships is that the interstellar objects are
radar sources, while aircraft and ships are radar reflectors.  Targets that
are passive reflectors need a radar transmitter to illuminate them.  If the
radar transmitter is coherent, radar imaging can be achieved with only a
single receiver sampling within a larger aperture.  In fact, if the
transmitter's oscillator is connected directly to the receiver, this may
serve as the phase reference for each measurement.  Then the transmitter
need be coherent over the time taken for a measurement at a single position
within the aperture, not for the whole of the imaging period.  

So far, this overview of the design principles behind radar imaging has
indicated how a high resolution may be obtained in the two spatial
dimensions perpendicular to the radar's line-of-sight.  The initial design
requiring a very large number of receiving elements spread over a large
two-dimensional aperture has been simplified to a single transmitter and
receiver moving over the large two-dimensional aperture.  Decreasing the
number of receiving elements has been offset by an increase in the time
taken to acquire the measurements for the radar image.  

However the design at this stage is still not practical for imaging moving 
targets such as ships and aircraft.  It takes too long to move the radar and
make the measurement at each position within the two-dimensional aperture.  
Moving the radar is the most time consuming part of this, so to make the
imaging feasible much of the radar's movement within the aperture has 
to be eliminated.

Two dimensions of measurements are needed to form a two-dimensional image,
but there is no reason why both have to be spatial dimensions.  Close
coupling between the radar transmitter and receiver---they often share the
antenna---means that time delay can be used as one of the measurement
dimensions.  If the transmitter sends out a short pulse, the signal measured
at the receiver is the superposition of the pulse's reflections from the
target's scatterers.  Each reflection is delayed by
$2r_i/c$, the time it takes the pulse, propagating at the speed of light $c$,
to travel the distance $r_i$ from the transmitter to the $i^{\rm th}$
scatterer, and return to the receiver.  

Therefore, the time delay from scatterers at different ranges can be used 
as one of the measurement dimensions.  The aperture
now need be large only in one spatial dimension, and the number of
positions within the aperture's one spatial dimension where measurements 
are made is the
square-root of the number needed with the two-dimensional aperture.
The image of the target lies in the plane containing the time-delay
measurements and the one-dimensional aperture.  This plane is parallel to
the radar's line-of-sight whereas before the image plane was normal to the
radar's line-of-sight.  This gives one important difference in the images from
a time-delay/linear synthetic aperture and a two-dimensional aperture such 
as an optical system or Steinberg's radio camera: the time-delay/linear
synthetic aperture's image is a plan view of the target, even though the
target is viewed from a shallow angle.

A very important example of radar imaging using a time-delay/linear synthetic 
aperture is synthetic aperture radar, or SAR, used for mapping
large areas of the Earth's surface.  The radar is carried on a moving
platform, such as an aircraft or a satellite, and the platform's motion
creates the synthetic aperture.  The Seasat-A synthetic aperture radar 
orbited at an altitude of 780~km and produced ground maps with a width of
100~km and a resolution of 25~m \cite[ch. 22]{Sko90}.

An important point to note about SAR imaging is that the synthetic aperture
is needed so that measurements can be made with the radar and the target at
different relative positions and orientations.  This requirement can be met
just as well if the radar is stationary and the target moves along the
synthetic aperture.  These are just the conditions for ISAR.

So, starting at a familiar optical imaging system, the trade-offs
necessary for high resolution radar imaging have been discussed and 
the conditions for ISAR imaging have been derived qualitatively.  Next, the
process of radar scattering will be analysed to obtain a mathematical model
for radar reflections.  This allows the constraints ISAR places on the radar 
system and the target's motion to be calculated in a more quantitative way.


%%%%%%%%%%%%%%%%%%%%%%%%%%%%%%%%%%%%%%%%%%%%%%%%%%%%%%%%%%%%%%%%%%%%%%%%%%%%%
\section{Radar Scattering Mechanisms}


When a target such as an aircraft is illuminated by a radar transmitter, the
reflected electromagnetic wave is a combination of reradiated fields due to
a number of complex electromagnetic phenomena.  If the target's conductivity
and permittivity are known perfectly, Maxwell's equations may be evaluated to
give an exact description of the reflected wave. For real targets---all but
a few simple geometric shapes---Maxwell's equations are too complex for an
exact solution.  Simplified methods such as the method of moments,
geometrical optics, physical optics or the geometrical theory of diffraction
may be used to estimate the target's reflections.

However in radar imaging, only a portion of the reflected wave is known and
the target's structure must be estimated.  This is too complicated, and the
measurements too incomplete, to solve even for the simplified methods
mentioned above.  

Radar imaging is made tractable by assuming an even simpler electromagnetic
model: a target composed of point scatterers.  After analysing reflections
from an isolated point scatterer, this model is generalized to a description
of a target in terms of its two-dimensional reflectivity function $g(x,y)$.


%%%%%%%%%%%%%%%%%%%%%%%%%%%%%%%%%%%%%%%%%%%%%%%%%%%%%%%%%%%%%%%%%%%%%%%%%%%%%
\subsection{Isolated Point Scatterers}


The strength of the radar reflection from a point scatterer is determined by
$\sigma$, its radar cross-section or RCS.  This is a measure of the
scatterer's effective cross-sectional area, with units of ${\rm m}^2$.
Suppose this point scatterer is located a distance $r$ from a radar that is
transmitting a sinusoidal signal $p_t(t)$ with a frequency $\omega/2\pi$
\begin{equation}\label{hrr eqn:pt(t)}
p_t(t)=A\cos\omega t
\end{equation}
The incident wave striking the scatterer, $p_i(t)$, is $p_t(t)$ delayed by 
$r/c$, where $c$ is the speed of light
\begin{equation}
p_i(t)=\frac{A}{\sqrt{4\pi r^2}}\cos\left(\omega\left[t-\frac{r}{c}\right]
\right)
\end{equation}
The amplitude is attenuated by $4\pi r^2$ because the transmitter's fixed 
power is being distributed over a spherical wavefront with surface area 
$4\pi r^2$.

When $p_i(t)$ is reflected from the scatterer, the backscattered waveform
has a power density of $\sigma$ times the incident power density.  There may 
also be a phase shift $\phi$ due to a discontinuity between the conductivity 
and permittivity of the scatterer and free space.  Therefore the 
backscattered wave is
\begin{equation}
p_b(t)=\frac{A\sqrt{\sigma}}{\sqrt{4\pi r^2}}
\cos\left(\omega\left[t-\frac{r}{c}\right]+\phi\right)
\end{equation}
By the time this reflection reaches the radar antenna, there has been an
additional time delay of $r/c$ and another attenuation of $\sqrt{4\pi r^2}$
because the point scatterer radiates in all directions.  The received
waveform is
\begin{equation}
p_r(t)=\frac{A\sqrt{\sigma}}{4\pi r^2}
\cos\left(\omega\left[t-\frac{2r}{c}\right]+\phi\right)
\end{equation}
which can be written as
\begin{equation}
p_r(t)=\real{\frac{g}{r^2}\,e^{j\omega\left[t-\frac{2r}{c}\right]}}
\end{equation}
where the scatterer's radar reflectivity 
\begin{equation}
g=\frac{A}{4\pi}\sqrt{\sigma}\,e^{j\phi}
\end{equation}
incorporates the scatterer's RCS and phase shift, and for notational
simplicity, the transmitter's amplitude and the factor of $4\pi$.


%%%%%%%%%%%%%%%%%%%%%%%%%%%%%%%%%%%%%%%%%%%%%%%%%%%%%%%%%%%%%%%%%%%%%%%%%%%%%
\subsection{Reflections From a Distributed Target}


This model of an isolated point scatterer will now be extended to the radar
reflectivity of a distributed target. Define $g(x,y)\,dx\,dy$ to be
the overall reflectivity of the differential region located at $(x,y)$ on 
the target with extent $(dx,dy)$.  All coordinates are measured in a
Cartesian coordinate system that has its origin at the target's centre and is
fixed to the target.  The received signal due to this differential area is
\begin{equation}
dp_r(t)=\real{\frac{g(x,y)}{r_{xy}^2}\,
e^{j\omega\left[t-\frac{2r_{xy}}{c}\right]}\,dx\,dy}
\end{equation}
where $r_{xy}$ is the distance from the radar to $(x,y)$.  By combining the
received contributions from all parts of the target, the total received
signal becomes
\begin{equation}\label{hrr eqn:pr(t)}
p_r(t)=\real{\int\!\!\int\frac{g(x,y)}{r_{xy}^2}\,
e^{j\omega\left[t-\frac{2r_{xy}}{c}\right]}\,dx\,dy}
\end{equation}
where the integration is taken over the region occupied by the target.

A number of assumptions are implicit in this model and its application to
radar imaging \cite[ch. 9]{Rih69}:
\begin{enumerate}
\item The target is uniformly illuminated by the radar transmitter.
\item $g(x,y)$ is independent of the target's orientation.
\item $g(x,y)$ is independent of the radar's frequency $\omega$.
\item The only scatterering mechanism is direct reflection; there are no
multiple reflections, creeping waves or diffraction phenomena.
\end{enumerate}

In general, all of these assumptions are invalid.  Since radar targets are
three-dimensional, part of the target may be obscured by other parts between
it and the radar.  Shadowing like this is orientation-dependent, which
immediately invalidates the second assumption.  Furthermore, even direct
reflections are orientation-dependent. Reflections from flat surfaces are
specular so a strong reflection is only apparent when the surface is normal
to the radar's line-of-sight.\footnote{This was the rationale behind the
faceted design of the F-117A Stealth Fighter.  The aircraft's skin is formed
from flat surfaces oriented in small number of different directions. 
Therefore an F-117A is nearly invisible unless viewed from one of 
these directions \cite{Goo92}.}

The third assumption is invalid because the reflections are only
frequency-independent when the features on the target responsible for the
scattering are large compared to the wavelength.  This can be seen in
figure~3-1 of \cite{Bla86}, where the RCS of a conducting sphere is very
oscillatory when the sphere's radius is smaller than about five wavelengths.

The final assumption is also invalid because radar wavelengths are long
enough to diffract around features on typical man-made targets.  Waves may
diffract around one side of the target, emerge on the other side and
travel back to the radar.  They may also interfere constructively or
destructively behind the target with
waves diffracting around the other side.  The incident radar wave also
induces currents on the target's surface which cause creeping waves and
other effects.  The result of some of these indirect propagation modes on an
ISAR image is clearly shown in figure~4.39 of \cite{Men91}.

Even though these four assumptions are generally not valid, they can be
quite accurate under restricted circumstances.  If the target is viewed over
a narrow range of viewing angles, $g(x,y)$ may be considered independent of
orientation.  This also justifies the first assumption because scatterers
that start in shadow are likely to remain shadowed if the target's
orientation changes only slightly.  The frequency independence of $g(x,y)$
is appropriate if the target is illuminated with waves whose wavelength
varies little.  In other words, a narrow relative bandwidth---the ratio of
bandwidth to centre frequency---should be used.  

The final assumption is harder to justify, because multiple reflections and
diffraction phenomena do occur.  It is difficult to distinguish reflections
due to these phenomena from ordinary reflections so they get treated like
direct reflections and appear as artifacts such as phantom scatterers at
locations where there is no structural justification for them.  In figure~6
of \cite{Zyw94}, multiple reflections inside the engine intakes of a Mirage 
aircraft are manifested in an ISAR image as extra scatterers located at
regularly increasing intervals in range from their true location.  Therefore
the final assumption is replaced with an assumption that the target's
estimated reflectivity combines the effects of direct reflections located at
their correct positions and the effects of indirect reflections offset from
their correct positions by amount attributable to the propagation phenomenon
interacting with a particular part of the target's structure.


In conclusion, it is assumed for the purposes of radar imaging that the 
target's reflection can be modelled as
\begin{equation}\label{hrr eqn:2D pr(t)}
p_r(t)=\real{\int\!\!\int\frac{g(x,y)}{r_{xy}^2}\,
e^{j\omega\left[t-\frac{2r_{xy}}{c}\right]}\,dx\,dy}
\end{equation}
where the spatial radar reflectivity function $g(x,y)$ is independent of the
target's orientation and the radar frequency, at least over small changes in
aspect angle and for frequencies within a small relative bandwidth.


%%%%%%%%%%%%%%%%%%%%%%%%%%%%%%%%%%%%%%%%%%%%%%%%%%%%%%%%%%%%%%%%%%%%%%%%%%%%%
\subsection{The Dynamic Range of Radar and Optical Images}


People viewing ISAR images for the first time often are disappointed by the
poor image quality compared to a photograph.  Partly this is because radar
images of targets are usually taken from further away than photographs.  But
partly it is because they do not appreciate the different information
content caused by a five order-of-magnitude change in wavelength.

Reflections in the visible spectrum from most objects are diffuse because
the surface roughness is larger than the wavelength of visible light.
Because even dark surfaces are diffuse scatterers, optical images have a
relatively low dynamic range.  Radar wavelengths are large in comparison to
the surface roughness, especially for man-made targets.  Therefore most
radar reflections are specular, and radar images have such a high dynamic
range that only the strongest specular scatterers are detectable above 
noise and background clutter.

Taking this reasoning to the extreme suggests an even simpler model.  This
model, used in the superresolution approaches to ISAR imaging discussed in 
section \ref{ii sec:sr}, leads to a model of a radar target as a small
collection of ideal point scatterers.  The target's reflection using this 
model is
\begin{equation}\label{hrr eqn:pt sc pr(t)}
p_r(t)=\real{\sum_{k=1}^K \frac{g_k}{r_k^2}\,
e^{j\omega\left[t-\frac{2r_k}{c}\right]}}
\end{equation}
where there are $K$ discrete scatterers and the $k^{\rm th}$ scatterer has
reflectivity $g_k$ and is located $r_k$ from the radar.


%\tracingall %% REMOVE

%%%%%%%%%%%%%%%%%%%%%%%%%%%%%%%%%%%%%%%%%%%%%%%%%%%%%%%%%%%%%%%%%%%%%%%%%%%%%
\section{Achieving a High Range Resolution}


Section~\ref{hrr sec:qual} established, at a qualitative level, that by
sampling the time-delay response of a moving target over a period of time, a
high resolution estimate of the target's radar reflectivity could be
obtained.  In this section and the next, quantitative aspects of
achieving a high resolution image in range and cross-range are examined.  
This indicates the constraints that ISAR imaging imposes on the radar
waveform and the target's motion.

Consider the situation illustrated in figure~\ref{hrr fig:r res}, where 
the distance between two radar scatterers, $A$ and $B$, and the radar are
$r_A=r_0-d/2$ and $r_B=r_0+d/2$ respectively.  

%============================================================================
\begin{figure}
\centering
\caption{Geometry of a radar and two scatterers, $A$ and $B$, used to
calculate the range resolution of a radar with bandwidth $\beta$.}
\label{hrr fig:r res}

\psset{unit=0.8cm}
\begin{pspicture}(-0.5,-2.5)(12,3)
% Dotted lines to indicate the axes
\psline[linecolor=lightgray,linewidth=1pt]{-}(0,0)(12,0)
\psline[linecolor=lightgray,linewidth=1pt]{-}(8,-2)(8,2)
% The two scatterers and their labels
\pscircle[linecolor=black,fillcolor=gray,fillstyle=solid](6,0){0.3}
\uput[u](6,0.3){$A$}
\pscircle[linecolor=black,fillcolor=darkgray,fillstyle=solid](10,0){0.3}
\uput[u](10,0.3){$B$}
% Finally the arrows and their labels
\pcline[linecolor=black,linewidth=1pt]{<->}(0,-1.5)(8,-1.5)
\naput{$r_0$} %% CHANGED FROM \lput*{:U}{$r_0$}
\pcline[linecolor=black,linewidth=1pt]{<->}(6,1.5)(8,1.5)
\naput{$\ds\frac{d}{2}$} %% CHANGED FROM \lput*{:U}{$\ds\frac{d}{2}$}
\pcline[linecolor=black,linewidth=1pt]{<->}(8,1.5)(10,1.5)
\naput{$\ds\frac{d}{2}$} %% CHANGED FROM \lput*{:U}{$\ds\frac{d}{2}$}
% Now something for the radar at the left
\psarc*[fillcolor=darkgray](0,0){0.625}{128}{232}
\psellipse[fillstyle=solid,fillcolor=gray](-0.375,0)(0.15,0.5)
\psline{cc-cc}(-0.375,-0.475)(0,0)
\psline{cc-cc}(-0.5,0.25)(0,0)
\psline{cc-cc}(-0.25,0.25)(0,0)
\end{pspicture}

\end{figure}
%============================================================================ 

If the radar transmits a short pulse $p_t(t)$, the received signal $p_r(t)$ is 
the superposition of the reflections from the two scatterers
\begin{equation}
p_r(t)=g_A\,p(t-\tau_A)+g_B\,p(t-\tau_B)
\end{equation}
where $\tau_A=2r_A/c$ and $\tau_B=2r_B/c$ are the times it takes the pulse
to reach each of the scatterers and return to the radar.  $g_A$ and $g_B$
are the scatterers' reflectivities scaled to account for the $1/r^2$
amplitude attenuation with range.  

The two pulses may be separated, hence the two scatterers distinguished, if
the trailing edge of the pulse reflected from $A$ arrives before the leading
edge of the pulse reflected from $B$.  For a pulse $p_t(t)$ with duration
$\Delta\tau$, this implies that
\begin{equation}
\tau_A+\Delta\tau<\tau_B
\end{equation}
which is equivalent to saying that 
\begin{equation}
\Delta\tau<\frac{2d}{c}
\end{equation}
or that scatterers separated by $d$ in range may only be resolved if
\begin{equation}
d>\frac{c\Delta\tau}{2}
\end{equation}

Since the bandwidth $\beta$ of a rectangular pulse is approximately equal to
the reciprocal of its duration $\Delta\tau$, the range resolution 
$\Delta r_r$ of such a pulsed radar is 
\begin{equation}\label{hrr eqn:pulsed res}
\Delta r_r=\frac{c\Delta\tau}{2}=\frac{c}{2\beta}
\end{equation}

ISAR's high range resolution requires very, very short pulses; a range 
resolution of $\Delta r_r=1$~m means a pulse width of $\Delta\tau=6$~ns.  As
the pulse width decreases, the peak transmitted power must increase so that
the signal-to-noise ratio is not adversely affected.  At pulse widths small
enough for ISAR imaging, the peak powers are very high.  In addition,
magnetrons that are specially designed for very short pulses still have
$\Delta\tau>50$~ns, so the best range resolution is $\Delta r_r=6$~m, which
is not sufficient for most ISAR applications \cite[p. 104]{Weh87}.

The solution is to transmit a waveform with a slowly changing instantaneous
frequency that covers the required bandwidth over a long period of time. If
the waveform's centre frequency is large compared with $\beta$, the relative
bandwidth is small, which justifies assuming that the target's reflectivity 
is independent of frequency.  As an example, a total bandwidth of 150~MHz
corresponds to a range resolution of 1~m.  With a centre frequency of 10~GHz,
the relative bandwidth is $\frac{150\ {\rm MHz}}{10\ {\rm GHz}}=1.5$\%. 
This is small enough for the target's reflectivity to be considered uniform,
and it is also small enough for an X-band antenna to radiate efficiently
over the whole frequency range. 

Two such wideband waveforms, chirps and stepped-frequency waveforms, will
now be examined.

%%%%%%%%%%%%%%%%%%%%%%%%%%%%%%%%%%%%%%%%%%%%%%%%%%%%%%%%%%%%%%%%%%%%%%%%%%%%%
\subsection{Chirp Waveforms}

A waveform such as $e^{j2\pi(f_ct+\frac{1}{2}Kt^2)}$ for 
$t\in[-T/2,T/2]$ is called a chirp \cite{Kla60} because the instantaneous 
frequency
\begin{equation}
\frac{d}{dt}\left[f_ct+{\textstyle\frac{1}{2}}Kt^2\right]=f_c+Kt
\end{equation}
increases linearly from $f_c-KT/2$ to $f_c+KT/2$.  This is illustrated in
figure~\ref{hrr fig:wideband waveforms}(a).  

Consider what happens when a one-dimensional target is illuminated with the 
chirp
\begin{equation}\label{hrr eqn:chirp pt(t)}
p_t(t)=\real{e^{j\left(\omega t+\alpha t^2\right)}}
\end{equation}
Here, following \cite{Mun83}, $\omega=2\pi f_c$ and $\alpha=\pi K$ 
to reduce the number of symbols.

%============================================================================
\begin{figure}\centering
\caption[Chirp and stepped-frequency waveforms.]{This shows two wideband 
waveforms commonly used to obtain a high range resolution.  (a) is a chirp 
waveform and (b) is a stepped-frequency waveform.  In both cases, the range
resolution is $c/2\beta$ where $\beta$ is the waveform's bandwidth.}
\label{hrr fig:wideband waveforms}

% Chirp picture
\psset{unit=0.6cm}
\begin{pspicture}(-2,-1.25)(14,10)
% First the axes and their labels
\psaxes[linecolor=black,linewidth=1.5pt,ticks=none,labels=none]{<->}(0,0)(12,8)
\uput[r](12,0){$t$}
\uput[u](0,8){$f$}
% Draw in the chirp
\psline[linecolor=gray,linewidth=1.5pt]{-}(0,5)(2,7)(2,1)(10,7)(10,1)(12,3)
% The dotted line at the centre of the chirp
\psline[linecolor=black,linewidth=1pt,linestyle=dashed,dash=4pt 4pt]{-}(0,4)(6,4)(6,0)
% And the labels with ticks
\uput[l](0,7){$f_c+K\frac{T}{2}$}
\psline[linecolor=black,linewidth=1pt]{-}(0,7)(0.15,7)
\uput[l](0,4){$f_c$}
\psline[linecolor=black,linewidth=1pt]{-}(0,4)(0.15,4)
\uput[l](0,1){$f_c-K\frac{T}{2}$}
\psline[linecolor=black,linewidth=1pt]{-}(0,1)(0.15,1)
\uput[d](2,0){$-\frac{T}{2}$}
\psline[linecolor=black,linewidth=1pt]{-}(2,0)(2,0.15)
\uput[d](6,0){$0$}
\psline[linecolor=black,linewidth=1pt]{-}(6,0)(6,0.15)
\uput[d](10,0){$\frac{T}{2}$}
\psline[linecolor=black,linewidth=1pt]{-}(10,0)(10,0.15)
\end{pspicture}

(a)\,\,A chirp waveform with duration $T$ and total bandwidth $\beta=KT$.


% Stepped-frequency picture
\psset{unit=0.6cm}
\begin{pspicture}(-2,-1.25)(14,10)
% First the axes and their labels
\psaxes[linecolor=black,linewidth=1.5pt,ticks=none,labels=none]{<->}(0,0)(12,8)
\uput[r](12,0){$t$}
\uput[u](0,8){$f$}
% Draw in the stepped-frequency
\psline[linecolor=gray,linewidth=1.5pt]{-}(0,6)(1,6)(1,7)(2,7)(2,1)(3,1)(3,2)(4,2)(4,3)(5,3)
\psline[linecolor=gray,linewidth=1.5pt]{-}(12,2)(11,2)(11,1)(10,1)(10,7)(9,7)(9,6)(8,6)
% The dotted line filling in the missing steps
\psline[linecolor=gray,linewidth=1.5pt,linestyle=dashed,dash=4pt 4pt]{-}(5,3)(8,6)
% And the labels with ticks
\uput[l](0,7){$f_0+(N-1)\Delta f$}
\psline[linecolor=black,linewidth=1pt]{-}(0,7)(0.15,7)
\uput[l](0,1){$f_0$}
\psline[linecolor=black,linewidth=1pt]{-}(0,1)(0.15,1)
\uput[d](2,0){$0$}
\psline[linecolor=black,linewidth=1pt]{-}(2,0)(2,0.15)
\uput[d](10,0){$NT$}
\psline[linecolor=black,linewidth=1pt]{-}(10,0)(10,0.15)
% Finally, an arrow to indicate \Delta f 
\psline[linecolor=black,linewidth=1pt]{<->}(5,2)(5,1)
\uput[r](5,1.5){$\Delta f$}
\end{pspicture}

(b)\,\,A stepped-frequency waveform with total bandwidth $\beta=N\Delta f$.
\end{figure}
%============================================================================ 


Repeating the derivation of (\ref{hrr eqn:pr(t)}), but with a one-dimensional 
target $g(r)$ instead of $g(x,y)$ and (\ref{hrr eqn:chirp pt(t)}) as $p_t(t)$ 
rather than (\ref{hrr eqn:pt(t)}), shows that
\begin{equation}\label{hrr eqn:chirp pr(t)}
p_r(t)=\real{\int\frac{g(r)}{r^2}\,
e^{j\left(\omega (t-2r/c)+\alpha(t-2r/c)^2\right)}\,dr}
\end{equation}

As shown in \cite{Mun89}, the target's reflectivity $g(r)$ may be recovered 
by mixing $p_r(t)$ either with $2e^{j\omega t}$ or with the chirp
itself.  The second method, known as deramp-FFT, is illustrated here.

The received signal $p_r(t)$ is mixed with $e^{j\left(\omega (t-t_0)+
\alpha(t-t_0)^2\right)}$, which is the complex form of the original chirp, 
in a quadrature modulator.  The delay $t_0$ is equal to $2r_0/c$ where $r_0$
is the target's range.  The output of the quadrature modulator is passed
through a low-pass filter (LPF) to give in-phase and quadrature components
$s_I(t)$ and $s_Q(t)$
\begin{eqnarray}
s_I(t)
&=&{\rm LPF}\left\{p_r(t)\cdot 
	2\cos\left(\omega (t-t_0)+\alpha(t-t_0)^2\right)\right\} \nn\\
&=&\real{\int\frac{g(r)}{r^2}\,e^{j\left[\omega\left(t_0-2\frac{r}{c}\right)
	-\alpha t_0^2+\alpha\left(2\frac{r}{c}\right)^2\right]}\,
	e^{j2\alpha\left(t_0-\frac{2r}{c}\right)}\,dr} \\
s_Q(t)
&=&{\rm LPF}\left\{p_r(t)\cdot 
	2\sin\left(\omega (t-t_0)+\alpha(t-t_0)^2\right)\right\} \nn\\
&=&\imag{\int\frac{g(r)}{r^2}\,e^{j\left[\omega\left(t_0-2\frac{r}{c}\right)
	-\alpha t_0^2+\alpha\left(2\frac{r}{c}\right)^2\right]}\,
	e^{j2\alpha\left(t_0-\frac{2r}{c}\right)t}\,dr} 
\end{eqnarray}
This shows that the in-phase and quadrature components form the real and
imaginary parts of $s(t)$ where
\begin{eqnarray}
s(t)
&=&s_I(t)+js_Q(t)\nn\\
&=&\int\frac{g(r)}{r^2}\,e^{j\left[\omega\left(t_0-2\frac{r}{c}\right)
	-\alpha t_0^2+\alpha\left(\frac{2r}{c}\right)^2\right]}\,
	e^{j2\alpha\left(t_0-\frac{2r}{c}\right)t}\,dr
\end{eqnarray}
which has the form of an inverse Fourier transform
\begin{equation}
s(t)
=\int\left[\frac{g(r(f))}{r^2(f)}\,e^{j\phi(r(f))}\,\frac{dr}{df}\right]\,
e^{j2\pi ft}\,df
\end{equation}
where
\begin{equation}\label{hrr eqn:f and r}
f=\frac{\alpha}{\pi}\left(t_0-\frac{2r}{c}\right)
\end{equation}
and 
\begin{equation}
\phi(r)=\omega\left(t_0-2\frac{r}{c}\right)
-\alpha t_0^2+\alpha\left(\frac{2r}{c}\right)^2
\end{equation}

The duration of the chirp is $T$, so $s(t)$ is sampled over an interval of
length $T$.  The Fourier transform's frequency resolution is $\Delta f=1/T$. 
Using (\ref{hrr eqn:f and r}) and writing $\alpha=\pi K$, the range 
resolution $\Delta r_r$ satisfies
\begin{equation}
\frac{1}{T}=\Delta f=K\frac{2\Delta r_r}{c}
\end{equation}
which is
\begin{equation}
\Delta r_r=\frac{c}{2KT}
\end{equation}
Now $KT$ is the spread of instantaneous frequencies swept out by the chirp
during $[-T/2,T/2]$.  Associating this with the bandwidth $\beta$ of the
chirp waveform, the range resolution can be expressed as
\begin{equation}\label{hrr eqn:chirp res}
\Delta r_r=\frac{c}{2\beta}
\end{equation}
which is the same as the expression obtained in (\ref{hrr eqn:pulsed res}).

Note that the chirp's duration is $T$, which is $KT^2$ longer than
the equivalent single pulse.  Therefore, for identical signal-to-noise
ratios, the chirp's average power is $KT^2$ times lower than a single 
pulse's peak power.

%%%%%%%%%%%%%%%%%%%%%%%%%%%%%%%%%%%%%%%%%%%%%%%%%%%%%%%%%%%%%%%%%%%%%%%%%%%%%
\subsection{Stepped-Frequency Waveforms}

Another type of wideband waveform is a stepped-frequency waveform
\cite{Weh87}.  As shown in figure~\ref{hrr fig:wideband waveforms}(b), this
consists of $N$ discrete frequencies $f_n$, where
\begin{equation}
f_n=f_0+n\Delta f\qquad\mbox{for $n=0,1,\ldots,N-1$}
\end{equation}
and each frequency is transmitted for the same duration. 

When frequency $f_n$ is transmitted, the transmitted waveform is
\begin{equation}
p_{n,t}(t)=\cos(2\pi f_n t)=\real{e^{j2\pi f_n t}}
\end{equation}
Following the same reasoning as equations (\ref{hrr eqn:pt(t)}) to
(\ref{hrr eqn:pr(t)}), the received waveform is
\begin{equation}
p_{n,r}(t)=\real{\int\frac{g(r)}{r^2}\,e^{j2\pi f_n(t-2r/c)}\,dr}
\end{equation}
This is passed through a quadrature modulator and low-pass filtered to give
in-phase and quadrature components that are independent of time once
reflections from the whole target have been received
\begin{eqnarray}
s_I(f_n)
&=&{\rm LPF}\left\{p_{n,r}(t)\cdot 
	2\cos\left(2\pi f_nt\right)\right\} \nn\\
&=&\real{\int\frac{g(r)}{r^2}\,e^{-j2\pi f_n\cdot 2\frac{r}{c}}\,dr} 
\end{eqnarray}
and 
\begin{eqnarray}
s_Q(f_n)
&=&{\rm LPF}\left\{p_{n,r}(t)\cdot 
	2\sin\left(2\pi f_nt\right)\right\} \nn\\
&=&\imag{\int\frac{g(r)}{r^2}\,e^{-j2\pi f_n\cdot 2\frac{r}{c}}\,dr} 
\end{eqnarray}
When these are added together in quadrature, 
\begin{eqnarray}
s(f_n)
&=&s_I(f_n)+js_Q(f_n)\nn\\
&=&\int\frac{g(r)}{r^2}\,e^{-j2\pi f_n\cdot \frac{2r}{c}}\,dr
\label{hrr eqn:sf s(fn)}
\end{eqnarray}
which is the Fourier transform of the target's reflectivity, sampled
at frequency $f_n$.

Therefore, if the target's frequency responses $\{s(f_n)\}$ are measured for
the $N$ evenly spaced frequencies in the stepped-frequency waveform, an
inverse discrete Fourier transform (DFT) may be used to estimate the
target's one-dimensional reflectivity $g(r)$.  

The temporal resolution $\Delta t$ of an inverse DFT where the $N$ frequency
samples are spaced by $\Delta f$ is $1/N\Delta f$.  The range resolution
corresponding to this time resolution is $\Delta r_r=c\Delta t/2$,
which is
\begin{equation}
\Delta r_r=\frac{c}{2N\Delta f}
\end{equation}
Since $N\Delta f$ is the total frequency covered by the stepped-frequency
waveform, its bandwidth may be written $\beta=N\Delta f$ and the
range resolution for a stepped-frequency waveform becomes
\begin{equation}\label{hrr eqn:sf res}
\Delta r_r=\frac{c}{2\beta}
\end{equation}
This is the same as the equivalent expressions for a single pulse and a 
chirp given in (\ref{hrr eqn:pulsed res}) and (\ref{hrr eqn:chirp res}).


%%%%%%%%%%%%%%%%%%%%%%%%%%%%%%%%%%%%%%%%%%%%%%%%%%%%%%%%%%%%%%%%%%%%%%%%%%%%%
\section{Achieving a High Cross-Range Resolution}


The second dimension for radar imaging is the cross-range dimension.  Since 
the radar is only capable of distinguishing scatterers by differences in
range, the target must move during the imaging interval in such a way that
the range of each of the scatterers changes by a different amount.  This
happens when the target's aspect angle changes over time, which is the
situation illustrated in figure~\ref{hrr fig:cr res}.  Here a target
composed of two radar scatterers, $A$ and $B$, separated by $d$ in
cross-range, are rotating at angular speed $\omega$ at a distance $r_0$ from
the radar.  With $\theta(t)=\omega t$ being the angle that $A$ and $B$ make
with the cross-range, and assuming that $r_0\gg d$, the ranges of the
scatterers are
\begin{eqnarray}
r_A(t)
&=&\sqrt{(r_0-{\textstyle\frac{d}{2}}\sin\omega t)^2
	+({\textstyle\frac{d}{2}}\cos\omega t)^2}		\nn\\
&\approx&r_0-{\textstyle\frac{d}{2}}\sin\omega t		\\
r_B(t)
&=&\sqrt{(r_0+{\textstyle\frac{d}{2}}\sin\omega t)^2
	+({\textstyle\frac{d}{2}}\cos\omega t)^2}		\nn\\
&\approx&r_0+{\textstyle\frac{d}{2}}\sin\omega t		
\end{eqnarray}

%============================================================================
\begin{figure}\centering
\caption{Geometry of a radar and two scatterers, $A$ and $B$, used to
calculate the cross-range resolution of a radar.}
\label{hrr fig:cr res}

\psset{unit=0.8cm}
\begin{pspicture}(-0.5,-2.5)(12,3)
% Dotted lines to indicate the axes
\psline[linecolor=lightgray,linewidth=1pt]{-}(0,0)(12,0)
\psline[linecolor=lightgray,linewidth=1pt]{-}(8,-2)(8,0.5)
\rput{-60}(8,0){
	\psline[linecolor=lightgray,linewidth=1pt]{-}(-2,0)(2,0)
	% The two scatterers and their labels
	\pscircle[linecolor=black,fillcolor=gray,fillstyle=solid](-2,0){0.3}
	\uput[u](-2,0.3){$A$}
	\pscircle[linecolor=black,fillcolor=darkgray,fillstyle=solid](2,0){0.3}
	\uput[u](2,0.3){$B$}
	% Finally the arrows and their labels
	\pcline[linecolor=black,linewidth=1pt]{<->}(-2,1.5)(2,1.5)
	\lput*{:U}{$d$}
}
% An arrow and a label to indicate the total rotation
\psarc[linecolor=black,linewidth=1pt]{->}(8,0){1}{270}{300}
\uput[l](8.2,-0.7){$\theta(t)=\omega t$}
% Now something for the radar at the left
\pcline[linecolor=black,linewidth=1pt]{<->}(0,-1.5)(8,-1.5)
\lput*{:U}{$r_0$}
% Now something for the radar at the left
\psarc*[fillcolor=darkgray](0,0){0.625}{128}{232}
\psellipse[fillstyle=solid,fillcolor=gray](-0.375,0)(0.15,0.5)
\psline{cc-cc}(-0.375,-0.475)(0,0)
\psline{cc-cc}(-0.5,0.25)(0,0)
\psline{cc-cc}(-0.25,0.25)(0,0)
\end{pspicture}
\end{figure}
%============================================================================ 

If the target is rotating sufficiently slowly, $\sin\omega t\approx\omega t$
during the period $t\in[-T/2,T/2]$.  When illuminated by a radar at a
frequency $f$, the difference in phase between the two scatterers'
reflections is
\begin{equation}
\Delta\theta(t)=2\pi f\frac{2d\sin\omega t}{c}\approx \fpc fd\omega t
\end{equation}
which has an instantaneous frequency of
\begin{equation}
\frac{1}{2\pi}\,\frac{d}{dt}\left[\fpc fd\omega t\right]=\frac{2fd\omega}{c}
\end{equation}
Since the rotating target's phase is observing for a total time $T$, a
Fourier transform can determine the rate at which the phase changes with an
accuracy of $\Delta f=1/T$.  Setting this to the phase's instantaneous
frequency shows that
\begin{equation}
\frac{1}{T}=\frac{2fd\omega}{c}
\end{equation}
Rearranging to make $d$ the subject then replacing it with the cross-range 
resolution $\Delta r_c$ gives
\begin{equation}
\Delta r_c=\frac{c}{2f\omega T}=\frac{c}{2f\Delta\theta}
\end{equation}
where $\Delta\theta$ is the target's total rotation during $[-T/2,T/2]$.
Therefore, the cross-range resolution for the radar image is inversely
proportional to the total change in the target's aspect angle during the
imaging period.

%%%%%%%%%%%%%%%%%%%%%%%%%%%%%%%%%%%%%%%%%%%%%%%%%%%%%%%%%%%%%%%%%%%%%%%%%%%%%
\subsection{Doppler Shifts}
\label{hrr sec:doppler}

When a target rotates, the scatterers on one side of the target move towards
the radar and those on the other side move away.  With the target
illuminated by a radar transmitting a single frequency, the phase of the
radar reflection from each scatterer is shifted linearly in time due to that
scatterer's changing range.  This appears as a shift in the frequency of
the reflected wave.  

It is a common misconception in the ISAR literature%
%%%%%
\footnote{See, for example, the criticisms of Harris \protect\cite{Har90a}
and Munson \protect\cite{Mun89}.  Many papers, for example 
\protect\cite{Aus84} and \protect\cite{Wal80}, and even Wehner's book
\protect\cite{Weh87}, use the Doppler shift as an intuitive explanation of
cross-range processing.  Even if they later analyse ISAR properly, this
perpetuates the myth that ISAR is a result of the Doppler effect.  While
some intuition is useful, false intuition hampers a proper understanding of
ISAR.  Even when range-Doppler processing is properly described without the
Doppler effect, the appearance of the word ``Doppler'' when there is no 
Doppler effect causes more confusion.}
%%%%%
that cross-range processing separates targets in cross-range because their
reflections have Doppler shifts due to each scatterer's velocity towards or
away from the radar.  However, the apparent frequency shifts are due to each
scatterers' changing position, not their velocities.  Soumekh \cite{Sou94}
calls this a spatial Doppler phenomenon, to emphasize that it is not 
velocity-related.  He also calls the Doppler effect a temporal Doppler
phenomenon to make the distinction absolutely clear.

Nevertheless, the scatterers are moving with radial components, so their
temporal Doppler shifts should be considered.  This is studied by Wehner in
\cite{Weh87} for ISAR and by Munson in \cite{Mun83} for SAR.  In the case of
ISAR, each scatterer's radial velocity has two components, a radial velocity
due to the translational movement of the target as a whole, and a radial
velocity due to its rotation.  The former causes a slight shift of the whole
target in the range direction, hence it can be ignored.  The latter
component causes phase shifts which degrade the ISAR image.  From equation
(30) of \cite{Mun83}, the Doppler effect can be neglected as long as $v_R$,
the radial component of any scatterer's velocity due solely to the target's 
rotation, satisfies
\begin{equation}
v_R<\frac{\lambda c}{8r_0}
\end{equation}

For a target 10~km from the radar imaged at X-band, this is a radial
velocity of over 100~ms${}^{-1}$ just from the target's rotation.
Therefore, the temporal Doppler effect may be neglected for many ISAR
targets.

%%%%%%%%%%%%%%%%%%%%%%%%%%%%%%%%%%%%%%%%%%%%%%%%%%%%%%%%%%%%%%%%%%%%%%%%%%%%%
\section{Imaging Two-Dimensional Moving Targets}

The quantitative and qualitative analyses of ISAR in this chapter indicate
that ISAR imaging requires a target whose motion causes a small change in its
aspect angle over a period during which its radar reflection is repeatedly
sampled by a wideband signal with a narrow relative bandwidth.

Either a stepped-frequency or a chirp waveform could be used.  For
simplicity, the remainder of this thesis will assume that a
stepped-frequency waveform is used to sample the target's reflectivity at
$N$ discrete frequencies
\begin{equation}
f_n=f_0+n\Delta f\qquad\mbox{for $n=0,1,\ldots,N-1$}
\end{equation}

Since the target is moving, its reflectivity $g(x,y)$ must be measured
in a coordinate system that moves with the target.  The distance between the
point $(x,y)$ on the target and the radar is now a function of time
$r_{xy}(t)$.  Equation (\ref{hrr eqn:2D pr(t)}) for the signal received from
a stationary two-dimensional target illuminated by a radar at frequency
$f_n$ becomes
\begin{equation}
p_r(f_n,t)=\real{\int\!\!\int\frac{g(x,y)}{r_{xy}^2(t)}\,
e^{j2\pi f_n\left[t-\frac{2r_{xy}(t)}{c}\right]}\,dx\,dy}
\end{equation}
for a moving target.

Because quadrature demodulation and low-pass filtering assume a carrier
of frequency $f_n$, phase shifts caused by changes in $r_{xy}(t)$ are
too low a frequency to be affected and the 
$e^{-j2\pi f_n\frac{2r_{xy}(t)}{c}}$ term passes through the demodulator
unchanged.  Using the same steps as (\ref{hrr eqn:sf s(fn)}) with
$r_{xy}(t)$ considered a constant, the frequency response of a moving
target at time $t$ and at frequency $f$ is
\begin{equation}\label{hrr eqn:s(f,t)}
s(f,t)=\int\!\!\int\frac{g(x,y)}{r_{xy}^2(t)}\,
e^{-j\fpc f r_{xy}(t)}\,dx\,dy
\end{equation}
where the region of integration includes the whole of the target.  Actually,
the region is determined by the location of the target within the target's 
mainlobe, the width of the mainlobe and the times at which the receiver's 
output is sampled, but these should be set to cover the entire target.

One final assumption about the target's frequency response will be made for
convenience, although the effects of it can easily be compensated for.  The
notation $s(f,t)$ for the target's frequency response at frequency $f$ and
at time $t$ should really be written $s(f(t),t)$ to emphasize that $f(t)$ is
actually a single-valued function determining which frequency in the
stepped-frequency waveform is transmitted at which time.  By suppressing the
``$(t)$'' in $f(t)$, it is implicitly stated that frequency response may be
measured at any number of frequencies in one single instant of time.  This is
not too unrealistic if the duration of each step in the stepped-frequency
waveform is very short so that the stepped-frequency waveform repeats with a
frequency that is high enough to capture the changes in the target's
reflections with time.  This requires a fast digital frequency synthesiser 
to generate the stepped-frequency waveform.

Now, all that ISAR requires is suitable algorithms to estimate $g(x,y)$ from
measurements of $s(f,t)$ taken over a set of discrete frequencies $\{f_n\}$ 
at each of a set of discrete times $\{t_m\}$.  A number of ISAR inversion 
algorithms suitable for different target motions are the subject of the next 
chapter.  

