%%%%%%%%%%%%%%%%%%%%%%%%%%%%%%%%%%%%%%%%%%%%%%%%%%%%%%%%%%%%%%%%%%%%%%%%%%%
%
%			INVERSE SYNTHETIC APERTURE RADAR
%
%				  PhD Thesis
%
%		Stephen Simmons		simmons@ee.mu.oz.au
%
%	    Department of Electrical and Electronic Engineering
%	    University of Melbourne, Parkville, 3052, Australia
%
% Preface (material before the table of contents)
%
%		started first draft:	Fri 6 Jan 1995
%		finished first draft:	Fri 6 Jan 1995
%		submitted:		Mon 9 Jan 1995
%
%%%%%%%%%%%%%%%%%%%%% Copyright (C) 1995 Stephen Simmons %%%%%%%%%%%%%%%%%%

\preface{Abstract}

Inverse synthetic aperture radar (ISAR) forms high resolution radar images
of a moving target by using the target's changing aspect angle to create a
synthetic aperture.  Estimates of changes in the target's position accurate
to a tenth of a wavelength are essential for sharply focussed ISAR images.

This thesis investigates the use of estimation theory to find optimal
estimators of a target's radial motion during stepped-frequency ISAR
imaging and to analyse their performance.  The aim of this is two-fold: to
apply a rigorously mathematical approach to a problem that has
traditionally been solved using more intuitive methods, and to develop a
unifying framework in whose terms the intuitive methods can be
reinterpreted and reappraised.

The maximum likelihood estimator of the radial distance moved by the target
is derived as the location of the global minimum of a cost function. The
cost function has a slowly varying envelope modulated by a high frequency
carrier.  The envelope shows the approximate distance the target has moved
and the high frequency modulation allows the distance to be determined
within a small fraction of a wavelength.  

Four efficient algorithms for minimizing the cost function are derived. Two
are approximate global algorithms based on the fast Fourier transform and
the chirp-Z transform.  The other two are more precise local minimization
algorithms.  The second of these is an iterative contraction mapping with
the rate of convergence equal to the stepped-frequency waveform's relative
bandwidth.  

The statistical properties of the maximum likelihood motion estimator are
examined and the estimator is shown to be unbiased.

To illustrate connections between conventional methods of range profile
alignment and phase compensation, the maximum likelihood motion estimator is shown
to be equivalent to the location of the maximum of the cross-correlation of
the target's complex range profiles.  Cross-correlating the magnitudes of
the range profiles, as in conventional motion estimation, is shown to be
less accurate but less susceptible to phase errors caused by the target
rotating.  

Phase alignment techniques of adaptive beamforming and the phase gradient
autofocus are also compared to the maximum likelihood estimator.  It is
found that the phase gradient part of phase gradient autofocus can be
implemented using slightly sub-optimal versions of the algorithm developed
for the local minimization of the estimator's cost function.

Finally, some common assumptions of ISAR imaging are analysed, including
the conditions under which polar reformatting may be neglected, the
assumption that the target's rotation may be neglected during motion
estimation, and the assumption that range profile alignment need only be
accurate to one range bin.  These assumptions are found to impose some new
sets of constraints on the target's motion and the ISAR image's resolution.

This work illustrates how a mathematical approach using estimation theory
can be used to obtain useful ISAR motion estimators and provide valuable
insights into related motion estimators whose derivations otherwise could
not be justified so easily.

%%%%%%%%%%%%%%%%%%%%%%%%%%%%%%%%%%%%%%%%%%%%%%%%%%%%%%%%%%%%%%%%%%%%%%%%%%%%%
\preface{Declaration}

This thesis contains fewer than 100,000 words.


\vspace{4cm}

%%%%%%%%%%%%%%%%%%%%%%%%%%%%%%%%%%%%%%%%%%%%%%%%%%%%%%%%%%%%%%%%%%%%%%%%%%%%%
\preface{Acknowledgements}

I would like to thank my supervisors, Professor Rob Evans, Dr Zhi-Qiang Liu
and Professor Terry Caelli, for their advice and encouragement, the Defence
Science and Technology Organisation for its support, and especially Dr David
Heilbronn, who introduced me to ISAR.
