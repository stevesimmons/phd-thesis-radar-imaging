%%%%%%%%%%%%%%%%%%%%%%%%%%%%%%%%%%%%%%%%%%%%%%%%%%%%%%%%%%%%%%%%%%%%%%%%%%%
%
%			INVERSE SYNTHETIC APERTURE RADAR
%
%				  PhD Thesis
%
%		Stephen Simmons		simmons@ee.mu.oz.au
%
%	    Department of Electrical and Electronic Engineering
%	    University of Melbourne, Parkville, 3052, Australia
%
% Chapter 2:	Mathematical Preliminaries
%
%		started first draft:	30 September
%		finished first draft:	1 October
%		second draft:		23 December
%		submitted:		9 January
%
%%%%%%%%%%%%%%%%%%%%% Copyright (C) 1995 Stephen Simmons %%%%%%%%%%%%%%%%%%

\chapter{Mathematical Preliminaries}
\label{mp chp}

\bigletter Statistical estimation theory is used throughout this thesis to
derive and analyse estimators of a radar target's motion.  This chapter
summarizes the main results of maximum likelihood estimation for 
real parameters, and extends it for use with complex parameters.

Full derivations of the results mentioned here may be found in the papers by
Fisher \cite{Fis25,Fis34b,Fis34a}, and the books by Cram\'{e}r~\cite{Cra46}, 
Kay~\cite{Kay93}, Kendall, Stuart and Ord \cite{Stu87,Stu91}, 
Rao~\cite{Rao65}, and Van Trees~\cite{Van68}.

%%%%%%%%%%%%%%%%%%%%%%%%%%%%%%%%%%%%%%%%%%%%%%%%%%%%%%%%%%%%%%%%%%%%%%%%%%%%%
\section{Statistical Estimation Theory}

As expressed by Fisher in {\em Theory of Statistical Estimation\/}
\cite{Fis25}, 
\begin{quote}\singlespaced
``Problems of estimation arise when we know, or are willing to assume, the
form of the frequency distribution of the population, as a mathematical
function involving one or more unknown parameters, and wish to estimate the
values of these parameters by means of the observational record available. 
A statistic may be defined as a function of the observations designed as an
estimate of any such parameter.''
\hspace*{\fill}{\em R. A. Fisher, 1925.}
\end{quote}

The population's known frequency distribution takes the form of a
probability density function $p(\md;\uvp)$.\footnote{To distinguish scalars,
vectors and matrices, scalars usually appear in a normal-weight typeface
$x$, vectors as bold lower-case letters $\vect{x}$ and matrices as bold
upper-case letters $\vect{X}$.}  This is one member of the family of
probability distributions $p(\md;\cdot)$ selected by the vector of unknown
parameters 
\begin{equation}
\uvp=\left[\up_1, \up_2, \cdots, \up_M\right]^T
\end{equation} 
The estimator $\uvpWH$ is a function of the measured data $\md$ used to 
estimate $\uvp$, and the estimate is the value of the estimator for one
particular realization of $\md$.\footnote{Some authors use an upper-case
letter for an estimator and a lower-case letter for an estimate.
Here both are indicated by the same symbol $\upWH$ and the context
indicates whether the estimator or an estimate is meant.}

In general, there are infinitely many ways in which $\md$ can be processed
to give $\uvpWH$.  Estimation theory is used to select one particular way of
processing the measurements to give the ``best'' estimate of $\uvp$, where
``best'' is defined in terms of some optimality criteria.  Three useful 
optimality criteria are consistency, bias and efficiency.

An estimator is consistent if the estimates tend to the parameter as
the number of measurements tends to infinity.  Formally, this convergence in
probability is written that for any $\epsilon>0$ and any $\nu>0$, there 
exists some $N$ such that
\begin{equation}
\Pr\left\{\left|\uvpWH_n-\uvp\right|<\epsilon\right\}>1-\nu \qquad\forall n>N
\end{equation}
where $n$ is the number of measurements in $\md$ combined to form the
estimate $\uvpWH_n$.  

An estimator is unbiased if its expectation is equal to the unknown
parameter
\begin{equation}
\E{\uvpWH}=\uvp
\end{equation}
This is a stronger condition than consistency, because a consistent
estimator with a finite expectation is asymptotically unbiased as the number 
of measurements tends to infinity.  A consistent estimator is not, in
general, unbiased for a finite number of measurements.

If an estimator is not unbiased, it is biased.  The
expectation of a biased estimator is 
\begin{equation}
\E{\uvpWH}=\uvp+\vect{b}(\uvp)
\end{equation}
where $\vect{b}(\uvp)$ is the bias.  Depending on the form of
$\vect{b}(\uvp)$, it may be possible to modify the estimator
so that it becomes unbiased.

Bias and consistency are important criteria, but they are not the only 
criteria by which estimators should be judged.  When $\uvp$ has a number of
unbiased estimators, the optimal estimator is the one which makes best use
of the information contained in the set of measurements.\footnote{Fisher was
critical of the method of moments for ignoring efficiency and using
consistency as the only criterion, saying ``estimates of the parameters may
be obtained $\ldots$ but they are often estimates of little value''
\protect\cite{Fis25}.}

Fisher measured an estimator's efficiency as the ratio of the estimator's
variance to the most efficient estimator's variance, expressed as a
percentage.  Then an efficient estimator, with an efficiency of $100\,\%$,
has the lowest variance of all consistent estimators.  Another estimator,
with an efficiency of $\eta\,\%$, requires approximately $100 N/\eta$
measurements to have the same variance as the efficient estimator has with $N$
measurements.

Terminology has evolved since Fisher's papers on statistical estimation
theory.  Optimal estimators are now minimum variance unbiased
estimators (MVUE) and efficient now means a MVUE which also
attains the \CR bound.  Rao \cite[p. 283]{Rao65} complains that ``logically,
this does not appear to be a good definition'' because the \CR bound is only 
one among many lower bounds.  The word ``efficient'' has such strong
non-mathematical connotations that it seems wrong not calling a
minimum-variance estimator efficient if the \CR bound is unobtainable by any
estimator.

The difficulty with using the MVUE criterion to select an estimator is that
there is no general method for minimizing $\Var{\uvpWH}$ over all unbiased
estimators.  All that can be done is analyse different estimators to compare
their biases and variances.

%%%%%%%%%%%%%%%%%%%%%%%%%%%%%%%%%%%%%%%%%%%%%%%%%%%%%%%%%%%%%%%%%%%%%%
\section{The Cram\'{e}r-Rao Bound}

The \CR bound provides a simple lower bound on the variance of any unbiased
estimator $\uvpWH$.  It is stated first for a scalar parameter $\up$, then 
for a vector parameter $\uvp$, and finally, it is restated as a lower bound
on the mean square error of any biased estimator.  The \CR bound is
sometimes called the minimum variance bound, or MVB \cite{Stu91}.

%%%%%%%%%%%%%%%%%%%%%%%%%%%%%%%%%%%%%%%%%%%%%%%%%%%%%%%%%%%%%%%%%%%%%%
\subsection{The Scalar \CR Bound}

Let $\md$ be a set of measurements used to estimate an unknown parameter
$\up$ where the probability density function $p(\md;\up)$ is known and the 
partial derivatives
$$\frac{\del p(\md;\up)}{\del\up}$$
and
$$\frac{\del^2p(\md;\up)}{\del \up^2}$$
exist and are absolutely integrable.
If $\upWH$ is an unbiased estimator of $\up$, the variance of $\upWH$
is always bounded below by the \CR bound
\begin{equation}
\Var{\upWH}\geq \left[\E{\left(\frac{\del\ln p(\md;\up)}
{\del \up}\right)^2}\right]^{-1}
=\left[-\E{\frac{\del^2\ln p(\md;\up)}
{\del\up^2}}\right]^{-1}
\end{equation}

The \CR bound is satisfied with equality exactly when 
\begin{equation}\label{mp eqn:NSCforCRB}
\frac{\del\ln p(\md;\up)}{\del\up}=k(\up)\left(\upWH-\up\right)
\end{equation}
for some function $k(\up)$ not dependent on $\md$.
If this is the case, the estimator has the lowest variance of
all unbiased estimators and is called efficient.

%%%%%%%%%%%%%%%%%%%%%%%%%%%%%%%%%%%%%%%%%%%%%%%%%%%%%%%%%%%%%%%%%%%%%%
\subsection{The Vector \CR Bound}

When the unknown parameter $\uvp$ is a vector, the \CR bound is expressed in
terms of the Fisher information matrix $\infmat{\uvp}$.  The elements of
$\infmat{\uvp}$ are
\begin{eqnarray}
I_{ij}&=&\E{\frac{\del\ln p(\md;\uvp)}{\del\ui{i}}
\frac{\del\ln p(\md;\uvp)}{\del \ui{j}}}\nn\\
&=&-\E{\frac{\del^2\ln p(\md;\uvp)}{\del\ui{i}\del\ui{j}}}
\end{eqnarray}
Then, following the notation used by Bhattacharyya in \cite{Bha46},
if $I^{ii}$ is the $i^{\rm th}$ element on the diagonal of 
$\invinfmat{\uvp}$, the vector \CR bound is
\begin{equation}
\Var{\uiWH{i}}\geq I^{ii}
\end{equation}
provided $\uiWH{i}$ is an unbiased estimator of $\ui{i}$.  This follows from 
the more general result that the covariance of $\uvpWH$,
defined by
\begin{equation}
\Cov{\uvpWH}=\E{(\uvpWH-\uvp)(\uvpWH-\uvp)^T}
\end{equation}
satisfies 
\begin{equation}
\Cov{\uvpWH}-\invinfmat{\uvp}\geq 0
\end{equation}
which is read as ``the matrix $\Cov{\uvpWH}-\invinfmat{\uvp}$
is positive semidefinite.''

The vector \CR bound is satisfied with equality if and only if 
\begin{equation}
\uiWH{i}-\ui{i}=\sum_{j=1}^M k_{ij}(\uvp)\frac{\del\ln p(\md;\uvp)}{\del
\ui{j}}
\end{equation}
for all $\md$ and all $i$ from $1$ to $M$ and for some set of functions 
$k_{ij}(\uvp)$ not dependent on $\md$.


%%%%%%%%%%%%%%%%%%%%%%%%%%%%%%%%%%%%%%%%%%%%%%%%%%%%%%%%%%%%%%%%%%%%%%
\subsection{The Scalar \CR Bound for a Biased Estimator}

When the estimator $\upWH$ is biased, the \CR bound becomes a lower bound 
on the mean square error
\begin{equation}
\E{\left(\upWH-\up\right)^2}\geq 
\frac{\ds\left(1+\frac{db(\up)}{d\up}\right)^2}
{\ds\E{\left(\frac{\del\ln p(\md;\up)}{\del \up}\right)^2}}
\end{equation}
Note that if $-2<db(\up)/d\up<0$, the mean square error of the 
biased estimator may be lower than the variance of an unbiased estimator.

%%%%%%%%%%%%%%%%%%%%%%%%%%%%%%%%%%%%%%%%%%%%%%%%%%%%%%%%%%%%%%%%%%%%%%
\subsection{Some Other Bounds}

If an efficient estimator does not exist, the \CR bound sets a lower limit
on an estimator's variance that cannot be attained.  This makes it difficult
to find how close a particular estimator is to the MVUE.

A number of other bounds may be used instead of the \CR bound to give
greater lower bounds on the variance of any unbiased estimator.  The
Bhattacharyya bound \cite{Bha46,Bha47,Bha48} gives a tighter bound than the
\CR bound by considering higher-order partial derivatives of $p(\md;\up)$.
The Barankin bound~\cite{Bar49} is a more general bound on the estimator's
$s^{\rm th}$ central moments that includes both the \CR and the Bhattacharyya
bounds as special cases.

One of the problems with these bounds is that improving the variance's 
lower bound comes at the price of increased complexity.  Therefore the \CR
bound is often the only bound considered, even when no efficient estimator
exists.

%%%%%%%%%%%%%%%%%%%%%%%%%%%%%%%%%%%%%%%%%%%%%%%%%%%%%%%%%%%%%%%%%%%%%%
\section{Maximum Likelihood Estimation}

In maximum likelihood estimation, the population's probability density 
function $p(\md;\uvp)$ dependent on the unknown parameters is called the likelihood
function\footnote{Fisher would not approve of this!  In \protect\cite{Fis30},
he complained ``In spite of the emphasis I have always laid upon the
difference between probability and likelihood there is still a tendency to
treat likelihood as though it were a sort of probability.'' Probability
relates a population with known parameters to an unknown sample.  Likelihood
relates a known sample to a population with unknown parameters.  The
conceptual difference is important, but mathematically it is just a scaling 
factor which usually may be ignored.}

Since there is no {\em a priori\/} knowledge about $\uvp$, the most likely 
value of $\uvp$ is that which maximizes $p(\md;\uvp)$.  If $p(\md;\uvp)$ 
has a unique maximum over the domain of $\uvp$, the maximum likelihood 
estimate of $\uvp$ is
\begin{equation}
\uvpML=\arg\max_{\uvp}\;p(\md;\uvp)
\end{equation}
The maximum likelihood estimator $\uvpML$ is also a solution to the 
likelihood equation
\begin{equation}
\left.\frac{\del p(\md;\uvp)}{\del\ui{i}}\right|_{\uvp=\uvpML}=0
\qquad\forall\, i=1,2,\ldots, M
\end{equation}

Since $\ln x$ is strictly monotonically increasing with $x$, the global 
maximum of $p(\md;\uvp)$ occurs at the same value of $\uvp$ as the
global maximum of $\ln p(\md;\uvp)$.  The maximum likelihood estimator may
be found by maximizing $\ln p(\md;\uvp)$
\begin{equation}
\uvpML=\arg\max_{\uvp}\;\ln p(\md;\uvp)
\end{equation}
or equivalently by solving the log-likelihood equation
\begin{equation}
\left.\frac{\del\ln p(\md;\uvp)}{\del\ui{i}}\right|_{\uvp=\uvpML}=0
\qquad\forall\, i=1,2,\ldots, M
\end{equation}

Note that the global maximum of $p(\md;\uvp)$ must be unique for the
maximum likelihood estimator to exist.

If the maximum likelihood estimator is unbiased and 
$\del\ln p(\md;\uvp)/\del\ui{i}$ can be written in the form of equation 
(\ref{mp eqn:NSCforCRB}), the maximum likelihood estimator satisfies the
\CR bound with equality.  

If $\del\ln p(\md;\uvp)/\del\ui{i}$ cannot be written in this form,
all that can be said is that the maximum likelihood estimator's variance is
greater than the \CR lower bound.  Other estimators may have lower variances 
than $\uvpML$, but there is no simple way of finding them.

Conversely, if an efficient estimator exists, then it is necessarily the
maximum likelihood solution $\uvpML$ which is the unique solution of the
likelihood equation.

In general, $\uvpML$ is neither unbiased nor efficient.  But as the number of
data samples tends to infinity,
\begin{equation}
\uvpML\to\uvp
\end{equation}
and
\begin{equation}
\Cov{\uvpML}\to\invinfmat{\uvp}
\end{equation}
so $\uvpML$ is asymptotically unbiased and asymptotically efficient,
and tends to a normal distribution with mean $\uvp$ and variance equal to 
the \CR lower bound
\begin{equation}
\uvpML\stackrel{d}{\sim}{\cal N}\left(\uvp,\invinfmat{\uvp}\right)
\end{equation}
as $N\rightarrow\infty$.  The $\stackrel{d}{\sim}$ indicates that
probability distribution of $\uvpML$ is asymptotically equal to the normal
distribution on the right-hand side.


In order for the maximum likelihood estimator to be asymptotically normal,
the likelihood function must satisfy a number of regularity conditions.  

One approach \cite[p. 175]{Ngu89b} combines the likelihood function
$p(\md;\uvp)$ and the regions of support of $\md$ and $\uvp$ to form a
model.   If the model satisfies seven regularity conditions, then it is
called a regular model.  Finally, a necessary and sufficient condition for
the asymptotic normality of the maximum likelihood estimator is that it is
formed from a regular model.

A simpler set of regularity conditions can be used if they are to be
sufficient but not necessarily necessary.  Examples of sufficient sets of
regularity conditions are given by \cite[section 5f.1]{Rao65} and 
\cite[appendix 7B]{Kay93}.  The following theorem from \cite{Kay93} is for a
scalar parameter.

%============================================================================
\begin{theorem}[Asymptotic Properties of the MLE]
\label{mp thm:normality}
Given a likelihood function $p(\md;\up)$, the maximum likelihood 
estimator $\upML$ of the unknown parameter $\up$ has a probability density
function for large data records that is asymptotically normal 
\begin{equation}
\upML\stackrel{d}{\sim}{\cal N}\left(\up,\invinfmat{\up}\right)
\end{equation}
if the first and second-order partial derivatives of the log-likelihood
function are well-defined and
\begin{equation}
\E{\frac{\del\ln\,p(\md;\up)}{\del\up}}=0
\end{equation}
\end{theorem}
%============================================================================

\begin{proof}
See appendix 7B of \cite{Kay93}.
\end{proof}

Similar regularity conditions apply for maximum likelihood estimators
$\uvpML$ of vector parameters $\uvp$ provided that the estimators of each of
the parameters average the measurements so that the central limit theorem
applies.

%%%%%%%%%%%%%%%%%%%%%%%%%%%%%%%%%%%%%%%%%%%%%%%%%%%%%%%%%%%%%%%%%%%%%%%%%%%%%
\section{Bayesian Estimation}


Maximum likelihood estimation assumes that nothing is known about the
unknown parameters $\uvp$ before the measurements $\md$ are made.  This
means that in order to use maximum likelihood estimation, any {\em a priori\/}
information must be discarded.

Bayesian estimation, a completely different estimation philosophy from maximum
likelihood estimation, assumes that the unknown parameters are random.  The
prior information takes the form of an {\em a priori\/} probability density
function $p(\uvp)$ on the unknown parameters.

This prior knowledge can be combined with the model's conditional
probability density $p(\md|\uvp)$ to give the joint probability density
function
\begin{equation}
p(\md,\uvp)=p(\md|\uvp)\,p(\uvp)
\end{equation}

The performance of a Bayesian estimator is measured a little differently
from a maximum likelihood estimator.  The minimum variance criterion is not
appropriate because the $\uvp$ in 
\begin{equation}
\Var{\uvpWH}=\E{(\uvpWH-\uvp)^2}
\end{equation}
is a random parameter, not a constant.  However, this concept can be 
generalized to random parameters using the Bayes risk $\cal R$.

Define the error $\bayeserror$ as the difference between the random parameters 
and their estimates
\begin{equation}
\bayeserror=\uvp-\uvpWH
\end{equation}
for one particular set of measurements $\md$ and realization of $\uvp$.  Then 
the seriousness of making a particular error is measured using a cost
function ${\cal C}(\bayeserror)$.  The Bayes risk associated with a
particular estimator $\uvpWH$ and cost function ${\cal C}(\bayeserror)$ is the
expected cost averaged over all values of $\md$ and $\uvp$
\begin{equation}
{\cal R}=\int\!\!\int {\cal C}(\uvp-\uvpWH)\,p(\md,\uvp)\,d\md\,d\uvp
\end{equation}
Minimizing $\cal R$ over all estimators gives the Bayesian estimator for
that particular cost function.

%When the cost function is quadratic, ${\cal C}(\bayeserror)=\bayeserror^2$, the
%Bayesian estimator is the the minimum mean-square error (MMSE),
%which is the mean of the posterior probability density
%\begin{equation}
%\uvpWH=\int \up\,p(\uvp|\md)\,d\uvp
%\end{equation}
%When the cost function is absolute, ${\cal C}(\bayeserror)=\left|\bayeserror\right|$,
%the Bayesian estimator is the median of the posterior probability density
%\begin{equation}
%\Pr\left(\uvp\leq\uvpWH\,|\,\md\right)=\frac{1}{2}
%\end{equation}
When the cost function is $1$ unless the error is less than a threshold 
$\delta$, 
\begin{equation}
{\cal C}(\bayeserror)=\cases{0 &  $\left|\bayeserror\right|<\delta$\cr
1 &  $\left|\bayeserror\right|\geq\delta$}
\end{equation}
the Bayesian estimator in the limit $\delta\rightarrow 0$ is the value of 
$\uvp$ that maximizes the posterior density function
\begin{equation}
\uvpWH=\arg\max_{\uvp}\,p(\uvp|\md)
\end{equation}
which is called the maximum {\em a posteriori\/} (MAP) estimator.

%It can be shown that if the cost function is symmetric and convex, and the
%{\em a posteriori\/} probability density $p(\uvp|\md)$ is unimodal and is
%symmetric about its conditional mean, then these three Bayesian estimators
%are equivalent.

%Therefore, when the prior distribution $p(\up)$ of $\up$ is Gaussian 
%and the measurement noise is Gaussian, and one of these three cost functions
%is used, the Bayesian estimator may be expressed as the MAP estimator
%$$\upH=\arg\max_{\up}\,p(\up|\md)$$

Using Bayes' rule,
\begin{equation}
p(\uvp|\md)=\frac{p(\md|\uvp)\,p(\uvp)}{p(\md)}
\end{equation}
Since the denominator is independent of $\uvp$, the MAP estimator can be
written
\begin{equation}
\uvpWH=\arg\max_{\uvp}\,p(\md|\uvp)\,p(\uvp)
\end{equation}

%%%%%%%%%%%%%%%%%%%%%%%%%%%%%%%%%%%%%%%%%%%%%%%%%%%%%%%%%%%%%%%%%%%%%%%%%%%%%
\subsection{Diffuse Priors}
\label{mp sec:dp}

If the {\em a priori\/} probability density is uniform 
\begin{equation}
p(\uvp)=\frac{1}{\epsilon}	
\end{equation}
over a sufficiently large region of volume $1/\epsilon$, and zero elsewhere,
the MAP estimator coincides with the maximum likelihood estimator in the limit that
$\epsilon\to 0$.  A $p(\uvp)$ like this is called a diffuse prior
\cite{Bar88}.

In the special case that $p(\uvp)$ is constant on some closed subset
$\UVP$, the MAP estimator becomes
\begin{equation}
\uvpWH=\arg\max_{\uvp\in\UVP}\,p(\md|\uvp)
\end{equation}
Provided that $\uvpML\in\UVP$, this has the same form as the maximum 
likelihood estimator
\begin{equation}
\uvpML=\arg\max_{\uvp}\,p(\md;\uvp)
\end{equation}
because the {\em a posteriori\/} probability density function $p(\md|\uvp)$
has the same form as the likelihood function $p(\md;\uvp)$.
This provides the justification for using maximum likelihood estimation over
a restricted domain $\UVP$ which is known {\em a priori\/} to contain $\uvp$.

%%%%%%%%%%%%%%%%%%%%%%%%%%%%%%%%%%%%%%%%%%%%%%%%%%%%%%%%%%%%%%%%%%%%%%%%%%%%%
\section{Estimation with Complex Parameters}
\label{mp sec:complex}

When the unknown parameter in an estimation problem is a complex vector 
$\uvp$, the maximum likelihood estimator $\uvpML$ and the Fisher information matrix 
$\infmat{\uvp}$ may be found by forming a new vector $\tilde{\uvp}$ of real 
parameters
\begin{equation}
\tilde{\uvp}=\left[\begin{array}{c}\real{\uvp}\\\imag{\uvp}\end{array}\right]
\end{equation}
and proceed as for the real case to find the maximum likelihood estimator and Fisher
information matrix of $\tilde{\uvp}$.

A more satisfying approach is to reformulate real parameter estimation 
for complex parameters.  This is a brief summary of the results
obtained by Kay in section 15.7 of \cite{Kay93}.

The derivative of a function $f(z)$ with respect to the complex argument $z$
is defined as
\begin{equation}\label{mp eqn:complex deriv}
\frac{\del f(z)}{\del z}=\frac{1}{2}
\left(\frac{\del f(x+jy)}{\del x}-j\frac{\del f(x+jy)}{\del y}\right)
\end{equation}
where $z=x+jy$ with $x$ and $y$ real.  Then it is easy to show that
\begin{equation}
\frac{\del z}{\del z}=1 \quad\quad\mbox{and}\quad\quad
\frac{\del\conj{z}}{\del z}=0
\end{equation}
One surprising consequence of this is that
\begin{equation}
\frac{\del\,|z|^2}{\del z}=\conj{z}
\end{equation}
with no factor of $2$ as in the real case.

The maximum likelihood estimator $\uvpML$ is, as for real $\uvp$,
\begin{equation}
\uvpML=\arg\max_{\uvp}\,p(\md;\uvp)
\end{equation}
If $\uvpWH$ is any unbiased estimator of $\uvp$, the \CR bound is
\begin{equation}
\Cov{\uvpWH}\geq \invinfmat{\uvp}
\end{equation}
which is interpreted to mean that $\Cov{\uvpWH}-\invinfmat{\uvp}$ is
positive semidefinite.  The elements of the covariance matrix are
\begin{equation}
\left(\Cov{\uvpWH}\right)_{ij}=\E{\left(\uiWH{i}-\ui{i}\right)
\conj{\left(\uiWH{j}-\ui{j}\right)}}
\end{equation}
and the elements of the Fisher information matrix $\infmat{\uvp}$ are
\begin{equation}
I_{ij}=\E{\frac{\del\ln p(\md;\uvp)}{\del \conj{\ui{i}}}
\frac{\del\ln p(\md;\uvp)}{\del \ui{j}}}
=-\E{\frac{\del^2\ln p(\md;\uvp)}{\del \conj{\ui{i}}\del \ui{j}}}
\end{equation}
