%%%%%%%%%%%%%%%%%%%%%%%%%%%%%%%%%%%%%%%%%%%%%%%%%%%%%%%%%%%%%%%%%%%%%%%%%%%
%
%			INVERSE SYNTHETIC APERTURE RADAR
%
%				  PhD Thesis
%
%		Stephen Simmons		simmons@ee.mu.oz.au
%
%	    Department of Electrical and Electronic Engineering
%	    University of Melbourne, Parkville, 3052, Australia
%
% Chapter 5:	Motion Estimation and Motion Compensation 
%
%		started first draft:	31 Dec 1994
%		finished first draft:	 4 Jan 1995
%		submitted:		
%
%%%%%%%%%%%%%%%%%%%%% Copyright (C) 1995 Stephen Simmons %%%%%%%%%%%%%%%%%%

\chapter{Motion Estimation and Motion Compensation}
\label{mc chp}


\typeout{Something about stepped-frequency ISAR}

\bigletter Since the target's motion creates the synthetic aperture in ISAR
imaging, a high cross-range resolution can only be obtained if the position
and orientation of the target matches that assumed by the ISAR inversion
algorithm.  Furthermore, the ISAR inversion can be implemented efficiently
only if the target moves in one of two basic ways; either rotating with
uniform angular velocity a fixed distance from the radar, or moving along a
straight line at constant speed.  If the target deviates by even a small
fraction of a wavelength from its presumed position, the ISAR image becomes
badly blurred in the cross-range direction.  

Consequently, the first and most delicate part of ISAR imaging is motion
estimation, the process of measuring the target's position and orientation
at all times from its radar reflections.  Once the inversion algorithm has
been chosen, motion compensation is performed.  This adjusts the radar
measurements to compensate for deviations from the target's presumed
location and aspect angle.  Finally, the compensated radar measurements
match those expected by the ISAR inversion algorithm and the ISAR image can
be formed.

In this chapter, these ideas about motion estimation and motion compensation
are explored in detail.  After describing how to apply motion compensation
to correct for deviations in the target's range and aspect angle, various
methods of motion estimation suitable for ISAR are discussed.   

In chapter \ref{ml chp}, a general model for radial motion estimation is
developed and this is used to find the maximum likelihood solution of the
ISAR motion estimation problem. After chapter \ref{ee chp} constructs
efficient algorithms for evaluating the maximum likelihood motion
estimator, and chapter \ref{sp chp} analyses the estimator's statistical
properties, chapter \ref{do chp} shows how some of the radial motion
estimators in this chapter are related to the maximum likelihood radial
motion estimator.

%%%%%%%%%%%%%%%%%%%%%%%%%%%%%%%%%%%%%%%%%%%%%%%%%%%%%%%%%%%%%%%%%%%%%%%%%%%%%
\section{Motion Errors}

When an ISAR inversion algorithm estimates a target's spatial radar 
reflectivity, $g(x,y)$, from a set of radar measurements 
\begin{equation}
s(k_n,t_m)\quad\mbox{for $n=0,1,\ldots,N-1$ and $m=0,1,\ldots,M-1$}
\end{equation}
the measurements are related to the target's unknown reflectivity $g(x,y)$ 
according to the model in equation (\ref{ii eqn:s(k,t) II})
\begin{equation}
s(k,t)=\int\!\!\int g(x,y)\,e^{-jkr_{xy}(t)}\,dx\,dy
\end{equation}
where $r_{xy}(t)$ is the distance between a scatterer at $(x,y)$ on the
target and the radar at time $t$.  The inversion algorithm assumes a 
certain form for $r_{xy}(t)$, and if the target's motion deviates from this,
the resulting ISAR image will be blurred.

If at time $t$, the target's centre is located a radial distance $r(t)$ from
the radar and is oriented so that its $x$ axis is at an angle $\theta(t)$
to the line between the target and the radar antenna, $r_{xy}(t)$ is
\begin{equation}\label{mc eqn:rxy(t)}
r_{xy}(t)=\sqrt{\left(r(t)+x\cos\theta(t)-y\sin\theta(t)\right)^2
+\left(x\sin\theta(t)+y\cos\theta(t)\right)^2}
\end{equation}
This shows that $r(t)$ and $\theta(t)$ are the important factors in ensuring
that the target's motion corresponds to the model.  

Suppose an ISAR inversion algorithm requires a target moving in such a way
that its radial position and orientation are $r_0(t)$ and $\theta_0(t)$ 
respectively.  But as the target may be moving slightly differently, its
actual radial position and orientation are given by the functions $r(t)$
and $\theta(t)$.  

The problem now is to estimate $r(t)$ and $\theta(t)$ sufficiently
accurately that $r_{xy}(t)$, the range of any point on the target at any
time, can be calculated with an accuracy of a fraction of a wavelength, and
then to adjust the radar measurements to account for any differences between
$r(t)$ and $r_0(t)$ and between $\theta(t)$ and $\theta_0(t)$.

The second part of this problem, that of motion compensation, is discussed
first because it is considerably simpler than motion estimation.

%%%%%%%%%%%%%%%%%%%%%%%%%%%%%%%%%%%%%%%%%%%%%%%%%%%%%%%%%%%%%%%%%%%%%%%%%%%%%
\section{Motion Compensation}


Motion compensation has to adjust the radar measurements $s(k,t)$ to
compensate for the target being a distance $r(t)$ from the radar when it
should be a distance $r_0(t)$ away, and having an orientation $\theta(t)$
rather than the $\theta_0(t)$ expected by the inversion algorithm.  

The corrections are made in two stages; first radial motion compensation is
used to correct for deviations in the target's range, then angular motion
compensation is used to correct for deviations in the target's aspect
angle.\footnote{Sometimes these stages are called translational motion
compensation (TMC) and rotational motion compensation (RMC)
\protect\cite{Boc91}.  These names are slightly misleading when the target
is moving in a straight line at constant speed because a deviation in the
target's translational motion changes both its range and aspect angle.  TMC
corrects errors in the target's range and RMC corrects errors in the
target's aspect angle.  However TMC's name suggests that it can correct
errors in the target's translational motion, which involves corrections  
both in range and in aspect angle.}

For radar imaging at X-band, radial motion compensation is mandatory because
a moving target's range is almost guaranteed to deviate from the ideal by
more than $\lambda/10=3$~mm.  Angular motion compensation will probably not
be necessary if the target is small and the ISAR image has a low enough
cross-range resolution that polar reformatting can be left out of the
inversion algorithm.  This is fortunate because radial motion affects the
whole target in the same way and is comparatively easy to estimate.  On the
other hand, the target's angular motion affects different parts of the
target in different ways and is much harder to estimate.


%%%%%%%%%%%%%%%%%%%%%%%%%%%%%%%%%%%%%%%%%%%%%%%%%%%%%%%%%%%%%%%%%%%%%%%%%%%%%
\subsection{Radial Compensation}
\label{mc sec:rc}

At first glance, $r_{xy}(t)$ in (\ref{mc eqn:rxy(t)}) is a complicated
function of $r(t)$ and $\theta(t)$, so correcting deviations from both
$r_0(t)$ and $\theta_0(t)$ appears very difficult.  However, using the 
simple assumptions that the size of the target is small in comparison to its 
range
\begin{equation}
\sqrt{x^2+y^2}\ll r(t)
\end{equation}
and that deviations from the assumed radial position are also relatively
small
\begin{equation}
\left|r_0(t)-r(t)\right|\ll r_0(t)
\end{equation}
the range to a scatterer at $(x,y)$ can be written in terms of $r_0(t)$ 
\begin{eqnarray}
r_{xy}(t)
&=&\sqrt{\left(r(t)+x\cos\theta(t)-y\sin\theta(t)\right)^2
+\left(x\sin\theta(t)+y\cos\theta(t)\right)^2}			   \nn\\
&\approx&\left[r(t)-r_0(t)\right]				   \\
&&\mbox{}+\sqrt{\left(r_0(t)+x\cos\theta(t)-y\sin\theta(t)\right)^2
+\left(x\sin\theta(t)+y\cos\theta(t)\right)^2}			   \nn
\end{eqnarray}
This means that when the target moves a small radial distance,
the ranges of all points on the target change by the same amount.

Since the reflection from a scatterer at $(x,y)$ undergoes a phase change of
$-kr_{xy}(t)$, a radial deviation of
\begin{equation}
\Delta r(t)=r(t)-r_0(t)
\end{equation}
causes an extra phase shift of $-k\Delta r(t)$.  This phase error is
common to the reflections from all scatterers on the target, so the 
measurement $s(k,t)$ at frequency $k$ has suffered an additional phase shift
of $-k\Delta r(t)$.  This can be corrected by multiplying $s(k,t)$ by
$e^{jk\Delta r(t)}$ to give
\begin{equation}
s'(k,t)=e^{jk\Delta r(t)}\,s(k,t)
\label{mc eqn:rmc}
\end{equation}
where $s'(k,t)$ is $s(k,t)$ with radial motion compensation applied.

%%%%%%%%%%%%%%%%%%%%%%%%%%%%%%%%%%%%%%%%%%%%%%%%%%%%%%%%%%%%%%%%%%%%%%%%%%%%%
\subsection{Angular Compensation}


Once radial motion compensation has been applied to the radar measurements,
they have to be corrected for deviations in the target's aspect angle.
The measurement $s(k,t)$ with radial motion compensation applied, $s'(k,t)$, 
is given by
\begin{equation}
s'(k,t)=\int\!\!\int g(x,y)\,e^{-jkr'_{xy}(t)}\,dx\,dy
\end{equation}
where $r'_{xy}(t)$ is $r_{xy}$ with radial motion compensation
\begin{eqnarray}
r'_{xy}(t)
&=&r_{xy}(t)-\Delta r(t)			\nn\\
&=&\sqrt{\left(r_0(t)+x\cos\theta(t)-y\sin\theta(t)\right)^2
+\left(x\sin\theta(t)+y\sin\theta(t)\right)^2}\qquad
% The quad at the end should move the equ. marginally to the left
\label{mc eqn:rrxy(t)}
\end{eqnarray}
Correcting the measurements so that the target's actual aspect angle 
$\theta(t)$ in (\ref{mc eqn:rrxy(t)}) becomes the presumed aspect angle 
$\theta_0(t)$ is more complicated because the dependence of $r'_{xy}(t)$ 
on $\theta(t)$ varies with $x$ and $y$.
  
The only way of rotating the target so that its aspect angle changes from 
$\theta(t)$ to $\theta_0(t)$ is to warp the time axis.  Then when the
inversion algorithm requires the target's frequency response at time $t$,
the actual frequency response used is $s'(k,t')$ where the target's aspect
angle at $t'$ is equal to the desired aspect angle at time $t$
\begin{equation}
\theta(t')=\theta_0(t)
\end{equation}
So $t'$ is related to $t$ by a function $t'=\tau(t)$ which accomplishes the
time-warping
\begin{equation}
\tau(t)=\theta^{-1}\left(\theta_0(t)\right)
\end{equation}

Therefore the radar measurements with both radial and angular motion
compensation are
\begin{equation}
s_0(k,t)=s'(k,\tau(t))
\end{equation}
Since the radar measurements $s(k,t)$ are only made at discrete times $t_m$,
the radially-compensated measurements $s'(k,t_m)$ generally have to be 
resampled in the time domain to give the fully motion-compensated $s_0(k,t)$.

For range-Doppler ISAR inversions, the angular coordinate of $s'(k,t)$ in
$(k_x,k_y)$-space is given by the orientation of the target at time $t$.  
Therefore angular motion compensation is the same as moving the angular
coordinates of each measurement $s'(k,t)$ by $\theta_0(t)-\theta(t)$.
This shows that angular motion compensation and polar reformatting are very
similar operations, so they can be performed together during the polar
reformatting stage of range-Doppler ISAR inversions.\footnote{Angular motion
compensation must be performed before ISAR inversions using wavenumber
processing because step 2 of the wavenumber inversion (algorithm 
\protect\ref{ii alg:w-k} on page \protect\pageref{ii alg:w-k}) is a Fourier
transform of $s(k,t)$ with respect to time.  This requires that the target
have the correct orientation $\theta_0(t)$ at each time.}

One slight complication arises from angular motion compensation because
radial motion compensation moves $s(k,t)$ to a range of $r_0(t)$.  However,
angular motion compensation pretends that $s(k,t)$ was actually sampled at
time $\tau^{-1}(t)$ so the measurement should have had its range adjusted to
$r_0(\tau^{-1}(t))$.  
When the radial motion compensation's phase correction is combined with 
angular motion compensation's time-warping, the desired measurements at time
$t$, $s_0(k,t)$, are found from the actual measurements, $s(k,t)$, according
to
\begin{equation}
s_0\left(k,t\right)=e^{jk\Delta r(t)}\,s\left(k,\tau(t)\right)
\end{equation}
where $\Delta r(t)$ has been modified to take into account the time-warping
\begin{equation}
\Delta r(t)=r(\tau(t))-r_0(t)
\end{equation}
This complication does not arise when the inversion algorithm presumes that
the target is rotating uniformly a fixed distance from the radar, but it is
important for the alternative model of a target moving in a straight line at
constant speed.

%%%%%%%%%%%%%%%%%%%%%%%%%%%%%%%%%%%%%%%%%%%%%%%%%%%%%%%%%%%%%%%%%%%%%%%%%%%%%
\section{Motion Estimation in the Range and Frequency Domains}

Motion estimation methods are classified as frequency domain or range domain 
methods depending on whether the target's frequency responses $s(k,t)$ or its
range profiles $s(r,t)$ are used.  As shown later in this section, range
profiles are the inverse Fourier transforms of the frequency responses with
respect to $k$ 
\begin{equation}
s(r,t)={\cal F}^{-1}_k\left\{s(k,t)\right\}
\end{equation}

This section introduces the terminology of radial motion estimation and
shows how a target's frequency responses and range profiles change as its
range from the radar changes.  The relationships developed in equations 
(\ref{mc eqn:fr change}) and (\ref{mc eqn:rp change}) are the basis for
the radial motion estimation algorithms which are described in detail in
section \ref{mc sec:me algs}.

%%%%%%%%%%%%%%%%%%%%%%%%%%%%%%%%%%%%%%%%%%%%%%%%%%%%%%%%%%%%%%%%%%%%%%%%%%%%%
\subsection{Frequency Domain Motion Estimation}

At each one of the $M$ discrete times $t_m$ during the ISAR imaging period
$[-T/2,T/2]$, the target's radar reflectivity $s(k_n,t_m)$ is measured at
the $N$ frequencies $k_n=4\pi f_n/c$ in the stepped-frequency waveform. 
These $N$ measurements, sometimes collectively called a sweep, constitute
one sample of the target's frequency response over the frequency range
$[k_0,k_{N-1}]$.

The goal of motion estimation is to measure the target's range $r(t)$
during $[-T/2,T/2]$.  The target's average range during this period only
needs to be known approximately but changes in its range must be known
within a small fraction of a wavelength.  So any motion estimation method
must be capable of estimating changes in range this accurately.  

From the discussion of radial motion compensation in section \ref{mc sec:rc}, 
a measurement $s(k,t)$ has a phase shift $e^{-jk\Delta r}$ if the target
moves a radial distance $\Delta r$.  This phase shift can be used for
estimating the radial distance an ISAR target moves over a period of time
providing the target rotates a negligible amount during this
period.\footnote{From here until section \protect\ref{rmc sec:me}, where
this assumption is properly tested, the target's rotation will be assumed 
negligible.}  If the moving target is rotating slowly enough,
\begin{equation}\label{mc eqn:fr change}
s(k,t_1)\approx e^{-jk[r(t_2)-r(t_1)]}\,s(k,t_2)
\end{equation}
and the target's radial motion can be detected via the phase shift in its
frequency responses measured at different times.

Motion estimation methods using the frequency responses like this to measure
$r(t_2)-r(t_1)$ are said to be operating in the frequency domain.  

%%%%%%%%%%%%%%%%%%%%%%%%%%%%%%%%%%%%%%%%%%%%%%%%%%%%%%%%%%%%%%%%%%%%%%%%%%%%%
\subsection{Range Domain Motion Estimation}

Motion estimation can also be performed using Fourier transforms of the
frequency responses and such methods of motion estimation are called
range domain methods.  This is because the inverse Fourier transform of
$s(k,t)$ with respect to $k$ gives $s(r,t)$, a one-dimensional image of the
target called a range profile.  

Each measurement $s(k,t)$ in the frequency response is related to the
target's spatial reflectivity $g(x,y)$ according to equation (\ref{ii
eqn:s(k,t) II}), which is an integral representation of $s(k,t)$ in terms of
the $(x,y)$ coordinates of scatterers on the target.  Radial motion
estimation is concerned with the radial position of the target, so $s(k,t)$
is more usefully expressed as an integral over $r$, the radial distance 
from the radar
\begin{equation}\label{mc eqn:s(k,t) via g(r,t)}
s(k,t)=\int g(r,t)\,e^{-jkr}\,dr
\end{equation}
Here $g(r,t)$ is the target's reflectivity at time $t$ as a function of 
radial position
\begin{equation}\label{mc eqn:g(r,t)}
g(r,t)=\int\!\!\int g(x,y)\,\delta(r_{xy}(t)-r)\,dx\,dy
\end{equation}
where the delta function $\delta(r_{xy}(t)-r)$ selects those points on the
target that are a radial distance $r$ from the radar.

From equation (\ref{mc eqn:s(k,t) via g(r,t)}), $s(k,t)$ is the Fourier 
transform with respect to $r$ of $g(r,t)$ so $g(r,t)$ can be recovered by 
taking the inverse Fourier transform with respect to $k$ of the $s(k,t)$. 
However, $s(k,t)$ is only known for the $N$ discrete frequencies in the
stepped-frequency waveform so $s(r,t)$, the inverse Fourier transform of
$s(k,t)$, is different from $g(r,t)$.
\begin{eqnarray}
s(r,t)
&=&\frac{1}{N}\sum_{n=0}^{N-1} s(k_n,t)\, e^{jk_nr}		  \nn\\
&=&\int g(r',t)\,\left[\frac{1}{N}\sum_{n=0}^{N-1} e^{jk_n(r-r')}\right]
	\,dr'  \nn\\
&=&\int g(r',t)\,h(r-r')\,dr'	\nn\\
&=&g(r',t)\,\conv\, h(r')
\label{mc eqn:s(r,t)}
\end{eqnarray}
which is a convolution with kernel $h(r)$ 
\begin{eqnarray}
h(r)&=&\frac{1}{N}\sum_{n=0}^{N-1} e^{jk_nr} 		\nn\\
    &=&e^{j\overline{k}r}\,\frac{\ds\sin\left(\frac{N\Delta kr}{2}\right)}
		{\ds N\sin\left(\frac{\Delta kr}{2}\right)}		
\end{eqnarray}
where $\overline{k}=(k_0+k_{N-1})/2$ is the average frequency of the
stepped-frequency waveform.  This shows that $s(r,t)$ is $g(r,t)$ smoothed
with a resolution equal to the stepped-frequency waveform's range resolution
\begin{equation}
\Delta r_r=\frac{2\pi}{N\Delta k}=\frac{c}{2N\Delta f}
\end{equation}
For efficiency, range profiles are usually calculated from their 
corresponding frequency responses using an inverse fast Fourier transform
(FFT).  This gives $s(r,t)$ sampled at $N$ discrete ranges $r_n$ 
\begin{equation}
r_n=r_0+n\Delta r_r
\end{equation}
where each discrete range is called a range cell or a range bin.
The range profiles $s(r_n,t)$ are cyclic because the FFT on which they are
based have a period of $N\Delta r_r=c/2\Delta f$.   The length of each range
profile is $W_r$, the length of what is called the range ambiguity window 
\begin{equation}
W_r=N\Delta r_r=\frac{c}{2\df}
\end{equation}
because scatterers separated in range by multiples of $W_r$ appear
at the same position in the range profile.  Because of this, the origin of
the range profile $r_0$ is arbitrary.  This usually does not matter because
motion estimation and motion compensation are concerned with estimating
changes in radial position, not absolute ranges.

This detailed examination of range profiles will now be used to show how
range profiles can be used for radial motion estimation.  If a target's 
range changes from $r(t_1)$ to $r(t_2)$ between times $t_1$ and $t_2$ 
respectively, its radial reflectivities at $t_1$ and $t_2$ are 
\begin{equation}
g(r+r(t_2),t_2)\approx g(r+r(t_1),t_1)
\end{equation}
provided its aspect angle changes by a negligible amount between $t_1$
and $t_2$.  Therefore, from equation (\ref{mc eqn:s(r,t)}), the two range 
profiles $s(r,t_1)$ and $s(r,t_2)$ are related by
\begin{equation}\label{mc eqn:rp change}
s(r,t_2)\approx e^{-j\overline{k}[r(t_2)-r(t_1)]}\,s(r-[r(t_2)-r(t_1)],t_1)
\end{equation}
Motion estimation methods based on differences between $s(r,t_1)$ and
$s(r,t_2)$ like this are called range domain motion estimators.


%%%%%%%%%%%%%%%%%%%%%%%%%%%%%%%%%%%%%%%%%%%%%%%%%%%%%%%%%%%%%%%%%%%%%%%%%%%%%
\section{Radial Motion Estimation}
\label{mc sec:me algs}

This section is a detailed examination of some radial motion estimation
algorithms that have been used for ISAR imaging.  They are described from
the perspective of a stepped-frequency radar, so the measurements for one
ISAR image takes the form of $M$ sets of frequency responses, one measured 
at each of 
$M$ uniformly spaced times $t_m$ during $[-T/2,T/2]$.  Each frequency
response contains $s(k_n,t_m)$ sampled at the $N$ frequencies $k_n$ in the
stepped-frequency waveform.\footnote{If the radar uses a chirp instead of a
stepped-frequency waveform, the measurements consist of $M$ sets of range
profiles $s(r_n,t_m)$, where each profile consists of $N$ discrete range bins
separated by $\Delta r_r$.}

None of the methods presented here is designed to estimate changes in a
target's radial position with an accuracy better than $\lambda/10$. 
Generally, radial motion estimators can be divided into coarse estimators,
which tend to have an accuracy of one range bin $\Delta r_r$, and fine
estimators, that are capable of measuring the fractional part of the number
of wavelengths that the target moves.  For reasons that will shortly become
apparent, motion compensation using the coarse estimators is often described 
as range profile alignment and motion compensation using the fine estimators is
called phase compensation.

%%%%%%%%%%%%%%%%%%%%%%%%%%%%%%%%%%%%%%%%%%%%%%%%%%%%%%%%%%%%%%%%%%%%%%%%%%%%%
\subsection{Range Profile Alignment and Phase Compensation}

From equation (\ref{mc eqn:rmc}), the measurement $s(k,t)$ with a radial
motion of $\Delta r(t)$ corrected is
\begin{equation}
s'(k,t)=e^{jk\Delta r(t)}\,s(k,t)
\end{equation}
When $k_n$ is written as $k_0+n\dk$, this equation becomes
\begin{equation}
s'(k_n,t)=e^{jk_0\Delta r(t)}\,e^{jn\dk\Delta r(t)}\,s(k_n,t)
\end{equation}
Since the radar has a small relative bandwidth, 
$k_0\gg n\dk$ for all frequencies in the stepped-frequency waveform 
and $e^{jn\dk\Delta r(t)}$ varies only a very small amount for $\Delta r(t)$
less than $\Delta r_r$, the size of one range bin.  On the other hand, 
$e^{jk_0\Delta r(t)}$ varies rapidly for changes in $\Delta r(t)$ that are 
less than a wavelength and is only sensitive to $\Delta r(t)$ modulo
$\lambda/2$.

This suggests that radial motion estimation needs to be able to estimate
changes in the target's position at the scale of a range bin and at the
scale of a fraction of a wavelength.  If motion compensation leaves a
residual motion error that is smaller than a range bin and an exact
half-integral multiple of a wavelength, this is probably sufficient for
ISAR imaging.\footnote{This assumption, one of the many ``self-evident''
truths of ISAR imaging, is tested properly in section \protect\ref{rmc
sec:ma}.}

On this basis, most radial motion estimators can be classified as either a
method of range profile alignment or a method of phase compensation
depending on whether they are accurate to the nearest multiple of a range
bin or accurate within a fraction of a wavelength.

Most methods of range profile alignment are unable to use the target's 
frequency responses directly but convert the frequency responses to range
profiles (or if the ISAR uses a chirp instead of a stepped-frequency
waveform, the range profiles are obtained directly) and measure the radial
motion using 
\begin{equation}
\left|s(r,t_2)\right|\approx 
\left|s(r-\dr,t_1)\right|
\end{equation}
which is the magnitude of equation (\ref{mc eqn:rp change}).  Therefore the
offset between the magnitudes of two range profiles at different times is 
equal to the radial distance moved by the target.  Range profile alignment in
the frequency domain is more difficult because $s(k_n,t_1)$ and $s(k_n,t_2)$
differ by $e^{jk_n\dr}$.  If $\dr>\lambda/2$, the phase changes are greater
than $2\pi$ and cannot be measured without potential ambiguities.

Phase compensation methods can operate equally well in the frequency domain
\begin{equation}
s(k_n,t_2)\approx e^{-jk_n\dr} s(k_n,t_1)
\end{equation}
or in the range domain
\begin{equation}
s(r,t_2)\approx e^{-j\overline{k}\dr} s(r-\dr,t_1)
\end{equation}
because $\left|s(r-\dr,t_1)\right|$ is approximately equal to 
$\left|s(r,t_1)\right|$ when $\dr<\lambda/2\ll\Delta r_r$.  

The following sections analyse two methods of range profile alignment,
cross-correlation and image feedback control, and two methods of phase
compensation, adaptive beamforming and the phase gradient
autofocus.\footnote{Of course there is also the possibility of ISAR imaging
without having to do any motion estimation or motion compensation.  Zhao
\protect\cite{Zha92} tried ISAR imaging using only the magnitude of the
target's frequency response.  Certainly this removes the phase shift due to
the target's radial motion but it also removes most of the information used
to resolve scatterers in the cross-range direction.  Zhao's examples were
only $8\times 8$ and $16\times 16$ pixels so it is hard to tell anything
meaningful from them.  However, common sense suggests that it would be more
sensible to develop effective motion estimation methods and then use the
target's phase information.}


%%%%%%%%%%%%%%%%%%%%%%%%%%%%%%%%%%%%%%%%%%%%%%%%%%%%%%%%%%%%%%%%%%%%%%%%%%%%%
\subsection{Cross-Correlation}
\label{mc sec:cc}

Cross-correlation is a simple method of estimating the radial offset between
range profiles at different times, hence estimating the radial distance that
the target moves.  This was first used by Chen and Andrews \cite{Che80a} and
later by Wehner \cite{Weh87}.  They noted that because
\begin{equation}
\left|s(r,t_2)\right|\approx\left|s(r-\dr,t_1)\right|
\end{equation}
$\dr$ could be estimated by cross-correlating the magnitudes of the two
range profiles
\begin{equation}
R_{12}(r)=\int\left|s(r',t_1)\right|\,\left|s(r'+r,t_2)\right|\,dr'
\end{equation}
Then by the Cauchy-Schwartz inequality,
\begin{equation}
R_{12}(r)\leq R_{12}(\dr)
\end{equation}
if there is no noise, so the location of the peak of the cross-correlation 
gives an estimate of $\dr$.

Since the range profiles are discrete, the cross-correlation is also
discrete with a resolution of $\Delta r_r$, or one ISAR range bin.  The
discrete correlation is usually implemented as a fast correlation via a fast
Fourier transform.  The correlations implemented in this way are cyclic, but
this is appropriate because the discrete range profiles themselves are
cyclic as they have been formed by transforming the target's frequency
responses.

The inherent resolution of the correlation is $\Delta r_r$ because this is
the spacing between samples of the discrete range profiles, but
interpolation can be used to give estimates of $\dr$ that lie 
between the discrete points at which the correlation is available.  Two
methods have been suggested by Xu {\em et al.\/} \cite{Xu89,Xu90a,Xu90b}.
One is interpolation by padding the range profiles with zeros before
applying the fast correlation.  Another is to measure the cross-correlation
at the discrete range bin where it is maximum, and at the range bins either
side of the maximum.  By fitting a quadratic curve between the value of the
cross-correlation at these three points, the location of the maximum of the 
quadratic curve
can be used as a better estimate of $\dr$.\footnote{It is surprising that
the chirp-Z transform (CZT) has not been used to zoom into any portion of the
cross-correlation between range profiles.  This approach is taken in 
algorithm \protect\ref{ee alg:min using CZT} in chapter \protect\ref{ee chp} 
to accomplish essentially the same task.}

Like any method of motion estimation, cross-correlating range profiles has 
advantages and drawbacks.  It is quick and simple and usually works well. 
However, the amplitude of the response from strong scatterers can vary
considerably with time if they are affected by scintillation or their
response varies as they move between resolution cells.  This may cause the
cross-correlation to match a prominant scatterer in one profile with a
different prominant scatterer in another profile, giving estimates
of $\dr$ with very large errors.

%%%%%%%%%%%%%%%%%%%%%%%%%%%%%%%%%%%%%%%%%%%%%%%%%%%%%%%%%%%%%%%%%%%%%%%%%%%%%
\subsection{Image Feedback Control}
\label{mc sec:en}

In \cite{Ste83}, Steinberg proposed using image feedback control, a
combination of motion estimation and motion compensation which uses the
sharpness of the resulting radar image as a measure of how well the
target's motion has been estimated.  Steinberg mentioned a number of
techniques that might potentially be useful for ISAR imaging, but concluded
that ``none has yet been satisfactorily developed for radarlike waveforms.''

Bocker {\em et al.\/} have proposed a similar method of ISAR motion
estimation that uses the quality of the resulting ISAR image to gauge the
target's motion \cite{Boc91}.  The target is assumed to be moving in such a
way that its range can be described by a low-degree polynomial such as
\begin{equation}
r_0(t)=r_0+v_0t+\frac{1}{2}a_0t^2+\frac{1}{6}j_0t^3
\end{equation}
where $r_0$, $v_0$ and $a_0$ are the target's radial position, velocity and
acceleration at time $t=0$.  The third derivative of the target's position
is given by $j_0$, which is sometimes called the jerk component.

Bocker's method tries to find the set of parameters $(v_0,a_0,j_0)$
describing the target's motion by choosing some $(v,a,j)$ and
reconstructing the ISAR image using those $(v,a,j)$ as the parameters of
the target's motion for motion compensation.  The best estimate of the
target's motion is found by searching over the three-dimensional $(v,a,j)$
parameter space for the values which give the highest quality ISAR image.

The quality measure used by Bocker was
\begin{equation}
\label{mc eqn:ent}
H(v,a,j)=-\int\!\!\int \left|\widehat{g}(x,y)\right|^2
\log\left|\widehat{g}(x,y)\right|^2\,dx\,dy
\end{equation}
where $\widehat{g}(x,y)$ is the ISAR image of the target reconstructed with
the specified velocity, acceleration and jerk.  The target's position $r_0$
cannot be estimated in this way because the quality of the image is
independent of its position.  The same measure was also used by Flores
\cite{Flo90} in an investigation of some of the ideas in Bocker's
paper.\footnote{Bocker and Flores call this an entropy-based method, but
this is a bit of a misnomer. 
It is true that (\protect\ref{mc eqn:ent}) has the mathematical form for
differential entropy
$$H(X,Y)=-\int\!\!\int f(x,y)\,\log f(x,y)\, dx\, dy$$
but in this context, $f(x,y)$ is the joint probability density function of
the random variables $X$ and $Y$.  

The ``entropy'' in (\protect\ref{mc eqn:ent}) is just one of many non-linear
functions that could have been used to measure the ISAR image's quality,
where the notion of quality is determined by the particular non-linear 
function used.}

While this method of motion estimation seems attractive, it has three
serious difficulties:
\begin{itemize}
\item The conditions under which the search converges to the correct 
value of $(v,a,j)$ have not been established. The few examples of motion
estimation using entropy methods in the ISAR literature use simple targets
composed of a few point scatterers where the target's radial motion is
precisely a cubic polynomial.
\item Because the multidimensional search through $(v,a,j)$ parameter
space requires that an ISAR image be formed and its entropy measured
for each estimate, this is a very slow method of motion estimation and
motion compensation.  
\item The algorithm makes no use of the information about the target's
radial motion that can be obtained much more efficiently by more direct 
methods of motion estimation.
\end{itemize}
In view of these problems, such entropy-based methods are not yet suitable
for motion estimation and motion compensation in ISAR imaging.


%%%%%%%%%%%%%%%%%%%%%%%%%%%%%%%%%%%%%%%%%%%%%%%%%%%%%%%%%%%%%%%%%%%%%%%%%%%%%
\subsection{Adaptive Beamforming}
\label{mc sec:ab}

Adaptive beamforming is a range-domain method of phase compensation.  Using
a number of range profiles that have been motion-compensated using some method of range
profile alignment, adaptive beamforming can correct for residual radial motions
of fractions of a wavelength. 

Once the range profiles have been aligned so that the target is at
essentially the same radial position in each range profile, the main
difference between range profiles is a phase shift due to small residual
errors in the target's corrected radial position or due to atmospheric
turbulence.  From equation (\ref{mc eqn:rp change}), range profiles at
times $t_m$ and $t_{M/2}$ (the range profile at $t_{M/2}$ is used as a
reference range profile) differ by a phase shift, $\phi_m$, that is common
to all pairs of range bins at time $t_m$
\begin{equation}
s(r_n,t_m)= e^{-j\phi_m}\,s(r_n,t_{M/2})\qquad\forall m,n
\end{equation}

Adaptive beamforming measures each $\phi_m$ by finding a range
resolution cell containing a single strong scatterer that can be used as a
phase reference.  With this reference range bin at $r_d$, for some $0\leq
d\leq N-1$, the phase shift of each range profile is estimated using
\begin{equation}
\widehat{\phi}_m=\arg\left(\frac{s(r_d,t_{M/2})}{s(r_d,t_m)}\right)
\end{equation}
Finally, each range profile's phase shift is corrected by multiplying
$s(r_n,t_m)$ in each range bin by $e^{-j\widehat{\phi}_m}$ to give
\begin{equation}
s'(r_n,t_m)=e^{-j\widehat{\phi}_m}\,s(r_n,t_m)
\end{equation}

The particular algorithm described here is that developed by
Steinberg \cite{Ste81,Ste83b} for focusing his radio camera \cite{Ste83}.
Sometimes the algorithm is called the dominant scatterer algorithm (DSA)
because the strongest scatterer is used to establish the phase reference.
Sometimes the algorithm is called the minium variance algorithm (MVA)
because the range bin containing the dominant scatterer is determined by
finding the range bin whose amplitude has the least variance over all range
profiles.  A similar form of phase compensation using the first strong
scatterer, such as an aircraft's leading wingtip, as the phase reference,
was described by Chen and Andrews in \cite{Che80a}.  

A slightly modified dominant scatterer algorithm has been used by Haywood
for phase compensation in ISAR imaging \cite{Hay89,Hay92a}.  Haywood
changed the criterion for finding the range bin containing the
dominant scatterer so that it is the range bin with the smallest
variance of all range bins whose average amplitudes are above a threshold.  The
addition of the threshold helps prevent the algorithm from selecting a
reference range bin that is not part of the target.

Adaptive beamforming methods of phase compensation are widely used in ISAR
imaging, but they do have some drawbacks.  One disadvantage is that there is
no guarantee that the reference range bin is suitable for use as a phase
reference.

A more serious disadvantage of adaptive beamforming is that it assumes that
the measured responses in the reference range bin have identical phases in
all range profiles.  Since the reference range bin is selected because its
magnitude has a low variance, the responses in this range bin after
beamforming are approximately constant across all range profiles.

Information for the target's cross-range profile is contained in both
the phases and the magnitudes of the evolution of each range bin over time.
By finding the range bin whose magnitude has the smallest variance, adaptive 
beamforming selects as its reference the range bin whose magnitude
contains the least information.  Then by removing all of the phase
information in the reference range bin, the result is no variation in phase
and minimal variation in magnitude.  

When the cross-range profile of the target is formed by taking the
Fourier transform of each range bin across all range profiles, the
cross-range profile in the reference range bin is nearly a delta function. 
While adaptive beamforming focusses the rest of the target, it artificially
increases the power of the dominant scatterer by suppressing other
scatterers in that range bin.\footnote{Distortions such as this may
unwittingly make adaptive beamforming appear to perform better than it
actually does.  ISAR images are often displayed with a fixed dynamic range
relative to the brightest scatterer in the image.  Because adaptive
beamforming increases the power of the dominant scatterer, the powers of
other scatterers and of the background noise decrease relative to that of
the dominant scatterer.  This makes the image appear sharper when really it
is just being displayed with an offset intensity scale.}

%%%%%%%%%%%%%%%%%%%%%%%%%%%%%%%%%%%%%%%%%%%%%%%%%%%%%%%%%%%%%%%%%%%%%%%%%%%%%
\subsection{Phase Gradient Autofocus}
\label{mc sec:pga}

The phase gradient autofocus (PGA) is an effective method of phase
compensation that has been developed for SAR imaging by Eichel, Ghiglia and
Jakowatz \cite{Eic89,Eic89b,Jak89}.  It estimates the modulo-$2\pi$
variation with time of phase errors due to slight residual errors in the
corrected position of the SAR antenna.

The basis of the phase gradient autofocus is the observation that if
$s_n(t)=s(r_n,t)$ is the time-history of the response in the $n^{\rm th}$
range bin,
\begin{equation}
s_n(t)=e^{j\phi(t)}\,g_n(t)
\end{equation}
where $\phi(t)$ is the time-varying phase error which affects all range bins
in a range profile in the same way.  $g_n(t)$ is the range bin's time-history whose Fourier
transform gives the cross-range profile $g_n(\omega)$ in the $n^{\rm th}$ 
range bin
\begin{equation}
g_n(\omega)={\cal F}\{g_n(t)\}
\end{equation}
If each range bin contains a single point scatterer, then
\begin{equation}
s_n(t)=A_ne^{j\omega_nt}\,e^{j\phi(t)}
\end{equation}
where $A_n$ is the scatterer's amplitude and $\omega_n$ determines its
cross-range position.  Therefore, the time-derivative of the phase of
$s_n(t)$ is
\begin{equation}\label{mc eqn:ddt arg sn(t) I}
\frac{d}{dt}\,\arg\left(s_n(t)\right)=\omega_n+\dot{\phi}(t)
\end{equation}
where $\dot{\phi}(t)$ is the derivative of the phase error.  This derivative
can be implemented using the identity
\begin{equation}\label{mc eqn:ddt arg sn(t) II}
\frac{d}{dt}\,\arg\left(s_n(t)\right)=
\frac{\imag{\dot{s}_n(t)\overline{s_n(t)}}}{\left|s_n(t)\right|^2}
\end{equation}
Using this, the non-constant part of (\ref{mc eqn:ddt arg sn(t) II}) can be
used to estimate $\dot{\phi}(t)$ within a constant.  Integrating this
gives an estimate of $\phi(t)$ within a linear term
\begin{equation}
\widehat{\phi}(t)=\int\widehat{\dot{\phi}}(t)\,dt\approx \phi(t)+at+b
\end{equation}
The linear term $at+b$ does not affect the SAR image's focus, only its
position.

This only uses the information contained in one range bin to estimate
$\phi(t)$, so a more accurate estimate can be obtained by using all range
bins.  From (\ref{mc eqn:ddt arg sn(t) I}), the time-derivative of $s_n(t)$
has a constant term of $\omega_n$, which is different for each range bin.
These can be removed by preprocessing each $s_n(t)$.  First, the scatterer's
location is found by forming the cross-range profile (which is slightly
blurred due to $\phi(t)$), and the cross-range profile is shifted so that
the scatterer is at the origin.  When the shifted cross-range profile is
transformed back to the time-domain, $s'_n(t)$, the preprocessed $s_n(t)$,
is just $s_n(t)$ with the linear shift due to the scatterer at $\omega_n$
removed.  Thus 
\begin{equation}
s'_n(t)=A_n\,e^{j\phi(t)}
\end{equation}
and $\dot{\phi}(t)$ may be estimated using
\begin{equation}\label{mc eqn:phidot sum}
\frac{d}{dt}\,\arg\left(s'_n(t)\right)=
\frac{\ds\sum_n\imag{\dot{s}'_n(t)\overline{s'_n(t)}}}
{\ds\sum_n\left|s'_n(t)\right|^2}
\end{equation}

The full phase gradient autofocus algorithm is more complicated than this
because SAR images contain more than  a single scatterer in each range
bin.  The actual time-history of each range bin is 
\begin{equation}
s_n(t)=\sum_pA_{pn}e^{j\omega_{pn}t}\,e^{j\phi(t)}
\end{equation}
where the scatterers in the range bin at $r_n$ have amplitudes $A_{pn}$ and
cross-range positions determined by $\omega_{pn}$.  Now the derivative of
the argument of $s_n(t)$ is
\begin{equation}\label{mc eqn:ddt arg sn(t) III}
\frac{d}{dt}\,\arg\left(s_n(t)\right)=\dot{\phi}(t)+
\frac{d}{dt}\,\arg\left(\sum_pA_{pn}e^{j\omega_{pn}t}\right)
\end{equation}
The second term of this is no longer constant, so it cannot be removed so
simply from $\dot{\phi}(t)$.  This problem is resolved by preprocessing
$s_n(t)$ slightly differently.  The cross-range profile is formed and the
scatterer with the greatest amplitude is moved to the origin (once again
this can only be done approximately because the phase error blurs the
cross-range profile).  The shifted cross-range profile is windowed so that
only the portion of the cross-range profile close to the origin is retained. 
Finally the windowed and shifted cross-range profile is transformed back to
the time domain to give $s'_n(t)$.

The combination of shifting and windowing is the crucial part of this
process because this changes the second term of (\ref{mc eqn:ddt arg sn(t)
III}) to
\begin{equation}
\frac{d}{dt}\,\arg\left(\sum_{p:|\omega_{pn}-\omega_n|<W}
A_{pn}e^{j(\omega_{pn}-\omega_n)t}\right)
\end{equation}
where $\omega_n$ is the cross-range position $\omega_{pn}$ of the dominant 
scatterer in the $n^{\rm th}$ range bin and $W$ is half the width of the 
window.

The width of the window has to be chosen carefully.  If the window is
too wide, the estimate of $\dot{\phi}(t)$ will be badly affected by the
second term of (\ref{mc eqn:ddt arg sn(t) III}).  If the window is too
narrow, the second term of (\ref{mc eqn:ddt arg sn(t) III}) will be small
but the high frequency content of $\phi(t)$ will also be lost.

A compromise between using a large and a small window is reached by running
the phase gradient autofocus as an iterative algorithm.  The window is
initially large.  At each successive iteration, the size of the window
is decreased and the estimate of $\phi(t)$ at that iteration used to correct
the phase of the range profiles before the next iteration.  Typically five
to ten iterations are required to focus the SAR image.

Therefore, the full phase gradient autofocus algorithm is:

%============================================================================
\begin{algorithm}[Phase Gradient Autofocus]
\label{mc alg:PGA}\mbox{}\par

To correct for a phase error $e^{j\phi(t)}$ affecting range profiles
$s(r_n,t)$:
\begin{enumerate}
\item Set $i$, the iteration counter, to 1 and set the initial width of
the window $W$ to $W_1$.

\item For each range bin at $r_n$, take the Fourier transform of $s(r_n,t)$
with respect to $t$ to give the cross-range profile $s(r_n,\omega)$.

\item In each range bin, $r_n$, locate the strongest scatterer at
$\omega=\omega_n$ and shift the cross-range profile so that the strongest
scatterer is at the origin, giving
\begin{equation}
s'(r_n,\omega)=s(r_n,\omega-\omega_n)
\end{equation}

\item Window each $s'(r_n,\omega)$, keeping only the portion of the shifted
profile with $\left|\omega\right|<W_i$, and transform back to the time
domain to give $s'(r_n,t)$.

\item Estimate the derivative of the remaining phase error using
\begin{equation}
\widehat{\dot{\phi}}_i(t)=
\frac{\ds\sum_n\imag{\dot{s'}(r_n,t)\overline{s'(r_n,t)}}}
{\ds\sum_n\left|s'(r_n,t)\right|^2}
\end{equation}

\item Integrate $\widehat{\dot{\phi}}_i(t)$ and remove any bias or linear
trend to give this iteration's estimate of the remaining phase error,
$\widehat{\phi}_i(t)$.

\item Correct for $\widehat{\phi}_i(t)$ in the range profiles, replacing
$s(r_n,t)$ by $e^{-j\widehat{\phi}_i(t)}s(r_n,t)$.

\item Decrease the size of the window from $W_i$ to $W_{i+1}$ and, after
incrementing $i$ for the next iteration, return to step 2.
\end{enumerate}
\end{algorithm}
%============================================================================


Examples of phase gradient autofocus's performance on SAR data are given in
the papers cited earlier and in \cite{Wah94}.  These seem to indicate that
the phase gradient autofocus works very well for autofocussing SAR images.

The phase gradient autofocus has also been applied to ISAR imaging by
Robertson and Munson \cite{Rob93}.  Their simulations were of a target 
composed of a few point scatterers, so the results are not necessarily
representative of what would happen with real ISAR data.  

The success that phase gradient autofocus has had with SAR suggests that it
should be considered for ISAR phase compensation, especially as it does not
suffer from some of the distortions that adaptive beamforming
introduces.\footnote{It may be that the phase gradient autofocus's shifting
and windowing can be left out when applied to ISAR targets because they
have a much narrower cross-range extent than the cross-range extent of a
SAR image.  This and many other questions about the statistical performance
and robustness of the phase gradient autofocus can only be answered by a
rigorous mathematical analysis, something that is currently lacking.}

%%%%%%%%%%%%%%%%%%%%%%%%%%%%%%%%%%%%%%%%%%%%%%%%%%%%%%%%%%%%%%%%%%%%%%%%%%%%%
\section{Angular Motion Estimation}

Angular motion estimation is a much more difficult problem than radial
motion estimation because each scatterer's phase history depends on its
location within the target as well as the target's overall rotation.

In \cite{Wer90}, Werness {\em et al.\/} propose an algorithm for imaging
moving targets in SAR data.  This involves tracking three point scatterers
on the target.  The first point scatterer is used to compensate for the
target's radial motion, the second point scatterer is used to detect
deviations from uniform rotation and the third point scatterer is used to
estimate the angular velocity to set the cross-range scaling for the whole
image.

While this is a nice idea, the examples in \cite{Wer90} were of very simple
simulated point scatterers which indicate little about how the algorithm
would perform with real radar data.  In \cite{Del94}, Delisle and Wu comment
that ``selection of multiple reference points and evaluation of the
quadratic phase component are difficult in practice because of the target's
scintillation, clutter and trajectory perturbations.'' 

A simpler problem than complete angular motion estimation is the fitting of
a low-degree polynomial 
\begin{equation}
\theta(t)=\omega t+\frac{\alpha t^2}{2}
\end{equation}
to the target's aspect angle over time.  Bocker {\em et al.\/} \cite{Boc91}
proposed a multidimensional search over the two-dimensional
$(\omega,\alpha)$ parameter space that is essentially the same as their
entropy method of image feedback control for radial motion estimation. By
measuring the sharpness of the reconstructed ISAR image for each value of
$(\omega,\alpha)$, the best estimate of the target's rotational motion is
given by the image with the least blurring.  A simplification of Bocker's
idea has been experimentally tested by Flores, who tried to estimate
$\omega$ for a target composed of three aluminimum cylinders rotating at a
known angular velocity \cite{Flo90}.  Flores' results indicate that for
this simple target under very controlled conditions, the rotation rate
could be measured accurately.

Chuang and Huang \cite{Chu92} used a similar scheme to estimate $\omega$ by
comparing tomographic inversions with different assumed rotation rates. They
reasoned that the image reconstructed with the correct value of $\omega$
should have minimum variance.  Since their experimental results consisted of
simulations involving a single point scatterer in no noise, few conclusions
can be drawn from it.

The difficulty of angular motion estimation is indicated by the fact that no
results have been published for angular motion estimation for real targets
under uncontrolled conditions.  

Since the subject of this thesis is radial motion estimation rather than 
angular motion estimation, because angular motion estimation is of
secondary importance to radial motion estimation in ISAR, and because no
demonstrably successful angular motion estimators have been published in
the unclassified literature, none of the angular motion estimators
mentioned above will be discussed in detail.

