%%%%%%%%%%%%%%%%%%%%%%%%%%%%%%%%%%%%%%%%%%%%%%%%%%%%%%%%%%%%%%%%%%%%%%%%%%%
%
%			INVERSE SYNTHETIC APERTURE RADAR
%
%				  PhD Thesis
%
%		Stephen Simmons		simmons@ee.mu.oz.au
%
%	    Department of Electrical and Electronic Engineering
%	    University of Melbourne, Parkville, 3052, Australia
%
% Chapter 7:	Evaluating the Maximum Likelihood Estimator
%
%		started first draft:	Thu 6 Oct 1994
%		finished first draft:	Sun 9 Oct 1994
%		started second draft:	Tue 3 Jan 1995
%		submitted:		Mon 9 Jan 1995
%
%%%%%%%%%%%%%%%%%%%%% Copyright (C) 1995 Stephen Simmons %%%%%%%%%%%%%%%%%%

\chapter[Evaluating the ML Motion Estimator]{Evaluating the Maximum Likelihood Motion Estimator}
\label{ee chp}

\bigletter One problem with maximum likelihood estimation is that the
estimator is found by maximizing the likelihood function.  In some cases,
this maximization problem gives the estimator as an explicit function of the
measurements
\begin{equation}
\uvpML=f(\md)
\end{equation}
Unfortunately, estimating a target's radial motion $\dr$ is not such a
case.  The maximum likelihood estimator $\rML$ is found by minimizing the 
cost function
\begin{equation}\label{ee eqn:J(r)}
J(\r)=\sum_{n=0}^{N-1} \left|a_n-b_n\e{jk_n\r}\right|^2
\end{equation}
but the machinery of maximum likelihood estimation does not reveal how to
minimize $J(\r)$, or even say whether $\rML$ can be found with
sufficient accuracy for ISAR motion estimation at an acceptably low cost in
both time and computational resources.

The straightforward approach of maximizing a function using Newton-Raphson
iteration or gradient descent is not practical because $J(\r)$ has many
closely-spaced local minima.  An order-of-magnitude calculation shows that 
the motion estimation for a $128\times 128$ pixel ISAR image taken at
X-band would require about one billion floating point arithmetic operations
if one of these methods were used.

Clearly, an exhaustive search is not suitable for finding the global minimum
of $J(\r)$.  This chapter examines the problem of minimizing $J(\r)$ and
proposes a number of efficient methods.   Each method uses a global 
minimization scheme to find the approximate location of $\rML$ followed by a
local minimization scheme which converges to $\rML$.

%%%%%%%%%%%%%%%%%%%%%%%%%%%%%%%%%%%%%%%%%%%%%%%%%%%%%%%%%%%%%%%%%%%%%%%%%%%%%
\section{Global Minimization of $J(\r)$}
\label{ee sec:agm}

Two algorithms for the approximate global minimization of $J(\r)$ are
presented here.  The first algorithm is based on a discrete Fourier
transform (DFT) and gives an approximation $\rWH$ to the global minimum of
$J(\r)$ that is at most $c/2N\Delta f$, or one ISAR range bin, away from
$\rML$.

The second algorithm uses a chirp-Z transform (CZT) \cite{Rab69b,Rab69a} 
to interpolate part of the Fourier spectrum of the first
method, and so determine $\rML$ with an accuracy much better than one ISAR range
bin.  The parameters of the chirp-Z transform can be adjusted to select the
resolution and the portion of the Fourier spectrum zoomed into.

These algorithms are efficient because they look only at the envelope of
$J(\r)$, ignoring the high frequency carrier.  They can be implemented
efficiently using the fast Fourier transform if the
frequencies in the stepped-frequency waveform are uniformly spaced, as in
(\ref{ml eqn:f0+ndf})
\begin{equation}\label{ee eqn:fn}
k_n=k_0+n\dk\qquad\mbox{for $n=0,1,\ldots,N-1$}
\end{equation}

Note that these global minimizations of $J(\r)$ are not using the discrete
Fourier transform as a spectral estimator.  Rather one form of the expression
for $J(\r)$ is as the real part of a continuous-range\footnote{\label{ee
ftn:crft}The
continuous-time Fourier series gives a continuous-time representation of a 
function with a discrete frequency spectrum.  Here, the two domains are
range and frequency, not time and frequency.  Hence $J(\r)$ is a
continuous-range Fourier series.} Fourier series.  The maximum likelihood
estimate of $\dr$ is found by locating the maximum of the Fourier spectrum.
Since this is really implementing a correlation, it is misguided to try
modern spectral estimators with the hope of achieving a higher resolution 
estimate of $\dr$.

%%%%%%%%%%%%%%%%%%%%%%%%%%%%%%%%%%%%%%%%%%%%%%%%%%%%%%%%%%%%%%%%%%%%%%%%%%%%%
\subsection{Global Minimization Using a DFT}

The following algorithm describes an efficient way to find the approximate
location $\rWH$ of the global minimum of $J(\r)$, which occurs when
$\r=\rML$.  The estimate $\rWH$ is guaranteed to be within
$c/2N\Delta f$ of the true global minimum at $\rML$.

%============================================================================
\begin{algorithm}[Global Minimization of $J(\r)$ using a DFT]
\label{ee alg:min using DFT}\mbox{}\par

To obtain a rough estimate $\rWH$ of the location of the global minimum of 
$J(\r)$:
\begin{enumerate}
\item From the set of measurements $\md$ in (\ref{ml eqn:md}), form 
\begin{equation}
c_n=a_n\overline{b_n}\qquad\forall n=0,1,\ldots, N-1
\end{equation}

\item Take the discrete Fourier transform (in the form of an FFT) of 
the $\{c_n\}$ to give the vector $\{C_m\}$
\begin{equation}
C_m={\cal F}\{c_n\}
\end{equation}

\item Find the index of the absolute maximum of the $\{C_m\}$
\begin{equation}
k=\arg\,\max_m\,\left|C_m\right|
\end{equation}

\item A rough estimate $\rWH$ of the location of the global minimum of
$J(\r)$ is
\begin{equation}
\rWH=k\,\frac{c}{2N\Delta f}
\end{equation}
\end{enumerate}
\end{algorithm}
%============================================================================

The next theorem is a formal statement of the accuracy of $\rWH$.  
Note that it is a theorem concerning the maximum distance between $\rWH$ and
$\rML$, not the distance between $\rWH$ and $\dr$.  That depends on the bias
and variance of $\rML$, and is discussed in chapter~\ref{sp chp}.

%============================================================================
\begin{theorem}[Global Minimization of $J(\r)$ using a DFT]
\label{ee thm:min using DFT}\mbox{}\par

Let $\rML=\arg\min_{\r}J(\r)$ where $J(\r)$ is the cost function given by
(\ref{ee eqn:J(r)}) and the stepped-frequency waveform is given by
(\ref{ee eqn:fn}).  Then $\rWH$, the rough estimate of $\rML$ calculated
using algorithm~\ref{ee alg:min using DFT}, is less than one ISAR range bin from
$\rML$
\begin{equation}
\left|\rWH-\rML\right|<\frac{c}{2N\Delta f}
\end{equation}
\end{theorem}
%============================================================================

%============================================================================
\begin{proof}
See appendix~\ref{ee app:min using DFT}.
\end{proof}
%============================================================================

%%%%%%%%%%%%%%%%%%%%%%%%%%%%%%%%%%%%%%%%%%%%%%%%%%%%%%%%%%%%%%%%%%%%%%%%%%%%%
\subsection{Global Minimization Using a CZT}


The procedure in algorithm~\ref{ee alg:min using DFT} determines $\rML$ to
within one ISAR range bin, which is typically orders of magnitude larger
than the small fraction of a wavelength required for ISAR motion
estimation.  The procedure is also inefficient because the minimum of
$J(\r)$ is sought over the whole ISAR range ambiguity window.  Most
targets, even those moving as swiftly as aircraft, can move only a few
range bins in the time between measuring consecutive frequency responses. 
It would make sense to examine only part of the range ambiguity window, and
do so with a higher resolution than algorithm~\ref{ee alg:min using DFT}.

The resolution of the DFT can be increased by appending zeros to the end of
the data sequence before transforming.  This interpolates the discrete 
frequency spectrum, giving a a discrete spectrum which more closely
approximates the data's continuous Fourier transform.  Note that the
discrete spectrum has more samples, so the spectral peaks may be found more
accurately, but the fundamental frequency resolution of the transform is
unchanged because this depends on the interval between the measurements,
which stays the same.  (This is illustrated convincingly in figure~4 of 
\cite{Kay81}.)

While zero padding can be used to interpolate the Fourier spectrum and so 
achieve more accurate estimates of $\rML$, the increase in resolution comes
at the cost of longer vectors to be transformed.  Another disadvantage of
zero-padding is that the Fourier spectrum is still calculated over the whole
of the range ambiguity window.

Both of these difficulties with the DFT may be overcome by using a chirp-Z
transform to compute the Fourier spectrum over an arbitrary portion of the
range ambiguity window with an arbitrary resolution.

The chirp-Z transform, described by Rabiner, Schafer and Rader in
\cite{Rab69b} and \cite{Rab69a}, is an efficient way of evaluating the 
Z transform\footnote{Incidently, the chirp-Z transform implemented in 
{\tt MATLAB} version 4.1 does not follow Rabiner's, Schafer's and Rader's
papers.  A revised version of {\tt czt.m} is included in
appendix~\protect\ref{ee app:Matlab CZT}.} 
\begin{equation}
X_k=\sum_{n=0}^{N-1}x_nz_k^{-n}
\end{equation}
along a spiral contour in the $Z$ plane where
\begin{equation}
z_k=AW^{-k}\qquad \mbox{for $k=0,1,\ldots,M-1$}
\end{equation}
$A$ is the start of the contour, $\arg W$ is the angular spacing between
samples and $\left|W\right|$ determines how quickly the contour moves 
towards or away from the origin.  The number of samples, $M$, in the contour
is independent of the number of data points, $N$.

When $M=N$, $A=1$, and $W=\e{j\frac{2\pi}{N}}$, the spiral contour is the
set of points spaced uniformly around the unit circle and the chirp-Z
transform becomes a discrete Fourier transform. 

Well-chosen values of $M$, $A$ and $W$ can achieve a higher resolution over 
a small part of the spectrum, and this is used in the following algorithm to
obtain a much better approximation to $\rML$ than 
algorithm~\ref{ee alg:min using DFT}. 

%============================================================================
\begin{algorithm}[Global Minimization of $J(\r)$ using a CZT]
\label{ee alg:min using CZT}\mbox{}\par

To obtain a rough estimate $\rWH$ of the location of the global minimum
$\rML$ of $J(\r)$ when it is known that $\rML$ lies within the range
$[\r_l,\r_u]$:
\begin{enumerate}
\item Choose the desired accuracy of the estimate $\rWH$ of $\rML$. This
accuracy $\epsilon_{\r}$ should be smaller than an ISAR range bin but
larger than half the radar wavelength
\begin{equation}
\frac{c}{2 f_0} < \epsilon_{\r} < \frac{c}{2 N \Delta f}
\end{equation}

\item Choose an integer $M$ (not necessarily a power of $2$) such that the 
resolution of the CZT is better than the desired accuracy $\epsilon_{\r}$
\begin{equation}
\frac{\r_u-\r_l}{M-1}<\epsilon_{\r}
\end{equation}

\item From the set of measurements $\md$ in (\ref{ml eqn:md}), form 
\begin{equation}
c_n=a_n\overline{b_n}\qquad\forall n=0,1,\ldots, N-1
\end{equation}

\item Given $M$, $\r_l$ and $\r_u$, define $A$ and $W$ by
\begin{equation}
A=\e{j\frac{4\pi}{c}\Delta f \r_l}
\end{equation}
and
\begin{equation}
W=\e{j\frac{4\pi}{c}\Delta f \frac{\r_u-\r_l}{M-1}}
\end{equation}

\item Take the CZT of the $\{c_n\}$ using the values of $M$, $A$ and $W$
just calculated to give the sequence $\{C_m\}$ for $m=0,1,\ldots, M-1$.

\item Find the index of the absolute maximum of the $\{C_m\}$
\begin{equation}
k=\arg\,\max_m\,\left|C_m\right|
\end{equation}

\item The estimate $\rWH$ of $\rML$ accurate to $\epsilon_{\r}$ is
\begin{equation}
\rWH= \r_l+k\frac{\r_u-\r_l}{M-1}
\end{equation}
\end{enumerate}
\end{algorithm}
%============================================================================

The correctness of the algorithm is stated formally in the following
theorem:

%============================================================================
\begin{theorem}[Global Minimization of $J(\r)$ using a CZT]
\label{ee thm:min using CZT}\mbox{}\par

Let $\rML=\arg\min_{\r}J(\r)$ where $J(\r)$ is the cost function given by
(\ref{ee eqn:J(r)}) and the stepped-frequency waveform is given by
(\ref{ee eqn:fn}).  Then $\rWH$, the estimate of $\rML$ calculated
using algorithm~\ref{ee alg:min using CZT}, attains the desired accuracy,
$\epsilon_{\r}$ providing
\begin{equation}
\epsilon_{\r} > \frac{c}{2f_0}
\end{equation}
\end{theorem}
%============================================================================

%============================================================================
\begin{proof}
This is an outline of the proof of theorem~\ref{ee thm:min using CZT}.  Most of
the additional steps needed to fill in the gaps and make it rigorous 
are given in the proof of theorem~\ref{ee thm:min using DFT} in 
appendix~\ref{ee app:min using DFT}.

Minimizing $J(\r)$ is equivalent to maximizing the bandpass signal 
\begin{equation}
g(\r)=\real{\e{-jk_0\r}\sum_{n=0}^{N-1}c_n\e{-jn\Delta k\r}}
\end{equation}
so that
\begin{equation}
\rML=\arg\max_{\r}g(\r)
\end{equation}
The complex envelope of $g(\r)$ is
\begin{equation}
g_{+}(\r)=\e{-jk_0\r}\sum_{n=0}^{N-1}c_n\e{-jn\Delta k\r}
\end{equation}
Because $g(\r)$ is a bandpass signal, the location of the global maximum of
$g(\r)$ differs from the location of the global maximum of
$\left|g_{+}(\r)\right|$ by at most half the period of the modulation of
$g(\r)$.  Therefore
\begin{equation}
\left|\rML-\arg\max_{\r}\left|g_{+}(\r)\right|\right|\leq\frac{c}{4f_0}
\end{equation}
Similarly, the peak in the CZT spectrum is at most half the resolution of
the CZT away from the global maximum of $\left|g_{+}(\r)\right|$
\begin{equation}
\left|\rWH-\arg\max_{\r}\left|g_{+}(\r)\right|\right|\leq\frac{1}{2}
\epsilon_{\r}
\end{equation}
Therefore using the triangle inequality
\begin{equation}
\left|\rWH-\rML\right|\leq \frac{c}{4f_0}+\frac{1}{2}\epsilon_{\r}
<\epsilon_{\r}
\end{equation}
and the accuracy in the estimate of $\rML$ is better than $\epsilon_{\r}$.
\end{proof}
%============================================================================

%%%%%%%%%%%%%%%%%%%%%%%%%%%%%%%%%%%%%%%%%%%%%%%%%%%%%%%%%%%%%%%%%%%%%%
\subsection{Justifying the Incorporation of Prior Knowledge}

Algorithm~\ref{ee alg:min using CZT} has suggested that one way to speed up 
the search for the global minimum of $J(\r)$ is to restrict the scope of the 
search by including some {\em a priori\/} information.  The theoretical
justification for this is discussed in the context of the maximum 
likelihood estimator.

Maximizing the likelihood function in maximum likelihood estimation is
justified on the basis that no {\em a priori\/} information about the
unknown parameter is available.  But when {\em a priori\/} information about
the unknown parameter is available, it should be used along with the
likelihood function.  In this situation, the appropriate estimator
of $\up$ is not the maximum likelihood estimator $\upML$ but the maximum {\em
a posteriori\/} estimator $\upMAP$ which maximizes the {\em a posteriori\/}
probability density function $\pr{\up|\md}$. 

The discussion of diffuse priors in section \ref{mp sec:dp} shows that when
the {\em a priori\/} information takes the form of a closed interval on
which the unknown parameter has uniform prior probability, the maximum {\em
a posteriori\/} estimator has the same form as the maximum likelihood
estimator restricted to that closed interval.

This justifies using {\em a priori\/} information to restrict the search
for the global minimum of $J(\r)$ to a particular portion of the ISAR range
ambiguity window in algorithm~\ref{ee alg:min using CZT}.  The estimate of
$\dr$ is no longer the maximum likelihood estimate but the maximum {\em a
posteriori\/} estimate.   Since the difference is one of name rather than
form, the estimate of $\dr$ will continue to be called the maximum likelihood
estimate, even though this is strictly speaking incorrect.

As an example, if a ship's maximum velocity is $v$, and two frequency
responses are measured separated by a time interval of length $T$, $\dr$
must lie in the interval $[-vT,vT]$.  If the prior distribution of $\dr$ is
uniform over this interval and the maximum likelihood estimator $\rML$ lies
within the interval, the maximum likelihood and the maximum {\em a
posteriori\/} estimates are identical.

%%%%%%%%%%%%%%%%%%%%%%%%%%%%%%%%%%%%%%%%%%%%%%%%%%%%%%%%%%%%%%%%%%%%%%%%%%%%%
\section{Exact Local Minimization}
\label{ee sec:elm}


Once one of the two approximate global minimization algorithms 
\ref{ee alg:min using DFT} or \ref{ee alg:min using CZT} have been used to obtain
a rough estimate of $\rML$, another step is needed to find the global
minimum of $J(\r)$ in the neighbourhood of the rough estimate $\rWH$.

The two exact local minimization algorithms presented here find the position
of $\r_0$, the local minimum of $J(\r)$ closest to any $\r$.  
Once one local minimum $\r_0$ has been located and $J(\r_0)$ found, this
process can be repeated for all the local minima of $J(\r)$ near $\rWH$.
Then the local minimum at which $J(\r_0)$ is smallest gives the location of
the global minimum of $J(\r)$, which is $\rML$.

The number of local minima which have to be examined depends on the accuracy
of the rough estimate $\rWH$.  For the global minimization algorithm using 
the DFT, this may mean $50$ or more local minima (depending on the radar's
parameters).  For the global minimization algorithm using the CZT,
this may mean only a few local minima.  Since the number of arithmetic
operations required to find one local minimum is about the same as the
number required to implement either of the approximate global minimization
algorithms, the maximum likelihood estimator can be found most efficiently 
by using algorithm~\ref{ee alg:min using CZT} with an accuracy that is 
at most a couple of wavelengths at frequency $f_0$.

The first local minimization algorithm assumes that because $J(\r)$ is
bandpass, the phase of a sinusoidal signal fitted to $J(\r)$ can be used to
locate $\r_0$.  The second method uses the likelihood equation for $\dr$ to
derive an iterative procedure that converges to $\r_0$.  This is shown to be
a contraction mapping and the rate of contraction is analysed to find
how many iterations are necessary to achieve any desired accuracy in $\r_0$.

%%%%%%%%%%%%%%%%%%%%%%%%%%%%%%%%%%%%%%%%%%%%%%%%%%%%%%%%%%%%%%%%%%%%%%%%%%%%%
\subsection{Minimization by Sinusoidal Parameter Estimation}

Given an initial value of $\r$, the approximate position $\r_0$ of
the local minimum of $J(\r)$ closest to $\r$ may be found using
the following algorithm.  

In practice, algorithm~\ref{ee alg:min with cm} on page 
\pageref{ee alg:min with cm} should be used as it is guaranteed to converge
to the exact location of the closest local minimum.  The statement and
derivation of algorithm \ref{ee alg:min with sinusoids} are included here
because the statements of the two algorithms are similar, and the
derivation of this algorithm conveys a degree of geometrical insight which
is absent in the purely analytical derivation of algorithm
\ref{ee alg:min with cm}.

%============================================================================
\begin{algorithm}[Approximate Local Minimization of $J(\r)$]
\label{ee alg:min with sinusoids}\mbox{}\par

Suppose that $\r_0$ is the position of the local minimum of $J(\r)$ closest
to $\r$.  Then an approximation to $\r_0$ may be obtained by solving
\begin{equation}
\tan\left(\overline{k}(\r_0-\r)\right)=
\frac{
  \imag{\ds\sum_{n=0}^{N-1}a_n\overline{b_n}\e{-jk_n\r}}
}{
  \real{\ds\sum_{n=0}^{N-1}a_n\overline{b_n}\e{-jk_n\r}}
}
\end{equation}
where $\overline{k}$ is $4\pi/c$ times the centre frequency of the 
stepped-frequency waveform. 

If the denominator of this equation is negative, a local maximum has been
found instead of a local minimum.  Let
\begin{equation}
\r'=\r_0+\frac{\pi}{\overline{k}}\sgn\left(\r-\r_0\right)
\end{equation}
and repeat the algorithm with $\r=\r'$ to find the closest local minimum.
\end{algorithm}
%============================================================================

An informal derivation of this algorithm will now be presented.  No attempt is
made to formalize how accurate the approximation to $\r_0$ may be; this will
be left for the next algorithm, which converges to the exact value of $\r_0$.

Given a value of $\r$, the local extremum of $J(\r)$ closest to $\r$ may be
found by assuming that the envelope of $J(\r)$ is constant in the
neighbourhood of $\r$.  This assumption is valid because the envelope varies
very slowly in comparison to the modulation since the stepped-frequency
waveform has a small relative bandwidth.
Therefore $J(\r)$ may be written
\begin{equation}\label{ee eqn:sin model}
J(\r)=A-B\cos\left(\overline{k}(\r_0-\r)\right)
\end{equation}
where $\r_0$ is the location of the local extremum closest to $\r$ and
$\lambda=2\pi/\overline{k}$ is the instantaneous period of $J(\r)$.  $A$ and
$B$ are constants.  $A$ is always positive and $B$ is positive if the local
extremum is a minimum, or negative if it is a maximum. 

Now
\begin{equation}
2B\sin\left(\overline{k}(\r_0-\r)\right)=J\left(\r-\frac{\lambda}{4}\right)
-J\left(\r+\frac{\lambda}{4}\right)
\end{equation}
and 
\begin{equation}
2B\cos\left(\overline{k}(\r_0-\r)\right)=
\frac{1}{2}\left[J\left(\r+\frac{\lambda}{2}\right)
+J\left(\r-\frac{\lambda}{2}\right)\right]
-J(\r)
\end{equation}
Cancelling out the unknown amplitude $B$ in these two equations shows that
\begin{equation}
\tan\left(\overline{k}(\r_0-\r)\right)=
\frac{
  J\left(\r-\frac{\lambda}{4}\right)-J\left(\r+\frac{\lambda}{4}\right)
}{
  \frac{1}{2}\left[J\left(\r+\frac{\lambda}{2}\right)
  +J\left(\r-\frac{\lambda}{2}\right)\right]-J(\r)
}
\end{equation}
Substitute in $J(\r)$ using (\ref{ee eqn:J(r)}) and expand each of the
$\left|\cdot\right|^2$.  After a small algebraic struggle,
\begin{equation}\label{ee eqn:sin tan}
\tan\left(\overline{k}(\r_0-\r)\right)=
\frac{
  \imag{
    \ds\sum_{n=0}^{N-1}\sin\left(\frac{\pi k_n}{2\overline{k}}\right)
    a_n\overline{b_n}\e{-jk_n\r}
  }
}{
  \real{
    \ds\sum_{n=0}^{N-1}\sin^2\left(\frac{\pi k_n}{2\overline{k}}\right)
    a_n\overline{b_n}\e{-jk_n\r}
  }
}
\end{equation}
Since the stepped-frequency waveform is assumed to have a narrow relative
bandwidth, $k_n/\overline{k}\approx 1$ so each of the 
$\sin\left(\pi k_n/2\overline{k}\right)$ terms is
\begin{equation}\label{ee eqn:neglect sin}
\sin\left(\frac{\pi k_n}{2\overline{k}}\right)\approx
\sin\left(\frac{\pi}{2}\right)=1
\end{equation}
Neglecting these in (\ref{ee eqn:sin tan}) gives
\begin{equation}\label{ee eqn:sin tan approx}
\tan\left(\overline{k}(\r_0-\r)\right)=
\frac{\imag{\ds\sum_{n=0}^{N-1}a_n\overline{b_n}\e{-jk_n\r}}}
{\real{\ds\sum_{n=0}^{N-1}a_n\overline{b_n}\e{-jk_n\r}}}
\end{equation}
as required.

This method of calculating $\r_0$ is independent of the sign of $B$, so
$\r_0$ is the extremum that is closest to $\r$, not necessarily 
the local minimum that is closest.  The extremum is a local
minimum if the denominator of (\ref{ee eqn:sin tan approx}) is positive.  
If the denominator is negative, increment the calculated value of $\r_0$ by
$\lambda/4$ and use it as $\r$ once again to find the local minimum.

The approximations inherent in equations (\ref{ee eqn:sin model}) and
(\ref{ee eqn:neglect sin}) are dependent on the ISAR's relative bandwidth.  
In the limit as
\begin{equation}
\frac{N\Delta k}{k_0}\rightarrow 0
\end{equation}
the approximations become exact, and the algorithm determines the
positions of the local minima of $J(\r)$ exactly.

%%%%%%%%%%%%%%%%%%%%%%%%%%%%%%%%%%%%%%%%%%%%%%%%%%%%%%%%%%%%%%%%%%%%%%%%%%%%%
\subsection{Minimization by Contraction Mapping}

Given an initial value of $\r$, the position $\r_0$ of the local minimum
of $J(\r)$ closest to $\r$ may be found to any desired accuracy using
the following iterative algorithm, whose derivation is given in 
appendix~\ref{ee app:min with cm}.

%============================================================================
\begin{algorithm}[Iterative Local Minimization of $J(\r)$]
\label{ee alg:min with cm}\mbox{}\par

Suppose that $\r_0$ is the position of the local extremum of $J(\r)$ closest
to $\r$.  Let $\r_1=\r$ and use the iterative equation
\begin{equation}
\tan\left(\overline{k}(\r_{k+1}-\r_k)\right)=
\frac{
  \imag{\ds\sum_{n=0}^{N-1}k_na_n\overline{b_n}\e{-jk_n\r_k}}
}{
  \real{\ds\sum_{n=0}^{N-1}k_na_n\overline{b_n}\e{-jk_n\r_k}}
}
\end{equation}
to generate a sequence $\{r_k\}$.  Here $\overline{k}$ is $4\pi/c$ times the 
centre frequency of the stepped-frequency waveform.   Then
\begin{equation}
\lim_{k\rightarrow\infty} \r_k=\r_0
\end{equation}

If the denominator of the iterative equation is negative, a local maximum 
has been found instead of a local minimum.  Let
\begin{equation}
\r_1=\r_0+\frac{\pi}{\overline{k}}\sgn\left(\r-\r_0\right)
\end{equation}
and start iterating again to find the closest local minimum.

If the denominator of the iterative equation is zero, $\r$ is exactly
mid-way between a local maximum and a local minimum.  Increase or decrease
$\r$ by $\pi/2\overline{k}$ if the numerator is respectively 
positive or negative.
\end{algorithm}
%============================================================================

The following theorem shows that the algorithm converges to the closest
local extremum to the initial estimate $\r$, and that the rate of
convergence is dependent on the ISAR's relative bandwidth.  

%============================================================================
\begin{theorem}[Iterative Local Extremum of $J(\r)$]
\label{ee thm:min with cm}\mbox{}\par

Suppose that $\r_0$ is the position of the local extremum of $J(\r)$ closest
to $\r$, where $J(\r)$ is the cost function in (\ref{ee eqn:J(r)}).  
Let $\r_1=\r$ and use the iterative equation
\begin{equation}\label{ee eqn:cm}
\tan\left(\overline{k}(\r_{k+1}-\r_k)\right)=
\frac{
  \imag{\ds\sum_{n=0}^{N-1}k_na_n\overline{b_n}\e{-jk_n\r_k}}
}{
  \real{\ds\sum_{n=0}^{N-1}k_na_n\overline{b_n}\e{-jk_n\r_k}}
}
\end{equation}
with $k_0\leq\overline{k}\leq k_{N-1}$ to generate a sequence $\{r_k\}$.  
Then $\r_k\rightarrow\r_0$ as $k\rightarrow\infty$ for all $\r$ whose
closest extremum of $J(\r)$ is at $\r_0$.

This is a contraction mapping whose fixed point is $\r_0$
\begin{equation}
\left|\r_{k+1}-\r_0\right|< \alpha^k \left|\r-\r_0\right|
\end{equation}
$\alpha$, the speed of convergence to $\r_0$ is given by
\begin{equation}
\alpha=\frac{\Delta_k}{\overline{k}}
\end{equation}
where
\begin{equation}
\Delta_k=\max_n\,\left|k_n-\overline{k}\right|
\end{equation}
\end{theorem}
%============================================================================

%============================================================================
\begin{proof}
See appendix~\ref{ee app:proof using cm}.
\end{proof}
%============================================================================


This proof shows that the algorithm is guaranteed to converge to $\r_0$ 
if $\Delta_k<\overline{k}$.  This is equivalent to
\begin{equation}
k_0\leq\overline{k}\leq k_{N-1}
\end{equation}

For a given stepped-frequency waveform and arbitrary target, 
the guaranteed rate of convergence $\alpha$ is maximized if
\begin{equation}
\overline{k}=\arg\min_{k_0\leq k\leq k_{N-1}}
\max_n \left|\frac{k_n}{k}-1\right|
\end{equation}
which implies, not unsurprisingly, that the best choice of $\overline{k}$ is
\begin{equation}
\overline{k}=\frac{1}{2}\left(k_0+k_{N-1}\right)
\end{equation}

For a target with a given geometry, orientation and radar scattering
characteristics, the optimal choice of $\overline{k}$ is much harder to
calculate.

In the limit as the stepped-frequency waveform's relative bandwidth tends to
zero, $\alpha\rightarrow 0$, the algorithm converges in a single step, and
the approximate algorithm~\ref{ee alg:min with sinusoids} becomes equivalent to
the iterative algorithm~\ref{ee alg:min with cm}.

 
To illustrate calculating the number of iterations required for a typical 
X-band ISAR image with parameters of $f_0=9.16$~GHz, $\Delta f=2$~MHz and 
$N=128$, 
\begin{equation}
\alpha=\frac{k_{N-1}-k_0}{k_{N-1}+k_0}
=\frac{f_{N-1}-f_0}{f_{N-1}+f_0}
=0.014
\end{equation}
Since the wavelength at X-band is about $30$~mm,
any initial guess $\r_1$ is at most $7.5$~mm from a local extremum of $J(\r)$.
One iteration determines $\r_0$ to an accuracy of $\pm0.1$~mm and two 
iterations gives an accuracy of $\pm1.5$~$\mu$m, which is more than good
enough for ISAR motion compensation.

Once one local minimum has been located, the uncertainty in the position
of the next one is much less than $7.5$~mm so only one iteration is all that
is required when hopping along adjacent local minima.

%%%%%%%%%%%%%%%%%%%%%%%%%%%%%%%%%%%%%%%%%%%%%%%%%%%%%%%%%%%%%%%%%%%%%%
\section{Summary of the Procedure to Obtain $\rML$}

In summary, the maximum likelihood estimator $\rML$ of the radial
distance $\dr$ a target moves in the interval between two measurements of
the target's frequency response, using a stepped-frequency radar operating at
$N$ uniformly spaced frequencies, is found as follows:

\begin{enumerate}
\item Use any prior knowledge to restrict $\dr$ to some subinterval
$[\r_l,\r_u]$ on which $\dr$ is uniformly probable.

\item Given the subinterval $[\r_l,\r_u]$ and the radar's initial frequency, 
$f_0$, use algorithm~\ref{ee alg:min using CZT} to determine a rough estimate
$\rWH$ of $\rML$ with an accuracy of $c/2f_0$.

\item Use algorithm~\ref{ee alg:min with cm} to determine the positions of the 
local minima of $J(\r)$ within a distance given by the accuracy in the 
previous step to $\rWH$.  The number of iterations of the contraction
mapping required for each local minimum may be calculated using
theorem~\ref{ee thm:min with cm}.

\item Find which of the local minima has the smallest $J(\r_0)$.  This is
now $\rML$, the global minimum of $J(\r)$ and the maximum likelihood
estimate of $\dr$
\end{enumerate}

%%%%%%%%%%%%%%%%%%%%%%%%%%%%%%%%%%%%%%%%%%%%%%%%%%%%%%%%%%%%%%%%%%%%%%%%%%%%%
\appendix{Proof of Theorem~\protect\ref{ee thm:min using DFT}}
\label{ee app:min using DFT}

The proof of theorem~\ref{ee thm:min using DFT} given here shows that the
rough estimates $\rWH$ of the location of the global minimum of $J(\r)$
satisfy
\begin{equation}
\left|\rWH-\rML\right|<\frac{c}{2N\Delta f}
\end{equation}
which means that $\rWH$ is less than one ISAR range bin away
from the global minimum of $J(\r)$.  

\begin{proof}
$J(\r)$ can be rewritten as
\begin{eqnarray}
J(\r)&=&\sum_{n=0}^{N-1} \left|a_n-b_n\e{jk_n\r}\right|^2\nn\\
&=&\sum_{n=0}^{N-1}\left(\left|a_n\right|^2+\left|b_n\right|^2\right)
-2\sum_{n=0}^{N-1}\real{a_n\conj{b_n}\e{-jk_n\r}}
\end{eqnarray}
so an equivalent definition of $\rML$ is
\begin{equation}
\rML=\arg\max_{\r}\,g(\r)
\end{equation}
where $g(\r)$ is 
\begin{eqnarray}
g(\r)&=&\sum_{n=0}^{N-1}\real{a_n\conj{b_n}\e{-jk_n\r}}\nn\\
&=&\real{\e{-jk_0\r}\sum_{n=0}^{N-1}c_n\e{-jn\Delta k\r}}
\end{eqnarray}
where $c_n=a_n\conj{b_n}$ and each $k_n$ has been split into 
\begin{equation}
k_n=k_0+n\Delta k
\end{equation}

From this, $g(\r)$ is a bandpass signal whose centre frequency is
$f_c=\frac{1}{2}f_0$ and bandwidth is $W=\frac{1}{2}N\Delta f$.%
\footnote{Strictly speaking, this is not the best choice for $f_c$ and $W$ 
because $f_0$ is the base frequency of the stepped-frequency waveform, not
the centre frequency.  The difference is not important providing 
$N\Delta f>f_0$, but the theorem remains true even if this condition is not 
satisfied.  Incidentally, $f_c$ and $W$ are half what may be expected
because of the factor of $2$ included in $k_n$ to account for all ranges
being measured as round-trip distances from radar to target and back again.} 
The envelope of $g(\r)$ is given by $\left|g_{+}(\r)\right|$, where
$g_{+}(\r)$ is the pre-envelope of $g(\r)$
\begin{equation}
g_{+}(\r)=\e{-jk_0\r}\sum_{n=0}^{N-1}c_n\e{-jn\Delta k\r}
\end{equation}
Formally, the pre-envelope is defined as the complex bandpass signal with
$g(\r)$ as its real part and the Hilbert transform of $g(\r)$ as its
imaginary part
\begin{equation}
g_{+}(\r)=g(\r)+j\check{g}(\r)
\end{equation}
Because $g(\r)$ is bandpass, the distance between the global maxima of 
$g(\r)$ and its envelope is at most half a carrier wavelength
\begin{equation}
\left|\rML-\arg\max_{\r}\left|g_{+}(\r)\right|\right|\leq\frac{c}{4f_0}
\end{equation}

Now
\begin{eqnarray}
\arg\max_{\r}\left|g_{+}(\r)\right|
&=&\arg\max_{\r}\left|\e{-jk_0\r}\sum_{n=0}^{N-1}c_n\e{-jn\Delta k\r}\right|\nn\\
&=&\arg\max_{\r}\left|\sum_{n=0}^{N-1}c_n\e{-jn\Delta k\r}\right|\nn\\
&=&\arg\max_{\r}\left|\sum_{n=0}^{N-1}c_n
\e{-j\frac{2\pi}{N}\,\frac{2N\Delta f}{c}\r\,n}\right|\nn\\
&=&\arg\max_{\r}\left|N{\cal F}\{c_n\}\right|
\end{eqnarray}
From this, the global maximum of the envelope of $g(\r)$ is found by the
peak in the magnitude of the Fourier transform of the $c_n$.  
This transform is implemented using a DFT, so the peak in
the discrete Fourier spectrum occurs at the discrete frequency that is
closest to the actual peak in the continuous spectrum.  The resolution is
$c/2N\Delta f$, therefore the error is at most half this, or
\begin{equation}
\left|\rWH-\arg\max_{\r}\left|g_{+}(\r)\right|\right|\leq\frac{c}{4N\Delta f}
\end{equation}
Finally, using the triangle inequality shows that
\begin{eqnarray}
\left|\rWH-\arg\max_{\r}\left|g_{+}(\r)\right|\right|
&=&\left|\left(\rWH-\rML\right)+\left(\rML-
\arg\max_{\r}\left|g_{+}(\r)\right|\right)\right|\nn\\
&\leq&\left|\rWH-\rML\right|
	+\left|\rML-\arg\max_{\r}\left|g_{+}(\r)\right|\right|\nn\\
&\leq&\frac{c}{4N\Delta f}+\frac{c}{4f_0}\nn\\
&<&\frac{c}{2N\Delta f}
\end{eqnarray}
as required.
\end{proof}

%%%%%%%%%%%%%%%%%%%%%%%%%%%%%%%%%%%%%%%%%%%%%%%%%%%%%%%%%%%%%%%%%%%%%%%%%%%%%
\appendix{Derivation of Algorithm~\protect\ref{ee alg:min with cm}}
\label{ee app:min with cm}

The derivation of algorithm~\ref{ee alg:min with cm} starts with the condition
for a local extremum of $J(\r)$ at $\r=\r_0$
\begin{equation}
\left.\frac{\del J(\r)}{\del\r}\right|_{\r=\r_0}=0
\end{equation}
Writing $J(\r)$ in (\ref{ee eqn:J(r)} as
\begin{equation}
J(\r)=\sum_{n=0}^{N-1} \left|a_n\right|^2+\sum_{n=0}^{N-1} \left|b_n\right|^2
-2\real{\sum_{n=0}^{N-1} a_n\overline{b_n}\e{-jk_n\r}}
\end{equation}
the existence of a local extremum at $\r=\r_0$ means that
\begin{equation}\label{ee eqn:temp1}
\imag{\sum_{n=0}^{N-1} k_na_n\overline{b_n}\e{-jk_n\r_0}}=0
\end{equation}
Using the relationship
\begin{equation}
k_n\r_0=\overline{k}(\r_0-\r)+k_n\r+(k_n-\overline{k})(\r_0-\r)
\end{equation}
equation (\ref{ee eqn:temp1}) can be written as
\begin{equation}\label{ee eqn:temp2}
\imag{\e{-j\overline{k}(\r_0-\r)}
\sum_{n=0}^{N-1} k_na_n\overline{b_n}\,\e{-jk_n\r}
\,\e{-j(k_n-\overline{k})(\r_0-\r)}}=0
\end{equation}
Expanding the imaginary component of (\ref{ee eqn:temp2}) explicitly gives
\begin{equation}\label{ee eqn:temp3}
\tan\left(\overline{k}(\r_0-\r)\right)=
\frac{\ds\imag{\sum_{n=0}^{N-1} k_na_n\overline{b_n}\e{-jk_n\r}
\e{-j(k_n-\overline{k})(\r_0-\r)}}}
{\ds\real{\sum_{n=0}^{N-1} k_na_n\overline{b_n}\e{-jk_n\r}
\e{-j(k_n-\overline{k})(\r_0-\r)}}}
\end{equation}
This cannot be solved for $\r_0$ directly because $\r_0$ appears on both 
the left and the right hand sides.  However, because $\r_0$ is the closest 
extremum to $\r$,
\begin{equation}\label{ee eqn:condit1}
\overline{k}\left|\r_0-\r\right|<\pi
\end{equation}
Furthermore, the stepped-frequency waveform has a narrow relative bandwidth
so that
\begin{equation}\label{ee eqn:condit2}
\left|k_n-\overline{k}\right|\ll\overline{k}
\end{equation}
Combining (\ref{ee eqn:condit1}) and (\ref{ee eqn:condit2}) shows that
\begin{equation}
\left|(k_n-\overline{k})(\r_0-\r)\right|\ll
\overline{k}\left|\r_0-\r\right|
<\pi
\end{equation}
Therefore each $\e{-j(k_n-\overline{k})(\r_0-\r)}$ in (\ref{ee eqn:temp3})
is very close to unity and (\ref{ee eqn:temp3}) may be written as the close
approximation
\begin{equation}
\tan\left(\overline{k}(\r_0-\r)\right)\approx
\frac{\ds\imag{\sum_{n=0}^{N-1} k_na_n\overline{b_n}\e{-jk_n\r}}}
{\ds\real{\sum_{n=0}^{N-1} k_na_n\overline{b_n}\e{-jk_n\r}}}
\end{equation}
which can be solved directly to give an approximate value for $\r_0$.

This is only an approximation, but it is an approximation which gets 
progressively better as $\r\rightarrow \r_0$.  So rewriting the
approximation as the iterative equation
\begin{equation}\label{ee eqn:cm soln}
\tan\left(\overline{k}(\r_{k+1}-\r_k)\right)=
\frac{\ds\imag{\sum_{n=0}^{N-1} k_na_n\overline{b_n}\e{-jk_n\r_k}}}
{\ds\real{\sum_{n=0}^{N-1} k_na_n\overline{b_n}\e{-jk_n\r_k}}}
\end{equation}
where $\r_1$ is an initial estimate of $\r_0$ and $\r_k$ is used to find
$\r_{k+1}$, gives a sequence $\{r_k\}$ which it is hoped converges to $\r_0$
\begin{equation}
\lim_{k\rightarrow\infty}\r_k=\r_0
\end{equation}

%%%%%%%%%%%%%%%%%%%%%%%%%%%%%%%%%%%%%%%%%%%%%%%%%%%%%%%%%%%%%%%%%%%%%%%%%%%%%
\appendix{Proof of Theorem~\protect\ref{ee thm:min with cm}}
\label{ee app:proof using cm}

Banach's Fixed Point Theorem will be used to show that the iterative 
equation (\ref{ee eqn:cm}) is a contraction mapping whose iterates $\{\r_k\}$ 
converge to the closest local extremum of $J(\r)$ at $\r=\r_0$.
This requires that when (\ref{ee eqn:cm}) is written as the mapping $T$
\begin{equation}
\r_{k+1}=T\left(\r_k\right)
\end{equation}
whose domain $R$ is the set of points that are less than an eighth of a
wavelength from $\r_0$
\begin{equation}
R=\left\{
\r\,\mbox{such that}\,\left|\r-\r_0\right|<\frac{\pi}{2\overline{k}}
\right\}
\end{equation}
the following two conditions be satisfied:
\begin{description}
\item[(i)]  $\r_0$ is a fixed point of $T$
\begin{equation}
T\left(\r_0\right)=\r_0
\end{equation}
and that
\item[(ii)] $T$ is a contraction for all $\r\in R$.  This means that
there exists some constant $0\leq\alpha<1$ such that
\begin{equation}
\left|T\left(\r\right)-\r_0\right|\leq\alpha \left|\r-\r_0\right|
\end{equation}
for all $\r\in R$.
\end{description}

The first condition is satisfied because if $\r_k=\r_0$, 
(\ref{ee eqn:temp1}) shows that 
\begin{equation}
\imag{\sum_{n=0}^{N-1} k_na_n\overline{b_n}\e{-jk_n\r_k}}=0
\end{equation}
But $\r_{k+1}=T(\r_k)$, which is
\begin{equation}
\imag{\e{-j\overline{k}(\r_{k+1}-\r_k)}
\sum_{n=0}^{N-1} k_na_n\overline{b_n}\e{-jk_n\r_k}}=0
\end{equation}
Therefore $\e{-j\overline{k}(\r_{k+1}-\r_k)}=\pm 1$ and
\begin{equation}
\r_{k+1}=\r_k+l\,\frac{\pi}{\overline{k}}
\end{equation}
for some integer $l$.  The only solution with $\r_{k+1}\in R$ occurs when
$l=0$.  Thus 
\begin{equation}
\r_{k+1}=\r_k
\end{equation}
and $\r_0$ is a fixed point of $T$.

To show that $T$ is a contraction for all $\r\in R$, define a new 
mapping $\r'_{k+1}=T'(\r'_k)$ from $T$ by changing variables
\begin{equation}
\r'_k=\r_k-\r_0
\end{equation}
Then $T'$ is given implicitly by
\begin{equation}\label{ee eqn:T'}
\imag{\e{-j\overline{k}(\r'_{k+1}-\r'_k)}
\sum_{n=0}^{N-1} c_n\e{-jk_n\r'_k}}=0
\end{equation}
where $c_n=k_na_n\overline{b_n}\e{-jk_n\r_0}$ and the domain of $T'$ is
\begin{equation}
R'=\left\{
\r'\,\mbox{such that}\,\left|\r'\right|<\frac{\pi}{2\overline{k}}
\right\}
\end{equation}
Also define $\Delta_k$ as
\begin{equation}
\Delta_k=\max_n\,\left|k_n-\overline{k}\right|
\end{equation}

Since $\r_0$ is a fixed point of $T$, $\r'_0=\r_0-\r_0=0$ is a fixed point
of $T'$.  One consequence of this is
\begin{equation}
\imag{\sum_{n=0}^{N-1}c_n}=0
\end{equation}
Now consider the image $\r'_{k+1}$ of some $\r'_k\in R'$ under $T'$.

Multiplying the complex constant $c_n$ by $\e{-jk_n\r'_k}$ rotates $c_n$ by
an angle $\theta_n=-k_n\r'_k$ which satisfies
\begin{equation}
\left|\theta_n\right|=
k_n\left|\r'_k\right|\leq\left(\overline{k}+\Delta_k\right)\left|\r'_k\right|
\end{equation}
and 
\begin{equation}
\left|\theta_n+\overline{k}\r'_k\right|
=\left|(k_n-\overline{k})\r'_k\right|\leq\Delta_k\left|\r'_k\right|
\end{equation}
Let $\theta$ be the angle that $\ds\sum_{n=0}^{N-1}c_n$ is rotated to give
$\ds\sum_{n=0}^{N-1}c_n\e{-jk_n\r'_k}$.  Then
\begin{equation}
\min_n\, \theta_n\leq \theta\leq\max_n\,\theta_n
\end{equation}
In particular,
\begin{equation}
\left|\theta\right|\leq\left(\overline{k}+\Delta_k\right)\left|\r'_k\right|
\end{equation}
and 
\begin{equation}
\left|\theta+\overline{k}\r'_k\right|\leq\Delta_k\left|\r'_k\right|
\end{equation}

From (\ref{ee eqn:T'}) and the definition of $\theta$,
\begin{equation}
\e{-j\overline{k}(\r'_{k+1}-\r'_k)}=\pm\e{j\theta}
\end{equation}
Therefore
\begin{equation}
\overline{k}(\r'_{k+1}-\r'_k)=\theta+l\pi
\end{equation}
for some integer $l$ which is set appropriately to ensure that $\r'_{k+1}$
is in $R'$.  Thus
\begin{eqnarray}
\left|l\right|\pi
&=&\left|\overline{k}(\r'_{k+1}-\r'_k)-\theta\right| \nn\\
&\leq&\overline{k}\left|\r'_{k+1}\right|+\left|\theta+\overline{k}\r'_k\right|\nn\\
&<&\frac{\pi}{2}\left(1+\frac{\Delta_k}{\overline{k}}\right)
\end{eqnarray}
This has the unique solution $l=0$ if $\Delta_k<\overline{k}$.  Therefore
\begin{equation}
\overline{k}(\r'_{k+1}-\r'_k)=\theta
\end{equation}
which can be rearranged into
\begin{equation}
\left|\r'_{k+1}\right|=\frac{1}{\overline{k}}\left|\theta+\overline{k}\r'_k
\right|\leq \frac{\Delta_k}{\overline{k}}\,\left|\r'_k\right|
\end{equation}
So setting $\alpha'=\Delta_k/\overline{k}$ shows that
\begin{equation}
\left|\r'_{k+1}\right|\leq \alpha'\left|\r'_k\right|
\end{equation}
and $T'$ is a contraction mapping.

Expressing this in terms of $T$ and $\r_k$, $T$ is a contraction mapping 
with fixed point $\r_0$
\begin{equation}
\left|\r_{k+1}-\r_0\right|\leq \alpha\left|\r_k-\r_0\right|
\end{equation}
providing
\begin{equation}
\alpha=\frac{\Delta_k}{\overline{k}}<1
\end{equation}

Finally, applying the contraction mapping $k$ times in succession
shows that
\begin{equation}
\left|\r_{k+1}-\r_0\right|< \alpha^k \left|\r-\r_0\right|
\end{equation}
to complete the proof.

%%%%%%%%%%%%%%%%%%%%%%%%%%%%%%%%%%%%%%%%%%%%%%%%%%%%%%%%%%%%%%%%%%%%%%%%%%%%%
\appendix{Corrected {\tt MATLAB} Code to Implement the CZT}
\label{ee app:Matlab CZT}

The chirp-Z transform routine that comes with {\tt MATLAB} version 4.1 does
not follow the notation used by Rabiner in \cite{Rab69b} and \cite{Rab69a}.
This is a version of the MathWorks' {\tt czt.m} routine that has been
rewritten to follow Rabiner's notation.  It also works for a chirp-Z 
transform giving fewer elements that the input vector.

\begingroup
\singlespaced\small
\begin{verbatim}
function g = czt(x, m, w, a)
%CZT    Chirp-z transform.
%       G = CZT(X, M, W, A) returns:
%       G   the M-element chirp-z transform of data X,
%       X   the input matrix of signal columns
%       M   the length of the chirp z-transform
%       W   is the ratio W between points along the
%           complex-plane spiral contour of interest
%       A   the complex starting point on that contour

% Reference:
%  L.R. Rabiner, R.W. Schafer and C. M. Rader, The Chirp-Z 
%  Transform and Its Application, Bell System Technical Journal, 
%  May-June 1969, pp. 1249-1292.

% The original MATLAB algorithm, dated 15-Aug-90, did not work
% for M<N, and the circular convolution was specified
% differently from Rabiner et al.'s original paper.  
% Rewritten by Stephen Simmons (simmons@ee.mu.oz.au), 16-Feb-94,
% to correct these problems and to follow Rabiner's notation.

[n, k] = size(x); oldn = n;
if n == 1, x = x(:); [n, k] = size(x); end

if nargin < 2, m = length(x); end
if nargin < 3, w = exp(-sqrt(-1) .* 2 .* pi ./ m); end
if nargin < 4, a = 1; end

%------- Length for power-of-two fft.
l = 1;
while l < (m + n - 1), l = 2 .* l; end

%------- Premultiply data.
aa=a.^([0:-1:-(n-1)].');
wn=w.^((([0:1:n-1].').^2)/2);
wm=w.^(-(([0:1:m-1].').^2)/2);
y=[x .* [(aa.*wn)*ones(1,k)]; zeros(l-n,k) ];

v=zeros(l,1);
v(1:m,1)=wm;
v(l-n+2:l,1)=1./wn([n-1:-1:1]);

%------- Fast convolution via FFT.
fy = fft(y);
fv = fft(v);
fy = fy .* ( fv * ones(1,k) );
g  = ifft( fy );

%------- Final multiply.
g = g(1:m,:) .* [ wm * ones(1,k) ];
if oldn == 1, g = g.'; end
\end{verbatim}
\endgroup
